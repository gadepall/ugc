Given, $\vec {v_1, v_2,..., v_n}$ are orthonormal and form basis.\\
So, when they form column vectors of matrix $\vec A $, we can say that $\vec A$ is also orthonormal.
\begin{align}
    \therefore \quad & \vec {A^T A} =\vec I\\
    \implies & \vec{A^T A A^{-1}}= \vec {I A^{-1}}\\
    \implies & \vec {A^T} = \vec {A^{-1}}
\end{align}
Clearly, option 3 is the correct answer.
Let us consider an orthonormal basis for $\vec R^2$.\\[1em]
We can check that $\vec S$= \{ $\myvec{\frac{1}{\sqrt{5}}\\ \frac{2}{\sqrt{5}}}$, $\myvec{-\frac{2}{\sqrt{5}}\\ \frac{1}{\sqrt{5}}}$ \} forms an orthonormal basis.\\
Thus the matrix
\begin{align}
    \vec Q = \myvec{\frac{1}{\sqrt{5}} & -\frac{2}{\sqrt{5}}\\ \frac{2}{\sqrt{5}} & \frac{1}{\sqrt{5}} }
\end{align}
is the orthonormal matrix whose column vectors are the basis of $\vec R^2$.
For an orthonormal matrix $\vec A$,
\begin{align}
    & \vec{A^TA} =\vec I\\
    \implies &\det \brak{\vec{A^TA}}=\det\brak{ \vec I}\\
    \implies & \det \brak{\vec A^T} \det\brak{\vec A}= 1\\
    \implies & {\det\brak{\vec A}}^2 = 1 \quad \because \det\brak{\vec A}= \det\brak{\vec{A^T}}\\
    \implies & \det\brak{\vec A} =\pm 1
\end{align}
Also, here a contradictory example:\\
Let,
\begin{align}
    \vec R = \myvec{-\frac{1}{\sqrt{5}} & -\frac{2}{\sqrt{5}}\\ -\frac{2}{\sqrt{5}} & \frac{1}{\sqrt{5}}}
\end{align}
Clearly, $\vec R$  is an orthonormal matrix and the column vectors of it form an orthonormal basis of $\vec R^2$.
But,
\begin{align}
    \det {\vec R} &= \mydet {-\frac{1}{\sqrt{5}} & -\frac{2}{\sqrt{5}}\\ -\frac{2}{\sqrt{5}} & \frac{1}{\sqrt{5}}}\\
    &= -1
\end{align}
From the above two arguments it is clear that option 4 cannot be true.
