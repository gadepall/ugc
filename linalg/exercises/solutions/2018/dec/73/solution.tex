See Table \ref{eq:solutions/2018/dec/73/table:1}

\onecolumn
\begin{longtable}{|l|l|}
\hline
\text{Given} & \parbox{10cm}{\begin{align}
    \vec{M}=\myvec{1 & -1 & 1\\2 & 1 & 4\\-2 & 1 & -4}
\end{align}}\\
& One of the eigenvalue of $\vec{M}$ is 1\\
\hline
\text{Solution} & \text{Let the eigenvalues of matrix $\vec{M}$ of order $3 \times 3$ be $\lambda_1,\lambda_2,\lambda_3$}\\
& \text{From given , let $\lambda_1=1$}.\\
& \text{We know that sum of the eigenvalues of matrix is Trace of the matrix and product of }\\
& \text{eigenvalues of matrix is Determinant of the matrix.}\\
& Trace of the square matrix(Tr($\vec{M}$)) is the sum of the elements in the main diagonal of $\vec{M}$.\\
& \parbox{10cm}{\begin{align}
    Tr(\vec{M})&=1+1-4\\
    \implies Tr(\vec{M})&=-2\\
    \implies \lambda_1+\lambda_2+\lambda_3&=-2\\
    \implies \lambda_2+\lambda_3&=-3\\
    \implies \lambda_2&=-3-\lambda_3 \label{eq:solutions/2018/dec/73/1}
\end{align}}\\
& By row reducing the matrix $\vec{M}$, we get ,\\
& \parbox{10cm}{\begin{align}
 \vec{M}=\myvec{1 & -1 & 1 \\ 0 & 3 & 2\\ 0 & 0 & -\frac{4}{3}}
\end{align}}\\
\hline
& \parbox{10cm}{\begin{align}
    Det(\vec{M})&=1\left(3\left(-\frac{4}{3}\right)\right)=-4\\
    \implies \lambda_1\lambda_2\lambda_3&=-4\\
    \implies \lambda_2\lambda_3&=-4 \label{eq:solutions/2018/dec/73/2}
\end{align}}\\
& \text{Solving equations \eqref{eq:solutions/2018/dec/73/1} and \eqref{eq:solutions/2018/dec/73/2} one of the possibilities we get, }\\
& \parbox{10cm}{\begin{align}
   \lambda_1&=1\\
   \lambda_2&=1\\
   \lambda_3&=-4
\end{align}}\\
\hline
& \text{Using the eigenvalues the characteristic polynomial of matrix $\vec{M}$ is given by,}\\
& \parbox{10cm}{\begin{align}
   c(x)&=x^3+2x^2-7x+4=0 \label{eq:solutions/2018/dec/73/cx}
\end{align}}\\
& \text{The Cayley Hamilton Theorem states that every square matrix satisfies its own characteristic}\\
& \text{equation}.\\
& Using the above theorem, the equation \eqref{eq:solutions/2018/dec/73/cx} can be written as,\\
& \parbox{10cm}{\begin{align}
   \vec{M}^3+2\vec{M}^2-7\vec{M}+4\vec{I}&=0 \label{eq:solutions/2018/dec/73/cheq}\\
   \vec{M}^2+2\vec{M}-7\vec{I}+4\vec{M}^{-1}&=0\\
   \implies \vec{M}^{-1}&=-\frac{1}{4}(\vec{M}^2+2\vec{M}-7\vec{I}) \label{eq:solutions/2018/dec/73/4}
\end{align}}\\
\hline
\textbf{Statement 1} & \text{The minimal polynomial of $\vec{M}$ is $(x-1)(x+4)$}\\
\hline
& \text{If (x-1)(x+4) is a minimal polynomial of $\vec{M}$ then,}\\
& \parbox{10cm}{\begin{align}
   (\vec{M}-\vec{I})(\vec{M}+4\vec{I})&=\vec{0}_{3\times3}
\end{align}}\\
& \text{But,}\\
& \parbox{10cm}{\begin{align}
  (\vec{M}-\vec{I})(\vec{M}+4\vec{I})&=\myvec{-4 & -4 & -4\\2 & 2 & 2\\ 2 & 2 & 2}\neq \vec{0}_{3\times 3}
\end{align}}\\
& \parbox{10cm}{\begin{center}
\textbf{False Statement }
\end{center}}\\
\hline 
\textbf{Statement 2} & \text{The minimal polynomial of $\vec{M}$ is $(x-1)^2(x+4)$}\\
\hline
& Let m(x) be the minimal polynomial\\
& \parbox{10cm}{\begin{align}
   m(x)&=(x-1)^2(x+4) \label{eq:solutions/2018/dec/73/mx}\\
   &=x^3+2x^2-7x+4\\
   &=c(x) \notag
\end{align}}\\
& \text{In this case both minimal polynomial and characteristic polynomial were same.} \\
& \text{Therefore wecould say that equation \eqref{eq:solutions/2018/dec/73/mx} is the minimal polynomial of $\vec{M}$ as it satisfies }\\
& equation \eqref{eq:solutions/2018/dec/73/cheq} by Cayley Hamilton Theorem. \\
& \parbox{10cm}{\begin{center}
\textbf{True Statement }
\end{center}}\\
\hline 
\textbf{Statement 3} & \text{$\vec{M}$ is not diagonalizable.} \\
\hline
& $\vec{M}$ is diagonalizable if and only if its minimal polynomial is a product of distinct monic linear\\
\hline
& factors. From equation \eqref{eq:solutions/2018/dec/73/mx} we could see that one of the factor of minimal polynomial is \\
& repeated and it is not a linear factor. Therefore, Matrix $\vec{M}$ is not diagonalizable.\\
& \parbox{10cm}{\begin{center}
\textbf{True Statement }
\end{center}}\\
\hline 
\textbf{Statement 4} & \parbox{10cm}{\begin{align}
    \vec{M}^{-1}=\frac{1}{4}(\vec{M}+3\vec{I}) \label{eq:solutions/2018/dec/73/minv}
\end{align}}\\
\hline
& \text{Comparing equation \eqref{eq:solutions/2018/dec/73/4} and \eqref{eq:solutions/2018/dec/73/minv} we could say that the given statement is}\\
& \text{\textbf{False Statement}}.\\
\hline
\caption{Explanation}
\label{eq:solutions/2018/dec/73/table:1}
\end{longtable}

\twocolumn
