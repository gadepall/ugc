\renewcommand{\theequation}{\theenumi}
\renewcommand{\thefigure}{\theenumi}
\begin{enumerate}[label=\thesection.\arabic*.,ref=\thesection.\theenumi]
\numberwithin{equation}{enumi}
\numberwithin{figure}{enumi}

\item Consider the subspaces $W_1$ and $W_2$ of $\mathbb{R}^3$ given by
\begin{align}
W_1 &= \cbrak{\vec{x} \in \mathbb{R}^3: \myvec{1 & 1 & 1}\vec{x} = 0}
\\
W_2 &= \cbrak{\vec{x} \in \mathbb{R}^3: \myvec{1 & -1 & 1}\vec{x} = 0}.
\end{align}
If $W \subseteq \mathbb{R}^3$, such that 
\begin{enumerate}
\item $W \cap W_2 =$ span $\cbrak{\myvec{0\\1\\1}}$
\item $\cbrak{W \cap W_1} \perp \cbrak{W \cap W_2}$, 
\end{enumerate}
then 
\begin{enumerate}
\item $W =$ span $\cbrak{\myvec{0\\1\\-1},\myvec{0\\1\\1}}$
\item $W =$ span $\cbrak{\myvec{1\\0\\-1},\myvec{0\\1\\-1}}$
\item $W =$ span $\cbrak{\myvec{1\\0\\-1},\myvec{0\\1\\1}}$
\item $W =$ span $\cbrak{\myvec{1\\0\\-1},\myvec{1\\0\\1}}$
\end{enumerate}
\item Let
\begin{align}
C = \cbrak{\myvec{1 \\ 2},\myvec{2 \\ 1}}
\end{align}
be a basis of $\mathbb{R}^2$ and 
\begin{align}
T\myvec{x\\y} = \myvec{x+y \\ x-2y}.
\end{align}
If $T\sbrak{C}$ represents the matrix of $T$ with respect to the basis C then
which among the following is true?
\begin{enumerate}
\item $T\sbrak{C} = \myvec{-3 & -2\\3 & 1}$
\item $T\sbrak{C} = \myvec{3 & -2\\-3 & 1}$
\item $T\sbrak{C} = \myvec{-3 & -1\\3 & 2}$
\item $T\sbrak{C} = \myvec{3 & -1\\-3 & 2}$
\end{enumerate}
\item Let $W_1 = \cbrak{\vec{x} \in \mathbb{R}^4:}$
\begin{align}
 \myvec{1 & 1 & 1 & 0}\vec{x} = 0
\\
 \myvec{0 & 2 & 0 & 1}\vec{x} = 0
\\
 \myvec{2 & 0 & 2 & -1}\vec{x} = 0
\end{align}
and
$W_2 = \cbrak{\vec{x} \in \mathbb{R}^4:}$
\begin{align}
 \myvec{1 & 1 & 0 & 1}\vec{x} &= 0
\\
 \myvec{1 & 0 & 1 & -2}\vec{x} &= 0
\\
 \myvec{0 & 1 & 0 & -1}\vec{x} &= 0.
\end{align}
Then which among the following is true?
\begin{enumerate}
\item $\text{dim}\brak{W_1} = 1$
\item $\text{dim}\brak{W_2} = 2$
\item $\text{dim}\brak{W_1 \cap W_2} = 1$
\item $\text{dim}\brak{W_1+W_2} = 3$
\end{enumerate}
%
\item Let $A$ be an $n \times n$ complex matrix.  Assume that $A$ is self-adjoint and let $B$ denote the inverse of $A + \j I$. Then all eigenvalues of $\brak{A-\j I}B$ are 
\begin{enumerate}
\item purely imaginary
\item of modulus one
\item real
\item of modulus less than one
\end{enumerate}  
%
\item Let $\cbrak{u_1,u_2,\dots, u_n}$ be an orthonormal basis of $\mathbb{C}^n$ as column vectors.Let 
\begin{align}
\vec{M} &= \myvec{\vec{u}_1 & \vec{u}_2 & \dots & \vec{u}_k},
\\
\vec{N} &= \myvec{\vec{u}_{k+1} & \vec{u}_{k+2} & \dots & \vec{u}_n}
\end{align}
%
and $\vec{P}$ be the diagonal $k \times k$ matrix with diagonal entries $\alpha_1,\alpha_2, \dots, \alpha_k \in \mathbb{R}$.  Then which of the following is true?
\begin{enumerate}
\item rank$\brak{\vec{M}\vec{P}\vec{M}^*} = k$ whenever $\alpha_i \ne \alpha_j, 1 \le i, j \le k$.
\item tr$\brak{\vec{M}\vec{P}\vec{M}^*} = \sum_{i=1}^{k}\alpha_i$
\item rank$\brak{\vec{M}^*\vec{N}} = \min\brak{k,n-k}$
\item rank$\brak{\vec{M}\vec{M}^*+\vec{N}\vec{N}^*}  < n$.
\end{enumerate}  
%
\item Let $B: \mathbb{R} \times \mathbb{R} \to \mathbb{R}$ be the function
\begin{align}
B(a,b) = ab
\end{align}
Which of the following is true?
\begin{enumerate}
\item $B$ is a linear transformation
\item $B$ is a positive definite bilinear form
\item $B$ is symmetric but not positive definite
\item $B$ is neither linear nor bilinear
\end{enumerate}  
%
\item Let $\vec{A}$ be an invertible real $n \times n$ matrix.  Define a function
\begin{align}
F: \mathbb{R}^n \times \mathbb{R}^n \to \mathbb{R}
\end{align}
by 
\begin{align}
F(\vec{x},\vec{y}) = \brak{F\vec{x}}^T\vec{y}
\end{align}
Let $DF(\vec{x},\vec{y}) $ denote the derivate of $F$ at $(\vec{x},\vec{y}) $ which is 
a linear transformation from 
\begin{align}
\mathbb{R}^n \times \mathbb{R}^n \to \mathbb{R}
\end{align}
%
Then, if 
\begin{enumerate}
\item $\vec{x} \ne 0, DF(\vec{x},\vec{0}) \ne 0$ 
\item $\vec{y} \ne 0, DF(\vec{0},\vec{y}) \ne 0$ 
\item $(\vec{x},\vec{y}) \ne (\vec{0},\vec{0}), DF(\vec{x},\vec{0}) \ne 0$ 
\item $\vec{x} = 0$ or  $\vec{y} = 0, DF(\vec{x},\vec{y}) = 0$ 
\end{enumerate}  
%
\item Let
\begin{align}
T: \mathbb{R}^n \to \mathbb{R}^n
\end{align}
%
be a linear map that satisfies 
\begin{align}
T^2 = T-I.
\end{align}
Then which of the following is true?
\begin{enumerate}
\item $T$ is invertible. 
\item $T-I$ is not invertible. 
\item $T$ has a real eigenvalue. 
\item $T^3 = -I$ . 
\end{enumerate}
%
\item Let
\begin{align}
\vec{M} = 
\myvec
{
2 & 0 & 3 & 2 & 0 & -2
\\
0 & 1 & 0 & -1 & 3 & 4
\\
0 & 0 & 1 & 0 & 4 & 4
\\
1 & 1 & 1 & 0 & 1 & 1
}
\\
\vec{b}_1 = \myvec{5 \\ 1 \\ 1 \\ 4},
\vec{b}_2 = \myvec{5 \\ 1 \\ 3 \\ 3}.
\end{align}  
Then which of the following are true?
\begin{enumerate}
\item both systems $\vec{M}\vec{x} = \vec{b}_1$ and $\vec{M}\vec{x} = \vec{b}_2$ are inconsistent.
\item both systems $\vec{M}\vec{x} = \vec{b}_1$ and $\vec{M}\vec{x} = \vec{b}_2$ are consistent. 
\item the system $\vec{M}\vec{x} = \vec{b}_1-\vec{b}_2$ is consistent. 
\item the system $\vec{M}\vec{x} = \vec{b}_1-\vec{b}_2$ is inconsistent. 
\end{enumerate}
%
\item Let 
\begin{align}
\vec{M} = \myvec
{
1 & -1 & 1 \\
2 & 1 & 4 \\
-2 & 1 & -4 
}.
\end{align}
Given that 1 is an eigenvalue of $\vec{M}$, then which among the following
are correct?
\begin{enumerate}
\item The minimal polynomial of  $\vec{M}$ is $\brak{x-1}\brak{x+4}$ 
\item The minimal polynomial of  $\vec{M}$ is $\brak{x-1}^2\brak{x+4}$ 
\item   $\vec{M}$ is not diagonalizable.
\item $\vec{M}^{-1} = \frac{1}{4}\brak{\vec{M}+3\vec{I}}$. 
\end{enumerate}
%
\item Let $\vec{A}$ be a real matrix with characteristic polynomial $\brak{x-1}^3$.  Pick the correct statements from below:
\begin{enumerate}
\item $\vec{A}$ is necessarily diagonalizable.
\item If the minimal polynomial of  $\vec{A}$ is $\brak{x-1}^3$, then  $\vec{A}$ is diagonalizable.
\item  The characteristic polynomial of  $\vec{A}^2$ is $\brak{x-1}^3$
\item If $\vec{A}$ has exactly two Jordan blocks, then $\brak{\vec{A}-\vec{I}}^2$ is diagonalizable. 
\end{enumerate}
%
\item Let $P_3$ be the vector space of polynomails with real coefficients and of degree at most 3.  Consider the linear map
\begin{align}
T:P_3\to P_3
\end{align}
defined by 
\begin{align}
T\brak{p(x)} = p(x-1)+p(x+1)
\end{align}
%
Which of the following properties does the matrix of $T$ with respect to the standard basis
$B = \cbrak{1,x,x^2,x^3}$ of $P_3$ satisfy?
\begin{enumerate}
\item $det T = 0$.
\item $\brak{T-2I}^4 = 0$ but $\brak{T-2I}^3 \ne 0$.
\item $\brak{T-2I}^3 = 0$ but $\brak{T-2I}^2 \ne 0$.
\item 2 is an eigenvalue with multiplicity 4.
\end{enumerate}
%
\item Let $\vec{M}$ be an $n \times n$ Hermitian matrix of rank $k, k \ne n$.  If $\lambda \ne = 0$is an eigenvalue of $\vec{M}$ with corresponding unit column vector $\vec{u}$, then which of the
following are true?
\begin{enumerate}
\item rank$\brak{\vec{M}- \lambda \vec{u}\vec{u}^*} = k-1$.
\item rank$\brak{\vec{M}- \lambda \vec{u}\vec{u}^*} = k$.
\item rank$\brak{\vec{M}- \lambda \vec{u}\vec{u}^*} = k+1$.
\item $\brak{\vec{M}- \lambda \vec{u}\vec{u}^*}^n = \vec{M}^n - \lambda^n \vec{u}\vec{u}^*$.
\end{enumerate}
%
\item Define a real valued function $B$ on $\mathbb{R}^2 \times \mathbb{R}^2 $ as 
\begin{align}
B\brak{\vec{x},\vec{y}} = x_1y_1 - x_1y_2-x_2y_1+4x_2y_2
\end{align}
Let $\vec{v}_0 = \myvec{1\\0}$ and 
\begin{align}
W = \cbrak{\vec{v} \in \mathbb{R}^2: B(\vec{v}_0,\vec{v}) =0}
\end{align}
Then $W$
\begin{enumerate}
\item is not a subspace of $\mathbb{R}^2$.
\item equals $\vec{0}$.
\item is the y axis
\item is the line passing through $\myvec{0 \\ 0}$ and $\myvec{1 \\ 1}$.
\end{enumerate}
%
\item Consider the Quadratic forms
\begin{align}
Q_1(x,y) = xy
\\
Q_2(x,y) = x^2+2xy+y^2
\\
Q_3(x,y) = x^2+3xy+2y^2
\end{align}
%
on $\mathbb{R}^2$.  Choose the correct statements from below
\begin{enumerate}
\item $Q_1$ and $Q_2$ are equivalent.
\item $Q_1$ and $Q_3$ are equivalent.
\item $Q_2$ and $Q_3$ are equivalent.
\item all are equivalent.
\end{enumerate}

\item Consider a Markov Chain with state space $\cbrak{0,1,2}$ and transition matrix
\begin{align}
P = 
\begin{blockarray}{c@{\hspace{1pt}}rrr@{\hspace{3pt}}}
         & 0   & 1   & 2 \\
        \begin{block}{r@{\hspace{3pt}}@{\hspace{1pt}}
    (@{\hspace{1pt}}rrr@{\hspace{1pt}}@{\hspace{1pt}})}
        0 & \frac{1}{2} & \frac{1}{2} & 0  \\
        1 & 0 &\frac{1}{2}  & \frac{3}{4}  \\
%
        2 &  \frac{1}{3} & \frac{1}{3} & \frac{1}{3}  \\
        \end{block}
    \end{blockarray}
\end{align}
For any two states $i$ and $j$, let $p_{ij}^{(n)}$ denote the $n$-step transition probability of going from $i$ to $j$.  Identify correct statements.
\begin{enumerate}
\item $\lim_{n \to \infty} p_{11}^{(n)} = \frac{2}{9}$
\item $\lim_{n \to \infty} p_{21}^{(n)} = 0$
\item $\lim_{n \to \infty} p_{32}^{(n)} = \frac{1}{3}$
\item $\lim_{n \to \infty} p_{13}^{(n)} = \frac{1}{3}$
\end{enumerate}

\end{enumerate}
