\renewcommand{\theequation}{\theenumi}
\renewcommand{\thefigure}{\theenumi}
\begin{enumerate}[label=\thesection.\arabic*.,ref=\thesection.\theenumi]
\numberwithin{equation}{enumi}
\numberwithin{figure}{enumi}
\numberwithin{table}{enumi}

\item Let $\vec{A}$,$\vec{B}$ be n × n matrices.Which of the following equals trace$(\vec{A}^2 \vec{B}^2)$?\\

\begin{enumerate}
\item (trace$(\vec{A} \vec{B}))^2.$
\item trace$(\vec{A} \vec{B}^2 \vec{A}).$
\item trace$((\vec{A} \vec{B})^2).$
\item trace$(\vec{B} \vec{A} \vec{B} \vec{A}).$
\end{enumerate}
%
%
%
\solution
See Tables \ref{eq:solutions/2018/dec/106/table0} and \ref{eq:solutions/2018/dec/106/table1}


\onecolumn
	\begin{longtable}{|l|l|}
		\hline
		\multirow{3}{*}{Irreducible Markov Chain} 
		& \\
		& A Markov chain is $\textbf{irreducible}$ if all the states communicate with each other,\\
		& i.e., if there is only one communication class.\\
		&\\
		\hline
		\multirow{3}{*}{Aperiodic Markov Chain} & \\
		& If there is a self-transition in the chain ($p^{ii}>0$ for some i), then the chain is\\
		& called as $\textbf{aperiodic}$\\
		& \\
		\hline
		\multirow{3}{*}{Stationary Distribution} & \\
		& A stationary distribution of a Markov chain is a probability distribution that\\
		& remains unchanged in the Markov chain as time progresses. Typically, it is\\
		& represented as a row vector $\Vec{\pi}$ whose entries are probabilities summing to 1,\\ 
		& and given transition matrix $\textbf{P}$, it satisfies\\
		& \\
		&  \qquad \qquad  \qquad$\Vec{\pi} = \Vec{\pi} \textbf{P}$\\
		& \\
		\hline
\caption{}
\label{eq:solutions/2018/dec/106/table0}
	\end{longtable}
	\begin{longtable}{|l|l|}
		\hline
		\multirow{3}{*}{Drawing Transition diagram} 
		& \\
		& 
		
		$\begin{tikzpicture}[shorten >=1pt,node distance=2cm, scale =3, auto]
			\tikzstyle{every state}=[fill={rgb:black,1;white,10}]
			
			\node[state]   (q_1)                          {$1$};
			\node[state]   (q_2)  [right of=q_1]          {$2$};
			\node[state]   (q_3)  [below right of=q_1]          {$3$};
			
			\path[->]
			(q_1) edge [loop above] node {$\frac{1}{2}$}    (   )
			edge [bend left]  node {$\frac{1}{2}$}    (q_2)
			(q_2) edge [bend left]  node {$\frac{1}{2}$}    (q_3)
			edge [loop above] node {$\frac{1}{2}$}    ()
			(q_3) edge [bend left]  node {$\frac{1}{3}$}    (q_2)
			edge [bend left]  node {$\frac{1}{3}$}    (q_1)
			edge [loop below] node {$\frac{1}{3}$}    ();
		\end{tikzpicture}$
		
		\\  
		&\\
		&\\
		\hline
		\multirow{3}{*}{Checking whether the  } & \\
		& Here,\\chain is Irreducible
		& All the states are accessible to one another. \\and Aperiodic
		& $\implies$ They are in the same communication class. So, it is Irreducible.\\
		& \\
		& There exists the non- zero self-transition, which means that the chain \\
		& is Aperiodic.\\
		&\\ 
		& We know that if the Markov Chain is irreducible and aperiodic then \\
		& \qquad \qquad \qquad $\Vec{\pi}_{j} = \lim_{n \to \infty}P\{X_{n} = j\}$, $j = 1,...,N$ \\
		& These are the stationary probabilities. \\
		&\\
		\hline
		\multirow{3}{*}{Finding the Stationary} & \\
		& Stationary Probability can be represented as\\Probability Distributions
		& \qquad \qquad \qquad $\Vec{\pi} = \Vec{\pi} \vec{P}$\\
		& \\
		& \qquad $\implies$ $\myvec{v_{1}&&v_{2}&&v_{3}} = \myvec{v_{1}&&v_{2}&&v_{3}}\vec{P}$ \\
		& \\
		& Equating the above equation we get \\
		& \\
		& \qquad \qquad \qquad $\frac{1}{2}v_{1}-\frac{1}{3}v_{3} = 0$ $\label{eq:solutions/2018/dec/106/eq}$\\
		& \\
		& \qquad \qquad \qquad $\frac{1}{2}v_{1}-\frac{1}{2}v_{2} + \frac{1}{3}v_{3} = 0$\\
		& \\
		& \qquad \qquad \qquad $\frac{1}{2}v_{2}-\frac{2}{3}v_{3} = 0$\\
		& \\\
		& We see that summation of second and the third equation gives us the \\
		& first equation only. \\
		& And we know that the probability distribution will sum up to 1. \\
		& \\
		& \qquad \qquad \qquad $v_{1}+v_{2}+v_{3} = 1$ \\
		& \\
		& Therefore, we get the equation form as \\
		& \\
		& \qquad \qquad \qquad $\myvec{1&1&1\\\frac{1}{2}&0&\frac{-1}{3}\\\frac{1}{2}&\frac{-1}{2}&\frac{1}{3}}\myvec{v_{1}\\v_{2}\\v_{3}} = \myvec{1\\0\\0}$ \\
		& \\
		\hline
		\multirow{3}{*}{Solving the linear} & \\
		& The above linear equation can be solved using Gauss-Jordan method as\\equtions
		& \\
		& \qquad \qquad \qquad $\myvec{1&1&1&\vrule&1\\\frac{1}{2}&0&\frac{-1}{3}&\vrule&0\\\frac{1}{2}&\frac{-1}{2}&\frac{1}{3}&\vrule&0}$\\
		& \\
		& \qquad $\xleftrightarrow[]{R_2 \leftarrow R_2 - \frac{1}{2}R_1}$
		$\myvec{1&1&1&\vrule&1\\0&\frac{-1}{2}&\frac{-5}{6}&\vrule&\frac{-1}{2}\\\frac{1}{2}&\frac{-1}{2}&\frac{1}{3}&\vrule&0}$\\
		&\\
		& \qquad $\xleftrightarrow[]{R_3 \leftarrow R_3 - \frac{1}{2}R_1}$
		$\myvec{1&1&1&\vrule&1\\0&\frac{-1}{2}&\frac{-5}{6}&\vrule&\frac{-1}{2}\\0&-1&\frac{-1}{6}&\vrule&\frac{-1}{2}}$\\
		&\\
		& \qquad $\xleftrightarrow[]{R_2 \leftarrow \frac{-1}{2}R_2}$
		$\myvec{1&1&1&\vrule&1\\0&1&\frac{5}{3}&\vrule&1\\0&-1&\frac{-1}{6}&\vrule&\frac{-1}{2}}$\\
		&\\
		& \qquad $\xleftrightarrow[]{R_3 \leftarrow R_3 + R_2}$
		$\myvec{1&1&1&\vrule&1\\0&1&\frac{5}{3}&\vrule&1\\0&0&\frac{3}{2}&\vrule&\frac{1}{2}}$\\
		&\\
		& \qquad $\xleftrightarrow[]{R_3 \leftarrow \frac{3}{2}R_3}$
		$\myvec{1&1&1&\vrule&1\\0&1&\frac{5}{3}&\vrule&1\\0&0&1&\vrule&\frac{1}{3}}$\\
		&\\
		& \qquad $\xleftrightarrow[]{R_2 \leftarrow R_2 - \frac{5}{3}R_3}$
		$\myvec{1&1&1&\vrule&1\\0&1&0&\vrule&\frac{4}{9}\\0&0&1&\vrule&\frac{1}{3}}$\\
		&\\
		& \qquad $\xleftrightarrow[]{R_1 \leftarrow R_1 - R_3}$
		$\myvec{1&1&0&\vrule&\frac{2}{3}\\0&1&0&\vrule&\frac{4}{9}\\0&0&1&\vrule&\frac{1}{3}}$\\
		&\\
		& \qquad $\xleftrightarrow[]{R_1 \leftarrow R_1 - R_2}$
		$\myvec{1&0&0&\vrule&\frac{2}{9}\\0&1&0&\vrule&\frac{4}{9}\\0&0&1&\vrule&\frac{1}{3}}$\\
		&\\
		& $\therefore$, stationary probability distribution $\pi$ is given by \\
		& \qquad \qquad $\pi = \myvec{\frac{2}{9} & \frac{4}{9} & \frac{1}{3}}$ \\
		& \\
		\hline
		\multirow{3}{*}{Observations} & \\
		
		
		& Since the given transition probability matrix $\vec{P}$ is irreducible and aperiodic, \\
		& then $\lim_{n \to \infty} \vec{P}^{n}$ converges to a matrix with all rows identical and equal to $\vec{\pi}$. \\
		& \\
		& We were able to find $\vec{\pi}$ as $\myvec{\frac{2}{9} & \frac{4}{9} & \frac{1}{3}}$ \\
		& \\
		& $\lim_{n \to \infty} \vec{P}^{n} = \myvec{\frac{2}{9}&\frac{4}{9}&\frac{1}{3}\\\frac{2}{9}&\frac{4}{9}&\frac{1}{3}\\\frac{2}{9}&\frac{4}{9}&\frac{1}{3}}$\\
		& \\
		& From the above matrix, we get \\
		& \\
		& $\lim_{n \to \infty} \vec{P}^{n}_{11} = \frac{2}{9}$ \\
		&\\
		& $\lim_{n \to \infty} \vec{P}^{n}_{21} = \frac{2}{9}$ \\
		&\\
		& $\lim_{n \to \infty} \vec{P}^{n}_{32} = \frac{4}{9}$ \\
		&\\
		& $\lim_{n \to \infty} \vec{P}^{n}_{13} = \frac{1}{3}$ \\
		&\\
		\hline
		\multirow{3}{*}{Conclusion} & \\
		& From our observation we see that \\
		&\\
		& Options 1) and 4) are True.\\
		& \\
		\hline
\caption{}
\label{eq:solutions/2018/dec/106/table1}
	\end{longtable}
\twocolumn

\item The row space of a $20 \times 50$ matrix $\vec{A}$ has dimension 13.What is the dimension of the space of solution $\vec{Ax}=0$?
\begin{enumerate}
\item{7}
\item{13}
\item{33}
\item{37}
\end{enumerate}
%
%
\solution
See Tables \ref{eq:solutions/2018/dec/106/table0} and \ref{eq:solutions/2018/dec/106/table1}


\onecolumn
	\begin{longtable}{|l|l|}
		\hline
		\multirow{3}{*}{Irreducible Markov Chain} 
		& \\
		& A Markov chain is $\textbf{irreducible}$ if all the states communicate with each other,\\
		& i.e., if there is only one communication class.\\
		&\\
		\hline
		\multirow{3}{*}{Aperiodic Markov Chain} & \\
		& If there is a self-transition in the chain ($p^{ii}>0$ for some i), then the chain is\\
		& called as $\textbf{aperiodic}$\\
		& \\
		\hline
		\multirow{3}{*}{Stationary Distribution} & \\
		& A stationary distribution of a Markov chain is a probability distribution that\\
		& remains unchanged in the Markov chain as time progresses. Typically, it is\\
		& represented as a row vector $\Vec{\pi}$ whose entries are probabilities summing to 1,\\ 
		& and given transition matrix $\textbf{P}$, it satisfies\\
		& \\
		&  \qquad \qquad  \qquad$\Vec{\pi} = \Vec{\pi} \textbf{P}$\\
		& \\
		\hline
\caption{}
\label{eq:solutions/2018/dec/106/table0}
	\end{longtable}
	\begin{longtable}{|l|l|}
		\hline
		\multirow{3}{*}{Drawing Transition diagram} 
		& \\
		& 
		
		$\begin{tikzpicture}[shorten >=1pt,node distance=2cm, scale =3, auto]
			\tikzstyle{every state}=[fill={rgb:black,1;white,10}]
			
			\node[state]   (q_1)                          {$1$};
			\node[state]   (q_2)  [right of=q_1]          {$2$};
			\node[state]   (q_3)  [below right of=q_1]          {$3$};
			
			\path[->]
			(q_1) edge [loop above] node {$\frac{1}{2}$}    (   )
			edge [bend left]  node {$\frac{1}{2}$}    (q_2)
			(q_2) edge [bend left]  node {$\frac{1}{2}$}    (q_3)
			edge [loop above] node {$\frac{1}{2}$}    ()
			(q_3) edge [bend left]  node {$\frac{1}{3}$}    (q_2)
			edge [bend left]  node {$\frac{1}{3}$}    (q_1)
			edge [loop below] node {$\frac{1}{3}$}    ();
		\end{tikzpicture}$
		
		\\  
		&\\
		&\\
		\hline
		\multirow{3}{*}{Checking whether the  } & \\
		& Here,\\chain is Irreducible
		& All the states are accessible to one another. \\and Aperiodic
		& $\implies$ They are in the same communication class. So, it is Irreducible.\\
		& \\
		& There exists the non- zero self-transition, which means that the chain \\
		& is Aperiodic.\\
		&\\ 
		& We know that if the Markov Chain is irreducible and aperiodic then \\
		& \qquad \qquad \qquad $\Vec{\pi}_{j} = \lim_{n \to \infty}P\{X_{n} = j\}$, $j = 1,...,N$ \\
		& These are the stationary probabilities. \\
		&\\
		\hline
		\multirow{3}{*}{Finding the Stationary} & \\
		& Stationary Probability can be represented as\\Probability Distributions
		& \qquad \qquad \qquad $\Vec{\pi} = \Vec{\pi} \vec{P}$\\
		& \\
		& \qquad $\implies$ $\myvec{v_{1}&&v_{2}&&v_{3}} = \myvec{v_{1}&&v_{2}&&v_{3}}\vec{P}$ \\
		& \\
		& Equating the above equation we get \\
		& \\
		& \qquad \qquad \qquad $\frac{1}{2}v_{1}-\frac{1}{3}v_{3} = 0$ $\label{eq:solutions/2018/dec/106/eq}$\\
		& \\
		& \qquad \qquad \qquad $\frac{1}{2}v_{1}-\frac{1}{2}v_{2} + \frac{1}{3}v_{3} = 0$\\
		& \\
		& \qquad \qquad \qquad $\frac{1}{2}v_{2}-\frac{2}{3}v_{3} = 0$\\
		& \\\
		& We see that summation of second and the third equation gives us the \\
		& first equation only. \\
		& And we know that the probability distribution will sum up to 1. \\
		& \\
		& \qquad \qquad \qquad $v_{1}+v_{2}+v_{3} = 1$ \\
		& \\
		& Therefore, we get the equation form as \\
		& \\
		& \qquad \qquad \qquad $\myvec{1&1&1\\\frac{1}{2}&0&\frac{-1}{3}\\\frac{1}{2}&\frac{-1}{2}&\frac{1}{3}}\myvec{v_{1}\\v_{2}\\v_{3}} = \myvec{1\\0\\0}$ \\
		& \\
		\hline
		\multirow{3}{*}{Solving the linear} & \\
		& The above linear equation can be solved using Gauss-Jordan method as\\equtions
		& \\
		& \qquad \qquad \qquad $\myvec{1&1&1&\vrule&1\\\frac{1}{2}&0&\frac{-1}{3}&\vrule&0\\\frac{1}{2}&\frac{-1}{2}&\frac{1}{3}&\vrule&0}$\\
		& \\
		& \qquad $\xleftrightarrow[]{R_2 \leftarrow R_2 - \frac{1}{2}R_1}$
		$\myvec{1&1&1&\vrule&1\\0&\frac{-1}{2}&\frac{-5}{6}&\vrule&\frac{-1}{2}\\\frac{1}{2}&\frac{-1}{2}&\frac{1}{3}&\vrule&0}$\\
		&\\
		& \qquad $\xleftrightarrow[]{R_3 \leftarrow R_3 - \frac{1}{2}R_1}$
		$\myvec{1&1&1&\vrule&1\\0&\frac{-1}{2}&\frac{-5}{6}&\vrule&\frac{-1}{2}\\0&-1&\frac{-1}{6}&\vrule&\frac{-1}{2}}$\\
		&\\
		& \qquad $\xleftrightarrow[]{R_2 \leftarrow \frac{-1}{2}R_2}$
		$\myvec{1&1&1&\vrule&1\\0&1&\frac{5}{3}&\vrule&1\\0&-1&\frac{-1}{6}&\vrule&\frac{-1}{2}}$\\
		&\\
		& \qquad $\xleftrightarrow[]{R_3 \leftarrow R_3 + R_2}$
		$\myvec{1&1&1&\vrule&1\\0&1&\frac{5}{3}&\vrule&1\\0&0&\frac{3}{2}&\vrule&\frac{1}{2}}$\\
		&\\
		& \qquad $\xleftrightarrow[]{R_3 \leftarrow \frac{3}{2}R_3}$
		$\myvec{1&1&1&\vrule&1\\0&1&\frac{5}{3}&\vrule&1\\0&0&1&\vrule&\frac{1}{3}}$\\
		&\\
		& \qquad $\xleftrightarrow[]{R_2 \leftarrow R_2 - \frac{5}{3}R_3}$
		$\myvec{1&1&1&\vrule&1\\0&1&0&\vrule&\frac{4}{9}\\0&0&1&\vrule&\frac{1}{3}}$\\
		&\\
		& \qquad $\xleftrightarrow[]{R_1 \leftarrow R_1 - R_3}$
		$\myvec{1&1&0&\vrule&\frac{2}{3}\\0&1&0&\vrule&\frac{4}{9}\\0&0&1&\vrule&\frac{1}{3}}$\\
		&\\
		& \qquad $\xleftrightarrow[]{R_1 \leftarrow R_1 - R_2}$
		$\myvec{1&0&0&\vrule&\frac{2}{9}\\0&1&0&\vrule&\frac{4}{9}\\0&0&1&\vrule&\frac{1}{3}}$\\
		&\\
		& $\therefore$, stationary probability distribution $\pi$ is given by \\
		& \qquad \qquad $\pi = \myvec{\frac{2}{9} & \frac{4}{9} & \frac{1}{3}}$ \\
		& \\
		\hline
		\multirow{3}{*}{Observations} & \\
		
		
		& Since the given transition probability matrix $\vec{P}$ is irreducible and aperiodic, \\
		& then $\lim_{n \to \infty} \vec{P}^{n}$ converges to a matrix with all rows identical and equal to $\vec{\pi}$. \\
		& \\
		& We were able to find $\vec{\pi}$ as $\myvec{\frac{2}{9} & \frac{4}{9} & \frac{1}{3}}$ \\
		& \\
		& $\lim_{n \to \infty} \vec{P}^{n} = \myvec{\frac{2}{9}&\frac{4}{9}&\frac{1}{3}\\\frac{2}{9}&\frac{4}{9}&\frac{1}{3}\\\frac{2}{9}&\frac{4}{9}&\frac{1}{3}}$\\
		& \\
		& From the above matrix, we get \\
		& \\
		& $\lim_{n \to \infty} \vec{P}^{n}_{11} = \frac{2}{9}$ \\
		&\\
		& $\lim_{n \to \infty} \vec{P}^{n}_{21} = \frac{2}{9}$ \\
		&\\
		& $\lim_{n \to \infty} \vec{P}^{n}_{32} = \frac{4}{9}$ \\
		&\\
		& $\lim_{n \to \infty} \vec{P}^{n}_{13} = \frac{1}{3}$ \\
		&\\
		\hline
		\multirow{3}{*}{Conclusion} & \\
		& From our observation we see that \\
		&\\
		& Options 1) and 4) are True.\\
		& \\
		\hline
\caption{}
\label{eq:solutions/2018/dec/106/table1}
	\end{longtable}
\twocolumn

\item Given a $4 \times 4$ matrix $\vec{A}$, let $T:\mathbb R^4 \rightarrow \mathbb R^4$ be the linear transformation defined by $\vec{T}\vec{v} = \vec{A}\vec{v}$, where we think of $\mathbb R^4$ as the set of real $4 \times 1$ matrices. For which choices of $\vec{A}$ given below, do Image$(\vec{T})$ and Image$(\vec{T}^2)$ have respective dimensions 2 and 1? ($*$ denotes a nonzero entry)
\begin{enumerate}
    \item $\vec{A} = \myvec{0 & 0 & * & * \\ 0 & 0 & * & * \\ 0 & 0 & 0 & * \\ 0 & 0 & 0 & 0}$
    \item $\vec{A} = \myvec{0 & 0 & * & 0 \\ 0 & 0 & * & 0 \\ 0 & 0 & 0 & * \\ 0 & 0 & 0 & *}$
    \item $\vec{A} = \myvec{0 & 0 & 0 & 0 \\ 0 & 0 & 0 & 0 \\ 0 & 0 & 0 & * \\ 0 & 0 & * & 0}$
    \item $\vec{A} = \myvec{0 & 0 & 0 & 0 \\ 0 & 0 & 0 & 0 \\ 0 & 0 & * & * \\ 0 & 0 & * & *}$
\end{enumerate}
%
%
\solution
See Tables \ref{eq:solutions/2018/dec/106/table0} and \ref{eq:solutions/2018/dec/106/table1}


\onecolumn
	\begin{longtable}{|l|l|}
		\hline
		\multirow{3}{*}{Irreducible Markov Chain} 
		& \\
		& A Markov chain is $\textbf{irreducible}$ if all the states communicate with each other,\\
		& i.e., if there is only one communication class.\\
		&\\
		\hline
		\multirow{3}{*}{Aperiodic Markov Chain} & \\
		& If there is a self-transition in the chain ($p^{ii}>0$ for some i), then the chain is\\
		& called as $\textbf{aperiodic}$\\
		& \\
		\hline
		\multirow{3}{*}{Stationary Distribution} & \\
		& A stationary distribution of a Markov chain is a probability distribution that\\
		& remains unchanged in the Markov chain as time progresses. Typically, it is\\
		& represented as a row vector $\Vec{\pi}$ whose entries are probabilities summing to 1,\\ 
		& and given transition matrix $\textbf{P}$, it satisfies\\
		& \\
		&  \qquad \qquad  \qquad$\Vec{\pi} = \Vec{\pi} \textbf{P}$\\
		& \\
		\hline
\caption{}
\label{eq:solutions/2018/dec/106/table0}
	\end{longtable}
	\begin{longtable}{|l|l|}
		\hline
		\multirow{3}{*}{Drawing Transition diagram} 
		& \\
		& 
		
		$\begin{tikzpicture}[shorten >=1pt,node distance=2cm, scale =3, auto]
			\tikzstyle{every state}=[fill={rgb:black,1;white,10}]
			
			\node[state]   (q_1)                          {$1$};
			\node[state]   (q_2)  [right of=q_1]          {$2$};
			\node[state]   (q_3)  [below right of=q_1]          {$3$};
			
			\path[->]
			(q_1) edge [loop above] node {$\frac{1}{2}$}    (   )
			edge [bend left]  node {$\frac{1}{2}$}    (q_2)
			(q_2) edge [bend left]  node {$\frac{1}{2}$}    (q_3)
			edge [loop above] node {$\frac{1}{2}$}    ()
			(q_3) edge [bend left]  node {$\frac{1}{3}$}    (q_2)
			edge [bend left]  node {$\frac{1}{3}$}    (q_1)
			edge [loop below] node {$\frac{1}{3}$}    ();
		\end{tikzpicture}$
		
		\\  
		&\\
		&\\
		\hline
		\multirow{3}{*}{Checking whether the  } & \\
		& Here,\\chain is Irreducible
		& All the states are accessible to one another. \\and Aperiodic
		& $\implies$ They are in the same communication class. So, it is Irreducible.\\
		& \\
		& There exists the non- zero self-transition, which means that the chain \\
		& is Aperiodic.\\
		&\\ 
		& We know that if the Markov Chain is irreducible and aperiodic then \\
		& \qquad \qquad \qquad $\Vec{\pi}_{j} = \lim_{n \to \infty}P\{X_{n} = j\}$, $j = 1,...,N$ \\
		& These are the stationary probabilities. \\
		&\\
		\hline
		\multirow{3}{*}{Finding the Stationary} & \\
		& Stationary Probability can be represented as\\Probability Distributions
		& \qquad \qquad \qquad $\Vec{\pi} = \Vec{\pi} \vec{P}$\\
		& \\
		& \qquad $\implies$ $\myvec{v_{1}&&v_{2}&&v_{3}} = \myvec{v_{1}&&v_{2}&&v_{3}}\vec{P}$ \\
		& \\
		& Equating the above equation we get \\
		& \\
		& \qquad \qquad \qquad $\frac{1}{2}v_{1}-\frac{1}{3}v_{3} = 0$ $\label{eq:solutions/2018/dec/106/eq}$\\
		& \\
		& \qquad \qquad \qquad $\frac{1}{2}v_{1}-\frac{1}{2}v_{2} + \frac{1}{3}v_{3} = 0$\\
		& \\
		& \qquad \qquad \qquad $\frac{1}{2}v_{2}-\frac{2}{3}v_{3} = 0$\\
		& \\\
		& We see that summation of second and the third equation gives us the \\
		& first equation only. \\
		& And we know that the probability distribution will sum up to 1. \\
		& \\
		& \qquad \qquad \qquad $v_{1}+v_{2}+v_{3} = 1$ \\
		& \\
		& Therefore, we get the equation form as \\
		& \\
		& \qquad \qquad \qquad $\myvec{1&1&1\\\frac{1}{2}&0&\frac{-1}{3}\\\frac{1}{2}&\frac{-1}{2}&\frac{1}{3}}\myvec{v_{1}\\v_{2}\\v_{3}} = \myvec{1\\0\\0}$ \\
		& \\
		\hline
		\multirow{3}{*}{Solving the linear} & \\
		& The above linear equation can be solved using Gauss-Jordan method as\\equtions
		& \\
		& \qquad \qquad \qquad $\myvec{1&1&1&\vrule&1\\\frac{1}{2}&0&\frac{-1}{3}&\vrule&0\\\frac{1}{2}&\frac{-1}{2}&\frac{1}{3}&\vrule&0}$\\
		& \\
		& \qquad $\xleftrightarrow[]{R_2 \leftarrow R_2 - \frac{1}{2}R_1}$
		$\myvec{1&1&1&\vrule&1\\0&\frac{-1}{2}&\frac{-5}{6}&\vrule&\frac{-1}{2}\\\frac{1}{2}&\frac{-1}{2}&\frac{1}{3}&\vrule&0}$\\
		&\\
		& \qquad $\xleftrightarrow[]{R_3 \leftarrow R_3 - \frac{1}{2}R_1}$
		$\myvec{1&1&1&\vrule&1\\0&\frac{-1}{2}&\frac{-5}{6}&\vrule&\frac{-1}{2}\\0&-1&\frac{-1}{6}&\vrule&\frac{-1}{2}}$\\
		&\\
		& \qquad $\xleftrightarrow[]{R_2 \leftarrow \frac{-1}{2}R_2}$
		$\myvec{1&1&1&\vrule&1\\0&1&\frac{5}{3}&\vrule&1\\0&-1&\frac{-1}{6}&\vrule&\frac{-1}{2}}$\\
		&\\
		& \qquad $\xleftrightarrow[]{R_3 \leftarrow R_3 + R_2}$
		$\myvec{1&1&1&\vrule&1\\0&1&\frac{5}{3}&\vrule&1\\0&0&\frac{3}{2}&\vrule&\frac{1}{2}}$\\
		&\\
		& \qquad $\xleftrightarrow[]{R_3 \leftarrow \frac{3}{2}R_3}$
		$\myvec{1&1&1&\vrule&1\\0&1&\frac{5}{3}&\vrule&1\\0&0&1&\vrule&\frac{1}{3}}$\\
		&\\
		& \qquad $\xleftrightarrow[]{R_2 \leftarrow R_2 - \frac{5}{3}R_3}$
		$\myvec{1&1&1&\vrule&1\\0&1&0&\vrule&\frac{4}{9}\\0&0&1&\vrule&\frac{1}{3}}$\\
		&\\
		& \qquad $\xleftrightarrow[]{R_1 \leftarrow R_1 - R_3}$
		$\myvec{1&1&0&\vrule&\frac{2}{3}\\0&1&0&\vrule&\frac{4}{9}\\0&0&1&\vrule&\frac{1}{3}}$\\
		&\\
		& \qquad $\xleftrightarrow[]{R_1 \leftarrow R_1 - R_2}$
		$\myvec{1&0&0&\vrule&\frac{2}{9}\\0&1&0&\vrule&\frac{4}{9}\\0&0&1&\vrule&\frac{1}{3}}$\\
		&\\
		& $\therefore$, stationary probability distribution $\pi$ is given by \\
		& \qquad \qquad $\pi = \myvec{\frac{2}{9} & \frac{4}{9} & \frac{1}{3}}$ \\
		& \\
		\hline
		\multirow{3}{*}{Observations} & \\
		
		
		& Since the given transition probability matrix $\vec{P}$ is irreducible and aperiodic, \\
		& then $\lim_{n \to \infty} \vec{P}^{n}$ converges to a matrix with all rows identical and equal to $\vec{\pi}$. \\
		& \\
		& We were able to find $\vec{\pi}$ as $\myvec{\frac{2}{9} & \frac{4}{9} & \frac{1}{3}}$ \\
		& \\
		& $\lim_{n \to \infty} \vec{P}^{n} = \myvec{\frac{2}{9}&\frac{4}{9}&\frac{1}{3}\\\frac{2}{9}&\frac{4}{9}&\frac{1}{3}\\\frac{2}{9}&\frac{4}{9}&\frac{1}{3}}$\\
		& \\
		& From the above matrix, we get \\
		& \\
		& $\lim_{n \to \infty} \vec{P}^{n}_{11} = \frac{2}{9}$ \\
		&\\
		& $\lim_{n \to \infty} \vec{P}^{n}_{21} = \frac{2}{9}$ \\
		&\\
		& $\lim_{n \to \infty} \vec{P}^{n}_{32} = \frac{4}{9}$ \\
		&\\
		& $\lim_{n \to \infty} \vec{P}^{n}_{13} = \frac{1}{3}$ \\
		&\\
		\hline
		\multirow{3}{*}{Conclusion} & \\
		& From our observation we see that \\
		&\\
		& Options 1) and 4) are True.\\
		& \\
		\hline
\caption{}
\label{eq:solutions/2018/dec/106/table1}
	\end{longtable}
\twocolumn

\item %
Let $\vec{F}:\mathbb{R}^n\times\mathbb{R}^n\rightarrow\mathbb{R}$ be the function $\vec{F}(\vec{x},\vec{y})=\langle\vec{Ax},\vec{y}\rangle$, where $\langle,\rangle$ is the standard inner product of $\mathbb{R}^n$ and $\vec{A}$ is a $n\times n$ real matrix. Here D denotes the total derivative. Which of the following statements are correct?
\begin{enumerate}
    \item $(D\vec{F}(\vec{x},\vec{y}))(\vec{u},\vec{v})=\langle\vec{Au},\vec{y}\rangle+\langle\vec{Ax},\vec{v}\rangle$.
    \item $(D\vec{F}(\vec{x},\vec{y}))(0,0)=0$.
    \item $D\vec{F}(\vec{x},\vec{y})$ may not exist for some $(\vec{x},\vec{y})\in\mathbb{R}^n\times\mathbb{R}^n$.
    \item $D\vec{F}(\vec{x},\vec{y})$ does not exist at $(\vec{x},\vec{y})=(0,0)$.
\end{enumerate}
%
%
\solution
See Tables \ref{eq:solutions/2018/dec/106/table0} and \ref{eq:solutions/2018/dec/106/table1}


\onecolumn
	\begin{longtable}{|l|l|}
		\hline
		\multirow{3}{*}{Irreducible Markov Chain} 
		& \\
		& A Markov chain is $\textbf{irreducible}$ if all the states communicate with each other,\\
		& i.e., if there is only one communication class.\\
		&\\
		\hline
		\multirow{3}{*}{Aperiodic Markov Chain} & \\
		& If there is a self-transition in the chain ($p^{ii}>0$ for some i), then the chain is\\
		& called as $\textbf{aperiodic}$\\
		& \\
		\hline
		\multirow{3}{*}{Stationary Distribution} & \\
		& A stationary distribution of a Markov chain is a probability distribution that\\
		& remains unchanged in the Markov chain as time progresses. Typically, it is\\
		& represented as a row vector $\Vec{\pi}$ whose entries are probabilities summing to 1,\\ 
		& and given transition matrix $\textbf{P}$, it satisfies\\
		& \\
		&  \qquad \qquad  \qquad$\Vec{\pi} = \Vec{\pi} \textbf{P}$\\
		& \\
		\hline
\caption{}
\label{eq:solutions/2018/dec/106/table0}
	\end{longtable}
	\begin{longtable}{|l|l|}
		\hline
		\multirow{3}{*}{Drawing Transition diagram} 
		& \\
		& 
		
		$\begin{tikzpicture}[shorten >=1pt,node distance=2cm, scale =3, auto]
			\tikzstyle{every state}=[fill={rgb:black,1;white,10}]
			
			\node[state]   (q_1)                          {$1$};
			\node[state]   (q_2)  [right of=q_1]          {$2$};
			\node[state]   (q_3)  [below right of=q_1]          {$3$};
			
			\path[->]
			(q_1) edge [loop above] node {$\frac{1}{2}$}    (   )
			edge [bend left]  node {$\frac{1}{2}$}    (q_2)
			(q_2) edge [bend left]  node {$\frac{1}{2}$}    (q_3)
			edge [loop above] node {$\frac{1}{2}$}    ()
			(q_3) edge [bend left]  node {$\frac{1}{3}$}    (q_2)
			edge [bend left]  node {$\frac{1}{3}$}    (q_1)
			edge [loop below] node {$\frac{1}{3}$}    ();
		\end{tikzpicture}$
		
		\\  
		&\\
		&\\
		\hline
		\multirow{3}{*}{Checking whether the  } & \\
		& Here,\\chain is Irreducible
		& All the states are accessible to one another. \\and Aperiodic
		& $\implies$ They are in the same communication class. So, it is Irreducible.\\
		& \\
		& There exists the non- zero self-transition, which means that the chain \\
		& is Aperiodic.\\
		&\\ 
		& We know that if the Markov Chain is irreducible and aperiodic then \\
		& \qquad \qquad \qquad $\Vec{\pi}_{j} = \lim_{n \to \infty}P\{X_{n} = j\}$, $j = 1,...,N$ \\
		& These are the stationary probabilities. \\
		&\\
		\hline
		\multirow{3}{*}{Finding the Stationary} & \\
		& Stationary Probability can be represented as\\Probability Distributions
		& \qquad \qquad \qquad $\Vec{\pi} = \Vec{\pi} \vec{P}$\\
		& \\
		& \qquad $\implies$ $\myvec{v_{1}&&v_{2}&&v_{3}} = \myvec{v_{1}&&v_{2}&&v_{3}}\vec{P}$ \\
		& \\
		& Equating the above equation we get \\
		& \\
		& \qquad \qquad \qquad $\frac{1}{2}v_{1}-\frac{1}{3}v_{3} = 0$ $\label{eq:solutions/2018/dec/106/eq}$\\
		& \\
		& \qquad \qquad \qquad $\frac{1}{2}v_{1}-\frac{1}{2}v_{2} + \frac{1}{3}v_{3} = 0$\\
		& \\
		& \qquad \qquad \qquad $\frac{1}{2}v_{2}-\frac{2}{3}v_{3} = 0$\\
		& \\\
		& We see that summation of second and the third equation gives us the \\
		& first equation only. \\
		& And we know that the probability distribution will sum up to 1. \\
		& \\
		& \qquad \qquad \qquad $v_{1}+v_{2}+v_{3} = 1$ \\
		& \\
		& Therefore, we get the equation form as \\
		& \\
		& \qquad \qquad \qquad $\myvec{1&1&1\\\frac{1}{2}&0&\frac{-1}{3}\\\frac{1}{2}&\frac{-1}{2}&\frac{1}{3}}\myvec{v_{1}\\v_{2}\\v_{3}} = \myvec{1\\0\\0}$ \\
		& \\
		\hline
		\multirow{3}{*}{Solving the linear} & \\
		& The above linear equation can be solved using Gauss-Jordan method as\\equtions
		& \\
		& \qquad \qquad \qquad $\myvec{1&1&1&\vrule&1\\\frac{1}{2}&0&\frac{-1}{3}&\vrule&0\\\frac{1}{2}&\frac{-1}{2}&\frac{1}{3}&\vrule&0}$\\
		& \\
		& \qquad $\xleftrightarrow[]{R_2 \leftarrow R_2 - \frac{1}{2}R_1}$
		$\myvec{1&1&1&\vrule&1\\0&\frac{-1}{2}&\frac{-5}{6}&\vrule&\frac{-1}{2}\\\frac{1}{2}&\frac{-1}{2}&\frac{1}{3}&\vrule&0}$\\
		&\\
		& \qquad $\xleftrightarrow[]{R_3 \leftarrow R_3 - \frac{1}{2}R_1}$
		$\myvec{1&1&1&\vrule&1\\0&\frac{-1}{2}&\frac{-5}{6}&\vrule&\frac{-1}{2}\\0&-1&\frac{-1}{6}&\vrule&\frac{-1}{2}}$\\
		&\\
		& \qquad $\xleftrightarrow[]{R_2 \leftarrow \frac{-1}{2}R_2}$
		$\myvec{1&1&1&\vrule&1\\0&1&\frac{5}{3}&\vrule&1\\0&-1&\frac{-1}{6}&\vrule&\frac{-1}{2}}$\\
		&\\
		& \qquad $\xleftrightarrow[]{R_3 \leftarrow R_3 + R_2}$
		$\myvec{1&1&1&\vrule&1\\0&1&\frac{5}{3}&\vrule&1\\0&0&\frac{3}{2}&\vrule&\frac{1}{2}}$\\
		&\\
		& \qquad $\xleftrightarrow[]{R_3 \leftarrow \frac{3}{2}R_3}$
		$\myvec{1&1&1&\vrule&1\\0&1&\frac{5}{3}&\vrule&1\\0&0&1&\vrule&\frac{1}{3}}$\\
		&\\
		& \qquad $\xleftrightarrow[]{R_2 \leftarrow R_2 - \frac{5}{3}R_3}$
		$\myvec{1&1&1&\vrule&1\\0&1&0&\vrule&\frac{4}{9}\\0&0&1&\vrule&\frac{1}{3}}$\\
		&\\
		& \qquad $\xleftrightarrow[]{R_1 \leftarrow R_1 - R_3}$
		$\myvec{1&1&0&\vrule&\frac{2}{3}\\0&1&0&\vrule&\frac{4}{9}\\0&0&1&\vrule&\frac{1}{3}}$\\
		&\\
		& \qquad $\xleftrightarrow[]{R_1 \leftarrow R_1 - R_2}$
		$\myvec{1&0&0&\vrule&\frac{2}{9}\\0&1&0&\vrule&\frac{4}{9}\\0&0&1&\vrule&\frac{1}{3}}$\\
		&\\
		& $\therefore$, stationary probability distribution $\pi$ is given by \\
		& \qquad \qquad $\pi = \myvec{\frac{2}{9} & \frac{4}{9} & \frac{1}{3}}$ \\
		& \\
		\hline
		\multirow{3}{*}{Observations} & \\
		
		
		& Since the given transition probability matrix $\vec{P}$ is irreducible and aperiodic, \\
		& then $\lim_{n \to \infty} \vec{P}^{n}$ converges to a matrix with all rows identical and equal to $\vec{\pi}$. \\
		& \\
		& We were able to find $\vec{\pi}$ as $\myvec{\frac{2}{9} & \frac{4}{9} & \frac{1}{3}}$ \\
		& \\
		& $\lim_{n \to \infty} \vec{P}^{n} = \myvec{\frac{2}{9}&\frac{4}{9}&\frac{1}{3}\\\frac{2}{9}&\frac{4}{9}&\frac{1}{3}\\\frac{2}{9}&\frac{4}{9}&\frac{1}{3}}$\\
		& \\
		& From the above matrix, we get \\
		& \\
		& $\lim_{n \to \infty} \vec{P}^{n}_{11} = \frac{2}{9}$ \\
		&\\
		& $\lim_{n \to \infty} \vec{P}^{n}_{21} = \frac{2}{9}$ \\
		&\\
		& $\lim_{n \to \infty} \vec{P}^{n}_{32} = \frac{4}{9}$ \\
		&\\
		& $\lim_{n \to \infty} \vec{P}^{n}_{13} = \frac{1}{3}$ \\
		&\\
		\hline
		\multirow{3}{*}{Conclusion} & \\
		& From our observation we see that \\
		&\\
		& Options 1) and 4) are True.\\
		& \\
		\hline
\caption{}
\label{eq:solutions/2018/dec/106/table1}
	\end{longtable}
\twocolumn

\item 	%
	An $n\times n$ complex matrix $\vec{A}$ satisfies $\vec{A}^{k} = \vec{I}_n$. the $n\times n$ identity matrix, where $k$ is a positive integer $>$ 1. Suppose 1 is not an eigenvalue of $\vec{A}$. Then which of the following statements are necessarily true?\\
	
	\begin{enumerate}
		\item $\vec{A}$ is diagonalizable. \\
		\item $\vec{A}+\Vec{A}^2+...+\vec{A}^{k-1} = 0$, the $n\times n$ zero matrix. \\
		\item $tr(\vec{A})+tr(\Vec{A}^2)+...+tr(\vec{A}^{k-1}) = -n$ \\
		\item $\vec{A}^{-1}+\Vec{A}^{-2}+...+\vec{A}^{-(k-1)} = -\vec{I}_n$
	\end{enumerate}
%
%
\solution
See Tables \ref{eq:solutions/2018/dec/106/table0} and \ref{eq:solutions/2018/dec/106/table1}


\onecolumn
	\begin{longtable}{|l|l|}
		\hline
		\multirow{3}{*}{Irreducible Markov Chain} 
		& \\
		& A Markov chain is $\textbf{irreducible}$ if all the states communicate with each other,\\
		& i.e., if there is only one communication class.\\
		&\\
		\hline
		\multirow{3}{*}{Aperiodic Markov Chain} & \\
		& If there is a self-transition in the chain ($p^{ii}>0$ for some i), then the chain is\\
		& called as $\textbf{aperiodic}$\\
		& \\
		\hline
		\multirow{3}{*}{Stationary Distribution} & \\
		& A stationary distribution of a Markov chain is a probability distribution that\\
		& remains unchanged in the Markov chain as time progresses. Typically, it is\\
		& represented as a row vector $\Vec{\pi}$ whose entries are probabilities summing to 1,\\ 
		& and given transition matrix $\textbf{P}$, it satisfies\\
		& \\
		&  \qquad \qquad  \qquad$\Vec{\pi} = \Vec{\pi} \textbf{P}$\\
		& \\
		\hline
\caption{}
\label{eq:solutions/2018/dec/106/table0}
	\end{longtable}
	\begin{longtable}{|l|l|}
		\hline
		\multirow{3}{*}{Drawing Transition diagram} 
		& \\
		& 
		
		$\begin{tikzpicture}[shorten >=1pt,node distance=2cm, scale =3, auto]
			\tikzstyle{every state}=[fill={rgb:black,1;white,10}]
			
			\node[state]   (q_1)                          {$1$};
			\node[state]   (q_2)  [right of=q_1]          {$2$};
			\node[state]   (q_3)  [below right of=q_1]          {$3$};
			
			\path[->]
			(q_1) edge [loop above] node {$\frac{1}{2}$}    (   )
			edge [bend left]  node {$\frac{1}{2}$}    (q_2)
			(q_2) edge [bend left]  node {$\frac{1}{2}$}    (q_3)
			edge [loop above] node {$\frac{1}{2}$}    ()
			(q_3) edge [bend left]  node {$\frac{1}{3}$}    (q_2)
			edge [bend left]  node {$\frac{1}{3}$}    (q_1)
			edge [loop below] node {$\frac{1}{3}$}    ();
		\end{tikzpicture}$
		
		\\  
		&\\
		&\\
		\hline
		\multirow{3}{*}{Checking whether the  } & \\
		& Here,\\chain is Irreducible
		& All the states are accessible to one another. \\and Aperiodic
		& $\implies$ They are in the same communication class. So, it is Irreducible.\\
		& \\
		& There exists the non- zero self-transition, which means that the chain \\
		& is Aperiodic.\\
		&\\ 
		& We know that if the Markov Chain is irreducible and aperiodic then \\
		& \qquad \qquad \qquad $\Vec{\pi}_{j} = \lim_{n \to \infty}P\{X_{n} = j\}$, $j = 1,...,N$ \\
		& These are the stationary probabilities. \\
		&\\
		\hline
		\multirow{3}{*}{Finding the Stationary} & \\
		& Stationary Probability can be represented as\\Probability Distributions
		& \qquad \qquad \qquad $\Vec{\pi} = \Vec{\pi} \vec{P}$\\
		& \\
		& \qquad $\implies$ $\myvec{v_{1}&&v_{2}&&v_{3}} = \myvec{v_{1}&&v_{2}&&v_{3}}\vec{P}$ \\
		& \\
		& Equating the above equation we get \\
		& \\
		& \qquad \qquad \qquad $\frac{1}{2}v_{1}-\frac{1}{3}v_{3} = 0$ $\label{eq:solutions/2018/dec/106/eq}$\\
		& \\
		& \qquad \qquad \qquad $\frac{1}{2}v_{1}-\frac{1}{2}v_{2} + \frac{1}{3}v_{3} = 0$\\
		& \\
		& \qquad \qquad \qquad $\frac{1}{2}v_{2}-\frac{2}{3}v_{3} = 0$\\
		& \\\
		& We see that summation of second and the third equation gives us the \\
		& first equation only. \\
		& And we know that the probability distribution will sum up to 1. \\
		& \\
		& \qquad \qquad \qquad $v_{1}+v_{2}+v_{3} = 1$ \\
		& \\
		& Therefore, we get the equation form as \\
		& \\
		& \qquad \qquad \qquad $\myvec{1&1&1\\\frac{1}{2}&0&\frac{-1}{3}\\\frac{1}{2}&\frac{-1}{2}&\frac{1}{3}}\myvec{v_{1}\\v_{2}\\v_{3}} = \myvec{1\\0\\0}$ \\
		& \\
		\hline
		\multirow{3}{*}{Solving the linear} & \\
		& The above linear equation can be solved using Gauss-Jordan method as\\equtions
		& \\
		& \qquad \qquad \qquad $\myvec{1&1&1&\vrule&1\\\frac{1}{2}&0&\frac{-1}{3}&\vrule&0\\\frac{1}{2}&\frac{-1}{2}&\frac{1}{3}&\vrule&0}$\\
		& \\
		& \qquad $\xleftrightarrow[]{R_2 \leftarrow R_2 - \frac{1}{2}R_1}$
		$\myvec{1&1&1&\vrule&1\\0&\frac{-1}{2}&\frac{-5}{6}&\vrule&\frac{-1}{2}\\\frac{1}{2}&\frac{-1}{2}&\frac{1}{3}&\vrule&0}$\\
		&\\
		& \qquad $\xleftrightarrow[]{R_3 \leftarrow R_3 - \frac{1}{2}R_1}$
		$\myvec{1&1&1&\vrule&1\\0&\frac{-1}{2}&\frac{-5}{6}&\vrule&\frac{-1}{2}\\0&-1&\frac{-1}{6}&\vrule&\frac{-1}{2}}$\\
		&\\
		& \qquad $\xleftrightarrow[]{R_2 \leftarrow \frac{-1}{2}R_2}$
		$\myvec{1&1&1&\vrule&1\\0&1&\frac{5}{3}&\vrule&1\\0&-1&\frac{-1}{6}&\vrule&\frac{-1}{2}}$\\
		&\\
		& \qquad $\xleftrightarrow[]{R_3 \leftarrow R_3 + R_2}$
		$\myvec{1&1&1&\vrule&1\\0&1&\frac{5}{3}&\vrule&1\\0&0&\frac{3}{2}&\vrule&\frac{1}{2}}$\\
		&\\
		& \qquad $\xleftrightarrow[]{R_3 \leftarrow \frac{3}{2}R_3}$
		$\myvec{1&1&1&\vrule&1\\0&1&\frac{5}{3}&\vrule&1\\0&0&1&\vrule&\frac{1}{3}}$\\
		&\\
		& \qquad $\xleftrightarrow[]{R_2 \leftarrow R_2 - \frac{5}{3}R_3}$
		$\myvec{1&1&1&\vrule&1\\0&1&0&\vrule&\frac{4}{9}\\0&0&1&\vrule&\frac{1}{3}}$\\
		&\\
		& \qquad $\xleftrightarrow[]{R_1 \leftarrow R_1 - R_3}$
		$\myvec{1&1&0&\vrule&\frac{2}{3}\\0&1&0&\vrule&\frac{4}{9}\\0&0&1&\vrule&\frac{1}{3}}$\\
		&\\
		& \qquad $\xleftrightarrow[]{R_1 \leftarrow R_1 - R_2}$
		$\myvec{1&0&0&\vrule&\frac{2}{9}\\0&1&0&\vrule&\frac{4}{9}\\0&0&1&\vrule&\frac{1}{3}}$\\
		&\\
		& $\therefore$, stationary probability distribution $\pi$ is given by \\
		& \qquad \qquad $\pi = \myvec{\frac{2}{9} & \frac{4}{9} & \frac{1}{3}}$ \\
		& \\
		\hline
		\multirow{3}{*}{Observations} & \\
		
		
		& Since the given transition probability matrix $\vec{P}$ is irreducible and aperiodic, \\
		& then $\lim_{n \to \infty} \vec{P}^{n}$ converges to a matrix with all rows identical and equal to $\vec{\pi}$. \\
		& \\
		& We were able to find $\vec{\pi}$ as $\myvec{\frac{2}{9} & \frac{4}{9} & \frac{1}{3}}$ \\
		& \\
		& $\lim_{n \to \infty} \vec{P}^{n} = \myvec{\frac{2}{9}&\frac{4}{9}&\frac{1}{3}\\\frac{2}{9}&\frac{4}{9}&\frac{1}{3}\\\frac{2}{9}&\frac{4}{9}&\frac{1}{3}}$\\
		& \\
		& From the above matrix, we get \\
		& \\
		& $\lim_{n \to \infty} \vec{P}^{n}_{11} = \frac{2}{9}$ \\
		&\\
		& $\lim_{n \to \infty} \vec{P}^{n}_{21} = \frac{2}{9}$ \\
		&\\
		& $\lim_{n \to \infty} \vec{P}^{n}_{32} = \frac{4}{9}$ \\
		&\\
		& $\lim_{n \to \infty} \vec{P}^{n}_{13} = \frac{1}{3}$ \\
		&\\
		\hline
		\multirow{3}{*}{Conclusion} & \\
		& From our observation we see that \\
		&\\
		& Options 1) and 4) are True.\\
		& \\
		\hline
\caption{}
\label{eq:solutions/2018/dec/106/table1}
	\end{longtable}
\twocolumn

\item Let $S$ be the set of 3x3 real matrices $\vec{A}$ with 
\begin{align}
    \vec{A}^T\vec{A}=\myvec{1&0&0\\0&0&0\\0&0&0}
\end{align}
Then the set contains:-\\
\begin{enumerate}
    \item a Nilpotent Matrix
    \item a matrix of rank one
    \item a matrix of rank two
    \item a non-zero skew symmetric matrix.
\end{enumerate}
%
%
\solution
See Tables \ref{eq:solutions/2018/dec/106/table0} and \ref{eq:solutions/2018/dec/106/table1}


\onecolumn
	\begin{longtable}{|l|l|}
		\hline
		\multirow{3}{*}{Irreducible Markov Chain} 
		& \\
		& A Markov chain is $\textbf{irreducible}$ if all the states communicate with each other,\\
		& i.e., if there is only one communication class.\\
		&\\
		\hline
		\multirow{3}{*}{Aperiodic Markov Chain} & \\
		& If there is a self-transition in the chain ($p^{ii}>0$ for some i), then the chain is\\
		& called as $\textbf{aperiodic}$\\
		& \\
		\hline
		\multirow{3}{*}{Stationary Distribution} & \\
		& A stationary distribution of a Markov chain is a probability distribution that\\
		& remains unchanged in the Markov chain as time progresses. Typically, it is\\
		& represented as a row vector $\Vec{\pi}$ whose entries are probabilities summing to 1,\\ 
		& and given transition matrix $\textbf{P}$, it satisfies\\
		& \\
		&  \qquad \qquad  \qquad$\Vec{\pi} = \Vec{\pi} \textbf{P}$\\
		& \\
		\hline
\caption{}
\label{eq:solutions/2018/dec/106/table0}
	\end{longtable}
	\begin{longtable}{|l|l|}
		\hline
		\multirow{3}{*}{Drawing Transition diagram} 
		& \\
		& 
		
		$\begin{tikzpicture}[shorten >=1pt,node distance=2cm, scale =3, auto]
			\tikzstyle{every state}=[fill={rgb:black,1;white,10}]
			
			\node[state]   (q_1)                          {$1$};
			\node[state]   (q_2)  [right of=q_1]          {$2$};
			\node[state]   (q_3)  [below right of=q_1]          {$3$};
			
			\path[->]
			(q_1) edge [loop above] node {$\frac{1}{2}$}    (   )
			edge [bend left]  node {$\frac{1}{2}$}    (q_2)
			(q_2) edge [bend left]  node {$\frac{1}{2}$}    (q_3)
			edge [loop above] node {$\frac{1}{2}$}    ()
			(q_3) edge [bend left]  node {$\frac{1}{3}$}    (q_2)
			edge [bend left]  node {$\frac{1}{3}$}    (q_1)
			edge [loop below] node {$\frac{1}{3}$}    ();
		\end{tikzpicture}$
		
		\\  
		&\\
		&\\
		\hline
		\multirow{3}{*}{Checking whether the  } & \\
		& Here,\\chain is Irreducible
		& All the states are accessible to one another. \\and Aperiodic
		& $\implies$ They are in the same communication class. So, it is Irreducible.\\
		& \\
		& There exists the non- zero self-transition, which means that the chain \\
		& is Aperiodic.\\
		&\\ 
		& We know that if the Markov Chain is irreducible and aperiodic then \\
		& \qquad \qquad \qquad $\Vec{\pi}_{j} = \lim_{n \to \infty}P\{X_{n} = j\}$, $j = 1,...,N$ \\
		& These are the stationary probabilities. \\
		&\\
		\hline
		\multirow{3}{*}{Finding the Stationary} & \\
		& Stationary Probability can be represented as\\Probability Distributions
		& \qquad \qquad \qquad $\Vec{\pi} = \Vec{\pi} \vec{P}$\\
		& \\
		& \qquad $\implies$ $\myvec{v_{1}&&v_{2}&&v_{3}} = \myvec{v_{1}&&v_{2}&&v_{3}}\vec{P}$ \\
		& \\
		& Equating the above equation we get \\
		& \\
		& \qquad \qquad \qquad $\frac{1}{2}v_{1}-\frac{1}{3}v_{3} = 0$ $\label{eq:solutions/2018/dec/106/eq}$\\
		& \\
		& \qquad \qquad \qquad $\frac{1}{2}v_{1}-\frac{1}{2}v_{2} + \frac{1}{3}v_{3} = 0$\\
		& \\
		& \qquad \qquad \qquad $\frac{1}{2}v_{2}-\frac{2}{3}v_{3} = 0$\\
		& \\\
		& We see that summation of second and the third equation gives us the \\
		& first equation only. \\
		& And we know that the probability distribution will sum up to 1. \\
		& \\
		& \qquad \qquad \qquad $v_{1}+v_{2}+v_{3} = 1$ \\
		& \\
		& Therefore, we get the equation form as \\
		& \\
		& \qquad \qquad \qquad $\myvec{1&1&1\\\frac{1}{2}&0&\frac{-1}{3}\\\frac{1}{2}&\frac{-1}{2}&\frac{1}{3}}\myvec{v_{1}\\v_{2}\\v_{3}} = \myvec{1\\0\\0}$ \\
		& \\
		\hline
		\multirow{3}{*}{Solving the linear} & \\
		& The above linear equation can be solved using Gauss-Jordan method as\\equtions
		& \\
		& \qquad \qquad \qquad $\myvec{1&1&1&\vrule&1\\\frac{1}{2}&0&\frac{-1}{3}&\vrule&0\\\frac{1}{2}&\frac{-1}{2}&\frac{1}{3}&\vrule&0}$\\
		& \\
		& \qquad $\xleftrightarrow[]{R_2 \leftarrow R_2 - \frac{1}{2}R_1}$
		$\myvec{1&1&1&\vrule&1\\0&\frac{-1}{2}&\frac{-5}{6}&\vrule&\frac{-1}{2}\\\frac{1}{2}&\frac{-1}{2}&\frac{1}{3}&\vrule&0}$\\
		&\\
		& \qquad $\xleftrightarrow[]{R_3 \leftarrow R_3 - \frac{1}{2}R_1}$
		$\myvec{1&1&1&\vrule&1\\0&\frac{-1}{2}&\frac{-5}{6}&\vrule&\frac{-1}{2}\\0&-1&\frac{-1}{6}&\vrule&\frac{-1}{2}}$\\
		&\\
		& \qquad $\xleftrightarrow[]{R_2 \leftarrow \frac{-1}{2}R_2}$
		$\myvec{1&1&1&\vrule&1\\0&1&\frac{5}{3}&\vrule&1\\0&-1&\frac{-1}{6}&\vrule&\frac{-1}{2}}$\\
		&\\
		& \qquad $\xleftrightarrow[]{R_3 \leftarrow R_3 + R_2}$
		$\myvec{1&1&1&\vrule&1\\0&1&\frac{5}{3}&\vrule&1\\0&0&\frac{3}{2}&\vrule&\frac{1}{2}}$\\
		&\\
		& \qquad $\xleftrightarrow[]{R_3 \leftarrow \frac{3}{2}R_3}$
		$\myvec{1&1&1&\vrule&1\\0&1&\frac{5}{3}&\vrule&1\\0&0&1&\vrule&\frac{1}{3}}$\\
		&\\
		& \qquad $\xleftrightarrow[]{R_2 \leftarrow R_2 - \frac{5}{3}R_3}$
		$\myvec{1&1&1&\vrule&1\\0&1&0&\vrule&\frac{4}{9}\\0&0&1&\vrule&\frac{1}{3}}$\\
		&\\
		& \qquad $\xleftrightarrow[]{R_1 \leftarrow R_1 - R_3}$
		$\myvec{1&1&0&\vrule&\frac{2}{3}\\0&1&0&\vrule&\frac{4}{9}\\0&0&1&\vrule&\frac{1}{3}}$\\
		&\\
		& \qquad $\xleftrightarrow[]{R_1 \leftarrow R_1 - R_2}$
		$\myvec{1&0&0&\vrule&\frac{2}{9}\\0&1&0&\vrule&\frac{4}{9}\\0&0&1&\vrule&\frac{1}{3}}$\\
		&\\
		& $\therefore$, stationary probability distribution $\pi$ is given by \\
		& \qquad \qquad $\pi = \myvec{\frac{2}{9} & \frac{4}{9} & \frac{1}{3}}$ \\
		& \\
		\hline
		\multirow{3}{*}{Observations} & \\
		
		
		& Since the given transition probability matrix $\vec{P}$ is irreducible and aperiodic, \\
		& then $\lim_{n \to \infty} \vec{P}^{n}$ converges to a matrix with all rows identical and equal to $\vec{\pi}$. \\
		& \\
		& We were able to find $\vec{\pi}$ as $\myvec{\frac{2}{9} & \frac{4}{9} & \frac{1}{3}}$ \\
		& \\
		& $\lim_{n \to \infty} \vec{P}^{n} = \myvec{\frac{2}{9}&\frac{4}{9}&\frac{1}{3}\\\frac{2}{9}&\frac{4}{9}&\frac{1}{3}\\\frac{2}{9}&\frac{4}{9}&\frac{1}{3}}$\\
		& \\
		& From the above matrix, we get \\
		& \\
		& $\lim_{n \to \infty} \vec{P}^{n}_{11} = \frac{2}{9}$ \\
		&\\
		& $\lim_{n \to \infty} \vec{P}^{n}_{21} = \frac{2}{9}$ \\
		&\\
		& $\lim_{n \to \infty} \vec{P}^{n}_{32} = \frac{4}{9}$ \\
		&\\
		& $\lim_{n \to \infty} \vec{P}^{n}_{13} = \frac{1}{3}$ \\
		&\\
		\hline
		\multirow{3}{*}{Conclusion} & \\
		& From our observation we see that \\
		&\\
		& Options 1) and 4) are True.\\
		& \\
		\hline
\caption{}
\label{eq:solutions/2018/dec/106/table1}
	\end{longtable}
\twocolumn

\item Let $\vec{S}: \mathbb R^n \rightarrow \mathbb R^n$ be given by $\vec{S}(\vec{v}) = \alpha\vec{v}$, for a fixed $\alpha \in \mathbb R, \alpha \neq 0$. Let $\vec{T}: \mathbb R^n \rightarrow \mathbb R^n$ be a linear transformation such that $\vec{B} = \{ \vec{v}_1,\ldots,\vec{v}_n \}$ is a set of linearly independent eigenvectors of $\vec{T}$. Then
\begin{enumerate}
    \item The matrix of $\vec{T}$ with respect to $\vec{B}$ is diagonal
    \item The matrix of $(\vec{T}-\vec{S})$ with respect to $\vec{B}$ is diagonal
    \item The matrix of $\vec{T}$ with respect to $\vec{B}$ is not necessarily diagonal, but is upper triangular
    \item The matrix of $\vec{T}$ with respect to $\vec{B}$ is diagonal but the matrix of $(\vec{T}-\vec{S})$ with respect to $\vec{B}$ is not diagonal.
\end{enumerate}
%
\solution
See Tables \ref{eq:solutions/2018/dec/106/table0} and \ref{eq:solutions/2018/dec/106/table1}


\onecolumn
	\begin{longtable}{|l|l|}
		\hline
		\multirow{3}{*}{Irreducible Markov Chain} 
		& \\
		& A Markov chain is $\textbf{irreducible}$ if all the states communicate with each other,\\
		& i.e., if there is only one communication class.\\
		&\\
		\hline
		\multirow{3}{*}{Aperiodic Markov Chain} & \\
		& If there is a self-transition in the chain ($p^{ii}>0$ for some i), then the chain is\\
		& called as $\textbf{aperiodic}$\\
		& \\
		\hline
		\multirow{3}{*}{Stationary Distribution} & \\
		& A stationary distribution of a Markov chain is a probability distribution that\\
		& remains unchanged in the Markov chain as time progresses. Typically, it is\\
		& represented as a row vector $\Vec{\pi}$ whose entries are probabilities summing to 1,\\ 
		& and given transition matrix $\textbf{P}$, it satisfies\\
		& \\
		&  \qquad \qquad  \qquad$\Vec{\pi} = \Vec{\pi} \textbf{P}$\\
		& \\
		\hline
\caption{}
\label{eq:solutions/2018/dec/106/table0}
	\end{longtable}
	\begin{longtable}{|l|l|}
		\hline
		\multirow{3}{*}{Drawing Transition diagram} 
		& \\
		& 
		
		$\begin{tikzpicture}[shorten >=1pt,node distance=2cm, scale =3, auto]
			\tikzstyle{every state}=[fill={rgb:black,1;white,10}]
			
			\node[state]   (q_1)                          {$1$};
			\node[state]   (q_2)  [right of=q_1]          {$2$};
			\node[state]   (q_3)  [below right of=q_1]          {$3$};
			
			\path[->]
			(q_1) edge [loop above] node {$\frac{1}{2}$}    (   )
			edge [bend left]  node {$\frac{1}{2}$}    (q_2)
			(q_2) edge [bend left]  node {$\frac{1}{2}$}    (q_3)
			edge [loop above] node {$\frac{1}{2}$}    ()
			(q_3) edge [bend left]  node {$\frac{1}{3}$}    (q_2)
			edge [bend left]  node {$\frac{1}{3}$}    (q_1)
			edge [loop below] node {$\frac{1}{3}$}    ();
		\end{tikzpicture}$
		
		\\  
		&\\
		&\\
		\hline
		\multirow{3}{*}{Checking whether the  } & \\
		& Here,\\chain is Irreducible
		& All the states are accessible to one another. \\and Aperiodic
		& $\implies$ They are in the same communication class. So, it is Irreducible.\\
		& \\
		& There exists the non- zero self-transition, which means that the chain \\
		& is Aperiodic.\\
		&\\ 
		& We know that if the Markov Chain is irreducible and aperiodic then \\
		& \qquad \qquad \qquad $\Vec{\pi}_{j} = \lim_{n \to \infty}P\{X_{n} = j\}$, $j = 1,...,N$ \\
		& These are the stationary probabilities. \\
		&\\
		\hline
		\multirow{3}{*}{Finding the Stationary} & \\
		& Stationary Probability can be represented as\\Probability Distributions
		& \qquad \qquad \qquad $\Vec{\pi} = \Vec{\pi} \vec{P}$\\
		& \\
		& \qquad $\implies$ $\myvec{v_{1}&&v_{2}&&v_{3}} = \myvec{v_{1}&&v_{2}&&v_{3}}\vec{P}$ \\
		& \\
		& Equating the above equation we get \\
		& \\
		& \qquad \qquad \qquad $\frac{1}{2}v_{1}-\frac{1}{3}v_{3} = 0$ $\label{eq:solutions/2018/dec/106/eq}$\\
		& \\
		& \qquad \qquad \qquad $\frac{1}{2}v_{1}-\frac{1}{2}v_{2} + \frac{1}{3}v_{3} = 0$\\
		& \\
		& \qquad \qquad \qquad $\frac{1}{2}v_{2}-\frac{2}{3}v_{3} = 0$\\
		& \\\
		& We see that summation of second and the third equation gives us the \\
		& first equation only. \\
		& And we know that the probability distribution will sum up to 1. \\
		& \\
		& \qquad \qquad \qquad $v_{1}+v_{2}+v_{3} = 1$ \\
		& \\
		& Therefore, we get the equation form as \\
		& \\
		& \qquad \qquad \qquad $\myvec{1&1&1\\\frac{1}{2}&0&\frac{-1}{3}\\\frac{1}{2}&\frac{-1}{2}&\frac{1}{3}}\myvec{v_{1}\\v_{2}\\v_{3}} = \myvec{1\\0\\0}$ \\
		& \\
		\hline
		\multirow{3}{*}{Solving the linear} & \\
		& The above linear equation can be solved using Gauss-Jordan method as\\equtions
		& \\
		& \qquad \qquad \qquad $\myvec{1&1&1&\vrule&1\\\frac{1}{2}&0&\frac{-1}{3}&\vrule&0\\\frac{1}{2}&\frac{-1}{2}&\frac{1}{3}&\vrule&0}$\\
		& \\
		& \qquad $\xleftrightarrow[]{R_2 \leftarrow R_2 - \frac{1}{2}R_1}$
		$\myvec{1&1&1&\vrule&1\\0&\frac{-1}{2}&\frac{-5}{6}&\vrule&\frac{-1}{2}\\\frac{1}{2}&\frac{-1}{2}&\frac{1}{3}&\vrule&0}$\\
		&\\
		& \qquad $\xleftrightarrow[]{R_3 \leftarrow R_3 - \frac{1}{2}R_1}$
		$\myvec{1&1&1&\vrule&1\\0&\frac{-1}{2}&\frac{-5}{6}&\vrule&\frac{-1}{2}\\0&-1&\frac{-1}{6}&\vrule&\frac{-1}{2}}$\\
		&\\
		& \qquad $\xleftrightarrow[]{R_2 \leftarrow \frac{-1}{2}R_2}$
		$\myvec{1&1&1&\vrule&1\\0&1&\frac{5}{3}&\vrule&1\\0&-1&\frac{-1}{6}&\vrule&\frac{-1}{2}}$\\
		&\\
		& \qquad $\xleftrightarrow[]{R_3 \leftarrow R_3 + R_2}$
		$\myvec{1&1&1&\vrule&1\\0&1&\frac{5}{3}&\vrule&1\\0&0&\frac{3}{2}&\vrule&\frac{1}{2}}$\\
		&\\
		& \qquad $\xleftrightarrow[]{R_3 \leftarrow \frac{3}{2}R_3}$
		$\myvec{1&1&1&\vrule&1\\0&1&\frac{5}{3}&\vrule&1\\0&0&1&\vrule&\frac{1}{3}}$\\
		&\\
		& \qquad $\xleftrightarrow[]{R_2 \leftarrow R_2 - \frac{5}{3}R_3}$
		$\myvec{1&1&1&\vrule&1\\0&1&0&\vrule&\frac{4}{9}\\0&0&1&\vrule&\frac{1}{3}}$\\
		&\\
		& \qquad $\xleftrightarrow[]{R_1 \leftarrow R_1 - R_3}$
		$\myvec{1&1&0&\vrule&\frac{2}{3}\\0&1&0&\vrule&\frac{4}{9}\\0&0&1&\vrule&\frac{1}{3}}$\\
		&\\
		& \qquad $\xleftrightarrow[]{R_1 \leftarrow R_1 - R_2}$
		$\myvec{1&0&0&\vrule&\frac{2}{9}\\0&1&0&\vrule&\frac{4}{9}\\0&0&1&\vrule&\frac{1}{3}}$\\
		&\\
		& $\therefore$, stationary probability distribution $\pi$ is given by \\
		& \qquad \qquad $\pi = \myvec{\frac{2}{9} & \frac{4}{9} & \frac{1}{3}}$ \\
		& \\
		\hline
		\multirow{3}{*}{Observations} & \\
		
		
		& Since the given transition probability matrix $\vec{P}$ is irreducible and aperiodic, \\
		& then $\lim_{n \to \infty} \vec{P}^{n}$ converges to a matrix with all rows identical and equal to $\vec{\pi}$. \\
		& \\
		& We were able to find $\vec{\pi}$ as $\myvec{\frac{2}{9} & \frac{4}{9} & \frac{1}{3}}$ \\
		& \\
		& $\lim_{n \to \infty} \vec{P}^{n} = \myvec{\frac{2}{9}&\frac{4}{9}&\frac{1}{3}\\\frac{2}{9}&\frac{4}{9}&\frac{1}{3}\\\frac{2}{9}&\frac{4}{9}&\frac{1}{3}}$\\
		& \\
		& From the above matrix, we get \\
		& \\
		& $\lim_{n \to \infty} \vec{P}^{n}_{11} = \frac{2}{9}$ \\
		&\\
		& $\lim_{n \to \infty} \vec{P}^{n}_{21} = \frac{2}{9}$ \\
		&\\
		& $\lim_{n \to \infty} \vec{P}^{n}_{32} = \frac{4}{9}$ \\
		&\\
		& $\lim_{n \to \infty} \vec{P}^{n}_{13} = \frac{1}{3}$ \\
		&\\
		\hline
		\multirow{3}{*}{Conclusion} & \\
		& From our observation we see that \\
		&\\
		& Options 1) and 4) are True.\\
		& \\
		\hline
\caption{}
\label{eq:solutions/2018/dec/106/table1}
	\end{longtable}
\twocolumn

\item Let $p_n\brak{x}=x^n$ for $x\in\mathbb{R}$ and let $\varrho=span\{p_0,p_1,p_2,...\}$. Then
\begin{enumerate}
    \item $\varrho$ is a vector space of all real valued continuous functions on $\mathbb{R}$.
    \item $\varrho$ is a subspace of all real valued continuous functions on $\mathbb{R}$.
    \item $\{p_0,p_1,p_2,...\}$ is a linearly independent set in the vector space of all real valued continuous functions on $\mathbb{R}$.
    \item Trigonometric functions belong to $\varrho$.
\end{enumerate}
%
%
\solution
See Tables \ref{eq:solutions/2018/dec/106/table0} and \ref{eq:solutions/2018/dec/106/table1}


\onecolumn
	\begin{longtable}{|l|l|}
		\hline
		\multirow{3}{*}{Irreducible Markov Chain} 
		& \\
		& A Markov chain is $\textbf{irreducible}$ if all the states communicate with each other,\\
		& i.e., if there is only one communication class.\\
		&\\
		\hline
		\multirow{3}{*}{Aperiodic Markov Chain} & \\
		& If there is a self-transition in the chain ($p^{ii}>0$ for some i), then the chain is\\
		& called as $\textbf{aperiodic}$\\
		& \\
		\hline
		\multirow{3}{*}{Stationary Distribution} & \\
		& A stationary distribution of a Markov chain is a probability distribution that\\
		& remains unchanged in the Markov chain as time progresses. Typically, it is\\
		& represented as a row vector $\Vec{\pi}$ whose entries are probabilities summing to 1,\\ 
		& and given transition matrix $\textbf{P}$, it satisfies\\
		& \\
		&  \qquad \qquad  \qquad$\Vec{\pi} = \Vec{\pi} \textbf{P}$\\
		& \\
		\hline
\caption{}
\label{eq:solutions/2018/dec/106/table0}
	\end{longtable}
	\begin{longtable}{|l|l|}
		\hline
		\multirow{3}{*}{Drawing Transition diagram} 
		& \\
		& 
		
		$\begin{tikzpicture}[shorten >=1pt,node distance=2cm, scale =3, auto]
			\tikzstyle{every state}=[fill={rgb:black,1;white,10}]
			
			\node[state]   (q_1)                          {$1$};
			\node[state]   (q_2)  [right of=q_1]          {$2$};
			\node[state]   (q_3)  [below right of=q_1]          {$3$};
			
			\path[->]
			(q_1) edge [loop above] node {$\frac{1}{2}$}    (   )
			edge [bend left]  node {$\frac{1}{2}$}    (q_2)
			(q_2) edge [bend left]  node {$\frac{1}{2}$}    (q_3)
			edge [loop above] node {$\frac{1}{2}$}    ()
			(q_3) edge [bend left]  node {$\frac{1}{3}$}    (q_2)
			edge [bend left]  node {$\frac{1}{3}$}    (q_1)
			edge [loop below] node {$\frac{1}{3}$}    ();
		\end{tikzpicture}$
		
		\\  
		&\\
		&\\
		\hline
		\multirow{3}{*}{Checking whether the  } & \\
		& Here,\\chain is Irreducible
		& All the states are accessible to one another. \\and Aperiodic
		& $\implies$ They are in the same communication class. So, it is Irreducible.\\
		& \\
		& There exists the non- zero self-transition, which means that the chain \\
		& is Aperiodic.\\
		&\\ 
		& We know that if the Markov Chain is irreducible and aperiodic then \\
		& \qquad \qquad \qquad $\Vec{\pi}_{j} = \lim_{n \to \infty}P\{X_{n} = j\}$, $j = 1,...,N$ \\
		& These are the stationary probabilities. \\
		&\\
		\hline
		\multirow{3}{*}{Finding the Stationary} & \\
		& Stationary Probability can be represented as\\Probability Distributions
		& \qquad \qquad \qquad $\Vec{\pi} = \Vec{\pi} \vec{P}$\\
		& \\
		& \qquad $\implies$ $\myvec{v_{1}&&v_{2}&&v_{3}} = \myvec{v_{1}&&v_{2}&&v_{3}}\vec{P}$ \\
		& \\
		& Equating the above equation we get \\
		& \\
		& \qquad \qquad \qquad $\frac{1}{2}v_{1}-\frac{1}{3}v_{3} = 0$ $\label{eq:solutions/2018/dec/106/eq}$\\
		& \\
		& \qquad \qquad \qquad $\frac{1}{2}v_{1}-\frac{1}{2}v_{2} + \frac{1}{3}v_{3} = 0$\\
		& \\
		& \qquad \qquad \qquad $\frac{1}{2}v_{2}-\frac{2}{3}v_{3} = 0$\\
		& \\\
		& We see that summation of second and the third equation gives us the \\
		& first equation only. \\
		& And we know that the probability distribution will sum up to 1. \\
		& \\
		& \qquad \qquad \qquad $v_{1}+v_{2}+v_{3} = 1$ \\
		& \\
		& Therefore, we get the equation form as \\
		& \\
		& \qquad \qquad \qquad $\myvec{1&1&1\\\frac{1}{2}&0&\frac{-1}{3}\\\frac{1}{2}&\frac{-1}{2}&\frac{1}{3}}\myvec{v_{1}\\v_{2}\\v_{3}} = \myvec{1\\0\\0}$ \\
		& \\
		\hline
		\multirow{3}{*}{Solving the linear} & \\
		& The above linear equation can be solved using Gauss-Jordan method as\\equtions
		& \\
		& \qquad \qquad \qquad $\myvec{1&1&1&\vrule&1\\\frac{1}{2}&0&\frac{-1}{3}&\vrule&0\\\frac{1}{2}&\frac{-1}{2}&\frac{1}{3}&\vrule&0}$\\
		& \\
		& \qquad $\xleftrightarrow[]{R_2 \leftarrow R_2 - \frac{1}{2}R_1}$
		$\myvec{1&1&1&\vrule&1\\0&\frac{-1}{2}&\frac{-5}{6}&\vrule&\frac{-1}{2}\\\frac{1}{2}&\frac{-1}{2}&\frac{1}{3}&\vrule&0}$\\
		&\\
		& \qquad $\xleftrightarrow[]{R_3 \leftarrow R_3 - \frac{1}{2}R_1}$
		$\myvec{1&1&1&\vrule&1\\0&\frac{-1}{2}&\frac{-5}{6}&\vrule&\frac{-1}{2}\\0&-1&\frac{-1}{6}&\vrule&\frac{-1}{2}}$\\
		&\\
		& \qquad $\xleftrightarrow[]{R_2 \leftarrow \frac{-1}{2}R_2}$
		$\myvec{1&1&1&\vrule&1\\0&1&\frac{5}{3}&\vrule&1\\0&-1&\frac{-1}{6}&\vrule&\frac{-1}{2}}$\\
		&\\
		& \qquad $\xleftrightarrow[]{R_3 \leftarrow R_3 + R_2}$
		$\myvec{1&1&1&\vrule&1\\0&1&\frac{5}{3}&\vrule&1\\0&0&\frac{3}{2}&\vrule&\frac{1}{2}}$\\
		&\\
		& \qquad $\xleftrightarrow[]{R_3 \leftarrow \frac{3}{2}R_3}$
		$\myvec{1&1&1&\vrule&1\\0&1&\frac{5}{3}&\vrule&1\\0&0&1&\vrule&\frac{1}{3}}$\\
		&\\
		& \qquad $\xleftrightarrow[]{R_2 \leftarrow R_2 - \frac{5}{3}R_3}$
		$\myvec{1&1&1&\vrule&1\\0&1&0&\vrule&\frac{4}{9}\\0&0&1&\vrule&\frac{1}{3}}$\\
		&\\
		& \qquad $\xleftrightarrow[]{R_1 \leftarrow R_1 - R_3}$
		$\myvec{1&1&0&\vrule&\frac{2}{3}\\0&1&0&\vrule&\frac{4}{9}\\0&0&1&\vrule&\frac{1}{3}}$\\
		&\\
		& \qquad $\xleftrightarrow[]{R_1 \leftarrow R_1 - R_2}$
		$\myvec{1&0&0&\vrule&\frac{2}{9}\\0&1&0&\vrule&\frac{4}{9}\\0&0&1&\vrule&\frac{1}{3}}$\\
		&\\
		& $\therefore$, stationary probability distribution $\pi$ is given by \\
		& \qquad \qquad $\pi = \myvec{\frac{2}{9} & \frac{4}{9} & \frac{1}{3}}$ \\
		& \\
		\hline
		\multirow{3}{*}{Observations} & \\
		
		
		& Since the given transition probability matrix $\vec{P}$ is irreducible and aperiodic, \\
		& then $\lim_{n \to \infty} \vec{P}^{n}$ converges to a matrix with all rows identical and equal to $\vec{\pi}$. \\
		& \\
		& We were able to find $\vec{\pi}$ as $\myvec{\frac{2}{9} & \frac{4}{9} & \frac{1}{3}}$ \\
		& \\
		& $\lim_{n \to \infty} \vec{P}^{n} = \myvec{\frac{2}{9}&\frac{4}{9}&\frac{1}{3}\\\frac{2}{9}&\frac{4}{9}&\frac{1}{3}\\\frac{2}{9}&\frac{4}{9}&\frac{1}{3}}$\\
		& \\
		& From the above matrix, we get \\
		& \\
		& $\lim_{n \to \infty} \vec{P}^{n}_{11} = \frac{2}{9}$ \\
		&\\
		& $\lim_{n \to \infty} \vec{P}^{n}_{21} = \frac{2}{9}$ \\
		&\\
		& $\lim_{n \to \infty} \vec{P}^{n}_{32} = \frac{4}{9}$ \\
		&\\
		& $\lim_{n \to \infty} \vec{P}^{n}_{13} = \frac{1}{3}$ \\
		&\\
		\hline
		\multirow{3}{*}{Conclusion} & \\
		& From our observation we see that \\
		&\\
		& Options 1) and 4) are True.\\
		& \\
		\hline
\caption{}
\label{eq:solutions/2018/dec/106/table1}
	\end{longtable}
\twocolumn


\item Let $\vec{A}$ be an invertible $4 \times 4$ real matrix. Which of the following are NOT true ? 
\begin{enumerate}
\item  Rank $\vec{A}$ =4
\item For every vector $\vec{b} \in \mathbb{R}$, $\vec{A}\vec{x}=\vec{b}$ has exactly one solution. 
\item dim(nullspace $\vec{A}$)$\geq$ 1
\item 0 is an eigenvalue of $\vec{A}$
\end{enumerate}
%
\solution
See Tables \ref{eq:solutions/2018/dec/106/table0} and \ref{eq:solutions/2018/dec/106/table1}


\onecolumn
	\begin{longtable}{|l|l|}
		\hline
		\multirow{3}{*}{Irreducible Markov Chain} 
		& \\
		& A Markov chain is $\textbf{irreducible}$ if all the states communicate with each other,\\
		& i.e., if there is only one communication class.\\
		&\\
		\hline
		\multirow{3}{*}{Aperiodic Markov Chain} & \\
		& If there is a self-transition in the chain ($p^{ii}>0$ for some i), then the chain is\\
		& called as $\textbf{aperiodic}$\\
		& \\
		\hline
		\multirow{3}{*}{Stationary Distribution} & \\
		& A stationary distribution of a Markov chain is a probability distribution that\\
		& remains unchanged in the Markov chain as time progresses. Typically, it is\\
		& represented as a row vector $\Vec{\pi}$ whose entries are probabilities summing to 1,\\ 
		& and given transition matrix $\textbf{P}$, it satisfies\\
		& \\
		&  \qquad \qquad  \qquad$\Vec{\pi} = \Vec{\pi} \textbf{P}$\\
		& \\
		\hline
\caption{}
\label{eq:solutions/2018/dec/106/table0}
	\end{longtable}
	\begin{longtable}{|l|l|}
		\hline
		\multirow{3}{*}{Drawing Transition diagram} 
		& \\
		& 
		
		$\begin{tikzpicture}[shorten >=1pt,node distance=2cm, scale =3, auto]
			\tikzstyle{every state}=[fill={rgb:black,1;white,10}]
			
			\node[state]   (q_1)                          {$1$};
			\node[state]   (q_2)  [right of=q_1]          {$2$};
			\node[state]   (q_3)  [below right of=q_1]          {$3$};
			
			\path[->]
			(q_1) edge [loop above] node {$\frac{1}{2}$}    (   )
			edge [bend left]  node {$\frac{1}{2}$}    (q_2)
			(q_2) edge [bend left]  node {$\frac{1}{2}$}    (q_3)
			edge [loop above] node {$\frac{1}{2}$}    ()
			(q_3) edge [bend left]  node {$\frac{1}{3}$}    (q_2)
			edge [bend left]  node {$\frac{1}{3}$}    (q_1)
			edge [loop below] node {$\frac{1}{3}$}    ();
		\end{tikzpicture}$
		
		\\  
		&\\
		&\\
		\hline
		\multirow{3}{*}{Checking whether the  } & \\
		& Here,\\chain is Irreducible
		& All the states are accessible to one another. \\and Aperiodic
		& $\implies$ They are in the same communication class. So, it is Irreducible.\\
		& \\
		& There exists the non- zero self-transition, which means that the chain \\
		& is Aperiodic.\\
		&\\ 
		& We know that if the Markov Chain is irreducible and aperiodic then \\
		& \qquad \qquad \qquad $\Vec{\pi}_{j} = \lim_{n \to \infty}P\{X_{n} = j\}$, $j = 1,...,N$ \\
		& These are the stationary probabilities. \\
		&\\
		\hline
		\multirow{3}{*}{Finding the Stationary} & \\
		& Stationary Probability can be represented as\\Probability Distributions
		& \qquad \qquad \qquad $\Vec{\pi} = \Vec{\pi} \vec{P}$\\
		& \\
		& \qquad $\implies$ $\myvec{v_{1}&&v_{2}&&v_{3}} = \myvec{v_{1}&&v_{2}&&v_{3}}\vec{P}$ \\
		& \\
		& Equating the above equation we get \\
		& \\
		& \qquad \qquad \qquad $\frac{1}{2}v_{1}-\frac{1}{3}v_{3} = 0$ $\label{eq:solutions/2018/dec/106/eq}$\\
		& \\
		& \qquad \qquad \qquad $\frac{1}{2}v_{1}-\frac{1}{2}v_{2} + \frac{1}{3}v_{3} = 0$\\
		& \\
		& \qquad \qquad \qquad $\frac{1}{2}v_{2}-\frac{2}{3}v_{3} = 0$\\
		& \\\
		& We see that summation of second and the third equation gives us the \\
		& first equation only. \\
		& And we know that the probability distribution will sum up to 1. \\
		& \\
		& \qquad \qquad \qquad $v_{1}+v_{2}+v_{3} = 1$ \\
		& \\
		& Therefore, we get the equation form as \\
		& \\
		& \qquad \qquad \qquad $\myvec{1&1&1\\\frac{1}{2}&0&\frac{-1}{3}\\\frac{1}{2}&\frac{-1}{2}&\frac{1}{3}}\myvec{v_{1}\\v_{2}\\v_{3}} = \myvec{1\\0\\0}$ \\
		& \\
		\hline
		\multirow{3}{*}{Solving the linear} & \\
		& The above linear equation can be solved using Gauss-Jordan method as\\equtions
		& \\
		& \qquad \qquad \qquad $\myvec{1&1&1&\vrule&1\\\frac{1}{2}&0&\frac{-1}{3}&\vrule&0\\\frac{1}{2}&\frac{-1}{2}&\frac{1}{3}&\vrule&0}$\\
		& \\
		& \qquad $\xleftrightarrow[]{R_2 \leftarrow R_2 - \frac{1}{2}R_1}$
		$\myvec{1&1&1&\vrule&1\\0&\frac{-1}{2}&\frac{-5}{6}&\vrule&\frac{-1}{2}\\\frac{1}{2}&\frac{-1}{2}&\frac{1}{3}&\vrule&0}$\\
		&\\
		& \qquad $\xleftrightarrow[]{R_3 \leftarrow R_3 - \frac{1}{2}R_1}$
		$\myvec{1&1&1&\vrule&1\\0&\frac{-1}{2}&\frac{-5}{6}&\vrule&\frac{-1}{2}\\0&-1&\frac{-1}{6}&\vrule&\frac{-1}{2}}$\\
		&\\
		& \qquad $\xleftrightarrow[]{R_2 \leftarrow \frac{-1}{2}R_2}$
		$\myvec{1&1&1&\vrule&1\\0&1&\frac{5}{3}&\vrule&1\\0&-1&\frac{-1}{6}&\vrule&\frac{-1}{2}}$\\
		&\\
		& \qquad $\xleftrightarrow[]{R_3 \leftarrow R_3 + R_2}$
		$\myvec{1&1&1&\vrule&1\\0&1&\frac{5}{3}&\vrule&1\\0&0&\frac{3}{2}&\vrule&\frac{1}{2}}$\\
		&\\
		& \qquad $\xleftrightarrow[]{R_3 \leftarrow \frac{3}{2}R_3}$
		$\myvec{1&1&1&\vrule&1\\0&1&\frac{5}{3}&\vrule&1\\0&0&1&\vrule&\frac{1}{3}}$\\
		&\\
		& \qquad $\xleftrightarrow[]{R_2 \leftarrow R_2 - \frac{5}{3}R_3}$
		$\myvec{1&1&1&\vrule&1\\0&1&0&\vrule&\frac{4}{9}\\0&0&1&\vrule&\frac{1}{3}}$\\
		&\\
		& \qquad $\xleftrightarrow[]{R_1 \leftarrow R_1 - R_3}$
		$\myvec{1&1&0&\vrule&\frac{2}{3}\\0&1&0&\vrule&\frac{4}{9}\\0&0&1&\vrule&\frac{1}{3}}$\\
		&\\
		& \qquad $\xleftrightarrow[]{R_1 \leftarrow R_1 - R_2}$
		$\myvec{1&0&0&\vrule&\frac{2}{9}\\0&1&0&\vrule&\frac{4}{9}\\0&0&1&\vrule&\frac{1}{3}}$\\
		&\\
		& $\therefore$, stationary probability distribution $\pi$ is given by \\
		& \qquad \qquad $\pi = \myvec{\frac{2}{9} & \frac{4}{9} & \frac{1}{3}}$ \\
		& \\
		\hline
		\multirow{3}{*}{Observations} & \\
		
		
		& Since the given transition probability matrix $\vec{P}$ is irreducible and aperiodic, \\
		& then $\lim_{n \to \infty} \vec{P}^{n}$ converges to a matrix with all rows identical and equal to $\vec{\pi}$. \\
		& \\
		& We were able to find $\vec{\pi}$ as $\myvec{\frac{2}{9} & \frac{4}{9} & \frac{1}{3}}$ \\
		& \\
		& $\lim_{n \to \infty} \vec{P}^{n} = \myvec{\frac{2}{9}&\frac{4}{9}&\frac{1}{3}\\\frac{2}{9}&\frac{4}{9}&\frac{1}{3}\\\frac{2}{9}&\frac{4}{9}&\frac{1}{3}}$\\
		& \\
		& From the above matrix, we get \\
		& \\
		& $\lim_{n \to \infty} \vec{P}^{n}_{11} = \frac{2}{9}$ \\
		&\\
		& $\lim_{n \to \infty} \vec{P}^{n}_{21} = \frac{2}{9}$ \\
		&\\
		& $\lim_{n \to \infty} \vec{P}^{n}_{32} = \frac{4}{9}$ \\
		&\\
		& $\lim_{n \to \infty} \vec{P}^{n}_{13} = \frac{1}{3}$ \\
		&\\
		\hline
		\multirow{3}{*}{Conclusion} & \\
		& From our observation we see that \\
		&\\
		& Options 1) and 4) are True.\\
		& \\
		\hline
\caption{}
\label{eq:solutions/2018/dec/106/table1}
	\end{longtable}
\twocolumn

\item 	Consider non-zero vector spaces $\vec{V_1}, \vec{V_2}, \vec{V_3}, \vec{V_4}$ and linear transformations $\phi_1 : \vec{V_1} \rightarrow \vec{V_2}$, $\phi_2 : \vec{V_2} \rightarrow \vec{V_3}$, $\phi_3 : \vec{V_3} \rightarrow \vec{V_4}$ such that $Ker(\phi_1) = \{0\}$, $Range(\phi_1) = Ker(\phi_2)$, $Range(\phi_2) = Ker(\phi_3)$, $Range(\phi_3) = \vec{V_4}$. Then \\
	
	\begin{enumerate}
		\item $\sum_{i=1}^{4} \ (-1)^{i} \ dim \ \vec{V_i} = 0$ \\
		\item $\sum_{i=2}^{4} \ (-1)^{i} \ dim \ \vec{V_i} > 0$ \\
		\item $\sum_{i=1}^{4} \ (-1)^{i} \ dim \ \vec{V_i} < 0$ \\
		\item $\sum_{i=1}^{4} \ (-1)^{i} \ dim \ \vec{V_i} \neq 0$
	\end{enumerate}
%
\solution
See Tables \ref{eq:solutions/2018/dec/106/table0} and \ref{eq:solutions/2018/dec/106/table1}


\onecolumn
	\begin{longtable}{|l|l|}
		\hline
		\multirow{3}{*}{Irreducible Markov Chain} 
		& \\
		& A Markov chain is $\textbf{irreducible}$ if all the states communicate with each other,\\
		& i.e., if there is only one communication class.\\
		&\\
		\hline
		\multirow{3}{*}{Aperiodic Markov Chain} & \\
		& If there is a self-transition in the chain ($p^{ii}>0$ for some i), then the chain is\\
		& called as $\textbf{aperiodic}$\\
		& \\
		\hline
		\multirow{3}{*}{Stationary Distribution} & \\
		& A stationary distribution of a Markov chain is a probability distribution that\\
		& remains unchanged in the Markov chain as time progresses. Typically, it is\\
		& represented as a row vector $\Vec{\pi}$ whose entries are probabilities summing to 1,\\ 
		& and given transition matrix $\textbf{P}$, it satisfies\\
		& \\
		&  \qquad \qquad  \qquad$\Vec{\pi} = \Vec{\pi} \textbf{P}$\\
		& \\
		\hline
\caption{}
\label{eq:solutions/2018/dec/106/table0}
	\end{longtable}
	\begin{longtable}{|l|l|}
		\hline
		\multirow{3}{*}{Drawing Transition diagram} 
		& \\
		& 
		
		$\begin{tikzpicture}[shorten >=1pt,node distance=2cm, scale =3, auto]
			\tikzstyle{every state}=[fill={rgb:black,1;white,10}]
			
			\node[state]   (q_1)                          {$1$};
			\node[state]   (q_2)  [right of=q_1]          {$2$};
			\node[state]   (q_3)  [below right of=q_1]          {$3$};
			
			\path[->]
			(q_1) edge [loop above] node {$\frac{1}{2}$}    (   )
			edge [bend left]  node {$\frac{1}{2}$}    (q_2)
			(q_2) edge [bend left]  node {$\frac{1}{2}$}    (q_3)
			edge [loop above] node {$\frac{1}{2}$}    ()
			(q_3) edge [bend left]  node {$\frac{1}{3}$}    (q_2)
			edge [bend left]  node {$\frac{1}{3}$}    (q_1)
			edge [loop below] node {$\frac{1}{3}$}    ();
		\end{tikzpicture}$
		
		\\  
		&\\
		&\\
		\hline
		\multirow{3}{*}{Checking whether the  } & \\
		& Here,\\chain is Irreducible
		& All the states are accessible to one another. \\and Aperiodic
		& $\implies$ They are in the same communication class. So, it is Irreducible.\\
		& \\
		& There exists the non- zero self-transition, which means that the chain \\
		& is Aperiodic.\\
		&\\ 
		& We know that if the Markov Chain is irreducible and aperiodic then \\
		& \qquad \qquad \qquad $\Vec{\pi}_{j} = \lim_{n \to \infty}P\{X_{n} = j\}$, $j = 1,...,N$ \\
		& These are the stationary probabilities. \\
		&\\
		\hline
		\multirow{3}{*}{Finding the Stationary} & \\
		& Stationary Probability can be represented as\\Probability Distributions
		& \qquad \qquad \qquad $\Vec{\pi} = \Vec{\pi} \vec{P}$\\
		& \\
		& \qquad $\implies$ $\myvec{v_{1}&&v_{2}&&v_{3}} = \myvec{v_{1}&&v_{2}&&v_{3}}\vec{P}$ \\
		& \\
		& Equating the above equation we get \\
		& \\
		& \qquad \qquad \qquad $\frac{1}{2}v_{1}-\frac{1}{3}v_{3} = 0$ $\label{eq:solutions/2018/dec/106/eq}$\\
		& \\
		& \qquad \qquad \qquad $\frac{1}{2}v_{1}-\frac{1}{2}v_{2} + \frac{1}{3}v_{3} = 0$\\
		& \\
		& \qquad \qquad \qquad $\frac{1}{2}v_{2}-\frac{2}{3}v_{3} = 0$\\
		& \\\
		& We see that summation of second and the third equation gives us the \\
		& first equation only. \\
		& And we know that the probability distribution will sum up to 1. \\
		& \\
		& \qquad \qquad \qquad $v_{1}+v_{2}+v_{3} = 1$ \\
		& \\
		& Therefore, we get the equation form as \\
		& \\
		& \qquad \qquad \qquad $\myvec{1&1&1\\\frac{1}{2}&0&\frac{-1}{3}\\\frac{1}{2}&\frac{-1}{2}&\frac{1}{3}}\myvec{v_{1}\\v_{2}\\v_{3}} = \myvec{1\\0\\0}$ \\
		& \\
		\hline
		\multirow{3}{*}{Solving the linear} & \\
		& The above linear equation can be solved using Gauss-Jordan method as\\equtions
		& \\
		& \qquad \qquad \qquad $\myvec{1&1&1&\vrule&1\\\frac{1}{2}&0&\frac{-1}{3}&\vrule&0\\\frac{1}{2}&\frac{-1}{2}&\frac{1}{3}&\vrule&0}$\\
		& \\
		& \qquad $\xleftrightarrow[]{R_2 \leftarrow R_2 - \frac{1}{2}R_1}$
		$\myvec{1&1&1&\vrule&1\\0&\frac{-1}{2}&\frac{-5}{6}&\vrule&\frac{-1}{2}\\\frac{1}{2}&\frac{-1}{2}&\frac{1}{3}&\vrule&0}$\\
		&\\
		& \qquad $\xleftrightarrow[]{R_3 \leftarrow R_3 - \frac{1}{2}R_1}$
		$\myvec{1&1&1&\vrule&1\\0&\frac{-1}{2}&\frac{-5}{6}&\vrule&\frac{-1}{2}\\0&-1&\frac{-1}{6}&\vrule&\frac{-1}{2}}$\\
		&\\
		& \qquad $\xleftrightarrow[]{R_2 \leftarrow \frac{-1}{2}R_2}$
		$\myvec{1&1&1&\vrule&1\\0&1&\frac{5}{3}&\vrule&1\\0&-1&\frac{-1}{6}&\vrule&\frac{-1}{2}}$\\
		&\\
		& \qquad $\xleftrightarrow[]{R_3 \leftarrow R_3 + R_2}$
		$\myvec{1&1&1&\vrule&1\\0&1&\frac{5}{3}&\vrule&1\\0&0&\frac{3}{2}&\vrule&\frac{1}{2}}$\\
		&\\
		& \qquad $\xleftrightarrow[]{R_3 \leftarrow \frac{3}{2}R_3}$
		$\myvec{1&1&1&\vrule&1\\0&1&\frac{5}{3}&\vrule&1\\0&0&1&\vrule&\frac{1}{3}}$\\
		&\\
		& \qquad $\xleftrightarrow[]{R_2 \leftarrow R_2 - \frac{5}{3}R_3}$
		$\myvec{1&1&1&\vrule&1\\0&1&0&\vrule&\frac{4}{9}\\0&0&1&\vrule&\frac{1}{3}}$\\
		&\\
		& \qquad $\xleftrightarrow[]{R_1 \leftarrow R_1 - R_3}$
		$\myvec{1&1&0&\vrule&\frac{2}{3}\\0&1&0&\vrule&\frac{4}{9}\\0&0&1&\vrule&\frac{1}{3}}$\\
		&\\
		& \qquad $\xleftrightarrow[]{R_1 \leftarrow R_1 - R_2}$
		$\myvec{1&0&0&\vrule&\frac{2}{9}\\0&1&0&\vrule&\frac{4}{9}\\0&0&1&\vrule&\frac{1}{3}}$\\
		&\\
		& $\therefore$, stationary probability distribution $\pi$ is given by \\
		& \qquad \qquad $\pi = \myvec{\frac{2}{9} & \frac{4}{9} & \frac{1}{3}}$ \\
		& \\
		\hline
		\multirow{3}{*}{Observations} & \\
		
		
		& Since the given transition probability matrix $\vec{P}$ is irreducible and aperiodic, \\
		& then $\lim_{n \to \infty} \vec{P}^{n}$ converges to a matrix with all rows identical and equal to $\vec{\pi}$. \\
		& \\
		& We were able to find $\vec{\pi}$ as $\myvec{\frac{2}{9} & \frac{4}{9} & \frac{1}{3}}$ \\
		& \\
		& $\lim_{n \to \infty} \vec{P}^{n} = \myvec{\frac{2}{9}&\frac{4}{9}&\frac{1}{3}\\\frac{2}{9}&\frac{4}{9}&\frac{1}{3}\\\frac{2}{9}&\frac{4}{9}&\frac{1}{3}}$\\
		& \\
		& From the above matrix, we get \\
		& \\
		& $\lim_{n \to \infty} \vec{P}^{n}_{11} = \frac{2}{9}$ \\
		&\\
		& $\lim_{n \to \infty} \vec{P}^{n}_{21} = \frac{2}{9}$ \\
		&\\
		& $\lim_{n \to \infty} \vec{P}^{n}_{32} = \frac{4}{9}$ \\
		&\\
		& $\lim_{n \to \infty} \vec{P}^{n}_{13} = \frac{1}{3}$ \\
		&\\
		\hline
		\multirow{3}{*}{Conclusion} & \\
		& From our observation we see that \\
		&\\
		& Options 1) and 4) are True.\\
		& \\
		\hline
\caption{}
\label{eq:solutions/2018/dec/106/table1}
	\end{longtable}
\twocolumn

\item Let $\vec{u}$ be a real n $\times$ 1 vector satisfying $\vec{u}^T\vec{u}=1$, where $\vec{u}^T$ is the transpose of $\vec{u}$.Define\\
$\vec{A}=\vec{I}-2\vec{u}\vec{u}^T$ where $\vec{I}$ is the $n^{th}$ order identity matrix.Which of the following statements are true?\\
1. $\vec{A}$ is singular\\
2. $\vec{A}^2=\vec{A}$\\
3. Trace($\vec{A}$)=n-2\\
4. $\vec{A}^2=\vec{I}$\\
%
\solution
See Tables \ref{eq:solutions/2018/dec/106/table0} and \ref{eq:solutions/2018/dec/106/table1}


\onecolumn
	\begin{longtable}{|l|l|}
		\hline
		\multirow{3}{*}{Irreducible Markov Chain} 
		& \\
		& A Markov chain is $\textbf{irreducible}$ if all the states communicate with each other,\\
		& i.e., if there is only one communication class.\\
		&\\
		\hline
		\multirow{3}{*}{Aperiodic Markov Chain} & \\
		& If there is a self-transition in the chain ($p^{ii}>0$ for some i), then the chain is\\
		& called as $\textbf{aperiodic}$\\
		& \\
		\hline
		\multirow{3}{*}{Stationary Distribution} & \\
		& A stationary distribution of a Markov chain is a probability distribution that\\
		& remains unchanged in the Markov chain as time progresses. Typically, it is\\
		& represented as a row vector $\Vec{\pi}$ whose entries are probabilities summing to 1,\\ 
		& and given transition matrix $\textbf{P}$, it satisfies\\
		& \\
		&  \qquad \qquad  \qquad$\Vec{\pi} = \Vec{\pi} \textbf{P}$\\
		& \\
		\hline
\caption{}
\label{eq:solutions/2018/dec/106/table0}
	\end{longtable}
	\begin{longtable}{|l|l|}
		\hline
		\multirow{3}{*}{Drawing Transition diagram} 
		& \\
		& 
		
		$\begin{tikzpicture}[shorten >=1pt,node distance=2cm, scale =3, auto]
			\tikzstyle{every state}=[fill={rgb:black,1;white,10}]
			
			\node[state]   (q_1)                          {$1$};
			\node[state]   (q_2)  [right of=q_1]          {$2$};
			\node[state]   (q_3)  [below right of=q_1]          {$3$};
			
			\path[->]
			(q_1) edge [loop above] node {$\frac{1}{2}$}    (   )
			edge [bend left]  node {$\frac{1}{2}$}    (q_2)
			(q_2) edge [bend left]  node {$\frac{1}{2}$}    (q_3)
			edge [loop above] node {$\frac{1}{2}$}    ()
			(q_3) edge [bend left]  node {$\frac{1}{3}$}    (q_2)
			edge [bend left]  node {$\frac{1}{3}$}    (q_1)
			edge [loop below] node {$\frac{1}{3}$}    ();
		\end{tikzpicture}$
		
		\\  
		&\\
		&\\
		\hline
		\multirow{3}{*}{Checking whether the  } & \\
		& Here,\\chain is Irreducible
		& All the states are accessible to one another. \\and Aperiodic
		& $\implies$ They are in the same communication class. So, it is Irreducible.\\
		& \\
		& There exists the non- zero self-transition, which means that the chain \\
		& is Aperiodic.\\
		&\\ 
		& We know that if the Markov Chain is irreducible and aperiodic then \\
		& \qquad \qquad \qquad $\Vec{\pi}_{j} = \lim_{n \to \infty}P\{X_{n} = j\}$, $j = 1,...,N$ \\
		& These are the stationary probabilities. \\
		&\\
		\hline
		\multirow{3}{*}{Finding the Stationary} & \\
		& Stationary Probability can be represented as\\Probability Distributions
		& \qquad \qquad \qquad $\Vec{\pi} = \Vec{\pi} \vec{P}$\\
		& \\
		& \qquad $\implies$ $\myvec{v_{1}&&v_{2}&&v_{3}} = \myvec{v_{1}&&v_{2}&&v_{3}}\vec{P}$ \\
		& \\
		& Equating the above equation we get \\
		& \\
		& \qquad \qquad \qquad $\frac{1}{2}v_{1}-\frac{1}{3}v_{3} = 0$ $\label{eq:solutions/2018/dec/106/eq}$\\
		& \\
		& \qquad \qquad \qquad $\frac{1}{2}v_{1}-\frac{1}{2}v_{2} + \frac{1}{3}v_{3} = 0$\\
		& \\
		& \qquad \qquad \qquad $\frac{1}{2}v_{2}-\frac{2}{3}v_{3} = 0$\\
		& \\\
		& We see that summation of second and the third equation gives us the \\
		& first equation only. \\
		& And we know that the probability distribution will sum up to 1. \\
		& \\
		& \qquad \qquad \qquad $v_{1}+v_{2}+v_{3} = 1$ \\
		& \\
		& Therefore, we get the equation form as \\
		& \\
		& \qquad \qquad \qquad $\myvec{1&1&1\\\frac{1}{2}&0&\frac{-1}{3}\\\frac{1}{2}&\frac{-1}{2}&\frac{1}{3}}\myvec{v_{1}\\v_{2}\\v_{3}} = \myvec{1\\0\\0}$ \\
		& \\
		\hline
		\multirow{3}{*}{Solving the linear} & \\
		& The above linear equation can be solved using Gauss-Jordan method as\\equtions
		& \\
		& \qquad \qquad \qquad $\myvec{1&1&1&\vrule&1\\\frac{1}{2}&0&\frac{-1}{3}&\vrule&0\\\frac{1}{2}&\frac{-1}{2}&\frac{1}{3}&\vrule&0}$\\
		& \\
		& \qquad $\xleftrightarrow[]{R_2 \leftarrow R_2 - \frac{1}{2}R_1}$
		$\myvec{1&1&1&\vrule&1\\0&\frac{-1}{2}&\frac{-5}{6}&\vrule&\frac{-1}{2}\\\frac{1}{2}&\frac{-1}{2}&\frac{1}{3}&\vrule&0}$\\
		&\\
		& \qquad $\xleftrightarrow[]{R_3 \leftarrow R_3 - \frac{1}{2}R_1}$
		$\myvec{1&1&1&\vrule&1\\0&\frac{-1}{2}&\frac{-5}{6}&\vrule&\frac{-1}{2}\\0&-1&\frac{-1}{6}&\vrule&\frac{-1}{2}}$\\
		&\\
		& \qquad $\xleftrightarrow[]{R_2 \leftarrow \frac{-1}{2}R_2}$
		$\myvec{1&1&1&\vrule&1\\0&1&\frac{5}{3}&\vrule&1\\0&-1&\frac{-1}{6}&\vrule&\frac{-1}{2}}$\\
		&\\
		& \qquad $\xleftrightarrow[]{R_3 \leftarrow R_3 + R_2}$
		$\myvec{1&1&1&\vrule&1\\0&1&\frac{5}{3}&\vrule&1\\0&0&\frac{3}{2}&\vrule&\frac{1}{2}}$\\
		&\\
		& \qquad $\xleftrightarrow[]{R_3 \leftarrow \frac{3}{2}R_3}$
		$\myvec{1&1&1&\vrule&1\\0&1&\frac{5}{3}&\vrule&1\\0&0&1&\vrule&\frac{1}{3}}$\\
		&\\
		& \qquad $\xleftrightarrow[]{R_2 \leftarrow R_2 - \frac{5}{3}R_3}$
		$\myvec{1&1&1&\vrule&1\\0&1&0&\vrule&\frac{4}{9}\\0&0&1&\vrule&\frac{1}{3}}$\\
		&\\
		& \qquad $\xleftrightarrow[]{R_1 \leftarrow R_1 - R_3}$
		$\myvec{1&1&0&\vrule&\frac{2}{3}\\0&1&0&\vrule&\frac{4}{9}\\0&0&1&\vrule&\frac{1}{3}}$\\
		&\\
		& \qquad $\xleftrightarrow[]{R_1 \leftarrow R_1 - R_2}$
		$\myvec{1&0&0&\vrule&\frac{2}{9}\\0&1&0&\vrule&\frac{4}{9}\\0&0&1&\vrule&\frac{1}{3}}$\\
		&\\
		& $\therefore$, stationary probability distribution $\pi$ is given by \\
		& \qquad \qquad $\pi = \myvec{\frac{2}{9} & \frac{4}{9} & \frac{1}{3}}$ \\
		& \\
		\hline
		\multirow{3}{*}{Observations} & \\
		
		
		& Since the given transition probability matrix $\vec{P}$ is irreducible and aperiodic, \\
		& then $\lim_{n \to \infty} \vec{P}^{n}$ converges to a matrix with all rows identical and equal to $\vec{\pi}$. \\
		& \\
		& We were able to find $\vec{\pi}$ as $\myvec{\frac{2}{9} & \frac{4}{9} & \frac{1}{3}}$ \\
		& \\
		& $\lim_{n \to \infty} \vec{P}^{n} = \myvec{\frac{2}{9}&\frac{4}{9}&\frac{1}{3}\\\frac{2}{9}&\frac{4}{9}&\frac{1}{3}\\\frac{2}{9}&\frac{4}{9}&\frac{1}{3}}$\\
		& \\
		& From the above matrix, we get \\
		& \\
		& $\lim_{n \to \infty} \vec{P}^{n}_{11} = \frac{2}{9}$ \\
		&\\
		& $\lim_{n \to \infty} \vec{P}^{n}_{21} = \frac{2}{9}$ \\
		&\\
		& $\lim_{n \to \infty} \vec{P}^{n}_{32} = \frac{4}{9}$ \\
		&\\
		& $\lim_{n \to \infty} \vec{P}^{n}_{13} = \frac{1}{3}$ \\
		&\\
		\hline
		\multirow{3}{*}{Conclusion} & \\
		& From our observation we see that \\
		&\\
		& Options 1) and 4) are True.\\
		& \\
		\hline
\caption{}
\label{eq:solutions/2018/dec/106/table1}
	\end{longtable}
\twocolumn


\end{enumerate}
