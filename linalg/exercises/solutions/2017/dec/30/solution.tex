{\em Property of eigen values of A: }
%\label{eq:solutions/2017/dec/30/section:prop}
Let $\vec{A}$ be an arbitary $n\times n$ matrix of complex numbers with eigen values $\lambda_1,\lambda_2,\hdots,\lambda_n$. Then  the eigen values of k\textsuperscript{th} power of $\vec{A}$, that is the eigen values of $\vec{A}^k$, for any positive integer k are $\lambda_1^k,\lambda_2^k,\hdots,\lambda_n^k$.
Let us calculate the eigen values of $\vec{A}$.
\begin{align}
\vec{A}=\myvec{0&1\\-1&1}\\
\det({\vec{A}-\lambda\vec{I}})=0\\
\mydet{-\lambda&1\\-1&1-\lambda}=0\\
-\lambda(1-\lambda)+1=0\\
\lambda^2-\lambda+1=0\\
\implies \lambda=\frac{-1\pm \sqrt{3}i}{2}
\end{align}
From the above property,
%\ref{eq:solutions/2017/dec/30/section:prop}, 
the eigen values of $\vec{A}^n$ are $\lambda^n$. Also as it is given that $\vec{A}^n=\vec{I}$, 
\begin{align}
\implies \lambda^n=1\\
\implies \brak{\frac{-1\pm \sqrt{3}i}{2}}^n=1
\end{align}
Clearly $n\ne 1$. For $n=2$,
\begin{align}
\brak{\frac{-1\pm \sqrt{3}i}{2}}^2=\frac{-1\mp \sqrt{3}i}{2}
\end{align}
For $n=4$,
\begin{align}
\brak{\frac{-1\pm \sqrt{3}i}{2}}^4=\frac{-1\pm \sqrt{3}i}{2}
\end{align}
For $n=6$,
\begin{align}
\brak{\frac{-1\pm \sqrt{3}i}{2}}^6=1
\end{align}
Hence $n=6$ is the smallest positive integer.
