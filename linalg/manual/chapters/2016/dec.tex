\renewcommand{\theequation}{\theenumi}
\renewcommand{\thefigure}{\theenumi}
\renewcommand{\thetable}{\theenumi}
\begin{enumerate}[label=\thesection.\arabic*.,ref=\thesection.\theenumi]
\numberwithin{equation}{enumi}
\numberwithin{figure}{enumi}
\numberwithin{table}{enumi}

\item The matrix
\begin{align}
\vec{A} = \myvec{3 && -1 && 0 \\ -1 && 2 && -1\\ 0 && -1 && 3}
\end{align}
is
\begin{enumerate}
\item positive definite.
\item non-negative definite but not positive definite.
\item negative definite. 
\item neither negative definite nor positive definite. .
\end{enumerate}
%
\solution
The nullspace is given by 
\begin{align}
	\myvec{1 & 1 & 1 & 0 \\ 1 & 1 & 0 & 1\\ 0 & 0 & 0 & 0\\0 & 0 & 0 & 0}\myvec{x\\y\\z\\w} = \myvec{0 \\ 0 \\ 0 \\ 0}
\end{align}	
Row reducing the above matrix we get,
\begin{align}
	\myvec{1 & 1 & 1 & 0 \\ 1 & 1 & 0 & 1\\ 0 & 0 & 0 & 0\\0 & 0 & 0 & 0}
	\xleftrightarrow[R_2 \leftarrow R_2 \times -1]{R_2 \leftarrow R_2 - R_1}
	\myvec{1 & 1 & 1 & 0 \\ 0 & 0 & 1 & -1\\ 0 & 0 & 0 & 0\\0 & 0 & 0 & 0}\\
	\xleftrightarrow{R_1 \leftarrow R_1- R_2}
	\myvec{1 & 1 & 0 & 1 \\ 0 & 0 & 1 & -1\\ 0 & 0 & 0 & 0\\0 & 0 & 0 & 0} \label{eq:solutions/2017/dec/27/eq:rref}
\end{align}
See Table \ref{eq:solutions/2017/dec/27/tab}

\begin{table*}[!ht]
	\begin{tabular}{|m{4.5cm}|l|}
		\hline
		&\\
		dim(C$(\vec{A})) = 1$ 
		& \textbf{False}. Because the number of pivot variables are 2 as obtained in \eqref{eq:solutions/2017/dec/27/eq:rref}\\
		&\\
		\hline
		&\\
		dim(C$(\vec{A})) = 2$
		& \textbf{True}. Since the number of pivot variables are 2, the rank of $\vec{A}$ is 2.\\
		&$\therefore dim(C(\vec{A})) = 2 \quad [\because dim(C(\vec{A})) = rank(\vec{A})]$ \\
		&\\
		\hline
		&\\
	     rank$(\vec{A}) = 1$
		& \textbf{False}. Because the rank$(\vec{A}) = 2$, as the number of pivot variables are 2\\
		&\\
		\hline
		&\\
		$\vec{S}$ = $\cbrak{(1, 1, 1, 0), (1, 1, 0, 1)}$ is a basis of $N(\vec{A})$
		& \textbf{False}. \\
		& Let, \\
		&  $\vec{u} = \myvec{1\\1\\1\\0}, \vec{v} = \myvec{1\\1\\0\\1}$\\ 
		&Consider, \\
		&$\myvec{1 & 1 & 1 & 0 \\ 1 & 1 & 0 & 1\\ 0 & 0 & 0 & 0\\0 & 0 & 0 & 0}\myvec{1\\1\\1\\0} = \myvec{3\\2\\0\\0} \not = \myvec{0\\0\\0\\0}$\\
		& Similarly,\\
		&$\myvec{1 & 1 & 1 & 0 \\ 1 & 1 & 0 & 1\\ 0 & 0 & 0 & 0\\0 & 0 & 0 & 0}\myvec{1\\1\\0\\1} = \myvec{2\\3\\0\\0} \not = \myvec{0\\0\\0\\0}$ \\
		&Hence, the given vectors do not form the basis.\\
		\hline
	\end{tabular}
\caption{}
\label{eq:solutions/2017/dec/27/tab}
\end{table*}

\item Which of the following sets of functions from $\Re$ to $\Re$ is a vector space over $\Re$?
\begin{align}
S_1 = \{f|\lim_{x\to3} f(x) = 0\}\\
S_2 = \{g|\lim_{x\to3} g(x) = 1\}\\
S_3 = \{h|\lim_{x\to3} h(x)~exists\}
\end{align}
is
\begin{enumerate}
\item Only $S_1$ 
\item Only $S_2$
\item $S_1$ and $S_3$ but not $S_2$ 
\item All the three are vector spaces
\end{enumerate}
%
\solution
The nullspace is given by 
\begin{align}
	\myvec{1 & 1 & 1 & 0 \\ 1 & 1 & 0 & 1\\ 0 & 0 & 0 & 0\\0 & 0 & 0 & 0}\myvec{x\\y\\z\\w} = \myvec{0 \\ 0 \\ 0 \\ 0}
\end{align}	
Row reducing the above matrix we get,
\begin{align}
	\myvec{1 & 1 & 1 & 0 \\ 1 & 1 & 0 & 1\\ 0 & 0 & 0 & 0\\0 & 0 & 0 & 0}
	\xleftrightarrow[R_2 \leftarrow R_2 \times -1]{R_2 \leftarrow R_2 - R_1}
	\myvec{1 & 1 & 1 & 0 \\ 0 & 0 & 1 & -1\\ 0 & 0 & 0 & 0\\0 & 0 & 0 & 0}\\
	\xleftrightarrow{R_1 \leftarrow R_1- R_2}
	\myvec{1 & 1 & 0 & 1 \\ 0 & 0 & 1 & -1\\ 0 & 0 & 0 & 0\\0 & 0 & 0 & 0} \label{eq:solutions/2017/dec/27/eq:rref}
\end{align}
See Table \ref{eq:solutions/2017/dec/27/tab}

\begin{table*}[!ht]
	\begin{tabular}{|m{4.5cm}|l|}
		\hline
		&\\
		dim(C$(\vec{A})) = 1$ 
		& \textbf{False}. Because the number of pivot variables are 2 as obtained in \eqref{eq:solutions/2017/dec/27/eq:rref}\\
		&\\
		\hline
		&\\
		dim(C$(\vec{A})) = 2$
		& \textbf{True}. Since the number of pivot variables are 2, the rank of $\vec{A}$ is 2.\\
		&$\therefore dim(C(\vec{A})) = 2 \quad [\because dim(C(\vec{A})) = rank(\vec{A})]$ \\
		&\\
		\hline
		&\\
	     rank$(\vec{A}) = 1$
		& \textbf{False}. Because the rank$(\vec{A}) = 2$, as the number of pivot variables are 2\\
		&\\
		\hline
		&\\
		$\vec{S}$ = $\cbrak{(1, 1, 1, 0), (1, 1, 0, 1)}$ is a basis of $N(\vec{A})$
		& \textbf{False}. \\
		& Let, \\
		&  $\vec{u} = \myvec{1\\1\\1\\0}, \vec{v} = \myvec{1\\1\\0\\1}$\\ 
		&Consider, \\
		&$\myvec{1 & 1 & 1 & 0 \\ 1 & 1 & 0 & 1\\ 0 & 0 & 0 & 0\\0 & 0 & 0 & 0}\myvec{1\\1\\1\\0} = \myvec{3\\2\\0\\0} \not = \myvec{0\\0\\0\\0}$\\
		& Similarly,\\
		&$\myvec{1 & 1 & 1 & 0 \\ 1 & 1 & 0 & 1\\ 0 & 0 & 0 & 0\\0 & 0 & 0 & 0}\myvec{1\\1\\0\\1} = \myvec{2\\3\\0\\0} \not = \myvec{0\\0\\0\\0}$ \\
		&Hence, the given vectors do not form the basis.\\
		\hline
	\end{tabular}
\caption{}
\label{eq:solutions/2017/dec/27/tab}
\end{table*}

\item Let $\vec{A}$ be an n $\times$ m matrix with each entry equal to +1,-1 or 0 such that every column has exactly one +1 and exactly one -1. We can conclude that\\
\begin{align}
    &\mbox{1. Rank } \vec{A}\leq n-1\\
    &\mbox{2. Rank } \vec{A}=m\\    
    &\mbox{3. }n\leq m\\
    &\mbox{4. }n-1\leq m
\end{align}
%
%
\solution
The nullspace is given by 
\begin{align}
	\myvec{1 & 1 & 1 & 0 \\ 1 & 1 & 0 & 1\\ 0 & 0 & 0 & 0\\0 & 0 & 0 & 0}\myvec{x\\y\\z\\w} = \myvec{0 \\ 0 \\ 0 \\ 0}
\end{align}	
Row reducing the above matrix we get,
\begin{align}
	\myvec{1 & 1 & 1 & 0 \\ 1 & 1 & 0 & 1\\ 0 & 0 & 0 & 0\\0 & 0 & 0 & 0}
	\xleftrightarrow[R_2 \leftarrow R_2 \times -1]{R_2 \leftarrow R_2 - R_1}
	\myvec{1 & 1 & 1 & 0 \\ 0 & 0 & 1 & -1\\ 0 & 0 & 0 & 0\\0 & 0 & 0 & 0}\\
	\xleftrightarrow{R_1 \leftarrow R_1- R_2}
	\myvec{1 & 1 & 0 & 1 \\ 0 & 0 & 1 & -1\\ 0 & 0 & 0 & 0\\0 & 0 & 0 & 0} \label{eq:solutions/2017/dec/27/eq:rref}
\end{align}
See Table \ref{eq:solutions/2017/dec/27/tab}

\begin{table*}[!ht]
	\begin{tabular}{|m{4.5cm}|l|}
		\hline
		&\\
		dim(C$(\vec{A})) = 1$ 
		& \textbf{False}. Because the number of pivot variables are 2 as obtained in \eqref{eq:solutions/2017/dec/27/eq:rref}\\
		&\\
		\hline
		&\\
		dim(C$(\vec{A})) = 2$
		& \textbf{True}. Since the number of pivot variables are 2, the rank of $\vec{A}$ is 2.\\
		&$\therefore dim(C(\vec{A})) = 2 \quad [\because dim(C(\vec{A})) = rank(\vec{A})]$ \\
		&\\
		\hline
		&\\
	     rank$(\vec{A}) = 1$
		& \textbf{False}. Because the rank$(\vec{A}) = 2$, as the number of pivot variables are 2\\
		&\\
		\hline
		&\\
		$\vec{S}$ = $\cbrak{(1, 1, 1, 0), (1, 1, 0, 1)}$ is a basis of $N(\vec{A})$
		& \textbf{False}. \\
		& Let, \\
		&  $\vec{u} = \myvec{1\\1\\1\\0}, \vec{v} = \myvec{1\\1\\0\\1}$\\ 
		&Consider, \\
		&$\myvec{1 & 1 & 1 & 0 \\ 1 & 1 & 0 & 1\\ 0 & 0 & 0 & 0\\0 & 0 & 0 & 0}\myvec{1\\1\\1\\0} = \myvec{3\\2\\0\\0} \not = \myvec{0\\0\\0\\0}$\\
		& Similarly,\\
		&$\myvec{1 & 1 & 1 & 0 \\ 1 & 1 & 0 & 1\\ 0 & 0 & 0 & 0\\0 & 0 & 0 & 0}\myvec{1\\1\\0\\1} = \myvec{2\\3\\0\\0} \not = \myvec{0\\0\\0\\0}$ \\
		&Hence, the given vectors do not form the basis.\\
		\hline
	\end{tabular}
\caption{}
\label{eq:solutions/2017/dec/27/tab}
\end{table*}


\item %
Let $\vec{A}=\myvec{1 & 1\\1 & 0}$ and let $\alpha_n$ and $\beta_n$ denote the two eigenvalues of $\vec{A}^n$ such that $\abs{\alpha_n}\geq\abs{\beta_n}$.\\
Then
\begin{enumerate}
    \item $\alpha_n\rightarrow \infty$ as $n\rightarrow \infty$
    \item $\beta_n\rightarrow 0$ as $n\rightarrow \infty$
    \item $\beta_n$ is positive if n is even.
    \item $\beta_n$ is negative if n is odd.
\end{enumerate}
%
\solution
The nullspace is given by 
\begin{align}
	\myvec{1 & 1 & 1 & 0 \\ 1 & 1 & 0 & 1\\ 0 & 0 & 0 & 0\\0 & 0 & 0 & 0}\myvec{x\\y\\z\\w} = \myvec{0 \\ 0 \\ 0 \\ 0}
\end{align}	
Row reducing the above matrix we get,
\begin{align}
	\myvec{1 & 1 & 1 & 0 \\ 1 & 1 & 0 & 1\\ 0 & 0 & 0 & 0\\0 & 0 & 0 & 0}
	\xleftrightarrow[R_2 \leftarrow R_2 \times -1]{R_2 \leftarrow R_2 - R_1}
	\myvec{1 & 1 & 1 & 0 \\ 0 & 0 & 1 & -1\\ 0 & 0 & 0 & 0\\0 & 0 & 0 & 0}\\
	\xleftrightarrow{R_1 \leftarrow R_1- R_2}
	\myvec{1 & 1 & 0 & 1 \\ 0 & 0 & 1 & -1\\ 0 & 0 & 0 & 0\\0 & 0 & 0 & 0} \label{eq:solutions/2017/dec/27/eq:rref}
\end{align}
See Table \ref{eq:solutions/2017/dec/27/tab}

\begin{table*}[!ht]
	\begin{tabular}{|m{4.5cm}|l|}
		\hline
		&\\
		dim(C$(\vec{A})) = 1$ 
		& \textbf{False}. Because the number of pivot variables are 2 as obtained in \eqref{eq:solutions/2017/dec/27/eq:rref}\\
		&\\
		\hline
		&\\
		dim(C$(\vec{A})) = 2$
		& \textbf{True}. Since the number of pivot variables are 2, the rank of $\vec{A}$ is 2.\\
		&$\therefore dim(C(\vec{A})) = 2 \quad [\because dim(C(\vec{A})) = rank(\vec{A})]$ \\
		&\\
		\hline
		&\\
	     rank$(\vec{A}) = 1$
		& \textbf{False}. Because the rank$(\vec{A}) = 2$, as the number of pivot variables are 2\\
		&\\
		\hline
		&\\
		$\vec{S}$ = $\cbrak{(1, 1, 1, 0), (1, 1, 0, 1)}$ is a basis of $N(\vec{A})$
		& \textbf{False}. \\
		& Let, \\
		&  $\vec{u} = \myvec{1\\1\\1\\0}, \vec{v} = \myvec{1\\1\\0\\1}$\\ 
		&Consider, \\
		&$\myvec{1 & 1 & 1 & 0 \\ 1 & 1 & 0 & 1\\ 0 & 0 & 0 & 0\\0 & 0 & 0 & 0}\myvec{1\\1\\1\\0} = \myvec{3\\2\\0\\0} \not = \myvec{0\\0\\0\\0}$\\
		& Similarly,\\
		&$\myvec{1 & 1 & 1 & 0 \\ 1 & 1 & 0 & 1\\ 0 & 0 & 0 & 0\\0 & 0 & 0 & 0}\myvec{1\\1\\0\\1} = \myvec{2\\3\\0\\0} \not = \myvec{0\\0\\0\\0}$ \\
		&Hence, the given vectors do not form the basis.\\
		\hline
	\end{tabular}
\caption{}
\label{eq:solutions/2017/dec/27/tab}
\end{table*}

\item Let $M_n$ denote the vector space of all $n\times n$ real matrices. Which of the following is a linear subspaces of $M_n$ :-
\begin{enumerate}
\item $ V_1 = \{  A \in M_n : \text{ A is nonsingular} \}$
\item $ V_2 = \{  A \in M_n : det(A) = 0 \}$
\item $ V_3 = \{  A \in M_n : trace(A) = 0 \}$
\item $ V_4 = \{  BA : A \in M_n\},$ where $ B$ is some fixed matrix in $ M_n$
\end{enumerate}
%
\solution
The nullspace is given by 
\begin{align}
	\myvec{1 & 1 & 1 & 0 \\ 1 & 1 & 0 & 1\\ 0 & 0 & 0 & 0\\0 & 0 & 0 & 0}\myvec{x\\y\\z\\w} = \myvec{0 \\ 0 \\ 0 \\ 0}
\end{align}	
Row reducing the above matrix we get,
\begin{align}
	\myvec{1 & 1 & 1 & 0 \\ 1 & 1 & 0 & 1\\ 0 & 0 & 0 & 0\\0 & 0 & 0 & 0}
	\xleftrightarrow[R_2 \leftarrow R_2 \times -1]{R_2 \leftarrow R_2 - R_1}
	\myvec{1 & 1 & 1 & 0 \\ 0 & 0 & 1 & -1\\ 0 & 0 & 0 & 0\\0 & 0 & 0 & 0}\\
	\xleftrightarrow{R_1 \leftarrow R_1- R_2}
	\myvec{1 & 1 & 0 & 1 \\ 0 & 0 & 1 & -1\\ 0 & 0 & 0 & 0\\0 & 0 & 0 & 0} \label{eq:solutions/2017/dec/27/eq:rref}
\end{align}
See Table \ref{eq:solutions/2017/dec/27/tab}

\begin{table*}[!ht]
	\begin{tabular}{|m{4.5cm}|l|}
		\hline
		&\\
		dim(C$(\vec{A})) = 1$ 
		& \textbf{False}. Because the number of pivot variables are 2 as obtained in \eqref{eq:solutions/2017/dec/27/eq:rref}\\
		&\\
		\hline
		&\\
		dim(C$(\vec{A})) = 2$
		& \textbf{True}. Since the number of pivot variables are 2, the rank of $\vec{A}$ is 2.\\
		&$\therefore dim(C(\vec{A})) = 2 \quad [\because dim(C(\vec{A})) = rank(\vec{A})]$ \\
		&\\
		\hline
		&\\
	     rank$(\vec{A}) = 1$
		& \textbf{False}. Because the rank$(\vec{A}) = 2$, as the number of pivot variables are 2\\
		&\\
		\hline
		&\\
		$\vec{S}$ = $\cbrak{(1, 1, 1, 0), (1, 1, 0, 1)}$ is a basis of $N(\vec{A})$
		& \textbf{False}. \\
		& Let, \\
		&  $\vec{u} = \myvec{1\\1\\1\\0}, \vec{v} = \myvec{1\\1\\0\\1}$\\ 
		&Consider, \\
		&$\myvec{1 & 1 & 1 & 0 \\ 1 & 1 & 0 & 1\\ 0 & 0 & 0 & 0\\0 & 0 & 0 & 0}\myvec{1\\1\\1\\0} = \myvec{3\\2\\0\\0} \not = \myvec{0\\0\\0\\0}$\\
		& Similarly,\\
		&$\myvec{1 & 1 & 1 & 0 \\ 1 & 1 & 0 & 1\\ 0 & 0 & 0 & 0\\0 & 0 & 0 & 0}\myvec{1\\1\\0\\1} = \myvec{2\\3\\0\\0} \not = \myvec{0\\0\\0\\0}$ \\
		&Hence, the given vectors do not form the basis.\\
		\hline
	\end{tabular}
\caption{}
\label{eq:solutions/2017/dec/27/tab}
\end{table*}


\item If $\vec{P}$ and $\vec{Q}$ are invertible matrices such that   
$\vec{P}\vec{Q} = -\vec{Q}\vec{P}$,then we can conclude that
\begin{enumerate}
\item  $Tr(\vec{P})=Tr(\vec{Q})=0$ \label{eq:solutions/2016/dec/75/1}
\item  $Tr(\vec{P})=Tr(\vec{Q})=1$ \label{eq:solutions/2016/dec/75/2}
\item  $Tr(\vec{P})=-Tr(\vec{Q})$ \label{eq:solutions/2016/dec/75/3}
\item  $Tr(\vec{P}) \neq Tr(\vec{Q})$ \label{eq:solutions/2016/dec/75/4}
\end{enumerate}
%
%
\solution
The nullspace is given by 
\begin{align}
	\myvec{1 & 1 & 1 & 0 \\ 1 & 1 & 0 & 1\\ 0 & 0 & 0 & 0\\0 & 0 & 0 & 0}\myvec{x\\y\\z\\w} = \myvec{0 \\ 0 \\ 0 \\ 0}
\end{align}	
Row reducing the above matrix we get,
\begin{align}
	\myvec{1 & 1 & 1 & 0 \\ 1 & 1 & 0 & 1\\ 0 & 0 & 0 & 0\\0 & 0 & 0 & 0}
	\xleftrightarrow[R_2 \leftarrow R_2 \times -1]{R_2 \leftarrow R_2 - R_1}
	\myvec{1 & 1 & 1 & 0 \\ 0 & 0 & 1 & -1\\ 0 & 0 & 0 & 0\\0 & 0 & 0 & 0}\\
	\xleftrightarrow{R_1 \leftarrow R_1- R_2}
	\myvec{1 & 1 & 0 & 1 \\ 0 & 0 & 1 & -1\\ 0 & 0 & 0 & 0\\0 & 0 & 0 & 0} \label{eq:solutions/2017/dec/27/eq:rref}
\end{align}
See Table \ref{eq:solutions/2017/dec/27/tab}

\begin{table*}[!ht]
	\begin{tabular}{|m{4.5cm}|l|}
		\hline
		&\\
		dim(C$(\vec{A})) = 1$ 
		& \textbf{False}. Because the number of pivot variables are 2 as obtained in \eqref{eq:solutions/2017/dec/27/eq:rref}\\
		&\\
		\hline
		&\\
		dim(C$(\vec{A})) = 2$
		& \textbf{True}. Since the number of pivot variables are 2, the rank of $\vec{A}$ is 2.\\
		&$\therefore dim(C(\vec{A})) = 2 \quad [\because dim(C(\vec{A})) = rank(\vec{A})]$ \\
		&\\
		\hline
		&\\
	     rank$(\vec{A}) = 1$
		& \textbf{False}. Because the rank$(\vec{A}) = 2$, as the number of pivot variables are 2\\
		&\\
		\hline
		&\\
		$\vec{S}$ = $\cbrak{(1, 1, 1, 0), (1, 1, 0, 1)}$ is a basis of $N(\vec{A})$
		& \textbf{False}. \\
		& Let, \\
		&  $\vec{u} = \myvec{1\\1\\1\\0}, \vec{v} = \myvec{1\\1\\0\\1}$\\ 
		&Consider, \\
		&$\myvec{1 & 1 & 1 & 0 \\ 1 & 1 & 0 & 1\\ 0 & 0 & 0 & 0\\0 & 0 & 0 & 0}\myvec{1\\1\\1\\0} = \myvec{3\\2\\0\\0} \not = \myvec{0\\0\\0\\0}$\\
		& Similarly,\\
		&$\myvec{1 & 1 & 1 & 0 \\ 1 & 1 & 0 & 1\\ 0 & 0 & 0 & 0\\0 & 0 & 0 & 0}\myvec{1\\1\\0\\1} = \myvec{2\\3\\0\\0} \not = \myvec{0\\0\\0\\0}$ \\
		&Hence, the given vectors do not form the basis.\\
		\hline
	\end{tabular}
\caption{}
\label{eq:solutions/2017/dec/27/tab}
\end{table*}

\twocolumn
\item Let $\vec{W_1}$, $\vec{W_2}$, $\vec{W_3}$ be 3 distinct subspaces of $\vec{R}^{10}$ such that each $\vec{W_i}$ has dimension of 9. Let $\vec{W} = \vec{W_1} \cap \vec{W_2} \cap \vec{W_3}$. Then we can conclude that\\
\begin{enumerate}
\item $\vec{W}$ may not be a subspace of $\vec{R}^{10}$\\
\item dim $\vec{W} \leq 8$\\
\item dim $\vec{W} \geq 7$\\
\item  dim $\vec{W} \leq 3$\\
\end{enumerate}
 %
\solution
The nullspace is given by 
\begin{align}
	\myvec{1 & 1 & 1 & 0 \\ 1 & 1 & 0 & 1\\ 0 & 0 & 0 & 0\\0 & 0 & 0 & 0}\myvec{x\\y\\z\\w} = \myvec{0 \\ 0 \\ 0 \\ 0}
\end{align}	
Row reducing the above matrix we get,
\begin{align}
	\myvec{1 & 1 & 1 & 0 \\ 1 & 1 & 0 & 1\\ 0 & 0 & 0 & 0\\0 & 0 & 0 & 0}
	\xleftrightarrow[R_2 \leftarrow R_2 \times -1]{R_2 \leftarrow R_2 - R_1}
	\myvec{1 & 1 & 1 & 0 \\ 0 & 0 & 1 & -1\\ 0 & 0 & 0 & 0\\0 & 0 & 0 & 0}\\
	\xleftrightarrow{R_1 \leftarrow R_1- R_2}
	\myvec{1 & 1 & 0 & 1 \\ 0 & 0 & 1 & -1\\ 0 & 0 & 0 & 0\\0 & 0 & 0 & 0} \label{eq:solutions/2017/dec/27/eq:rref}
\end{align}
See Table \ref{eq:solutions/2017/dec/27/tab}

\begin{table*}[!ht]
	\begin{tabular}{|m{4.5cm}|l|}
		\hline
		&\\
		dim(C$(\vec{A})) = 1$ 
		& \textbf{False}. Because the number of pivot variables are 2 as obtained in \eqref{eq:solutions/2017/dec/27/eq:rref}\\
		&\\
		\hline
		&\\
		dim(C$(\vec{A})) = 2$
		& \textbf{True}. Since the number of pivot variables are 2, the rank of $\vec{A}$ is 2.\\
		&$\therefore dim(C(\vec{A})) = 2 \quad [\because dim(C(\vec{A})) = rank(\vec{A})]$ \\
		&\\
		\hline
		&\\
	     rank$(\vec{A}) = 1$
		& \textbf{False}. Because the rank$(\vec{A}) = 2$, as the number of pivot variables are 2\\
		&\\
		\hline
		&\\
		$\vec{S}$ = $\cbrak{(1, 1, 1, 0), (1, 1, 0, 1)}$ is a basis of $N(\vec{A})$
		& \textbf{False}. \\
		& Let, \\
		&  $\vec{u} = \myvec{1\\1\\1\\0}, \vec{v} = \myvec{1\\1\\0\\1}$\\ 
		&Consider, \\
		&$\myvec{1 & 1 & 1 & 0 \\ 1 & 1 & 0 & 1\\ 0 & 0 & 0 & 0\\0 & 0 & 0 & 0}\myvec{1\\1\\1\\0} = \myvec{3\\2\\0\\0} \not = \myvec{0\\0\\0\\0}$\\
		& Similarly,\\
		&$\myvec{1 & 1 & 1 & 0 \\ 1 & 1 & 0 & 1\\ 0 & 0 & 0 & 0\\0 & 0 & 0 & 0}\myvec{1\\1\\0\\1} = \myvec{2\\3\\0\\0} \not = \myvec{0\\0\\0\\0}$ \\
		&Hence, the given vectors do not form the basis.\\
		\hline
	\end{tabular}
\caption{}
\label{eq:solutions/2017/dec/27/tab}
\end{table*}

\twocolumn


%\item Consider a Markov Chain with state space $\cbrak{0,1,2}$ and transition matrix
%\begin{align}
%P = 
%\begin{blockarray}{c@{\hspace{1pt}}rrr@{\hspace{3pt}}}
%         & 0   & 1   & 2 \\
%        \begin{block}{r@{\hspace{3pt}}@{\hspace{1pt}}
%    (@{\hspace{1pt}}rrr@{\hspace{1pt}}@{\hspace{1pt}})}
%        0 & \frac{1}{2} & \frac{1}{2} & 0  \\
%        1 & 0 &\frac{1}{2}  & \frac{3}{4}  \\
%%
%        2 &  \frac{1}{3} & \frac{1}{3} & \frac{1}{3}  \\
%        \end{block}
%    \end{blockarray}
%\end{align}
%For any two states $i$ and $j$, let $p_{ij}^{(n)}$ denote the $n$-step transition probability of going from $i$ to $j$.  Identify correct statements.
%\begin{enumerate}
%\item $\lim_{n \to \infty} p_{11}^{(n)} = \frac{2}{9}$
%\item $\lim_{n \to \infty} p_{21}^{(n)} = 0$
%\item $\lim_{n \to \infty} p_{32}^{(n)} = \frac{1}{3}$
%\item $\lim_{n \to \infty} p_{13}^{(n)} = \frac{1}{3}$
%\end{enumerate}

\end{enumerate}
