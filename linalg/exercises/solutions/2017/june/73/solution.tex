See Table \ref{eq:solutions/2017/june/73/1}
\begin{table*}[!ht]
        \centering
	\begin{tabular}{|m{2.0in}|m{5.0in}|} \hline
		\textbf{Objective} & \textbf{Explanation} \\ \hline
		Eigenvalues of $\vec{A}$ & Since 
		\begin{align}
			\vec{A}^2=\vec{A} \\
		        \implies \vec{A}^2-\vec{A}=\vec{O}
		\end{align} 
From Cayley-Hamilton Theorem we have,
\begin{align}
        \lambda^2-\lambda=0 \\
        \implies \lambda\brak{\lambda-1}=0 \\
        \implies \lambda=0,1
\end{align} 
	A matrix $\vec{A}$ satisfying $\vec{A}^2=\vec{A}$ is an idempotent matrix with eigen values
equal to 0 or 1. 	\\ \hline
Check if $\vec{A}$ is necessary diagonal & Consider
                \begin{align}
                        \vec{A}=\myvec{1&-1\\0&0}\\
                \end{align}
                Then,
                \begin{align}
                        \vec{A}^2=\myvec{1&-1\\0&0}\myvec{1&-1\\0&0}\\
                        =\myvec{1&-1\\0&0} \\
                        =\vec{A}
                \end{align}
                Hence $\vec{A}$ is idempotent but not diagonal. \\ \hline
Relation between rank and trace of $\vec{A}$ & Rank of matrix is defined as the number of non-zero eigenvalues. Since number of non-zero eigenvalues is 1,  
\begin{align} 
	rank(\vec{A})=1 \\
	trace(\vec{A})=\sum_i \lambda_i = 0+1 =1 \\
	\implies rank(\vec{A})=trace(\vec{A})
\end{align} \\ \hline
Relation between rank and trace of $\vec{I}-\vec{A}$ & Now for the matrix $\vec{I}-\vec{A}$ we have,
\begin{align}
	\brak{\vec{I}-\vec{A}}^2 = \brak{\vec{I}-\vec{A}}\brak{\vec{I}-\vec{A}}\\
	= \vec{I}^2-\vec{IA}-\vec{AI}+\vec{A}^2 \\
	= \vec{I}-\vec{A}-\vec{A}+\vec{A} \\
	= \vec{I}-\vec{A}
\end{align}
Hence $\vec{I}-\vec{A}$ is an idempotent matrix. Therefore we conclude,
\begin{align}
        rank(\vec{I}-\vec{A})=trace(\vec{I}-\vec{A})
\end{align} \\ \hline
		Answer& (1),(3) and (4) \\ \hline
        \end{tabular}
        \caption{} \label{eq:solutions/2017/june/73/1}
\end{table*}
