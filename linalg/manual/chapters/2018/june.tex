\renewcommand{\theequation}{\theenumi}
\renewcommand{\thefigure}{\theenumi}
\begin{enumerate}[label=\thesection.\arabic*.,ref=\thesection.\theenumi]
\numberwithin{equation}{enumi}
\numberwithin{figure}{enumi}

\item Let $\vec{A}$ be a $\brak{m \times n}$ matrix 
and $\vec{B}$ be a $\brak{n \times m}$ matrix over real numbers with $m < n$.  Then
\begin{enumerate}
\item $\vec{A}\vec{B}$ is always nonsingular.
\item $\vec{A}\vec{B}$ is always singular.
\item $\vec{B}\vec{A}$ is always nonsingular.
\item $\vec{B}\vec{A}$ is always singular.
\end{enumerate}
%
\item If $\vec{A}$ is a $\brak{2\times 2}$ matrix over $\mathbb{R}$ with $det\brak{\vec{A}+\vec{I}} 
= 1 + det\brak{\vec{A}}$.  Then we can conclude that
\begin{enumerate}
\item $det\brak{\vec{A}} = 0$.
\item $\vec{A} = 0$.
\item $tr\brak{\vec{A}} = 0$.
\item $\vec{A}$ is nonsingular.
\end{enumerate}
%
\item The system of equations
\begin{align}
x+2x^2+3xy = 6 \\
x+x^2+3xy + y = 5 \\
x-x^2+y = 7
\end{align}
\begin{enumerate}
\item has solutions in rational numbers.
\item has solutions in real numbers.
\item has solutions in complex numbers.
\item has no solutions.
\end{enumerate}
%
\item The trace of the matrix
\begin{align}
\myvec
{
2 & 1 & 0
\\
0 & 2 & 0
\\
0 & 0 & 3
}^{20}
\end{align}
is
\begin{enumerate}
\item $7^{20}$.
\item $2^{20}+3^{20}$.
\item $2^{21}+3^{20}$.
\item $2^{20}+3^{20}+1$.
\end{enumerate}
%
\item Given that there are real constants $a,b,c,d$ such that the identity
\begin{multline}
\lambda x^2 + 2xy + y^2 = \brak{ax+by}^2 + \brak{cx+dy}^2, 
\\
 \forall x,y \in \mathbb{R}
\end{multline}
This implies that
\begin{enumerate}
\item $\lambda = -5$
\item $\lambda \ge 1$
\item $0 < \lambda < 1$
\item There is no such $\lambda \in \mathbb{R}$
\end{enumerate}
%
\item Let $\mathbb{R}, n \ge 2$, be equipped with the standard inner product.  Let
$\vec{v}_1,\vec{v}_2,\dots,\vec{v}_n$ be $n$ column vectors forming an orthonormal
basis of $\mathbb{R}^n$.  Let $A$ be the $n \times n$ matrix formed by the column vectors
$\vec{v}_1,\vec{v}_2,\dots,\vec{v}_n$.  Then 
\begin{enumerate}
\begin{multicols}{2}
\item $\vec{A}=\vec{A}^{-1}$
\item $\vec{A}=\vec{A}^{\top}$
\item $\vec{A}^{-1}=\vec{A}^{\top}$
\item $det\brak{\vec{A}}=1$
\end{multicols}
\end{enumerate}
%
\item Consider a Markov Chain with state space $\cbrak{1,2, 3, 4}$ and transition matrix
\begin{align}
P = 
\begin{blockarray}{c@{\hspace{1pt}}rrrr@{\hspace{3pt}}}
            & 1   & 2 & 3 & 4\\
        \begin{block}{r@{\hspace{3pt}}@{\hspace{1pt}}
    (@{\hspace{1pt}}rrrr@{\hspace{1pt}}@{\hspace{1pt}})}
        1 & \frac{1}{2} & 0 & \frac{1}{2} & 0  \\
        2 & \frac{1}{4}  & \frac{1}{4} & \frac{1}{4} & \frac{1}{4}  \\
        3 & \frac{1}{3}  & 0 & \frac{1}{3} & \frac{1}{3}  \\
%
        4 & \frac{1}{2} & 0 & \frac{1}{2} & 0  \\
        \end{block}
    \end{blockarray}
\end{align}
Then,
\begin{enumerate}
\item $\lim_{n \to \infty} p_{22}^{(n)} = 0, \sum_{n=0}^{\infty}p_{22}^{(n)} = \infty$
\item $\lim_{n \to \infty} p_{22}^{(n)} = 0, \sum_{n=0}^{\infty}p_{22}^{(n)} < \infty$
\item $\lim_{n \to \infty} p_{22}^{(n)} = 1, \sum_{n=0}^{\infty}p_{22}^{(n)} = \infty$
\item $\lim_{n \to \infty} p_{22}^{(n)} = 1, \sum_{n=0}^{\infty}p_{22}^{(n)} < \infty$
\end{enumerate}
\item Let $V$ denote the vector space of all sequences $\vec{a} = \brak{a_1,a_2,\dots }$ of real numbers such that 
\begin{align}
\sum_{n}2^n\abs{a}_n
\end{align}
converges.  Define
\begin{align}
\norm{\cdot}: V \to \mathbb{R}
\end{align}
by
\begin{align}
\norm{\vec{a}} = \sum_{n}2^n\abs{a}_n.
\end{align}
%
Which of the following are true?
\begin{enumerate}
\item $V$ contains onlythe sequence $\brak{0,0,\dots}$
\item $V$ is finite dimensional
\item $V$ has a countable linear basis
\item $V$ is a complete normed space
\end{enumerate}
%
\item Let $V$ be a vector space over $\mathbb{C}$ with dimension $n$.  Let $T:V\to V$ be a linear transformation with
only1 as eigenvalue.  Then which of the following must be true?
\begin{enumerate}
\item $T-I = 0$ 
\item $\brak{T-I}^{n-1} = 0$ 
\item $\brak{T-I}^{n} = 0$ 
\item $\brak{T-I}^{2n} = 0$ 
\end{enumerate}
%
\item If $\vec{A}$ is a $5\times 5$ matrix and the dimension of the solution space of $\vec{A}\vec{x} = 0$ is at least two, then
\begin{enumerate}
\item rank$\brak{\vec{A}^2} \le 3$ 
\item rank$\brak{\vec{A}^2} \ge 3$ 
\item rank$\brak{\vec{A}^2} = 3$ 
\item det$\brak{\vec{A}^2} = 0$ 
\end{enumerate}
%
\item Let $\vec{A} \in M_3\brak{\mathbb{R}}$ be such that $\vec{A}^3 = \vec{I}_{3\times 3}$.  Then
\begin{enumerate}
\item minimal polynomial of $\vec{A}$ can only be of degree 2
\item minimal polynomial of $\vec{A}$ can only be of degree 3
\item either $\vec{A} = \vec{I}$ or $\vec{A} = -\vec{I}$
\item there can be uncountably many $\vec{A}$ satisfying the above.
\end{enumerate}
%
\item Let $\vec{A}$ be an $n \times n, n > 1$ matrix satisfying
\begin{align}
\vec{A}^2 - 7\vec{A} + 12\vec{I} = \vec{0}
\end{align}
%
Then which of the following statements is true?
\begin{enumerate}
\item  $\vec{A}$ is invertible
\item $t^2-7t+12n = 0$ where $t = tr\brak{\vec{A}}$
\item $d^2-7d+12 = 0$ where $d = det\brak{\vec{A}}$
\item $\lambda^2-7\lambda+12 = 0$ where $\lambda$ is an eigenvalue of $\vec{A}$
\end{enumerate}
%
\item Let $\vec{A}$ be a $6 \times 6$ matrix over $\mathbb{R}$ with characteristic polynomial
%
\begin{align}
\brak{x-3}^2
\brak{x-2}^4
\end{align}
%
and minimal polynomial
%
\begin{align}
\brak{x-3}
\brak{x-2}^2
\end{align}
%
Then the Jordan canonical form of $\vec{A}$ can be
\begin{enumerate}
\item  
$
\myvec
{
3 & 0 & 0 & 0 & 0 & 0 
\\
0 & 3 & 0 & 0 & 0 & 0 
\\
0 & 0 & 2 & 1 & 0 & 0 
\\
0 & 0 & 0 & 2 & 1 & 0 
\\
0 & 0 & 0 & 0 & 2 & 1 
\\
0 & 0 & 0 & 0 & 0 & 2
}
$ 
\item  
$
\myvec
{
3 & 0 & 0 & 0 & 0 & 0 
\\
0 & 3 & 0 & 0 & 0 & 0 
\\
0 & 0 & 2 & 1 & 0 & 0 
\\
0 & 0 & 0 & 2 & 0 & 0 
\\
0 & 0 & 0 & 0 & 2 & 0 
\\
0 & 0 & 0 & 0 & 0 & 2
}
$ 
\item  
$
\myvec
{
3 & 0 & 0 & 0 & 0 & 0 
\\
0 & 3 & 0 & 0 & 0 & 0 
\\
0 & 0 & 2 & 1 & 0 & 0 
\\
0 & 0 & 0 & 2 & 0 & 0 
\\
0 & 0 & 0 & 0 & 2 & 1 
\\
0 & 0 & 0 & 0 & 0 & 2
}
$ 
\item  
$
\myvec
{
3 & 1 & 0 & 0 & 0 & 0 
\\
0 & 3 & 0 & 0 & 0 & 0 
\\
0 & 0 & 2 & 1 & 0 & 0 
\\
0 & 0 & 0 & 2 & 0 & 0 
\\
0 & 0 & 0 & 0 & 2 & 1 
\\
0 & 0 & 0 & 0 & 0 & 2
}
$ 
\end{enumerate}
\item Let $V$ be an inner product space and $S$ be a
subset of $V$.  Let $\bar{S}$ denote the closure of $S$
in $V$ with respect to the topology induced by the metric
given by the inner product.  Which of the following statements is true?
\begin{enumerate}
\item $S = \brak{S^{\perp}}^{\perp}$
\item $\bar{S} = \brak{S^{\perp}}^{\perp}$
\item $\overline{span\brak{S}} = \brak{S^{\perp}}^{\perp}$
\item $S^{\perp} = \brak{\brak{S^{\perp}}^{\perp}}^{\perp}$
\end{enumerate}
%
\item Let
\begin{align}
\vec{A} =
\myvec
{
1 & 2 & 0 \\
0 & 0 & -2 \\
0 & 0 & 1
}
\end{align}
and 
\begin{align}
Q\brak{\vec{x}} = \vec{x}^T\vec{A}\vec{x}
\end{align}
%
Which of the following statements is true?
\begin{enumerate}
\item The matrix of second order partial derivatives of the quadratic form $Q$ is $2\vec{A}$
\item The rank of the quadratic form $Q$ is $2$
\item The signature  of the quadratic form $Q$ is $++0$
\item The quadratic form $Q$ take the value 0 for some non-zero vector $\vec{x}$
\end{enumerate}
\item Assume that a non-singular matrix
\begin{align}
\vec{A} = \vec{L}+\vec{D}+\vec{U}
\end{align}
%
where $\vec{L}$ and $\vec{U}$ are lower and upper triangular matrices respectively with all
diagonal entries are zero, and $\vec{D}$ si a diagonal matrix.  Let $\vec{x}^{*}$ be the solution of
$\vec{A}\vec{x} = \vec{b}$.  Then the Gauss-Seidel iteration method 
\begin{align}
\vec{x}_{k+1} = \vec{H}\vec{x}_{k}+\vec{c}, k = 0,1,2,\dots
\end{align}
%
with $\norm{\vec{H}} < 1$ converges to $\vec{x}^{*}$ provided $\vec{H}$ is equal to
\begin{enumerate}
\item $-\vec{D}^{-1}\brak{\vec{L}+\vec{U}}$
\item $-\brak{\vec{D}+\vec{L}}^{-1}\vec{U}$
\item $-\vec{D}\brak{\vec{L}+\vec{U}}^{-1}$
\item $-\brak{\vec{L}-\vec{D}}^{-1}\vec{U}$
\end{enumerate}
\item Consider a Markov Chain with state space $S = \cbrak{1,2, 3}$ and transition matrix
\begin{align}
P = 
\begin{blockarray}{c@{\hspace{1pt}}rrr@{\hspace{3pt}}}
            & 1   & 2 & 3 \\
        \begin{block}{r@{\hspace{3pt}}@{\hspace{1pt}}
    (@{\hspace{1pt}}rrr@{\hspace{1pt}}@{\hspace{1pt}})}
        1 &  0 & \frac{1}{2} & \frac{1}{2}   \\ [2mm]
        2 & \frac{1}{2}  & 0 & \frac{1}{2}\\ [2mm]
        3 & \frac{1}{2}  &  \frac{1}{2} & 0  \\ [2mm]
%
        \end{block}
    \end{blockarray}
\end{align}
%
Let $\vec{\pi}$ be a stationary distribution of the Markov chain and $d(1)$ denote the
period of state 1.  Which of the following statements are correct?
\begin{enumerate}
\item $d(1) = 1$
\item $d(1) = 2$
\item $\pi_1 = \frac{1}{2}$
\item $\pi_1 = \frac{1}{3}$
\end{enumerate}
\end{enumerate}
