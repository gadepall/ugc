See Table \ref{eq:solutions/2014/dec/27/Table1:}

\onecolumn
\begin{longtable}{|p{5cm}|p{13cm}|}
\hline
\textbf{Statement} &\textbf{Solution}\\
\hline
Given Condition&
\parbox{12cm}{\begin{align}
BA+ B^2= I-BA^2 \label{eq:solutions/2014/dec/27/eq1}
\end{align}}\\
\hline
Solution by Theory&
\parbox{12cm}{We will first provide theoretical proof }\\
\hline
Theory&
\parbox{12cm}{\text{As per definition of invertible matrix,A matrix 'B' in our case is defined}\\
 \text{as invertible if there exists left and right inverse of B such that BC=CB=I } \\
\text{In that case C is called the two sided inverse of B and B is said to be }\\
\text{invertible.}\\
\text{Now refer\eqref{eq:solutions/2014/dec/27/eq1} we get}
\begin{align}
 BA+ B^2= I-BA^2\\
 \implies BA+B^2+BA^2=I\\
 \implies I=B\brak{A+B+A^2}\label{eq:solutions/2014/dec/27/eqB}\\
 \end{align}
 \text{Let C= \brak{A+B+A^2} rewrite \eqref{eq:solutions/2014/dec/27/eqB} as}
  \begin{align}
  I=BC\label{eq:solutions/2014/dec/27/eqC}
  \end{align}
  \text{Also}\\
 \begin{align}
 I=\brak{A+B+A^2}B\label{eq:solutions/2014/dec/27/eqD}
 \end{align}
 \text{Let D= \brak{A+B+A^2} rewrite \eqref{eq:solutions/2014/dec/27/eqD} as}\\
 \begin{align}
 I=DB\label{eq:solutions/2014/dec/27/eqE}
 \end{align}
 \text{Now we can write}
 \begin{align}
    D= DI
   \end{align}
  \text{ Ref \eqref{eq:solutions/2014/dec/27/eqC}}
  \begin{align}
     =D\brak{BC}\\
    =\brak{DB}C\\
  \end{align}
  \text{ Ref \eqref{eq:solutions/2014/dec/27/eqE}}
  \begin{align}
      =IC\\
      =C\\
      \implies D=C
  \end{align}
  \text{Hence by definition stated above we imply that }\\
  \text{ Left inverse=Right inverse.}\\
  \text{So by looking at \eqref{eq:solutions/2014/dec/27/eqB},we imply that B has a left and right inverse}\\
 \begin{align}
 \implies I=BB^{-1}\\
 \implies \text{B is invertible}
 \end{align}
 \text{$\therefore$ B is non singular.\\Hence Option 2 is correct}
}
\\
\hline
Solution by examples&
\parbox{12cm}{We will check each respective options through examples}\\
\hline
Option 3&
\parbox{12cm}{\text{Let us take}
\begin{align}
A=\myvec{1&0\\0&1}\\
B=\myvec{-1&0\\0&-1}
\end{align}
\text{ Take L.H.S of \eqref{eq:solutions/2014/dec/27/eq1}}
\begin{align}
\myvec{-1&0\\0&-1}\myvec{1&0\\0&1}+\myvec{-1&0\\0&-1}\myvec{-1&0\\0&-1}\\
=\myvec{0&0\\0&0}\label{eq:solutions/2014/dec/27/eq2}
\end{align}
\text{ Take R.H.S of \eqref{eq:solutions/2014/dec/27/eq1}}
\begin{align}
 \myvec{1&0\\0&1}-\myvec{-1&0\\0&-1}\myvec{1&0\\0&1}\myvec{1&0\\0&1}\\
 =\myvec{0&0\\0&0}\label{eq:solutions/2014/dec/27/eq3}
\end{align}
\text{ Our assumption satisfies \eqref{eq:solutions/2014/dec/27/eq1}.}
\text{Now}
\begin{align}
    A+B=  \myvec{1&0\\0&1} + \myvec{-1&0\\0&-1}\\
    = \myvec{0&0\\0&0}
\end{align}
\text{$\therefore \mydet{A+B}=0$  the respective option is Singular. Hence Option 3 is incorrect}
}
\\
\hline
Option 1&
\parbox{12cm}{\text{ Now let us take} 
\begin{align}
 A=\myvec{0&0\\0&0}
 B=\myvec{-1&0\\0&-1} \label{eq:solutions/2014/dec/27/eq4}
\end{align}
\text{Substituting\eqref{eq:solutions/2014/dec/27/eq4} in \eqref{eq:solutions/2014/dec/27/eq1}}\\
\text{Take L.H.S of \eqref{eq:solutions/2014/dec/27/eq1}}
\begin{align}
\myvec{-1&0\\0&-1}\myvec{0&0\\0&0}+\myvec{-1&0\\0&-1}\myvec{-1&0\\0&-1}\\
 = \myvec{1&0\\0&1}
\end{align}
\text{Take R.H.S of \eqref{eq:solutions/2014/dec/27/eq1}}
\begin{align}
 \myvec{1&0\\0&1}-\myvec{-1&0\\0&-1}\myvec{0&0\\0&0}\myvec{0&0\\0&0}\\
 =\myvec{1&0\\0&1}
%\label{eq:solutions/2014/dec/27/eq3}
\end{align}
\text{Our assumption satisfies \eqref{eq:solutions/2014/dec/27/eq1}}\\
\text{But $\mydet{A}=0$}\\
\text{$\therefore$  the respective option is Singular. Hence Option 1 is incorrect}
}
\\
\hline
 Option 4&
\parbox{12cm}{\text{Similarly} 
\begin{align}
AB= \myvec{0&0\\0&0}\myvec{-1&0\\0&-1}\\
 = \myvec{0&0\\0&0}
\end{align}
\text{Here also $\mydet{AB}=0$}\\
\text{$\therefore$  the AB option is also Singular. Hence Option 4 is incorrect also}
}
\\
\hline
Correct Answer&
\parbox{12cm}{\text{So we conclude that Option 2 is correct by eliminating other options}}\\
\hline
\caption{Solution}
\label{eq:solutions/2014/dec/27/Table1:}
\end{longtable}
\twocolumn
