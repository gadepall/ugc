\renewcommand{\theequation}{\theenumi}
\renewcommand{\thefigure}{\theenumi}
\renewcommand{\thetable}{\theenumi}
\begin{enumerate}[label=\thesection.\arabic*.,ref=\thesection.\theenumi]
\numberwithin{equation}{enumi}
\numberwithin{figure}{enumi}
\numberwithin{table}{enumi}

\item The matrix
\begin{align}
\vec{A} = \myvec{3 && -1 && 0 \\ -1 && 2 && -1\\ 0 && -1 && 3}
\end{align}
is
\begin{enumerate}
\item positive definite.
\item non-negative definite but not positive definite.
\item negative definite. 
\item neither negative definite nor positive definite. .
\end{enumerate}
%
\solution
See Tables \ref{eq:solutions/2018/dec/106/table0} and \ref{eq:solutions/2018/dec/106/table1}


\onecolumn
	\begin{longtable}{|l|l|}
		\hline
		\multirow{3}{*}{Irreducible Markov Chain} 
		& \\
		& A Markov chain is $\textbf{irreducible}$ if all the states communicate with each other,\\
		& i.e., if there is only one communication class.\\
		&\\
		\hline
		\multirow{3}{*}{Aperiodic Markov Chain} & \\
		& If there is a self-transition in the chain ($p^{ii}>0$ for some i), then the chain is\\
		& called as $\textbf{aperiodic}$\\
		& \\
		\hline
		\multirow{3}{*}{Stationary Distribution} & \\
		& A stationary distribution of a Markov chain is a probability distribution that\\
		& remains unchanged in the Markov chain as time progresses. Typically, it is\\
		& represented as a row vector $\Vec{\pi}$ whose entries are probabilities summing to 1,\\ 
		& and given transition matrix $\textbf{P}$, it satisfies\\
		& \\
		&  \qquad \qquad  \qquad$\Vec{\pi} = \Vec{\pi} \textbf{P}$\\
		& \\
		\hline
\caption{}
\label{eq:solutions/2018/dec/106/table0}
	\end{longtable}
	\begin{longtable}{|l|l|}
		\hline
		\multirow{3}{*}{Drawing Transition diagram} 
		& \\
		& 
		
		$\begin{tikzpicture}[shorten >=1pt,node distance=2cm, scale =3, auto]
			\tikzstyle{every state}=[fill={rgb:black,1;white,10}]
			
			\node[state]   (q_1)                          {$1$};
			\node[state]   (q_2)  [right of=q_1]          {$2$};
			\node[state]   (q_3)  [below right of=q_1]          {$3$};
			
			\path[->]
			(q_1) edge [loop above] node {$\frac{1}{2}$}    (   )
			edge [bend left]  node {$\frac{1}{2}$}    (q_2)
			(q_2) edge [bend left]  node {$\frac{1}{2}$}    (q_3)
			edge [loop above] node {$\frac{1}{2}$}    ()
			(q_3) edge [bend left]  node {$\frac{1}{3}$}    (q_2)
			edge [bend left]  node {$\frac{1}{3}$}    (q_1)
			edge [loop below] node {$\frac{1}{3}$}    ();
		\end{tikzpicture}$
		
		\\  
		&\\
		&\\
		\hline
		\multirow{3}{*}{Checking whether the  } & \\
		& Here,\\chain is Irreducible
		& All the states are accessible to one another. \\and Aperiodic
		& $\implies$ They are in the same communication class. So, it is Irreducible.\\
		& \\
		& There exists the non- zero self-transition, which means that the chain \\
		& is Aperiodic.\\
		&\\ 
		& We know that if the Markov Chain is irreducible and aperiodic then \\
		& \qquad \qquad \qquad $\Vec{\pi}_{j} = \lim_{n \to \infty}P\{X_{n} = j\}$, $j = 1,...,N$ \\
		& These are the stationary probabilities. \\
		&\\
		\hline
		\multirow{3}{*}{Finding the Stationary} & \\
		& Stationary Probability can be represented as\\Probability Distributions
		& \qquad \qquad \qquad $\Vec{\pi} = \Vec{\pi} \vec{P}$\\
		& \\
		& \qquad $\implies$ $\myvec{v_{1}&&v_{2}&&v_{3}} = \myvec{v_{1}&&v_{2}&&v_{3}}\vec{P}$ \\
		& \\
		& Equating the above equation we get \\
		& \\
		& \qquad \qquad \qquad $\frac{1}{2}v_{1}-\frac{1}{3}v_{3} = 0$ $\label{eq:solutions/2018/dec/106/eq}$\\
		& \\
		& \qquad \qquad \qquad $\frac{1}{2}v_{1}-\frac{1}{2}v_{2} + \frac{1}{3}v_{3} = 0$\\
		& \\
		& \qquad \qquad \qquad $\frac{1}{2}v_{2}-\frac{2}{3}v_{3} = 0$\\
		& \\\
		& We see that summation of second and the third equation gives us the \\
		& first equation only. \\
		& And we know that the probability distribution will sum up to 1. \\
		& \\
		& \qquad \qquad \qquad $v_{1}+v_{2}+v_{3} = 1$ \\
		& \\
		& Therefore, we get the equation form as \\
		& \\
		& \qquad \qquad \qquad $\myvec{1&1&1\\\frac{1}{2}&0&\frac{-1}{3}\\\frac{1}{2}&\frac{-1}{2}&\frac{1}{3}}\myvec{v_{1}\\v_{2}\\v_{3}} = \myvec{1\\0\\0}$ \\
		& \\
		\hline
		\multirow{3}{*}{Solving the linear} & \\
		& The above linear equation can be solved using Gauss-Jordan method as\\equtions
		& \\
		& \qquad \qquad \qquad $\myvec{1&1&1&\vrule&1\\\frac{1}{2}&0&\frac{-1}{3}&\vrule&0\\\frac{1}{2}&\frac{-1}{2}&\frac{1}{3}&\vrule&0}$\\
		& \\
		& \qquad $\xleftrightarrow[]{R_2 \leftarrow R_2 - \frac{1}{2}R_1}$
		$\myvec{1&1&1&\vrule&1\\0&\frac{-1}{2}&\frac{-5}{6}&\vrule&\frac{-1}{2}\\\frac{1}{2}&\frac{-1}{2}&\frac{1}{3}&\vrule&0}$\\
		&\\
		& \qquad $\xleftrightarrow[]{R_3 \leftarrow R_3 - \frac{1}{2}R_1}$
		$\myvec{1&1&1&\vrule&1\\0&\frac{-1}{2}&\frac{-5}{6}&\vrule&\frac{-1}{2}\\0&-1&\frac{-1}{6}&\vrule&\frac{-1}{2}}$\\
		&\\
		& \qquad $\xleftrightarrow[]{R_2 \leftarrow \frac{-1}{2}R_2}$
		$\myvec{1&1&1&\vrule&1\\0&1&\frac{5}{3}&\vrule&1\\0&-1&\frac{-1}{6}&\vrule&\frac{-1}{2}}$\\
		&\\
		& \qquad $\xleftrightarrow[]{R_3 \leftarrow R_3 + R_2}$
		$\myvec{1&1&1&\vrule&1\\0&1&\frac{5}{3}&\vrule&1\\0&0&\frac{3}{2}&\vrule&\frac{1}{2}}$\\
		&\\
		& \qquad $\xleftrightarrow[]{R_3 \leftarrow \frac{3}{2}R_3}$
		$\myvec{1&1&1&\vrule&1\\0&1&\frac{5}{3}&\vrule&1\\0&0&1&\vrule&\frac{1}{3}}$\\
		&\\
		& \qquad $\xleftrightarrow[]{R_2 \leftarrow R_2 - \frac{5}{3}R_3}$
		$\myvec{1&1&1&\vrule&1\\0&1&0&\vrule&\frac{4}{9}\\0&0&1&\vrule&\frac{1}{3}}$\\
		&\\
		& \qquad $\xleftrightarrow[]{R_1 \leftarrow R_1 - R_3}$
		$\myvec{1&1&0&\vrule&\frac{2}{3}\\0&1&0&\vrule&\frac{4}{9}\\0&0&1&\vrule&\frac{1}{3}}$\\
		&\\
		& \qquad $\xleftrightarrow[]{R_1 \leftarrow R_1 - R_2}$
		$\myvec{1&0&0&\vrule&\frac{2}{9}\\0&1&0&\vrule&\frac{4}{9}\\0&0&1&\vrule&\frac{1}{3}}$\\
		&\\
		& $\therefore$, stationary probability distribution $\pi$ is given by \\
		& \qquad \qquad $\pi = \myvec{\frac{2}{9} & \frac{4}{9} & \frac{1}{3}}$ \\
		& \\
		\hline
		\multirow{3}{*}{Observations} & \\
		
		
		& Since the given transition probability matrix $\vec{P}$ is irreducible and aperiodic, \\
		& then $\lim_{n \to \infty} \vec{P}^{n}$ converges to a matrix with all rows identical and equal to $\vec{\pi}$. \\
		& \\
		& We were able to find $\vec{\pi}$ as $\myvec{\frac{2}{9} & \frac{4}{9} & \frac{1}{3}}$ \\
		& \\
		& $\lim_{n \to \infty} \vec{P}^{n} = \myvec{\frac{2}{9}&\frac{4}{9}&\frac{1}{3}\\\frac{2}{9}&\frac{4}{9}&\frac{1}{3}\\\frac{2}{9}&\frac{4}{9}&\frac{1}{3}}$\\
		& \\
		& From the above matrix, we get \\
		& \\
		& $\lim_{n \to \infty} \vec{P}^{n}_{11} = \frac{2}{9}$ \\
		&\\
		& $\lim_{n \to \infty} \vec{P}^{n}_{21} = \frac{2}{9}$ \\
		&\\
		& $\lim_{n \to \infty} \vec{P}^{n}_{32} = \frac{4}{9}$ \\
		&\\
		& $\lim_{n \to \infty} \vec{P}^{n}_{13} = \frac{1}{3}$ \\
		&\\
		\hline
		\multirow{3}{*}{Conclusion} & \\
		& From our observation we see that \\
		&\\
		& Options 1) and 4) are True.\\
		& \\
		\hline
\caption{}
\label{eq:solutions/2018/dec/106/table1}
	\end{longtable}
\twocolumn

\item Which of the following sets of functions from $\Re$ to $\Re$ is a vector space over $\Re$?
\begin{align}
S_1 = \{f|\lim_{x\to3} f(x) = 0\}\\
S_2 = \{g|\lim_{x\to3} g(x) = 1\}\\
S_3 = \{h|\lim_{x\to3} h(x)~exists\}
\end{align}
is
\begin{enumerate}
\item Only $S_1$ 
\item Only $S_2$
\item $S_1$ and $S_3$ but not $S_2$ 
\item All the three are vector spaces
\end{enumerate}
%
\solution
See Tables \ref{eq:solutions/2018/dec/106/table0} and \ref{eq:solutions/2018/dec/106/table1}


\onecolumn
	\begin{longtable}{|l|l|}
		\hline
		\multirow{3}{*}{Irreducible Markov Chain} 
		& \\
		& A Markov chain is $\textbf{irreducible}$ if all the states communicate with each other,\\
		& i.e., if there is only one communication class.\\
		&\\
		\hline
		\multirow{3}{*}{Aperiodic Markov Chain} & \\
		& If there is a self-transition in the chain ($p^{ii}>0$ for some i), then the chain is\\
		& called as $\textbf{aperiodic}$\\
		& \\
		\hline
		\multirow{3}{*}{Stationary Distribution} & \\
		& A stationary distribution of a Markov chain is a probability distribution that\\
		& remains unchanged in the Markov chain as time progresses. Typically, it is\\
		& represented as a row vector $\Vec{\pi}$ whose entries are probabilities summing to 1,\\ 
		& and given transition matrix $\textbf{P}$, it satisfies\\
		& \\
		&  \qquad \qquad  \qquad$\Vec{\pi} = \Vec{\pi} \textbf{P}$\\
		& \\
		\hline
\caption{}
\label{eq:solutions/2018/dec/106/table0}
	\end{longtable}
	\begin{longtable}{|l|l|}
		\hline
		\multirow{3}{*}{Drawing Transition diagram} 
		& \\
		& 
		
		$\begin{tikzpicture}[shorten >=1pt,node distance=2cm, scale =3, auto]
			\tikzstyle{every state}=[fill={rgb:black,1;white,10}]
			
			\node[state]   (q_1)                          {$1$};
			\node[state]   (q_2)  [right of=q_1]          {$2$};
			\node[state]   (q_3)  [below right of=q_1]          {$3$};
			
			\path[->]
			(q_1) edge [loop above] node {$\frac{1}{2}$}    (   )
			edge [bend left]  node {$\frac{1}{2}$}    (q_2)
			(q_2) edge [bend left]  node {$\frac{1}{2}$}    (q_3)
			edge [loop above] node {$\frac{1}{2}$}    ()
			(q_3) edge [bend left]  node {$\frac{1}{3}$}    (q_2)
			edge [bend left]  node {$\frac{1}{3}$}    (q_1)
			edge [loop below] node {$\frac{1}{3}$}    ();
		\end{tikzpicture}$
		
		\\  
		&\\
		&\\
		\hline
		\multirow{3}{*}{Checking whether the  } & \\
		& Here,\\chain is Irreducible
		& All the states are accessible to one another. \\and Aperiodic
		& $\implies$ They are in the same communication class. So, it is Irreducible.\\
		& \\
		& There exists the non- zero self-transition, which means that the chain \\
		& is Aperiodic.\\
		&\\ 
		& We know that if the Markov Chain is irreducible and aperiodic then \\
		& \qquad \qquad \qquad $\Vec{\pi}_{j} = \lim_{n \to \infty}P\{X_{n} = j\}$, $j = 1,...,N$ \\
		& These are the stationary probabilities. \\
		&\\
		\hline
		\multirow{3}{*}{Finding the Stationary} & \\
		& Stationary Probability can be represented as\\Probability Distributions
		& \qquad \qquad \qquad $\Vec{\pi} = \Vec{\pi} \vec{P}$\\
		& \\
		& \qquad $\implies$ $\myvec{v_{1}&&v_{2}&&v_{3}} = \myvec{v_{1}&&v_{2}&&v_{3}}\vec{P}$ \\
		& \\
		& Equating the above equation we get \\
		& \\
		& \qquad \qquad \qquad $\frac{1}{2}v_{1}-\frac{1}{3}v_{3} = 0$ $\label{eq:solutions/2018/dec/106/eq}$\\
		& \\
		& \qquad \qquad \qquad $\frac{1}{2}v_{1}-\frac{1}{2}v_{2} + \frac{1}{3}v_{3} = 0$\\
		& \\
		& \qquad \qquad \qquad $\frac{1}{2}v_{2}-\frac{2}{3}v_{3} = 0$\\
		& \\\
		& We see that summation of second and the third equation gives us the \\
		& first equation only. \\
		& And we know that the probability distribution will sum up to 1. \\
		& \\
		& \qquad \qquad \qquad $v_{1}+v_{2}+v_{3} = 1$ \\
		& \\
		& Therefore, we get the equation form as \\
		& \\
		& \qquad \qquad \qquad $\myvec{1&1&1\\\frac{1}{2}&0&\frac{-1}{3}\\\frac{1}{2}&\frac{-1}{2}&\frac{1}{3}}\myvec{v_{1}\\v_{2}\\v_{3}} = \myvec{1\\0\\0}$ \\
		& \\
		\hline
		\multirow{3}{*}{Solving the linear} & \\
		& The above linear equation can be solved using Gauss-Jordan method as\\equtions
		& \\
		& \qquad \qquad \qquad $\myvec{1&1&1&\vrule&1\\\frac{1}{2}&0&\frac{-1}{3}&\vrule&0\\\frac{1}{2}&\frac{-1}{2}&\frac{1}{3}&\vrule&0}$\\
		& \\
		& \qquad $\xleftrightarrow[]{R_2 \leftarrow R_2 - \frac{1}{2}R_1}$
		$\myvec{1&1&1&\vrule&1\\0&\frac{-1}{2}&\frac{-5}{6}&\vrule&\frac{-1}{2}\\\frac{1}{2}&\frac{-1}{2}&\frac{1}{3}&\vrule&0}$\\
		&\\
		& \qquad $\xleftrightarrow[]{R_3 \leftarrow R_3 - \frac{1}{2}R_1}$
		$\myvec{1&1&1&\vrule&1\\0&\frac{-1}{2}&\frac{-5}{6}&\vrule&\frac{-1}{2}\\0&-1&\frac{-1}{6}&\vrule&\frac{-1}{2}}$\\
		&\\
		& \qquad $\xleftrightarrow[]{R_2 \leftarrow \frac{-1}{2}R_2}$
		$\myvec{1&1&1&\vrule&1\\0&1&\frac{5}{3}&\vrule&1\\0&-1&\frac{-1}{6}&\vrule&\frac{-1}{2}}$\\
		&\\
		& \qquad $\xleftrightarrow[]{R_3 \leftarrow R_3 + R_2}$
		$\myvec{1&1&1&\vrule&1\\0&1&\frac{5}{3}&\vrule&1\\0&0&\frac{3}{2}&\vrule&\frac{1}{2}}$\\
		&\\
		& \qquad $\xleftrightarrow[]{R_3 \leftarrow \frac{3}{2}R_3}$
		$\myvec{1&1&1&\vrule&1\\0&1&\frac{5}{3}&\vrule&1\\0&0&1&\vrule&\frac{1}{3}}$\\
		&\\
		& \qquad $\xleftrightarrow[]{R_2 \leftarrow R_2 - \frac{5}{3}R_3}$
		$\myvec{1&1&1&\vrule&1\\0&1&0&\vrule&\frac{4}{9}\\0&0&1&\vrule&\frac{1}{3}}$\\
		&\\
		& \qquad $\xleftrightarrow[]{R_1 \leftarrow R_1 - R_3}$
		$\myvec{1&1&0&\vrule&\frac{2}{3}\\0&1&0&\vrule&\frac{4}{9}\\0&0&1&\vrule&\frac{1}{3}}$\\
		&\\
		& \qquad $\xleftrightarrow[]{R_1 \leftarrow R_1 - R_2}$
		$\myvec{1&0&0&\vrule&\frac{2}{9}\\0&1&0&\vrule&\frac{4}{9}\\0&0&1&\vrule&\frac{1}{3}}$\\
		&\\
		& $\therefore$, stationary probability distribution $\pi$ is given by \\
		& \qquad \qquad $\pi = \myvec{\frac{2}{9} & \frac{4}{9} & \frac{1}{3}}$ \\
		& \\
		\hline
		\multirow{3}{*}{Observations} & \\
		
		
		& Since the given transition probability matrix $\vec{P}$ is irreducible and aperiodic, \\
		& then $\lim_{n \to \infty} \vec{P}^{n}$ converges to a matrix with all rows identical and equal to $\vec{\pi}$. \\
		& \\
		& We were able to find $\vec{\pi}$ as $\myvec{\frac{2}{9} & \frac{4}{9} & \frac{1}{3}}$ \\
		& \\
		& $\lim_{n \to \infty} \vec{P}^{n} = \myvec{\frac{2}{9}&\frac{4}{9}&\frac{1}{3}\\\frac{2}{9}&\frac{4}{9}&\frac{1}{3}\\\frac{2}{9}&\frac{4}{9}&\frac{1}{3}}$\\
		& \\
		& From the above matrix, we get \\
		& \\
		& $\lim_{n \to \infty} \vec{P}^{n}_{11} = \frac{2}{9}$ \\
		&\\
		& $\lim_{n \to \infty} \vec{P}^{n}_{21} = \frac{2}{9}$ \\
		&\\
		& $\lim_{n \to \infty} \vec{P}^{n}_{32} = \frac{4}{9}$ \\
		&\\
		& $\lim_{n \to \infty} \vec{P}^{n}_{13} = \frac{1}{3}$ \\
		&\\
		\hline
		\multirow{3}{*}{Conclusion} & \\
		& From our observation we see that \\
		&\\
		& Options 1) and 4) are True.\\
		& \\
		\hline
\caption{}
\label{eq:solutions/2018/dec/106/table1}
	\end{longtable}
\twocolumn

\item Let $\vec{A}$ be an n $\times$ m matrix with each entry equal to +1,-1 or 0 such that every column has exactly one +1 and exactly one -1. We can conclude that\\
\begin{align}
    &\mbox{1. Rank } \vec{A}\leq n-1\\
    &\mbox{2. Rank } \vec{A}=m\\    
    &\mbox{3. }n\leq m\\
    &\mbox{4. }n-1\leq m
\end{align}
%
%
\solution
See Tables \ref{eq:solutions/2018/dec/106/table0} and \ref{eq:solutions/2018/dec/106/table1}


\onecolumn
	\begin{longtable}{|l|l|}
		\hline
		\multirow{3}{*}{Irreducible Markov Chain} 
		& \\
		& A Markov chain is $\textbf{irreducible}$ if all the states communicate with each other,\\
		& i.e., if there is only one communication class.\\
		&\\
		\hline
		\multirow{3}{*}{Aperiodic Markov Chain} & \\
		& If there is a self-transition in the chain ($p^{ii}>0$ for some i), then the chain is\\
		& called as $\textbf{aperiodic}$\\
		& \\
		\hline
		\multirow{3}{*}{Stationary Distribution} & \\
		& A stationary distribution of a Markov chain is a probability distribution that\\
		& remains unchanged in the Markov chain as time progresses. Typically, it is\\
		& represented as a row vector $\Vec{\pi}$ whose entries are probabilities summing to 1,\\ 
		& and given transition matrix $\textbf{P}$, it satisfies\\
		& \\
		&  \qquad \qquad  \qquad$\Vec{\pi} = \Vec{\pi} \textbf{P}$\\
		& \\
		\hline
\caption{}
\label{eq:solutions/2018/dec/106/table0}
	\end{longtable}
	\begin{longtable}{|l|l|}
		\hline
		\multirow{3}{*}{Drawing Transition diagram} 
		& \\
		& 
		
		$\begin{tikzpicture}[shorten >=1pt,node distance=2cm, scale =3, auto]
			\tikzstyle{every state}=[fill={rgb:black,1;white,10}]
			
			\node[state]   (q_1)                          {$1$};
			\node[state]   (q_2)  [right of=q_1]          {$2$};
			\node[state]   (q_3)  [below right of=q_1]          {$3$};
			
			\path[->]
			(q_1) edge [loop above] node {$\frac{1}{2}$}    (   )
			edge [bend left]  node {$\frac{1}{2}$}    (q_2)
			(q_2) edge [bend left]  node {$\frac{1}{2}$}    (q_3)
			edge [loop above] node {$\frac{1}{2}$}    ()
			(q_3) edge [bend left]  node {$\frac{1}{3}$}    (q_2)
			edge [bend left]  node {$\frac{1}{3}$}    (q_1)
			edge [loop below] node {$\frac{1}{3}$}    ();
		\end{tikzpicture}$
		
		\\  
		&\\
		&\\
		\hline
		\multirow{3}{*}{Checking whether the  } & \\
		& Here,\\chain is Irreducible
		& All the states are accessible to one another. \\and Aperiodic
		& $\implies$ They are in the same communication class. So, it is Irreducible.\\
		& \\
		& There exists the non- zero self-transition, which means that the chain \\
		& is Aperiodic.\\
		&\\ 
		& We know that if the Markov Chain is irreducible and aperiodic then \\
		& \qquad \qquad \qquad $\Vec{\pi}_{j} = \lim_{n \to \infty}P\{X_{n} = j\}$, $j = 1,...,N$ \\
		& These are the stationary probabilities. \\
		&\\
		\hline
		\multirow{3}{*}{Finding the Stationary} & \\
		& Stationary Probability can be represented as\\Probability Distributions
		& \qquad \qquad \qquad $\Vec{\pi} = \Vec{\pi} \vec{P}$\\
		& \\
		& \qquad $\implies$ $\myvec{v_{1}&&v_{2}&&v_{3}} = \myvec{v_{1}&&v_{2}&&v_{3}}\vec{P}$ \\
		& \\
		& Equating the above equation we get \\
		& \\
		& \qquad \qquad \qquad $\frac{1}{2}v_{1}-\frac{1}{3}v_{3} = 0$ $\label{eq:solutions/2018/dec/106/eq}$\\
		& \\
		& \qquad \qquad \qquad $\frac{1}{2}v_{1}-\frac{1}{2}v_{2} + \frac{1}{3}v_{3} = 0$\\
		& \\
		& \qquad \qquad \qquad $\frac{1}{2}v_{2}-\frac{2}{3}v_{3} = 0$\\
		& \\\
		& We see that summation of second and the third equation gives us the \\
		& first equation only. \\
		& And we know that the probability distribution will sum up to 1. \\
		& \\
		& \qquad \qquad \qquad $v_{1}+v_{2}+v_{3} = 1$ \\
		& \\
		& Therefore, we get the equation form as \\
		& \\
		& \qquad \qquad \qquad $\myvec{1&1&1\\\frac{1}{2}&0&\frac{-1}{3}\\\frac{1}{2}&\frac{-1}{2}&\frac{1}{3}}\myvec{v_{1}\\v_{2}\\v_{3}} = \myvec{1\\0\\0}$ \\
		& \\
		\hline
		\multirow{3}{*}{Solving the linear} & \\
		& The above linear equation can be solved using Gauss-Jordan method as\\equtions
		& \\
		& \qquad \qquad \qquad $\myvec{1&1&1&\vrule&1\\\frac{1}{2}&0&\frac{-1}{3}&\vrule&0\\\frac{1}{2}&\frac{-1}{2}&\frac{1}{3}&\vrule&0}$\\
		& \\
		& \qquad $\xleftrightarrow[]{R_2 \leftarrow R_2 - \frac{1}{2}R_1}$
		$\myvec{1&1&1&\vrule&1\\0&\frac{-1}{2}&\frac{-5}{6}&\vrule&\frac{-1}{2}\\\frac{1}{2}&\frac{-1}{2}&\frac{1}{3}&\vrule&0}$\\
		&\\
		& \qquad $\xleftrightarrow[]{R_3 \leftarrow R_3 - \frac{1}{2}R_1}$
		$\myvec{1&1&1&\vrule&1\\0&\frac{-1}{2}&\frac{-5}{6}&\vrule&\frac{-1}{2}\\0&-1&\frac{-1}{6}&\vrule&\frac{-1}{2}}$\\
		&\\
		& \qquad $\xleftrightarrow[]{R_2 \leftarrow \frac{-1}{2}R_2}$
		$\myvec{1&1&1&\vrule&1\\0&1&\frac{5}{3}&\vrule&1\\0&-1&\frac{-1}{6}&\vrule&\frac{-1}{2}}$\\
		&\\
		& \qquad $\xleftrightarrow[]{R_3 \leftarrow R_3 + R_2}$
		$\myvec{1&1&1&\vrule&1\\0&1&\frac{5}{3}&\vrule&1\\0&0&\frac{3}{2}&\vrule&\frac{1}{2}}$\\
		&\\
		& \qquad $\xleftrightarrow[]{R_3 \leftarrow \frac{3}{2}R_3}$
		$\myvec{1&1&1&\vrule&1\\0&1&\frac{5}{3}&\vrule&1\\0&0&1&\vrule&\frac{1}{3}}$\\
		&\\
		& \qquad $\xleftrightarrow[]{R_2 \leftarrow R_2 - \frac{5}{3}R_3}$
		$\myvec{1&1&1&\vrule&1\\0&1&0&\vrule&\frac{4}{9}\\0&0&1&\vrule&\frac{1}{3}}$\\
		&\\
		& \qquad $\xleftrightarrow[]{R_1 \leftarrow R_1 - R_3}$
		$\myvec{1&1&0&\vrule&\frac{2}{3}\\0&1&0&\vrule&\frac{4}{9}\\0&0&1&\vrule&\frac{1}{3}}$\\
		&\\
		& \qquad $\xleftrightarrow[]{R_1 \leftarrow R_1 - R_2}$
		$\myvec{1&0&0&\vrule&\frac{2}{9}\\0&1&0&\vrule&\frac{4}{9}\\0&0&1&\vrule&\frac{1}{3}}$\\
		&\\
		& $\therefore$, stationary probability distribution $\pi$ is given by \\
		& \qquad \qquad $\pi = \myvec{\frac{2}{9} & \frac{4}{9} & \frac{1}{3}}$ \\
		& \\
		\hline
		\multirow{3}{*}{Observations} & \\
		
		
		& Since the given transition probability matrix $\vec{P}$ is irreducible and aperiodic, \\
		& then $\lim_{n \to \infty} \vec{P}^{n}$ converges to a matrix with all rows identical and equal to $\vec{\pi}$. \\
		& \\
		& We were able to find $\vec{\pi}$ as $\myvec{\frac{2}{9} & \frac{4}{9} & \frac{1}{3}}$ \\
		& \\
		& $\lim_{n \to \infty} \vec{P}^{n} = \myvec{\frac{2}{9}&\frac{4}{9}&\frac{1}{3}\\\frac{2}{9}&\frac{4}{9}&\frac{1}{3}\\\frac{2}{9}&\frac{4}{9}&\frac{1}{3}}$\\
		& \\
		& From the above matrix, we get \\
		& \\
		& $\lim_{n \to \infty} \vec{P}^{n}_{11} = \frac{2}{9}$ \\
		&\\
		& $\lim_{n \to \infty} \vec{P}^{n}_{21} = \frac{2}{9}$ \\
		&\\
		& $\lim_{n \to \infty} \vec{P}^{n}_{32} = \frac{4}{9}$ \\
		&\\
		& $\lim_{n \to \infty} \vec{P}^{n}_{13} = \frac{1}{3}$ \\
		&\\
		\hline
		\multirow{3}{*}{Conclusion} & \\
		& From our observation we see that \\
		&\\
		& Options 1) and 4) are True.\\
		& \\
		\hline
\caption{}
\label{eq:solutions/2018/dec/106/table1}
	\end{longtable}
\twocolumn


\item %
Let $\vec{A}=\myvec{1 & 1\\1 & 0}$ and let $\alpha_n$ and $\beta_n$ denote the two eigenvalues of $\vec{A}^n$ such that $\abs{\alpha_n}\geq\abs{\beta_n}$.\\
Then
\begin{enumerate}
    \item $\alpha_n\rightarrow \infty$ as $n\rightarrow \infty$
    \item $\beta_n\rightarrow 0$ as $n\rightarrow \infty$
    \item $\beta_n$ is positive if n is even.
    \item $\beta_n$ is negative if n is odd.
\end{enumerate}
%
\solution
See Tables \ref{eq:solutions/2018/dec/106/table0} and \ref{eq:solutions/2018/dec/106/table1}


\onecolumn
	\begin{longtable}{|l|l|}
		\hline
		\multirow{3}{*}{Irreducible Markov Chain} 
		& \\
		& A Markov chain is $\textbf{irreducible}$ if all the states communicate with each other,\\
		& i.e., if there is only one communication class.\\
		&\\
		\hline
		\multirow{3}{*}{Aperiodic Markov Chain} & \\
		& If there is a self-transition in the chain ($p^{ii}>0$ for some i), then the chain is\\
		& called as $\textbf{aperiodic}$\\
		& \\
		\hline
		\multirow{3}{*}{Stationary Distribution} & \\
		& A stationary distribution of a Markov chain is a probability distribution that\\
		& remains unchanged in the Markov chain as time progresses. Typically, it is\\
		& represented as a row vector $\Vec{\pi}$ whose entries are probabilities summing to 1,\\ 
		& and given transition matrix $\textbf{P}$, it satisfies\\
		& \\
		&  \qquad \qquad  \qquad$\Vec{\pi} = \Vec{\pi} \textbf{P}$\\
		& \\
		\hline
\caption{}
\label{eq:solutions/2018/dec/106/table0}
	\end{longtable}
	\begin{longtable}{|l|l|}
		\hline
		\multirow{3}{*}{Drawing Transition diagram} 
		& \\
		& 
		
		$\begin{tikzpicture}[shorten >=1pt,node distance=2cm, scale =3, auto]
			\tikzstyle{every state}=[fill={rgb:black,1;white,10}]
			
			\node[state]   (q_1)                          {$1$};
			\node[state]   (q_2)  [right of=q_1]          {$2$};
			\node[state]   (q_3)  [below right of=q_1]          {$3$};
			
			\path[->]
			(q_1) edge [loop above] node {$\frac{1}{2}$}    (   )
			edge [bend left]  node {$\frac{1}{2}$}    (q_2)
			(q_2) edge [bend left]  node {$\frac{1}{2}$}    (q_3)
			edge [loop above] node {$\frac{1}{2}$}    ()
			(q_3) edge [bend left]  node {$\frac{1}{3}$}    (q_2)
			edge [bend left]  node {$\frac{1}{3}$}    (q_1)
			edge [loop below] node {$\frac{1}{3}$}    ();
		\end{tikzpicture}$
		
		\\  
		&\\
		&\\
		\hline
		\multirow{3}{*}{Checking whether the  } & \\
		& Here,\\chain is Irreducible
		& All the states are accessible to one another. \\and Aperiodic
		& $\implies$ They are in the same communication class. So, it is Irreducible.\\
		& \\
		& There exists the non- zero self-transition, which means that the chain \\
		& is Aperiodic.\\
		&\\ 
		& We know that if the Markov Chain is irreducible and aperiodic then \\
		& \qquad \qquad \qquad $\Vec{\pi}_{j} = \lim_{n \to \infty}P\{X_{n} = j\}$, $j = 1,...,N$ \\
		& These are the stationary probabilities. \\
		&\\
		\hline
		\multirow{3}{*}{Finding the Stationary} & \\
		& Stationary Probability can be represented as\\Probability Distributions
		& \qquad \qquad \qquad $\Vec{\pi} = \Vec{\pi} \vec{P}$\\
		& \\
		& \qquad $\implies$ $\myvec{v_{1}&&v_{2}&&v_{3}} = \myvec{v_{1}&&v_{2}&&v_{3}}\vec{P}$ \\
		& \\
		& Equating the above equation we get \\
		& \\
		& \qquad \qquad \qquad $\frac{1}{2}v_{1}-\frac{1}{3}v_{3} = 0$ $\label{eq:solutions/2018/dec/106/eq}$\\
		& \\
		& \qquad \qquad \qquad $\frac{1}{2}v_{1}-\frac{1}{2}v_{2} + \frac{1}{3}v_{3} = 0$\\
		& \\
		& \qquad \qquad \qquad $\frac{1}{2}v_{2}-\frac{2}{3}v_{3} = 0$\\
		& \\\
		& We see that summation of second and the third equation gives us the \\
		& first equation only. \\
		& And we know that the probability distribution will sum up to 1. \\
		& \\
		& \qquad \qquad \qquad $v_{1}+v_{2}+v_{3} = 1$ \\
		& \\
		& Therefore, we get the equation form as \\
		& \\
		& \qquad \qquad \qquad $\myvec{1&1&1\\\frac{1}{2}&0&\frac{-1}{3}\\\frac{1}{2}&\frac{-1}{2}&\frac{1}{3}}\myvec{v_{1}\\v_{2}\\v_{3}} = \myvec{1\\0\\0}$ \\
		& \\
		\hline
		\multirow{3}{*}{Solving the linear} & \\
		& The above linear equation can be solved using Gauss-Jordan method as\\equtions
		& \\
		& \qquad \qquad \qquad $\myvec{1&1&1&\vrule&1\\\frac{1}{2}&0&\frac{-1}{3}&\vrule&0\\\frac{1}{2}&\frac{-1}{2}&\frac{1}{3}&\vrule&0}$\\
		& \\
		& \qquad $\xleftrightarrow[]{R_2 \leftarrow R_2 - \frac{1}{2}R_1}$
		$\myvec{1&1&1&\vrule&1\\0&\frac{-1}{2}&\frac{-5}{6}&\vrule&\frac{-1}{2}\\\frac{1}{2}&\frac{-1}{2}&\frac{1}{3}&\vrule&0}$\\
		&\\
		& \qquad $\xleftrightarrow[]{R_3 \leftarrow R_3 - \frac{1}{2}R_1}$
		$\myvec{1&1&1&\vrule&1\\0&\frac{-1}{2}&\frac{-5}{6}&\vrule&\frac{-1}{2}\\0&-1&\frac{-1}{6}&\vrule&\frac{-1}{2}}$\\
		&\\
		& \qquad $\xleftrightarrow[]{R_2 \leftarrow \frac{-1}{2}R_2}$
		$\myvec{1&1&1&\vrule&1\\0&1&\frac{5}{3}&\vrule&1\\0&-1&\frac{-1}{6}&\vrule&\frac{-1}{2}}$\\
		&\\
		& \qquad $\xleftrightarrow[]{R_3 \leftarrow R_3 + R_2}$
		$\myvec{1&1&1&\vrule&1\\0&1&\frac{5}{3}&\vrule&1\\0&0&\frac{3}{2}&\vrule&\frac{1}{2}}$\\
		&\\
		& \qquad $\xleftrightarrow[]{R_3 \leftarrow \frac{3}{2}R_3}$
		$\myvec{1&1&1&\vrule&1\\0&1&\frac{5}{3}&\vrule&1\\0&0&1&\vrule&\frac{1}{3}}$\\
		&\\
		& \qquad $\xleftrightarrow[]{R_2 \leftarrow R_2 - \frac{5}{3}R_3}$
		$\myvec{1&1&1&\vrule&1\\0&1&0&\vrule&\frac{4}{9}\\0&0&1&\vrule&\frac{1}{3}}$\\
		&\\
		& \qquad $\xleftrightarrow[]{R_1 \leftarrow R_1 - R_3}$
		$\myvec{1&1&0&\vrule&\frac{2}{3}\\0&1&0&\vrule&\frac{4}{9}\\0&0&1&\vrule&\frac{1}{3}}$\\
		&\\
		& \qquad $\xleftrightarrow[]{R_1 \leftarrow R_1 - R_2}$
		$\myvec{1&0&0&\vrule&\frac{2}{9}\\0&1&0&\vrule&\frac{4}{9}\\0&0&1&\vrule&\frac{1}{3}}$\\
		&\\
		& $\therefore$, stationary probability distribution $\pi$ is given by \\
		& \qquad \qquad $\pi = \myvec{\frac{2}{9} & \frac{4}{9} & \frac{1}{3}}$ \\
		& \\
		\hline
		\multirow{3}{*}{Observations} & \\
		
		
		& Since the given transition probability matrix $\vec{P}$ is irreducible and aperiodic, \\
		& then $\lim_{n \to \infty} \vec{P}^{n}$ converges to a matrix with all rows identical and equal to $\vec{\pi}$. \\
		& \\
		& We were able to find $\vec{\pi}$ as $\myvec{\frac{2}{9} & \frac{4}{9} & \frac{1}{3}}$ \\
		& \\
		& $\lim_{n \to \infty} \vec{P}^{n} = \myvec{\frac{2}{9}&\frac{4}{9}&\frac{1}{3}\\\frac{2}{9}&\frac{4}{9}&\frac{1}{3}\\\frac{2}{9}&\frac{4}{9}&\frac{1}{3}}$\\
		& \\
		& From the above matrix, we get \\
		& \\
		& $\lim_{n \to \infty} \vec{P}^{n}_{11} = \frac{2}{9}$ \\
		&\\
		& $\lim_{n \to \infty} \vec{P}^{n}_{21} = \frac{2}{9}$ \\
		&\\
		& $\lim_{n \to \infty} \vec{P}^{n}_{32} = \frac{4}{9}$ \\
		&\\
		& $\lim_{n \to \infty} \vec{P}^{n}_{13} = \frac{1}{3}$ \\
		&\\
		\hline
		\multirow{3}{*}{Conclusion} & \\
		& From our observation we see that \\
		&\\
		& Options 1) and 4) are True.\\
		& \\
		\hline
\caption{}
\label{eq:solutions/2018/dec/106/table1}
	\end{longtable}
\twocolumn

\item Let $M_n$ denote the vector space of all $n\times n$ real matrices. Which of the following is a linear subspaces of $M_n$ :-
\begin{enumerate}
\item $ V_1 = \{  A \in M_n : \text{ A is nonsingular} \}$
\item $ V_2 = \{  A \in M_n : det(A) = 0 \}$
\item $ V_3 = \{  A \in M_n : trace(A) = 0 \}$
\item $ V_4 = \{  BA : A \in M_n\},$ where $ B$ is some fixed matrix in $ M_n$
\end{enumerate}
%
\solution
See Tables \ref{eq:solutions/2018/dec/106/table0} and \ref{eq:solutions/2018/dec/106/table1}


\onecolumn
	\begin{longtable}{|l|l|}
		\hline
		\multirow{3}{*}{Irreducible Markov Chain} 
		& \\
		& A Markov chain is $\textbf{irreducible}$ if all the states communicate with each other,\\
		& i.e., if there is only one communication class.\\
		&\\
		\hline
		\multirow{3}{*}{Aperiodic Markov Chain} & \\
		& If there is a self-transition in the chain ($p^{ii}>0$ for some i), then the chain is\\
		& called as $\textbf{aperiodic}$\\
		& \\
		\hline
		\multirow{3}{*}{Stationary Distribution} & \\
		& A stationary distribution of a Markov chain is a probability distribution that\\
		& remains unchanged in the Markov chain as time progresses. Typically, it is\\
		& represented as a row vector $\Vec{\pi}$ whose entries are probabilities summing to 1,\\ 
		& and given transition matrix $\textbf{P}$, it satisfies\\
		& \\
		&  \qquad \qquad  \qquad$\Vec{\pi} = \Vec{\pi} \textbf{P}$\\
		& \\
		\hline
\caption{}
\label{eq:solutions/2018/dec/106/table0}
	\end{longtable}
	\begin{longtable}{|l|l|}
		\hline
		\multirow{3}{*}{Drawing Transition diagram} 
		& \\
		& 
		
		$\begin{tikzpicture}[shorten >=1pt,node distance=2cm, scale =3, auto]
			\tikzstyle{every state}=[fill={rgb:black,1;white,10}]
			
			\node[state]   (q_1)                          {$1$};
			\node[state]   (q_2)  [right of=q_1]          {$2$};
			\node[state]   (q_3)  [below right of=q_1]          {$3$};
			
			\path[->]
			(q_1) edge [loop above] node {$\frac{1}{2}$}    (   )
			edge [bend left]  node {$\frac{1}{2}$}    (q_2)
			(q_2) edge [bend left]  node {$\frac{1}{2}$}    (q_3)
			edge [loop above] node {$\frac{1}{2}$}    ()
			(q_3) edge [bend left]  node {$\frac{1}{3}$}    (q_2)
			edge [bend left]  node {$\frac{1}{3}$}    (q_1)
			edge [loop below] node {$\frac{1}{3}$}    ();
		\end{tikzpicture}$
		
		\\  
		&\\
		&\\
		\hline
		\multirow{3}{*}{Checking whether the  } & \\
		& Here,\\chain is Irreducible
		& All the states are accessible to one another. \\and Aperiodic
		& $\implies$ They are in the same communication class. So, it is Irreducible.\\
		& \\
		& There exists the non- zero self-transition, which means that the chain \\
		& is Aperiodic.\\
		&\\ 
		& We know that if the Markov Chain is irreducible and aperiodic then \\
		& \qquad \qquad \qquad $\Vec{\pi}_{j} = \lim_{n \to \infty}P\{X_{n} = j\}$, $j = 1,...,N$ \\
		& These are the stationary probabilities. \\
		&\\
		\hline
		\multirow{3}{*}{Finding the Stationary} & \\
		& Stationary Probability can be represented as\\Probability Distributions
		& \qquad \qquad \qquad $\Vec{\pi} = \Vec{\pi} \vec{P}$\\
		& \\
		& \qquad $\implies$ $\myvec{v_{1}&&v_{2}&&v_{3}} = \myvec{v_{1}&&v_{2}&&v_{3}}\vec{P}$ \\
		& \\
		& Equating the above equation we get \\
		& \\
		& \qquad \qquad \qquad $\frac{1}{2}v_{1}-\frac{1}{3}v_{3} = 0$ $\label{eq:solutions/2018/dec/106/eq}$\\
		& \\
		& \qquad \qquad \qquad $\frac{1}{2}v_{1}-\frac{1}{2}v_{2} + \frac{1}{3}v_{3} = 0$\\
		& \\
		& \qquad \qquad \qquad $\frac{1}{2}v_{2}-\frac{2}{3}v_{3} = 0$\\
		& \\\
		& We see that summation of second and the third equation gives us the \\
		& first equation only. \\
		& And we know that the probability distribution will sum up to 1. \\
		& \\
		& \qquad \qquad \qquad $v_{1}+v_{2}+v_{3} = 1$ \\
		& \\
		& Therefore, we get the equation form as \\
		& \\
		& \qquad \qquad \qquad $\myvec{1&1&1\\\frac{1}{2}&0&\frac{-1}{3}\\\frac{1}{2}&\frac{-1}{2}&\frac{1}{3}}\myvec{v_{1}\\v_{2}\\v_{3}} = \myvec{1\\0\\0}$ \\
		& \\
		\hline
		\multirow{3}{*}{Solving the linear} & \\
		& The above linear equation can be solved using Gauss-Jordan method as\\equtions
		& \\
		& \qquad \qquad \qquad $\myvec{1&1&1&\vrule&1\\\frac{1}{2}&0&\frac{-1}{3}&\vrule&0\\\frac{1}{2}&\frac{-1}{2}&\frac{1}{3}&\vrule&0}$\\
		& \\
		& \qquad $\xleftrightarrow[]{R_2 \leftarrow R_2 - \frac{1}{2}R_1}$
		$\myvec{1&1&1&\vrule&1\\0&\frac{-1}{2}&\frac{-5}{6}&\vrule&\frac{-1}{2}\\\frac{1}{2}&\frac{-1}{2}&\frac{1}{3}&\vrule&0}$\\
		&\\
		& \qquad $\xleftrightarrow[]{R_3 \leftarrow R_3 - \frac{1}{2}R_1}$
		$\myvec{1&1&1&\vrule&1\\0&\frac{-1}{2}&\frac{-5}{6}&\vrule&\frac{-1}{2}\\0&-1&\frac{-1}{6}&\vrule&\frac{-1}{2}}$\\
		&\\
		& \qquad $\xleftrightarrow[]{R_2 \leftarrow \frac{-1}{2}R_2}$
		$\myvec{1&1&1&\vrule&1\\0&1&\frac{5}{3}&\vrule&1\\0&-1&\frac{-1}{6}&\vrule&\frac{-1}{2}}$\\
		&\\
		& \qquad $\xleftrightarrow[]{R_3 \leftarrow R_3 + R_2}$
		$\myvec{1&1&1&\vrule&1\\0&1&\frac{5}{3}&\vrule&1\\0&0&\frac{3}{2}&\vrule&\frac{1}{2}}$\\
		&\\
		& \qquad $\xleftrightarrow[]{R_3 \leftarrow \frac{3}{2}R_3}$
		$\myvec{1&1&1&\vrule&1\\0&1&\frac{5}{3}&\vrule&1\\0&0&1&\vrule&\frac{1}{3}}$\\
		&\\
		& \qquad $\xleftrightarrow[]{R_2 \leftarrow R_2 - \frac{5}{3}R_3}$
		$\myvec{1&1&1&\vrule&1\\0&1&0&\vrule&\frac{4}{9}\\0&0&1&\vrule&\frac{1}{3}}$\\
		&\\
		& \qquad $\xleftrightarrow[]{R_1 \leftarrow R_1 - R_3}$
		$\myvec{1&1&0&\vrule&\frac{2}{3}\\0&1&0&\vrule&\frac{4}{9}\\0&0&1&\vrule&\frac{1}{3}}$\\
		&\\
		& \qquad $\xleftrightarrow[]{R_1 \leftarrow R_1 - R_2}$
		$\myvec{1&0&0&\vrule&\frac{2}{9}\\0&1&0&\vrule&\frac{4}{9}\\0&0&1&\vrule&\frac{1}{3}}$\\
		&\\
		& $\therefore$, stationary probability distribution $\pi$ is given by \\
		& \qquad \qquad $\pi = \myvec{\frac{2}{9} & \frac{4}{9} & \frac{1}{3}}$ \\
		& \\
		\hline
		\multirow{3}{*}{Observations} & \\
		
		
		& Since the given transition probability matrix $\vec{P}$ is irreducible and aperiodic, \\
		& then $\lim_{n \to \infty} \vec{P}^{n}$ converges to a matrix with all rows identical and equal to $\vec{\pi}$. \\
		& \\
		& We were able to find $\vec{\pi}$ as $\myvec{\frac{2}{9} & \frac{4}{9} & \frac{1}{3}}$ \\
		& \\
		& $\lim_{n \to \infty} \vec{P}^{n} = \myvec{\frac{2}{9}&\frac{4}{9}&\frac{1}{3}\\\frac{2}{9}&\frac{4}{9}&\frac{1}{3}\\\frac{2}{9}&\frac{4}{9}&\frac{1}{3}}$\\
		& \\
		& From the above matrix, we get \\
		& \\
		& $\lim_{n \to \infty} \vec{P}^{n}_{11} = \frac{2}{9}$ \\
		&\\
		& $\lim_{n \to \infty} \vec{P}^{n}_{21} = \frac{2}{9}$ \\
		&\\
		& $\lim_{n \to \infty} \vec{P}^{n}_{32} = \frac{4}{9}$ \\
		&\\
		& $\lim_{n \to \infty} \vec{P}^{n}_{13} = \frac{1}{3}$ \\
		&\\
		\hline
		\multirow{3}{*}{Conclusion} & \\
		& From our observation we see that \\
		&\\
		& Options 1) and 4) are True.\\
		& \\
		\hline
\caption{}
\label{eq:solutions/2018/dec/106/table1}
	\end{longtable}
\twocolumn


\item If $\vec{P}$ and $\vec{Q}$ are invertible matrices such that   
$\vec{P}\vec{Q} = -\vec{Q}\vec{P}$,then we can conclude that
\begin{enumerate}
\item  $Tr(\vec{P})=Tr(\vec{Q})=0$ \label{eq:solutions/2016/dec/75/1}
\item  $Tr(\vec{P})=Tr(\vec{Q})=1$ \label{eq:solutions/2016/dec/75/2}
\item  $Tr(\vec{P})=-Tr(\vec{Q})$ \label{eq:solutions/2016/dec/75/3}
\item  $Tr(\vec{P}) \neq Tr(\vec{Q})$ \label{eq:solutions/2016/dec/75/4}
\end{enumerate}
%
%
\solution
See Tables \ref{eq:solutions/2018/dec/106/table0} and \ref{eq:solutions/2018/dec/106/table1}


\onecolumn
	\begin{longtable}{|l|l|}
		\hline
		\multirow{3}{*}{Irreducible Markov Chain} 
		& \\
		& A Markov chain is $\textbf{irreducible}$ if all the states communicate with each other,\\
		& i.e., if there is only one communication class.\\
		&\\
		\hline
		\multirow{3}{*}{Aperiodic Markov Chain} & \\
		& If there is a self-transition in the chain ($p^{ii}>0$ for some i), then the chain is\\
		& called as $\textbf{aperiodic}$\\
		& \\
		\hline
		\multirow{3}{*}{Stationary Distribution} & \\
		& A stationary distribution of a Markov chain is a probability distribution that\\
		& remains unchanged in the Markov chain as time progresses. Typically, it is\\
		& represented as a row vector $\Vec{\pi}$ whose entries are probabilities summing to 1,\\ 
		& and given transition matrix $\textbf{P}$, it satisfies\\
		& \\
		&  \qquad \qquad  \qquad$\Vec{\pi} = \Vec{\pi} \textbf{P}$\\
		& \\
		\hline
\caption{}
\label{eq:solutions/2018/dec/106/table0}
	\end{longtable}
	\begin{longtable}{|l|l|}
		\hline
		\multirow{3}{*}{Drawing Transition diagram} 
		& \\
		& 
		
		$\begin{tikzpicture}[shorten >=1pt,node distance=2cm, scale =3, auto]
			\tikzstyle{every state}=[fill={rgb:black,1;white,10}]
			
			\node[state]   (q_1)                          {$1$};
			\node[state]   (q_2)  [right of=q_1]          {$2$};
			\node[state]   (q_3)  [below right of=q_1]          {$3$};
			
			\path[->]
			(q_1) edge [loop above] node {$\frac{1}{2}$}    (   )
			edge [bend left]  node {$\frac{1}{2}$}    (q_2)
			(q_2) edge [bend left]  node {$\frac{1}{2}$}    (q_3)
			edge [loop above] node {$\frac{1}{2}$}    ()
			(q_3) edge [bend left]  node {$\frac{1}{3}$}    (q_2)
			edge [bend left]  node {$\frac{1}{3}$}    (q_1)
			edge [loop below] node {$\frac{1}{3}$}    ();
		\end{tikzpicture}$
		
		\\  
		&\\
		&\\
		\hline
		\multirow{3}{*}{Checking whether the  } & \\
		& Here,\\chain is Irreducible
		& All the states are accessible to one another. \\and Aperiodic
		& $\implies$ They are in the same communication class. So, it is Irreducible.\\
		& \\
		& There exists the non- zero self-transition, which means that the chain \\
		& is Aperiodic.\\
		&\\ 
		& We know that if the Markov Chain is irreducible and aperiodic then \\
		& \qquad \qquad \qquad $\Vec{\pi}_{j} = \lim_{n \to \infty}P\{X_{n} = j\}$, $j = 1,...,N$ \\
		& These are the stationary probabilities. \\
		&\\
		\hline
		\multirow{3}{*}{Finding the Stationary} & \\
		& Stationary Probability can be represented as\\Probability Distributions
		& \qquad \qquad \qquad $\Vec{\pi} = \Vec{\pi} \vec{P}$\\
		& \\
		& \qquad $\implies$ $\myvec{v_{1}&&v_{2}&&v_{3}} = \myvec{v_{1}&&v_{2}&&v_{3}}\vec{P}$ \\
		& \\
		& Equating the above equation we get \\
		& \\
		& \qquad \qquad \qquad $\frac{1}{2}v_{1}-\frac{1}{3}v_{3} = 0$ $\label{eq:solutions/2018/dec/106/eq}$\\
		& \\
		& \qquad \qquad \qquad $\frac{1}{2}v_{1}-\frac{1}{2}v_{2} + \frac{1}{3}v_{3} = 0$\\
		& \\
		& \qquad \qquad \qquad $\frac{1}{2}v_{2}-\frac{2}{3}v_{3} = 0$\\
		& \\\
		& We see that summation of second and the third equation gives us the \\
		& first equation only. \\
		& And we know that the probability distribution will sum up to 1. \\
		& \\
		& \qquad \qquad \qquad $v_{1}+v_{2}+v_{3} = 1$ \\
		& \\
		& Therefore, we get the equation form as \\
		& \\
		& \qquad \qquad \qquad $\myvec{1&1&1\\\frac{1}{2}&0&\frac{-1}{3}\\\frac{1}{2}&\frac{-1}{2}&\frac{1}{3}}\myvec{v_{1}\\v_{2}\\v_{3}} = \myvec{1\\0\\0}$ \\
		& \\
		\hline
		\multirow{3}{*}{Solving the linear} & \\
		& The above linear equation can be solved using Gauss-Jordan method as\\equtions
		& \\
		& \qquad \qquad \qquad $\myvec{1&1&1&\vrule&1\\\frac{1}{2}&0&\frac{-1}{3}&\vrule&0\\\frac{1}{2}&\frac{-1}{2}&\frac{1}{3}&\vrule&0}$\\
		& \\
		& \qquad $\xleftrightarrow[]{R_2 \leftarrow R_2 - \frac{1}{2}R_1}$
		$\myvec{1&1&1&\vrule&1\\0&\frac{-1}{2}&\frac{-5}{6}&\vrule&\frac{-1}{2}\\\frac{1}{2}&\frac{-1}{2}&\frac{1}{3}&\vrule&0}$\\
		&\\
		& \qquad $\xleftrightarrow[]{R_3 \leftarrow R_3 - \frac{1}{2}R_1}$
		$\myvec{1&1&1&\vrule&1\\0&\frac{-1}{2}&\frac{-5}{6}&\vrule&\frac{-1}{2}\\0&-1&\frac{-1}{6}&\vrule&\frac{-1}{2}}$\\
		&\\
		& \qquad $\xleftrightarrow[]{R_2 \leftarrow \frac{-1}{2}R_2}$
		$\myvec{1&1&1&\vrule&1\\0&1&\frac{5}{3}&\vrule&1\\0&-1&\frac{-1}{6}&\vrule&\frac{-1}{2}}$\\
		&\\
		& \qquad $\xleftrightarrow[]{R_3 \leftarrow R_3 + R_2}$
		$\myvec{1&1&1&\vrule&1\\0&1&\frac{5}{3}&\vrule&1\\0&0&\frac{3}{2}&\vrule&\frac{1}{2}}$\\
		&\\
		& \qquad $\xleftrightarrow[]{R_3 \leftarrow \frac{3}{2}R_3}$
		$\myvec{1&1&1&\vrule&1\\0&1&\frac{5}{3}&\vrule&1\\0&0&1&\vrule&\frac{1}{3}}$\\
		&\\
		& \qquad $\xleftrightarrow[]{R_2 \leftarrow R_2 - \frac{5}{3}R_3}$
		$\myvec{1&1&1&\vrule&1\\0&1&0&\vrule&\frac{4}{9}\\0&0&1&\vrule&\frac{1}{3}}$\\
		&\\
		& \qquad $\xleftrightarrow[]{R_1 \leftarrow R_1 - R_3}$
		$\myvec{1&1&0&\vrule&\frac{2}{3}\\0&1&0&\vrule&\frac{4}{9}\\0&0&1&\vrule&\frac{1}{3}}$\\
		&\\
		& \qquad $\xleftrightarrow[]{R_1 \leftarrow R_1 - R_2}$
		$\myvec{1&0&0&\vrule&\frac{2}{9}\\0&1&0&\vrule&\frac{4}{9}\\0&0&1&\vrule&\frac{1}{3}}$\\
		&\\
		& $\therefore$, stationary probability distribution $\pi$ is given by \\
		& \qquad \qquad $\pi = \myvec{\frac{2}{9} & \frac{4}{9} & \frac{1}{3}}$ \\
		& \\
		\hline
		\multirow{3}{*}{Observations} & \\
		
		
		& Since the given transition probability matrix $\vec{P}$ is irreducible and aperiodic, \\
		& then $\lim_{n \to \infty} \vec{P}^{n}$ converges to a matrix with all rows identical and equal to $\vec{\pi}$. \\
		& \\
		& We were able to find $\vec{\pi}$ as $\myvec{\frac{2}{9} & \frac{4}{9} & \frac{1}{3}}$ \\
		& \\
		& $\lim_{n \to \infty} \vec{P}^{n} = \myvec{\frac{2}{9}&\frac{4}{9}&\frac{1}{3}\\\frac{2}{9}&\frac{4}{9}&\frac{1}{3}\\\frac{2}{9}&\frac{4}{9}&\frac{1}{3}}$\\
		& \\
		& From the above matrix, we get \\
		& \\
		& $\lim_{n \to \infty} \vec{P}^{n}_{11} = \frac{2}{9}$ \\
		&\\
		& $\lim_{n \to \infty} \vec{P}^{n}_{21} = \frac{2}{9}$ \\
		&\\
		& $\lim_{n \to \infty} \vec{P}^{n}_{32} = \frac{4}{9}$ \\
		&\\
		& $\lim_{n \to \infty} \vec{P}^{n}_{13} = \frac{1}{3}$ \\
		&\\
		\hline
		\multirow{3}{*}{Conclusion} & \\
		& From our observation we see that \\
		&\\
		& Options 1) and 4) are True.\\
		& \\
		\hline
\caption{}
\label{eq:solutions/2018/dec/106/table1}
	\end{longtable}
\twocolumn

\twocolumn
\item Let $\vec{W_1}$, $\vec{W_2}$, $\vec{W_3}$ be 3 distinct subspaces of $\vec{R}^{10}$ such that each $\vec{W_i}$ has dimension of 9. Let $\vec{W} = \vec{W_1} \cap \vec{W_2} \cap \vec{W_3}$. Then we can conclude that\\
\begin{enumerate}
\item $\vec{W}$ may not be a subspace of $\vec{R}^{10}$\\
\item dim $\vec{W} \leq 8$\\
\item dim $\vec{W} \geq 7$\\
\item  dim $\vec{W} \leq 3$\\
\end{enumerate}
 %
\solution
See Tables \ref{eq:solutions/2018/dec/106/table0} and \ref{eq:solutions/2018/dec/106/table1}


\onecolumn
	\begin{longtable}{|l|l|}
		\hline
		\multirow{3}{*}{Irreducible Markov Chain} 
		& \\
		& A Markov chain is $\textbf{irreducible}$ if all the states communicate with each other,\\
		& i.e., if there is only one communication class.\\
		&\\
		\hline
		\multirow{3}{*}{Aperiodic Markov Chain} & \\
		& If there is a self-transition in the chain ($p^{ii}>0$ for some i), then the chain is\\
		& called as $\textbf{aperiodic}$\\
		& \\
		\hline
		\multirow{3}{*}{Stationary Distribution} & \\
		& A stationary distribution of a Markov chain is a probability distribution that\\
		& remains unchanged in the Markov chain as time progresses. Typically, it is\\
		& represented as a row vector $\Vec{\pi}$ whose entries are probabilities summing to 1,\\ 
		& and given transition matrix $\textbf{P}$, it satisfies\\
		& \\
		&  \qquad \qquad  \qquad$\Vec{\pi} = \Vec{\pi} \textbf{P}$\\
		& \\
		\hline
\caption{}
\label{eq:solutions/2018/dec/106/table0}
	\end{longtable}
	\begin{longtable}{|l|l|}
		\hline
		\multirow{3}{*}{Drawing Transition diagram} 
		& \\
		& 
		
		$\begin{tikzpicture}[shorten >=1pt,node distance=2cm, scale =3, auto]
			\tikzstyle{every state}=[fill={rgb:black,1;white,10}]
			
			\node[state]   (q_1)                          {$1$};
			\node[state]   (q_2)  [right of=q_1]          {$2$};
			\node[state]   (q_3)  [below right of=q_1]          {$3$};
			
			\path[->]
			(q_1) edge [loop above] node {$\frac{1}{2}$}    (   )
			edge [bend left]  node {$\frac{1}{2}$}    (q_2)
			(q_2) edge [bend left]  node {$\frac{1}{2}$}    (q_3)
			edge [loop above] node {$\frac{1}{2}$}    ()
			(q_3) edge [bend left]  node {$\frac{1}{3}$}    (q_2)
			edge [bend left]  node {$\frac{1}{3}$}    (q_1)
			edge [loop below] node {$\frac{1}{3}$}    ();
		\end{tikzpicture}$
		
		\\  
		&\\
		&\\
		\hline
		\multirow{3}{*}{Checking whether the  } & \\
		& Here,\\chain is Irreducible
		& All the states are accessible to one another. \\and Aperiodic
		& $\implies$ They are in the same communication class. So, it is Irreducible.\\
		& \\
		& There exists the non- zero self-transition, which means that the chain \\
		& is Aperiodic.\\
		&\\ 
		& We know that if the Markov Chain is irreducible and aperiodic then \\
		& \qquad \qquad \qquad $\Vec{\pi}_{j} = \lim_{n \to \infty}P\{X_{n} = j\}$, $j = 1,...,N$ \\
		& These are the stationary probabilities. \\
		&\\
		\hline
		\multirow{3}{*}{Finding the Stationary} & \\
		& Stationary Probability can be represented as\\Probability Distributions
		& \qquad \qquad \qquad $\Vec{\pi} = \Vec{\pi} \vec{P}$\\
		& \\
		& \qquad $\implies$ $\myvec{v_{1}&&v_{2}&&v_{3}} = \myvec{v_{1}&&v_{2}&&v_{3}}\vec{P}$ \\
		& \\
		& Equating the above equation we get \\
		& \\
		& \qquad \qquad \qquad $\frac{1}{2}v_{1}-\frac{1}{3}v_{3} = 0$ $\label{eq:solutions/2018/dec/106/eq}$\\
		& \\
		& \qquad \qquad \qquad $\frac{1}{2}v_{1}-\frac{1}{2}v_{2} + \frac{1}{3}v_{3} = 0$\\
		& \\
		& \qquad \qquad \qquad $\frac{1}{2}v_{2}-\frac{2}{3}v_{3} = 0$\\
		& \\\
		& We see that summation of second and the third equation gives us the \\
		& first equation only. \\
		& And we know that the probability distribution will sum up to 1. \\
		& \\
		& \qquad \qquad \qquad $v_{1}+v_{2}+v_{3} = 1$ \\
		& \\
		& Therefore, we get the equation form as \\
		& \\
		& \qquad \qquad \qquad $\myvec{1&1&1\\\frac{1}{2}&0&\frac{-1}{3}\\\frac{1}{2}&\frac{-1}{2}&\frac{1}{3}}\myvec{v_{1}\\v_{2}\\v_{3}} = \myvec{1\\0\\0}$ \\
		& \\
		\hline
		\multirow{3}{*}{Solving the linear} & \\
		& The above linear equation can be solved using Gauss-Jordan method as\\equtions
		& \\
		& \qquad \qquad \qquad $\myvec{1&1&1&\vrule&1\\\frac{1}{2}&0&\frac{-1}{3}&\vrule&0\\\frac{1}{2}&\frac{-1}{2}&\frac{1}{3}&\vrule&0}$\\
		& \\
		& \qquad $\xleftrightarrow[]{R_2 \leftarrow R_2 - \frac{1}{2}R_1}$
		$\myvec{1&1&1&\vrule&1\\0&\frac{-1}{2}&\frac{-5}{6}&\vrule&\frac{-1}{2}\\\frac{1}{2}&\frac{-1}{2}&\frac{1}{3}&\vrule&0}$\\
		&\\
		& \qquad $\xleftrightarrow[]{R_3 \leftarrow R_3 - \frac{1}{2}R_1}$
		$\myvec{1&1&1&\vrule&1\\0&\frac{-1}{2}&\frac{-5}{6}&\vrule&\frac{-1}{2}\\0&-1&\frac{-1}{6}&\vrule&\frac{-1}{2}}$\\
		&\\
		& \qquad $\xleftrightarrow[]{R_2 \leftarrow \frac{-1}{2}R_2}$
		$\myvec{1&1&1&\vrule&1\\0&1&\frac{5}{3}&\vrule&1\\0&-1&\frac{-1}{6}&\vrule&\frac{-1}{2}}$\\
		&\\
		& \qquad $\xleftrightarrow[]{R_3 \leftarrow R_3 + R_2}$
		$\myvec{1&1&1&\vrule&1\\0&1&\frac{5}{3}&\vrule&1\\0&0&\frac{3}{2}&\vrule&\frac{1}{2}}$\\
		&\\
		& \qquad $\xleftrightarrow[]{R_3 \leftarrow \frac{3}{2}R_3}$
		$\myvec{1&1&1&\vrule&1\\0&1&\frac{5}{3}&\vrule&1\\0&0&1&\vrule&\frac{1}{3}}$\\
		&\\
		& \qquad $\xleftrightarrow[]{R_2 \leftarrow R_2 - \frac{5}{3}R_3}$
		$\myvec{1&1&1&\vrule&1\\0&1&0&\vrule&\frac{4}{9}\\0&0&1&\vrule&\frac{1}{3}}$\\
		&\\
		& \qquad $\xleftrightarrow[]{R_1 \leftarrow R_1 - R_3}$
		$\myvec{1&1&0&\vrule&\frac{2}{3}\\0&1&0&\vrule&\frac{4}{9}\\0&0&1&\vrule&\frac{1}{3}}$\\
		&\\
		& \qquad $\xleftrightarrow[]{R_1 \leftarrow R_1 - R_2}$
		$\myvec{1&0&0&\vrule&\frac{2}{9}\\0&1&0&\vrule&\frac{4}{9}\\0&0&1&\vrule&\frac{1}{3}}$\\
		&\\
		& $\therefore$, stationary probability distribution $\pi$ is given by \\
		& \qquad \qquad $\pi = \myvec{\frac{2}{9} & \frac{4}{9} & \frac{1}{3}}$ \\
		& \\
		\hline
		\multirow{3}{*}{Observations} & \\
		
		
		& Since the given transition probability matrix $\vec{P}$ is irreducible and aperiodic, \\
		& then $\lim_{n \to \infty} \vec{P}^{n}$ converges to a matrix with all rows identical and equal to $\vec{\pi}$. \\
		& \\
		& We were able to find $\vec{\pi}$ as $\myvec{\frac{2}{9} & \frac{4}{9} & \frac{1}{3}}$ \\
		& \\
		& $\lim_{n \to \infty} \vec{P}^{n} = \myvec{\frac{2}{9}&\frac{4}{9}&\frac{1}{3}\\\frac{2}{9}&\frac{4}{9}&\frac{1}{3}\\\frac{2}{9}&\frac{4}{9}&\frac{1}{3}}$\\
		& \\
		& From the above matrix, we get \\
		& \\
		& $\lim_{n \to \infty} \vec{P}^{n}_{11} = \frac{2}{9}$ \\
		&\\
		& $\lim_{n \to \infty} \vec{P}^{n}_{21} = \frac{2}{9}$ \\
		&\\
		& $\lim_{n \to \infty} \vec{P}^{n}_{32} = \frac{4}{9}$ \\
		&\\
		& $\lim_{n \to \infty} \vec{P}^{n}_{13} = \frac{1}{3}$ \\
		&\\
		\hline
		\multirow{3}{*}{Conclusion} & \\
		& From our observation we see that \\
		&\\
		& Options 1) and 4) are True.\\
		& \\
		\hline
\caption{}
\label{eq:solutions/2018/dec/106/table1}
	\end{longtable}
\twocolumn

\twocolumn


%\item Consider a Markov Chain with state space $\cbrak{0,1,2}$ and transition matrix
%\begin{align}
%P = 
%\begin{blockarray}{c@{\hspace{1pt}}rrr@{\hspace{3pt}}}
%         & 0   & 1   & 2 \\
%        \begin{block}{r@{\hspace{3pt}}@{\hspace{1pt}}
%    (@{\hspace{1pt}}rrr@{\hspace{1pt}}@{\hspace{1pt}})}
%        0 & \frac{1}{2} & \frac{1}{2} & 0  \\
%        1 & 0 &\frac{1}{2}  & \frac{3}{4}  \\
%%
%        2 &  \frac{1}{3} & \frac{1}{3} & \frac{1}{3}  \\
%        \end{block}
%    \end{blockarray}
%\end{align}
%For any two states $i$ and $j$, let $p_{ij}^{(n)}$ denote the $n$-step transition probability of going from $i$ to $j$.  Identify correct statements.
%\begin{enumerate}
%\item $\lim_{n \to \infty} p_{11}^{(n)} = \frac{2}{9}$
%\item $\lim_{n \to \infty} p_{21}^{(n)} = 0$
%\item $\lim_{n \to \infty} p_{32}^{(n)} = \frac{1}{3}$
%\item $\lim_{n \to \infty} p_{13}^{(n)} = \frac{1}{3}$
%\end{enumerate}

\end{enumerate}
