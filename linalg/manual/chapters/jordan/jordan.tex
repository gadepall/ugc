\subsection{Motivation}
\renewcommand{\theequation}{\theenumi}
\renewcommand{\thefigure}{\theenumi}
\begin{enumerate}[label=\thesection.\arabic*.,ref=\thesection.\theenumi]
\numberwithin{equation}{enumi}
\numberwithin{figure}{enumi}

\item Consider the matrix

\begin{align}
\vec{A} = 
\myvec
{
  5 &  4 &  2 &  1 \\[2pt]
  0 &  1 & -1 & -1 \\[2pt]
 -1 & -1 &  3 &  0 \\[2pt]
  1 &  1 & -1 &  2
}
\label{eq:diag_exam_jord}
\end{align}
%
%
%
The matrix $\vec{A}$ in \eqref{eq:diag_exam_jord} has the characteristic polynomial
\begin{align} 
p(x) & = \det(x \vec{I} - \vec{A}) \\ &  = x^4 - 11 x^3 + 42 x^2 - 64 x + 32  \\ & = (x-1)(x-2)(x-4)^2. \, 
\end{align}
Thus, the eigenvalues are given by $x = 1,2,4$.
\item The eigenvectors corresponding to the above eigenvalues are
\begin{align} 
\vec{v}_1 = \myvec{-1\\ 1\\ 0\\ 0}, 
\vec{v}_2 = \myvec{1\\ -1\\ 0\\ 1}, 
\vec{v}_3 = \myvec{1\\ 0\\ -1\\ 1}, 
\end{align}
\item Hence the geometric multiplicity of each eigenvalue is 1.

\item Clearly, the dimension of the eigenspace is $< 4$.  Hence, the matrix $\vec{A}$ is not diagonalizable.
\item The minimal polynomial 
\begin{align} 
m(x) & = (x-1)(x-2)(x-4)^2. \, 
\end{align}
\end{enumerate}
\subsection{Jordan Form}
\renewcommand{\theequation}{\theenumi}
\renewcommand{\thefigure}{\theenumi}
\begin{enumerate}[label=\thesection.\arabic*.,ref=\thesection.\theenumi]
\numberwithin{equation}{enumi}
\numberwithin{figure}{enumi}

\item To find the jordan form for a matrix $\vec{A}$,
\begin{enumerate}
\item The eigenvalues $\lambda_i$ are the entries down the diagonal 
\item If the minimal polynomial is
\begin{align}
m_{\vec{A}}(x) = \prod_{i=1}^{k}\brak{x - \lambda_i}^{s_i}
\end{align}
then $s_i$ is the size of the largest $\lambda_i$ block in $\vec{A}$.
\item If the characteristic polynomial is
\begin{align}
p_{\vec{A}}(x) = \prod_{i=1}^{k}\brak{x - \lambda_i}^{r_i}
\end{align}
then $r_i$ is the number of occurences of $\lambda_i$ on the diagonal.
\item The geometric multiplicity of $\lambda_i$ is the number of $\lambda_i$-blocks in $\vec{A}$.
\end{enumerate}
\item Since the power of $\lambda_3 = 4$ in the minimal polynomial is 2, the size of its largest
Jordan block is 2 and given by
\begin{align}
\myvec{4 & 1 \\ 0 & 4}
\end{align}
Also, the geometric multiplicity of $\lambda_3 = 4$ is 1,  so there is only one such block.
\item The Jordan decomposition of $\vec{A}$ is given by
\begin{align}
\vec{A} = \vec{P}^{-1}\vec{J}\vec{P}
\label{eq:jord_mat}
\end{align}
%
where
%
\begin{align}
\vec{J} = 
\myvec
{
  1 &  0 &  0 &  0 \\[2pt]
  0 &  2 & 0 & 0 \\[2pt]
 0 & 0 &  4 &  1 \\[2pt]
  0 &  0 & 0 &  4
}
\label{eq:diag_exam_jord_mat}
\end{align}
\item The matrix $\vec{P}$ in \label{eq:exam_jord_mat} is given by 
\begin{align}
\vec{P} = \myvec{\vec{v}_1 & \vec{v}_2 & \vec{v}_3 & \vec{v}_4 }
\end{align}
%
where 
\begin{align}
\vec{A}\vec{v}_4 = \vec{v}_3   
\end{align}
$\vec{v}_4$ is defined to be the {\em generalized} eigenvector of $\vec{A}$.
\end{enumerate}
