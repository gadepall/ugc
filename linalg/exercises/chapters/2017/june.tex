\renewcommand{\theequation}{\theenumi}
\renewcommand{\thefigure}{\theenumi}
\begin{enumerate}[label=\thesection.\arabic*.,ref=\thesection.\theenumi]
\numberwithin{equation}{enumi}
\numberwithin{figure}{enumi}

\item 	Let $\vec{A}$ be a $4 \times 4$ matrix. Suppose that the null space $N(\vec{A})$ of $\vec{A}$ is
	\begin{align}
		\cbrak{(x,y,z,w) \in \vec{R}^4 : x + y + z = 0 , x + y + w = 0}
	\end{align}
Then which one of the following is correct
\begin{enumerate}
\item  dim(column space$(\vec{A})) = 1$ 
\item dim(column space$(\vec{A})) = 2$
\item rank$(\vec{A}) = 1$
\item $\vec{S}= \cbrak{(1, 1, 1, 0), (1, 1, 0, 1)}$ is a basis of $N(\vec{A})$
\end{enumerate}
%
%
\solution
The nullspace is given by 
\begin{align}
	\myvec{1 & 1 & 1 & 0 \\ 1 & 1 & 0 & 1\\ 0 & 0 & 0 & 0\\0 & 0 & 0 & 0}\myvec{x\\y\\z\\w} = \myvec{0 \\ 0 \\ 0 \\ 0}
\end{align}	
Row reducing the above matrix we get,
\begin{align}
	\myvec{1 & 1 & 1 & 0 \\ 1 & 1 & 0 & 1\\ 0 & 0 & 0 & 0\\0 & 0 & 0 & 0}
	\xleftrightarrow[R_2 \leftarrow R_2 \times -1]{R_2 \leftarrow R_2 - R_1}
	\myvec{1 & 1 & 1 & 0 \\ 0 & 0 & 1 & -1\\ 0 & 0 & 0 & 0\\0 & 0 & 0 & 0}\\
	\xleftrightarrow{R_1 \leftarrow R_1- R_2}
	\myvec{1 & 1 & 0 & 1 \\ 0 & 0 & 1 & -1\\ 0 & 0 & 0 & 0\\0 & 0 & 0 & 0} \label{eq:solutions/2017/dec/27/eq:rref}
\end{align}
See Table \ref{eq:solutions/2017/dec/27/tab}

\begin{table*}[!ht]
	\begin{tabular}{|m{4.5cm}|l|}
		\hline
		&\\
		dim(C$(\vec{A})) = 1$ 
		& \textbf{False}. Because the number of pivot variables are 2 as obtained in \eqref{eq:solutions/2017/dec/27/eq:rref}\\
		&\\
		\hline
		&\\
		dim(C$(\vec{A})) = 2$
		& \textbf{True}. Since the number of pivot variables are 2, the rank of $\vec{A}$ is 2.\\
		&$\therefore dim(C(\vec{A})) = 2 \quad [\because dim(C(\vec{A})) = rank(\vec{A})]$ \\
		&\\
		\hline
		&\\
	     rank$(\vec{A}) = 1$
		& \textbf{False}. Because the rank$(\vec{A}) = 2$, as the number of pivot variables are 2\\
		&\\
		\hline
		&\\
		$\vec{S}$ = $\cbrak{(1, 1, 1, 0), (1, 1, 0, 1)}$ is a basis of $N(\vec{A})$
		& \textbf{False}. \\
		& Let, \\
		&  $\vec{u} = \myvec{1\\1\\1\\0}, \vec{v} = \myvec{1\\1\\0\\1}$\\ 
		&Consider, \\
		&$\myvec{1 & 1 & 1 & 0 \\ 1 & 1 & 0 & 1\\ 0 & 0 & 0 & 0\\0 & 0 & 0 & 0}\myvec{1\\1\\1\\0} = \myvec{3\\2\\0\\0} \not = \myvec{0\\0\\0\\0}$\\
		& Similarly,\\
		&$\myvec{1 & 1 & 1 & 0 \\ 1 & 1 & 0 & 1\\ 0 & 0 & 0 & 0\\0 & 0 & 0 & 0}\myvec{1\\1\\0\\1} = \myvec{2\\3\\0\\0} \not = \myvec{0\\0\\0\\0}$ \\
		&Hence, the given vectors do not form the basis.\\
		\hline
	\end{tabular}
\caption{}
\label{eq:solutions/2017/dec/27/tab}
\end{table*}

\item Let $\vec{A}$ and $\vec{B}$ be real invertible matrices such that 
\begin{align}
    \vec{AB}=-\vec{BA}\label{eq:eq:solutions/2017/june/28/eq1}.
\end{align}
Then
\begin{enumerate}
    \item trace{$\vec{A}$} = trace($\vec{B}$) = 0
    \item trace{$\vec{A}$} = trace($\vec{B}$) = 1
    \item trace{$\vec{A}$} = 0, trace($\vec{B}$) = 1
    \item trace($\vec{A}$) = 1, trace($\vec{B}$) = 0
\end{enumerate}
%
\solution
The nullspace is given by 
\begin{align}
	\myvec{1 & 1 & 1 & 0 \\ 1 & 1 & 0 & 1\\ 0 & 0 & 0 & 0\\0 & 0 & 0 & 0}\myvec{x\\y\\z\\w} = \myvec{0 \\ 0 \\ 0 \\ 0}
\end{align}	
Row reducing the above matrix we get,
\begin{align}
	\myvec{1 & 1 & 1 & 0 \\ 1 & 1 & 0 & 1\\ 0 & 0 & 0 & 0\\0 & 0 & 0 & 0}
	\xleftrightarrow[R_2 \leftarrow R_2 \times -1]{R_2 \leftarrow R_2 - R_1}
	\myvec{1 & 1 & 1 & 0 \\ 0 & 0 & 1 & -1\\ 0 & 0 & 0 & 0\\0 & 0 & 0 & 0}\\
	\xleftrightarrow{R_1 \leftarrow R_1- R_2}
	\myvec{1 & 1 & 0 & 1 \\ 0 & 0 & 1 & -1\\ 0 & 0 & 0 & 0\\0 & 0 & 0 & 0} \label{eq:solutions/2017/dec/27/eq:rref}
\end{align}
See Table \ref{eq:solutions/2017/dec/27/tab}

\begin{table*}[!ht]
	\begin{tabular}{|m{4.5cm}|l|}
		\hline
		&\\
		dim(C$(\vec{A})) = 1$ 
		& \textbf{False}. Because the number of pivot variables are 2 as obtained in \eqref{eq:solutions/2017/dec/27/eq:rref}\\
		&\\
		\hline
		&\\
		dim(C$(\vec{A})) = 2$
		& \textbf{True}. Since the number of pivot variables are 2, the rank of $\vec{A}$ is 2.\\
		&$\therefore dim(C(\vec{A})) = 2 \quad [\because dim(C(\vec{A})) = rank(\vec{A})]$ \\
		&\\
		\hline
		&\\
	     rank$(\vec{A}) = 1$
		& \textbf{False}. Because the rank$(\vec{A}) = 2$, as the number of pivot variables are 2\\
		&\\
		\hline
		&\\
		$\vec{S}$ = $\cbrak{(1, 1, 1, 0), (1, 1, 0, 1)}$ is a basis of $N(\vec{A})$
		& \textbf{False}. \\
		& Let, \\
		&  $\vec{u} = \myvec{1\\1\\1\\0}, \vec{v} = \myvec{1\\1\\0\\1}$\\ 
		&Consider, \\
		&$\myvec{1 & 1 & 1 & 0 \\ 1 & 1 & 0 & 1\\ 0 & 0 & 0 & 0\\0 & 0 & 0 & 0}\myvec{1\\1\\1\\0} = \myvec{3\\2\\0\\0} \not = \myvec{0\\0\\0\\0}$\\
		& Similarly,\\
		&$\myvec{1 & 1 & 1 & 0 \\ 1 & 1 & 0 & 1\\ 0 & 0 & 0 & 0\\0 & 0 & 0 & 0}\myvec{1\\1\\0\\1} = \myvec{2\\3\\0\\0} \not = \myvec{0\\0\\0\\0}$ \\
		&Hence, the given vectors do not form the basis.\\
		\hline
	\end{tabular}
\caption{}
\label{eq:solutions/2017/dec/27/tab}
\end{table*}

\item Let $\vec{A}$ be an n $\times$ n self-adjoint matrix with eigenvalues $\lambda_1, \cdots, \lambda_2$.
Let,\begin{align} \norm{\vec{X}}_2=\sqrt{\vert\vec{X}_{1}^{2}\vert+\cdots+\vert\vec{X}_{n}^{2}\vert}\end{align} for $\vec{X}$=$(\vec{X}_{1},\cdots,\vec{X}_{n})\in \mathbb{C}^n$. If \begin{align}
p(\vec{A})=a_0\vec{I}+a_1\vec{A}+\cdots+a_n\vec{A}^n
\end{align}
then $sup_{\norm{\vec{X}}_{2}=1}\norm{p(\vec{A})\vec{X}}_2$ is equal to
%
\\
\solution
The nullspace is given by 
\begin{align}
	\myvec{1 & 1 & 1 & 0 \\ 1 & 1 & 0 & 1\\ 0 & 0 & 0 & 0\\0 & 0 & 0 & 0}\myvec{x\\y\\z\\w} = \myvec{0 \\ 0 \\ 0 \\ 0}
\end{align}	
Row reducing the above matrix we get,
\begin{align}
	\myvec{1 & 1 & 1 & 0 \\ 1 & 1 & 0 & 1\\ 0 & 0 & 0 & 0\\0 & 0 & 0 & 0}
	\xleftrightarrow[R_2 \leftarrow R_2 \times -1]{R_2 \leftarrow R_2 - R_1}
	\myvec{1 & 1 & 1 & 0 \\ 0 & 0 & 1 & -1\\ 0 & 0 & 0 & 0\\0 & 0 & 0 & 0}\\
	\xleftrightarrow{R_1 \leftarrow R_1- R_2}
	\myvec{1 & 1 & 0 & 1 \\ 0 & 0 & 1 & -1\\ 0 & 0 & 0 & 0\\0 & 0 & 0 & 0} \label{eq:solutions/2017/dec/27/eq:rref}
\end{align}
See Table \ref{eq:solutions/2017/dec/27/tab}

\begin{table*}[!ht]
	\begin{tabular}{|m{4.5cm}|l|}
		\hline
		&\\
		dim(C$(\vec{A})) = 1$ 
		& \textbf{False}. Because the number of pivot variables are 2 as obtained in \eqref{eq:solutions/2017/dec/27/eq:rref}\\
		&\\
		\hline
		&\\
		dim(C$(\vec{A})) = 2$
		& \textbf{True}. Since the number of pivot variables are 2, the rank of $\vec{A}$ is 2.\\
		&$\therefore dim(C(\vec{A})) = 2 \quad [\because dim(C(\vec{A})) = rank(\vec{A})]$ \\
		&\\
		\hline
		&\\
	     rank$(\vec{A}) = 1$
		& \textbf{False}. Because the rank$(\vec{A}) = 2$, as the number of pivot variables are 2\\
		&\\
		\hline
		&\\
		$\vec{S}$ = $\cbrak{(1, 1, 1, 0), (1, 1, 0, 1)}$ is a basis of $N(\vec{A})$
		& \textbf{False}. \\
		& Let, \\
		&  $\vec{u} = \myvec{1\\1\\1\\0}, \vec{v} = \myvec{1\\1\\0\\1}$\\ 
		&Consider, \\
		&$\myvec{1 & 1 & 1 & 0 \\ 1 & 1 & 0 & 1\\ 0 & 0 & 0 & 0\\0 & 0 & 0 & 0}\myvec{1\\1\\1\\0} = \myvec{3\\2\\0\\0} \not = \myvec{0\\0\\0\\0}$\\
		& Similarly,\\
		&$\myvec{1 & 1 & 1 & 0 \\ 1 & 1 & 0 & 1\\ 0 & 0 & 0 & 0\\0 & 0 & 0 & 0}\myvec{1\\1\\0\\1} = \myvec{2\\3\\0\\0} \not = \myvec{0\\0\\0\\0}$ \\
		&Hence, the given vectors do not form the basis.\\
		\hline
	\end{tabular}
\caption{}
\label{eq:solutions/2017/dec/27/tab}
\end{table*}

\item Let $p \brak{x}= \alpha x^2+\beta x + \gamma$ be a polynomial, where $\alpha,\beta,\gamma \epsilon R$. Fix $X_0 \epsilon R$. Let $S=\{\brak{a,b,c}  \epsilon R^3: p \brak{x}= a \brak{x-x_0}^2+b \brak{x-x_0}+ c\}$ for all $x\epsilon R$. Find the number of elements in S is
\begin{enumerate}
    \item 0
    \item 1
    \item Strictly greater than 1 but finite
    \item Infinite
\end{enumerate}
%
\solution
The nullspace is given by 
\begin{align}
	\myvec{1 & 1 & 1 & 0 \\ 1 & 1 & 0 & 1\\ 0 & 0 & 0 & 0\\0 & 0 & 0 & 0}\myvec{x\\y\\z\\w} = \myvec{0 \\ 0 \\ 0 \\ 0}
\end{align}	
Row reducing the above matrix we get,
\begin{align}
	\myvec{1 & 1 & 1 & 0 \\ 1 & 1 & 0 & 1\\ 0 & 0 & 0 & 0\\0 & 0 & 0 & 0}
	\xleftrightarrow[R_2 \leftarrow R_2 \times -1]{R_2 \leftarrow R_2 - R_1}
	\myvec{1 & 1 & 1 & 0 \\ 0 & 0 & 1 & -1\\ 0 & 0 & 0 & 0\\0 & 0 & 0 & 0}\\
	\xleftrightarrow{R_1 \leftarrow R_1- R_2}
	\myvec{1 & 1 & 0 & 1 \\ 0 & 0 & 1 & -1\\ 0 & 0 & 0 & 0\\0 & 0 & 0 & 0} \label{eq:solutions/2017/dec/27/eq:rref}
\end{align}
See Table \ref{eq:solutions/2017/dec/27/tab}

\begin{table*}[!ht]
	\begin{tabular}{|m{4.5cm}|l|}
		\hline
		&\\
		dim(C$(\vec{A})) = 1$ 
		& \textbf{False}. Because the number of pivot variables are 2 as obtained in \eqref{eq:solutions/2017/dec/27/eq:rref}\\
		&\\
		\hline
		&\\
		dim(C$(\vec{A})) = 2$
		& \textbf{True}. Since the number of pivot variables are 2, the rank of $\vec{A}$ is 2.\\
		&$\therefore dim(C(\vec{A})) = 2 \quad [\because dim(C(\vec{A})) = rank(\vec{A})]$ \\
		&\\
		\hline
		&\\
	     rank$(\vec{A}) = 1$
		& \textbf{False}. Because the rank$(\vec{A}) = 2$, as the number of pivot variables are 2\\
		&\\
		\hline
		&\\
		$\vec{S}$ = $\cbrak{(1, 1, 1, 0), (1, 1, 0, 1)}$ is a basis of $N(\vec{A})$
		& \textbf{False}. \\
		& Let, \\
		&  $\vec{u} = \myvec{1\\1\\1\\0}, \vec{v} = \myvec{1\\1\\0\\1}$\\ 
		&Consider, \\
		&$\myvec{1 & 1 & 1 & 0 \\ 1 & 1 & 0 & 1\\ 0 & 0 & 0 & 0\\0 & 0 & 0 & 0}\myvec{1\\1\\1\\0} = \myvec{3\\2\\0\\0} \not = \myvec{0\\0\\0\\0}$\\
		& Similarly,\\
		&$\myvec{1 & 1 & 1 & 0 \\ 1 & 1 & 0 & 1\\ 0 & 0 & 0 & 0\\0 & 0 & 0 & 0}\myvec{1\\1\\0\\1} = \myvec{2\\3\\0\\0} \not = \myvec{0\\0\\0\\0}$ \\
		&Hence, the given vectors do not form the basis.\\
		\hline
	\end{tabular}
\caption{}
\label{eq:solutions/2017/dec/27/tab}
\end{table*}

\item Let
\begin{align}
\vec{A}=\myvec{1&0&2\\1&-2&0\\0&0&-3}
\end{align}
and $\vec{I}$ be the $3\times3$ identity matrix. If 
\begin{align}
6\vec{A}^{-1}=a\vec{A}^2+b\vec{A}+c\vec{I} \label{eq:solutions/2017/june/31/eq:1}
\end{align} for $a,b,c \in \mathbb{R}$ then (a,b,c) equals
\begin{enumerate}
\item (1,2,1)\\
\item (1,-1,2)\\
\item (4,1,1)\\
\item (1,4,1)
\end{enumerate}
\solution
The nullspace is given by 
\begin{align}
	\myvec{1 & 1 & 1 & 0 \\ 1 & 1 & 0 & 1\\ 0 & 0 & 0 & 0\\0 & 0 & 0 & 0}\myvec{x\\y\\z\\w} = \myvec{0 \\ 0 \\ 0 \\ 0}
\end{align}	
Row reducing the above matrix we get,
\begin{align}
	\myvec{1 & 1 & 1 & 0 \\ 1 & 1 & 0 & 1\\ 0 & 0 & 0 & 0\\0 & 0 & 0 & 0}
	\xleftrightarrow[R_2 \leftarrow R_2 \times -1]{R_2 \leftarrow R_2 - R_1}
	\myvec{1 & 1 & 1 & 0 \\ 0 & 0 & 1 & -1\\ 0 & 0 & 0 & 0\\0 & 0 & 0 & 0}\\
	\xleftrightarrow{R_1 \leftarrow R_1- R_2}
	\myvec{1 & 1 & 0 & 1 \\ 0 & 0 & 1 & -1\\ 0 & 0 & 0 & 0\\0 & 0 & 0 & 0} \label{eq:solutions/2017/dec/27/eq:rref}
\end{align}
See Table \ref{eq:solutions/2017/dec/27/tab}

\begin{table*}[!ht]
	\begin{tabular}{|m{4.5cm}|l|}
		\hline
		&\\
		dim(C$(\vec{A})) = 1$ 
		& \textbf{False}. Because the number of pivot variables are 2 as obtained in \eqref{eq:solutions/2017/dec/27/eq:rref}\\
		&\\
		\hline
		&\\
		dim(C$(\vec{A})) = 2$
		& \textbf{True}. Since the number of pivot variables are 2, the rank of $\vec{A}$ is 2.\\
		&$\therefore dim(C(\vec{A})) = 2 \quad [\because dim(C(\vec{A})) = rank(\vec{A})]$ \\
		&\\
		\hline
		&\\
	     rank$(\vec{A}) = 1$
		& \textbf{False}. Because the rank$(\vec{A}) = 2$, as the number of pivot variables are 2\\
		&\\
		\hline
		&\\
		$\vec{S}$ = $\cbrak{(1, 1, 1, 0), (1, 1, 0, 1)}$ is a basis of $N(\vec{A})$
		& \textbf{False}. \\
		& Let, \\
		&  $\vec{u} = \myvec{1\\1\\1\\0}, \vec{v} = \myvec{1\\1\\0\\1}$\\ 
		&Consider, \\
		&$\myvec{1 & 1 & 1 & 0 \\ 1 & 1 & 0 & 1\\ 0 & 0 & 0 & 0\\0 & 0 & 0 & 0}\myvec{1\\1\\1\\0} = \myvec{3\\2\\0\\0} \not = \myvec{0\\0\\0\\0}$\\
		& Similarly,\\
		&$\myvec{1 & 1 & 1 & 0 \\ 1 & 1 & 0 & 1\\ 0 & 0 & 0 & 0\\0 & 0 & 0 & 0}\myvec{1\\1\\0\\1} = \myvec{2\\3\\0\\0} \not = \myvec{0\\0\\0\\0}$ \\
		&Hence, the given vectors do not form the basis.\\
		\hline
	\end{tabular}
\caption{}
\label{eq:solutions/2017/dec/27/tab}
\end{table*}


%
\item Find the Eigenvalues of the matrix,
\begin{align}
\vec{A} = \myvec{1 & 1 & 2 \\ 1 & -2 & 5 \\ 2 & 5 & -3 }\label{eq:solutions/2017/june/32/eq:1}
\end{align}
\begin{enumerate}
\item -4, 3, -3
\item 4, 3, 1
\item 4, -4$\pm\sqrt{13}$
\item 4, -2$\pm\sqrt{7}$
\end{enumerate}
%
%
\solution
The nullspace is given by 
\begin{align}
	\myvec{1 & 1 & 1 & 0 \\ 1 & 1 & 0 & 1\\ 0 & 0 & 0 & 0\\0 & 0 & 0 & 0}\myvec{x\\y\\z\\w} = \myvec{0 \\ 0 \\ 0 \\ 0}
\end{align}	
Row reducing the above matrix we get,
\begin{align}
	\myvec{1 & 1 & 1 & 0 \\ 1 & 1 & 0 & 1\\ 0 & 0 & 0 & 0\\0 & 0 & 0 & 0}
	\xleftrightarrow[R_2 \leftarrow R_2 \times -1]{R_2 \leftarrow R_2 - R_1}
	\myvec{1 & 1 & 1 & 0 \\ 0 & 0 & 1 & -1\\ 0 & 0 & 0 & 0\\0 & 0 & 0 & 0}\\
	\xleftrightarrow{R_1 \leftarrow R_1- R_2}
	\myvec{1 & 1 & 0 & 1 \\ 0 & 0 & 1 & -1\\ 0 & 0 & 0 & 0\\0 & 0 & 0 & 0} \label{eq:solutions/2017/dec/27/eq:rref}
\end{align}
See Table \ref{eq:solutions/2017/dec/27/tab}

\begin{table*}[!ht]
	\begin{tabular}{|m{4.5cm}|l|}
		\hline
		&\\
		dim(C$(\vec{A})) = 1$ 
		& \textbf{False}. Because the number of pivot variables are 2 as obtained in \eqref{eq:solutions/2017/dec/27/eq:rref}\\
		&\\
		\hline
		&\\
		dim(C$(\vec{A})) = 2$
		& \textbf{True}. Since the number of pivot variables are 2, the rank of $\vec{A}$ is 2.\\
		&$\therefore dim(C(\vec{A})) = 2 \quad [\because dim(C(\vec{A})) = rank(\vec{A})]$ \\
		&\\
		\hline
		&\\
	     rank$(\vec{A}) = 1$
		& \textbf{False}. Because the rank$(\vec{A}) = 2$, as the number of pivot variables are 2\\
		&\\
		\hline
		&\\
		$\vec{S}$ = $\cbrak{(1, 1, 1, 0), (1, 1, 0, 1)}$ is a basis of $N(\vec{A})$
		& \textbf{False}. \\
		& Let, \\
		&  $\vec{u} = \myvec{1\\1\\1\\0}, \vec{v} = \myvec{1\\1\\0\\1}$\\ 
		&Consider, \\
		&$\myvec{1 & 1 & 1 & 0 \\ 1 & 1 & 0 & 1\\ 0 & 0 & 0 & 0\\0 & 0 & 0 & 0}\myvec{1\\1\\1\\0} = \myvec{3\\2\\0\\0} \not = \myvec{0\\0\\0\\0}$\\
		& Similarly,\\
		&$\myvec{1 & 1 & 1 & 0 \\ 1 & 1 & 0 & 1\\ 0 & 0 & 0 & 0\\0 & 0 & 0 & 0}\myvec{1\\1\\0\\1} = \myvec{2\\3\\0\\0} \not = \myvec{0\\0\\0\\0}$ \\
		&Hence, the given vectors do not form the basis.\\
		\hline
	\end{tabular}
\caption{}
\label{eq:solutions/2017/dec/27/tab}
\end{table*}

\item Consider the vector space V of real polynomials of degree less than or equal to n. Fix distinct real numbers $a_0, a_1, \cdots, a_k$. For $p \in V$
\begin{align}
    max\cbrak{\abs{p(a_j)}: 0\leq j \leq k}
\end{align}
defines a norm on V
\begin{enumerate}
    \item only if $k<n$
    \item only if $k\ge n$
    \item if $ k+1\leq n$ 
    \item if $k \ge n+1$
\end{enumerate}
%
\solution
The nullspace is given by 
\begin{align}
	\myvec{1 & 1 & 1 & 0 \\ 1 & 1 & 0 & 1\\ 0 & 0 & 0 & 0\\0 & 0 & 0 & 0}\myvec{x\\y\\z\\w} = \myvec{0 \\ 0 \\ 0 \\ 0}
\end{align}	
Row reducing the above matrix we get,
\begin{align}
	\myvec{1 & 1 & 1 & 0 \\ 1 & 1 & 0 & 1\\ 0 & 0 & 0 & 0\\0 & 0 & 0 & 0}
	\xleftrightarrow[R_2 \leftarrow R_2 \times -1]{R_2 \leftarrow R_2 - R_1}
	\myvec{1 & 1 & 1 & 0 \\ 0 & 0 & 1 & -1\\ 0 & 0 & 0 & 0\\0 & 0 & 0 & 0}\\
	\xleftrightarrow{R_1 \leftarrow R_1- R_2}
	\myvec{1 & 1 & 0 & 1 \\ 0 & 0 & 1 & -1\\ 0 & 0 & 0 & 0\\0 & 0 & 0 & 0} \label{eq:solutions/2017/dec/27/eq:rref}
\end{align}
See Table \ref{eq:solutions/2017/dec/27/tab}

\begin{table*}[!ht]
	\begin{tabular}{|m{4.5cm}|l|}
		\hline
		&\\
		dim(C$(\vec{A})) = 1$ 
		& \textbf{False}. Because the number of pivot variables are 2 as obtained in \eqref{eq:solutions/2017/dec/27/eq:rref}\\
		&\\
		\hline
		&\\
		dim(C$(\vec{A})) = 2$
		& \textbf{True}. Since the number of pivot variables are 2, the rank of $\vec{A}$ is 2.\\
		&$\therefore dim(C(\vec{A})) = 2 \quad [\because dim(C(\vec{A})) = rank(\vec{A})]$ \\
		&\\
		\hline
		&\\
	     rank$(\vec{A}) = 1$
		& \textbf{False}. Because the rank$(\vec{A}) = 2$, as the number of pivot variables are 2\\
		&\\
		\hline
		&\\
		$\vec{S}$ = $\cbrak{(1, 1, 1, 0), (1, 1, 0, 1)}$ is a basis of $N(\vec{A})$
		& \textbf{False}. \\
		& Let, \\
		&  $\vec{u} = \myvec{1\\1\\1\\0}, \vec{v} = \myvec{1\\1\\0\\1}$\\ 
		&Consider, \\
		&$\myvec{1 & 1 & 1 & 0 \\ 1 & 1 & 0 & 1\\ 0 & 0 & 0 & 0\\0 & 0 & 0 & 0}\myvec{1\\1\\1\\0} = \myvec{3\\2\\0\\0} \not = \myvec{0\\0\\0\\0}$\\
		& Similarly,\\
		&$\myvec{1 & 1 & 1 & 0 \\ 1 & 1 & 0 & 1\\ 0 & 0 & 0 & 0\\0 & 0 & 0 & 0}\myvec{1\\1\\0\\1} = \myvec{2\\3\\0\\0} \not = \myvec{0\\0\\0\\0}$ \\
		&Hence, the given vectors do not form the basis.\\
		\hline
	\end{tabular}
\caption{}
\label{eq:solutions/2017/dec/27/tab}
\end{table*}

\item Let \textbf{V} be the vector space of polynomials of degree at most 3 in a variable x with coefficients in $\mathbb{R}$. Let \textbf{T}=d/dx be the linear transformation of \textbf{V} to itself given by differentiation.\\

Which of the following are correct?\\
\begin{enumerate}
\item $\vec{T}$ is invertible
\item 0 is an eigenvalue of $\vec{T}$
\item There is a basis with respect to which the matrix of \textbf{T} is nilpotent.
\item The matrix of \textbf{T} with respect to the basis \myvec{1,1+x,1+x+x^2,1+x+x^2+x^3} is diagonal.
\end{enumerate}
\solution
The nullspace is given by 
\begin{align}
	\myvec{1 & 1 & 1 & 0 \\ 1 & 1 & 0 & 1\\ 0 & 0 & 0 & 0\\0 & 0 & 0 & 0}\myvec{x\\y\\z\\w} = \myvec{0 \\ 0 \\ 0 \\ 0}
\end{align}	
Row reducing the above matrix we get,
\begin{align}
	\myvec{1 & 1 & 1 & 0 \\ 1 & 1 & 0 & 1\\ 0 & 0 & 0 & 0\\0 & 0 & 0 & 0}
	\xleftrightarrow[R_2 \leftarrow R_2 \times -1]{R_2 \leftarrow R_2 - R_1}
	\myvec{1 & 1 & 1 & 0 \\ 0 & 0 & 1 & -1\\ 0 & 0 & 0 & 0\\0 & 0 & 0 & 0}\\
	\xleftrightarrow{R_1 \leftarrow R_1- R_2}
	\myvec{1 & 1 & 0 & 1 \\ 0 & 0 & 1 & -1\\ 0 & 0 & 0 & 0\\0 & 0 & 0 & 0} \label{eq:solutions/2017/dec/27/eq:rref}
\end{align}
See Table \ref{eq:solutions/2017/dec/27/tab}

\begin{table*}[!ht]
	\begin{tabular}{|m{4.5cm}|l|}
		\hline
		&\\
		dim(C$(\vec{A})) = 1$ 
		& \textbf{False}. Because the number of pivot variables are 2 as obtained in \eqref{eq:solutions/2017/dec/27/eq:rref}\\
		&\\
		\hline
		&\\
		dim(C$(\vec{A})) = 2$
		& \textbf{True}. Since the number of pivot variables are 2, the rank of $\vec{A}$ is 2.\\
		&$\therefore dim(C(\vec{A})) = 2 \quad [\because dim(C(\vec{A})) = rank(\vec{A})]$ \\
		&\\
		\hline
		&\\
	     rank$(\vec{A}) = 1$
		& \textbf{False}. Because the rank$(\vec{A}) = 2$, as the number of pivot variables are 2\\
		&\\
		\hline
		&\\
		$\vec{S}$ = $\cbrak{(1, 1, 1, 0), (1, 1, 0, 1)}$ is a basis of $N(\vec{A})$
		& \textbf{False}. \\
		& Let, \\
		&  $\vec{u} = \myvec{1\\1\\1\\0}, \vec{v} = \myvec{1\\1\\0\\1}$\\ 
		&Consider, \\
		&$\myvec{1 & 1 & 1 & 0 \\ 1 & 1 & 0 & 1\\ 0 & 0 & 0 & 0\\0 & 0 & 0 & 0}\myvec{1\\1\\1\\0} = \myvec{3\\2\\0\\0} \not = \myvec{0\\0\\0\\0}$\\
		& Similarly,\\
		&$\myvec{1 & 1 & 1 & 0 \\ 1 & 1 & 0 & 1\\ 0 & 0 & 0 & 0\\0 & 0 & 0 & 0}\myvec{1\\1\\0\\1} = \myvec{2\\3\\0\\0} \not = \myvec{0\\0\\0\\0}$ \\
		&Hence, the given vectors do not form the basis.\\
		\hline
	\end{tabular}
\caption{}
\label{eq:solutions/2017/dec/27/tab}
\end{table*}

\item Let $m,n,r$ be natural numbers. Let $A$ be an $m\times n$ matrix with real entries such that $(AA^t)^r = I$, where $I$ is the $m \times m$ is identity matrix and $A^t$ is the transpose of the matrix $A$. We can conclude that\\
\begin{enumerate}
\item
$m = n$\\
\item
$AA^t$ is invertible\\
\item
$A^tA$ is invertible\\
\item
if $m=n$, then $A$ is invertible
\end{enumerate}
%
\solution
The nullspace is given by 
\begin{align}
	\myvec{1 & 1 & 1 & 0 \\ 1 & 1 & 0 & 1\\ 0 & 0 & 0 & 0\\0 & 0 & 0 & 0}\myvec{x\\y\\z\\w} = \myvec{0 \\ 0 \\ 0 \\ 0}
\end{align}	
Row reducing the above matrix we get,
\begin{align}
	\myvec{1 & 1 & 1 & 0 \\ 1 & 1 & 0 & 1\\ 0 & 0 & 0 & 0\\0 & 0 & 0 & 0}
	\xleftrightarrow[R_2 \leftarrow R_2 \times -1]{R_2 \leftarrow R_2 - R_1}
	\myvec{1 & 1 & 1 & 0 \\ 0 & 0 & 1 & -1\\ 0 & 0 & 0 & 0\\0 & 0 & 0 & 0}\\
	\xleftrightarrow{R_1 \leftarrow R_1- R_2}
	\myvec{1 & 1 & 0 & 1 \\ 0 & 0 & 1 & -1\\ 0 & 0 & 0 & 0\\0 & 0 & 0 & 0} \label{eq:solutions/2017/dec/27/eq:rref}
\end{align}
See Table \ref{eq:solutions/2017/dec/27/tab}

\begin{table*}[!ht]
	\begin{tabular}{|m{4.5cm}|l|}
		\hline
		&\\
		dim(C$(\vec{A})) = 1$ 
		& \textbf{False}. Because the number of pivot variables are 2 as obtained in \eqref{eq:solutions/2017/dec/27/eq:rref}\\
		&\\
		\hline
		&\\
		dim(C$(\vec{A})) = 2$
		& \textbf{True}. Since the number of pivot variables are 2, the rank of $\vec{A}$ is 2.\\
		&$\therefore dim(C(\vec{A})) = 2 \quad [\because dim(C(\vec{A})) = rank(\vec{A})]$ \\
		&\\
		\hline
		&\\
	     rank$(\vec{A}) = 1$
		& \textbf{False}. Because the rank$(\vec{A}) = 2$, as the number of pivot variables are 2\\
		&\\
		\hline
		&\\
		$\vec{S}$ = $\cbrak{(1, 1, 1, 0), (1, 1, 0, 1)}$ is a basis of $N(\vec{A})$
		& \textbf{False}. \\
		& Let, \\
		&  $\vec{u} = \myvec{1\\1\\1\\0}, \vec{v} = \myvec{1\\1\\0\\1}$\\ 
		&Consider, \\
		&$\myvec{1 & 1 & 1 & 0 \\ 1 & 1 & 0 & 1\\ 0 & 0 & 0 & 0\\0 & 0 & 0 & 0}\myvec{1\\1\\1\\0} = \myvec{3\\2\\0\\0} \not = \myvec{0\\0\\0\\0}$\\
		& Similarly,\\
		&$\myvec{1 & 1 & 1 & 0 \\ 1 & 1 & 0 & 1\\ 0 & 0 & 0 & 0\\0 & 0 & 0 & 0}\myvec{1\\1\\0\\1} = \myvec{2\\3\\0\\0} \not = \myvec{0\\0\\0\\0}$ \\
		&Hence, the given vectors do not form the basis.\\
		\hline
	\end{tabular}
\caption{}
\label{eq:solutions/2017/dec/27/tab}
\end{table*}

\item Let $\vec{A}$ be a $n\times n$ real matrix with $\vec{A}^2=\vec{A}$. Then
\begin{enumerate}
	\item the eigenvalues of $\vec{A}$ are either 0 or 1
	\item $\vec{A}$ is a diagonal matrix with diagonal entries 0 or 1
	\item $rank(\vec{A})=trace(\vec{A})$
	\item if $rank(\vec{I-A})=trace(\vec{I-A})$
\end{enumerate}
%
%
\solution
The nullspace is given by 
\begin{align}
	\myvec{1 & 1 & 1 & 0 \\ 1 & 1 & 0 & 1\\ 0 & 0 & 0 & 0\\0 & 0 & 0 & 0}\myvec{x\\y\\z\\w} = \myvec{0 \\ 0 \\ 0 \\ 0}
\end{align}	
Row reducing the above matrix we get,
\begin{align}
	\myvec{1 & 1 & 1 & 0 \\ 1 & 1 & 0 & 1\\ 0 & 0 & 0 & 0\\0 & 0 & 0 & 0}
	\xleftrightarrow[R_2 \leftarrow R_2 \times -1]{R_2 \leftarrow R_2 - R_1}
	\myvec{1 & 1 & 1 & 0 \\ 0 & 0 & 1 & -1\\ 0 & 0 & 0 & 0\\0 & 0 & 0 & 0}\\
	\xleftrightarrow{R_1 \leftarrow R_1- R_2}
	\myvec{1 & 1 & 0 & 1 \\ 0 & 0 & 1 & -1\\ 0 & 0 & 0 & 0\\0 & 0 & 0 & 0} \label{eq:solutions/2017/dec/27/eq:rref}
\end{align}
See Table \ref{eq:solutions/2017/dec/27/tab}

\begin{table*}[!ht]
	\begin{tabular}{|m{4.5cm}|l|}
		\hline
		&\\
		dim(C$(\vec{A})) = 1$ 
		& \textbf{False}. Because the number of pivot variables are 2 as obtained in \eqref{eq:solutions/2017/dec/27/eq:rref}\\
		&\\
		\hline
		&\\
		dim(C$(\vec{A})) = 2$
		& \textbf{True}. Since the number of pivot variables are 2, the rank of $\vec{A}$ is 2.\\
		&$\therefore dim(C(\vec{A})) = 2 \quad [\because dim(C(\vec{A})) = rank(\vec{A})]$ \\
		&\\
		\hline
		&\\
	     rank$(\vec{A}) = 1$
		& \textbf{False}. Because the rank$(\vec{A}) = 2$, as the number of pivot variables are 2\\
		&\\
		\hline
		&\\
		$\vec{S}$ = $\cbrak{(1, 1, 1, 0), (1, 1, 0, 1)}$ is a basis of $N(\vec{A})$
		& \textbf{False}. \\
		& Let, \\
		&  $\vec{u} = \myvec{1\\1\\1\\0}, \vec{v} = \myvec{1\\1\\0\\1}$\\ 
		&Consider, \\
		&$\myvec{1 & 1 & 1 & 0 \\ 1 & 1 & 0 & 1\\ 0 & 0 & 0 & 0\\0 & 0 & 0 & 0}\myvec{1\\1\\1\\0} = \myvec{3\\2\\0\\0} \not = \myvec{0\\0\\0\\0}$\\
		& Similarly,\\
		&$\myvec{1 & 1 & 1 & 0 \\ 1 & 1 & 0 & 1\\ 0 & 0 & 0 & 0\\0 & 0 & 0 & 0}\myvec{1\\1\\0\\1} = \myvec{2\\3\\0\\0} \not = \myvec{0\\0\\0\\0}$ \\
		&Hence, the given vectors do not form the basis.\\
		\hline
	\end{tabular}
\caption{}
\label{eq:solutions/2017/dec/27/tab}
\end{table*}

\item For any $n\times n$ matrix $B$, let $N(B)=\{X\in \mathbb{R}^n:BX=0\}$ be the null space of $B$. Let $A$ be a $4\times 4$ matrix with $dim(N(A-4I))=2, dim(N(A-2I))=1$ and $rank(A)=3$
Which of the following are true?
\begin{enumerate}
\item 0,2 and 4 are eigenvalues of A
\item determinant(A)=0
\item A is not diagonalizable
\item trace(A)=8
\end{enumerate}
%
\solution
The nullspace is given by 
\begin{align}
	\myvec{1 & 1 & 1 & 0 \\ 1 & 1 & 0 & 1\\ 0 & 0 & 0 & 0\\0 & 0 & 0 & 0}\myvec{x\\y\\z\\w} = \myvec{0 \\ 0 \\ 0 \\ 0}
\end{align}	
Row reducing the above matrix we get,
\begin{align}
	\myvec{1 & 1 & 1 & 0 \\ 1 & 1 & 0 & 1\\ 0 & 0 & 0 & 0\\0 & 0 & 0 & 0}
	\xleftrightarrow[R_2 \leftarrow R_2 \times -1]{R_2 \leftarrow R_2 - R_1}
	\myvec{1 & 1 & 1 & 0 \\ 0 & 0 & 1 & -1\\ 0 & 0 & 0 & 0\\0 & 0 & 0 & 0}\\
	\xleftrightarrow{R_1 \leftarrow R_1- R_2}
	\myvec{1 & 1 & 0 & 1 \\ 0 & 0 & 1 & -1\\ 0 & 0 & 0 & 0\\0 & 0 & 0 & 0} \label{eq:solutions/2017/dec/27/eq:rref}
\end{align}
See Table \ref{eq:solutions/2017/dec/27/tab}

\begin{table*}[!ht]
	\begin{tabular}{|m{4.5cm}|l|}
		\hline
		&\\
		dim(C$(\vec{A})) = 1$ 
		& \textbf{False}. Because the number of pivot variables are 2 as obtained in \eqref{eq:solutions/2017/dec/27/eq:rref}\\
		&\\
		\hline
		&\\
		dim(C$(\vec{A})) = 2$
		& \textbf{True}. Since the number of pivot variables are 2, the rank of $\vec{A}$ is 2.\\
		&$\therefore dim(C(\vec{A})) = 2 \quad [\because dim(C(\vec{A})) = rank(\vec{A})]$ \\
		&\\
		\hline
		&\\
	     rank$(\vec{A}) = 1$
		& \textbf{False}. Because the rank$(\vec{A}) = 2$, as the number of pivot variables are 2\\
		&\\
		\hline
		&\\
		$\vec{S}$ = $\cbrak{(1, 1, 1, 0), (1, 1, 0, 1)}$ is a basis of $N(\vec{A})$
		& \textbf{False}. \\
		& Let, \\
		&  $\vec{u} = \myvec{1\\1\\1\\0}, \vec{v} = \myvec{1\\1\\0\\1}$\\ 
		&Consider, \\
		&$\myvec{1 & 1 & 1 & 0 \\ 1 & 1 & 0 & 1\\ 0 & 0 & 0 & 0\\0 & 0 & 0 & 0}\myvec{1\\1\\1\\0} = \myvec{3\\2\\0\\0} \not = \myvec{0\\0\\0\\0}$\\
		& Similarly,\\
		&$\myvec{1 & 1 & 1 & 0 \\ 1 & 1 & 0 & 1\\ 0 & 0 & 0 & 0\\0 & 0 & 0 & 0}\myvec{1\\1\\0\\1} = \myvec{2\\3\\0\\0} \not = \myvec{0\\0\\0\\0}$ \\
		&Hence, the given vectors do not form the basis.\\
		\hline
	\end{tabular}
\caption{}
\label{eq:solutions/2017/dec/27/tab}
\end{table*}

\item Which of the following 3x3 matrices are diagonizable over $\mathbb{R}?$\\
\begin{enumerate}
    \item \myvec{1&2&3\\0&4&5\\0&0&6}
    \item \myvec{0&1&0\\-1&0&0\\0&0&1}
    \item \myvec{1&2&3\\2&1&4\\3&4&1}
    \item \myvec{0&1&2\\0&0&1\\0&0&0}
\end{enumerate}
%
\solution
The nullspace is given by 
\begin{align}
	\myvec{1 & 1 & 1 & 0 \\ 1 & 1 & 0 & 1\\ 0 & 0 & 0 & 0\\0 & 0 & 0 & 0}\myvec{x\\y\\z\\w} = \myvec{0 \\ 0 \\ 0 \\ 0}
\end{align}	
Row reducing the above matrix we get,
\begin{align}
	\myvec{1 & 1 & 1 & 0 \\ 1 & 1 & 0 & 1\\ 0 & 0 & 0 & 0\\0 & 0 & 0 & 0}
	\xleftrightarrow[R_2 \leftarrow R_2 \times -1]{R_2 \leftarrow R_2 - R_1}
	\myvec{1 & 1 & 1 & 0 \\ 0 & 0 & 1 & -1\\ 0 & 0 & 0 & 0\\0 & 0 & 0 & 0}\\
	\xleftrightarrow{R_1 \leftarrow R_1- R_2}
	\myvec{1 & 1 & 0 & 1 \\ 0 & 0 & 1 & -1\\ 0 & 0 & 0 & 0\\0 & 0 & 0 & 0} \label{eq:solutions/2017/dec/27/eq:rref}
\end{align}
See Table \ref{eq:solutions/2017/dec/27/tab}

\begin{table*}[!ht]
	\begin{tabular}{|m{4.5cm}|l|}
		\hline
		&\\
		dim(C$(\vec{A})) = 1$ 
		& \textbf{False}. Because the number of pivot variables are 2 as obtained in \eqref{eq:solutions/2017/dec/27/eq:rref}\\
		&\\
		\hline
		&\\
		dim(C$(\vec{A})) = 2$
		& \textbf{True}. Since the number of pivot variables are 2, the rank of $\vec{A}$ is 2.\\
		&$\therefore dim(C(\vec{A})) = 2 \quad [\because dim(C(\vec{A})) = rank(\vec{A})]$ \\
		&\\
		\hline
		&\\
	     rank$(\vec{A}) = 1$
		& \textbf{False}. Because the rank$(\vec{A}) = 2$, as the number of pivot variables are 2\\
		&\\
		\hline
		&\\
		$\vec{S}$ = $\cbrak{(1, 1, 1, 0), (1, 1, 0, 1)}$ is a basis of $N(\vec{A})$
		& \textbf{False}. \\
		& Let, \\
		&  $\vec{u} = \myvec{1\\1\\1\\0}, \vec{v} = \myvec{1\\1\\0\\1}$\\ 
		&Consider, \\
		&$\myvec{1 & 1 & 1 & 0 \\ 1 & 1 & 0 & 1\\ 0 & 0 & 0 & 0\\0 & 0 & 0 & 0}\myvec{1\\1\\1\\0} = \myvec{3\\2\\0\\0} \not = \myvec{0\\0\\0\\0}$\\
		& Similarly,\\
		&$\myvec{1 & 1 & 1 & 0 \\ 1 & 1 & 0 & 1\\ 0 & 0 & 0 & 0\\0 & 0 & 0 & 0}\myvec{1\\1\\0\\1} = \myvec{2\\3\\0\\0} \not = \myvec{0\\0\\0\\0}$ \\
		&Hence, the given vectors do not form the basis.\\
		\hline
	\end{tabular}
\caption{}
\label{eq:solutions/2017/dec/27/tab}
\end{table*}

\twocolumn
\item Let $\vec{A} = \myvec{3 & 1 & 2 \\ 1 & 2 & 3 \\ 2 & 3 & 1  }$ and $\vec{Q(X) = X^TAX}$ for $\vec{X} \in \mathbb{R}^{3}$. Then
\begin{enumerate}
	\item $\vec{A}$ has exactly two positive eigen values.
	\item all the eigen values of $\vec{A}$ are positive.
	\item $\vec{Q(X)} \geq 0 $ $\forall$ $\vec{X}$ $\in$ $\mathbb{R}^3$
	\item $\vec{Q(X)} < 0 $ for some $\vec{X}$ $\in$ $\mathbb{R}^3$
\end{enumerate}
%
%
\solution
The nullspace is given by 
\begin{align}
	\myvec{1 & 1 & 1 & 0 \\ 1 & 1 & 0 & 1\\ 0 & 0 & 0 & 0\\0 & 0 & 0 & 0}\myvec{x\\y\\z\\w} = \myvec{0 \\ 0 \\ 0 \\ 0}
\end{align}	
Row reducing the above matrix we get,
\begin{align}
	\myvec{1 & 1 & 1 & 0 \\ 1 & 1 & 0 & 1\\ 0 & 0 & 0 & 0\\0 & 0 & 0 & 0}
	\xleftrightarrow[R_2 \leftarrow R_2 \times -1]{R_2 \leftarrow R_2 - R_1}
	\myvec{1 & 1 & 1 & 0 \\ 0 & 0 & 1 & -1\\ 0 & 0 & 0 & 0\\0 & 0 & 0 & 0}\\
	\xleftrightarrow{R_1 \leftarrow R_1- R_2}
	\myvec{1 & 1 & 0 & 1 \\ 0 & 0 & 1 & -1\\ 0 & 0 & 0 & 0\\0 & 0 & 0 & 0} \label{eq:solutions/2017/dec/27/eq:rref}
\end{align}
See Table \ref{eq:solutions/2017/dec/27/tab}

\begin{table*}[!ht]
	\begin{tabular}{|m{4.5cm}|l|}
		\hline
		&\\
		dim(C$(\vec{A})) = 1$ 
		& \textbf{False}. Because the number of pivot variables are 2 as obtained in \eqref{eq:solutions/2017/dec/27/eq:rref}\\
		&\\
		\hline
		&\\
		dim(C$(\vec{A})) = 2$
		& \textbf{True}. Since the number of pivot variables are 2, the rank of $\vec{A}$ is 2.\\
		&$\therefore dim(C(\vec{A})) = 2 \quad [\because dim(C(\vec{A})) = rank(\vec{A})]$ \\
		&\\
		\hline
		&\\
	     rank$(\vec{A}) = 1$
		& \textbf{False}. Because the rank$(\vec{A}) = 2$, as the number of pivot variables are 2\\
		&\\
		\hline
		&\\
		$\vec{S}$ = $\cbrak{(1, 1, 1, 0), (1, 1, 0, 1)}$ is a basis of $N(\vec{A})$
		& \textbf{False}. \\
		& Let, \\
		&  $\vec{u} = \myvec{1\\1\\1\\0}, \vec{v} = \myvec{1\\1\\0\\1}$\\ 
		&Consider, \\
		&$\myvec{1 & 1 & 1 & 0 \\ 1 & 1 & 0 & 1\\ 0 & 0 & 0 & 0\\0 & 0 & 0 & 0}\myvec{1\\1\\1\\0} = \myvec{3\\2\\0\\0} \not = \myvec{0\\0\\0\\0}$\\
		& Similarly,\\
		&$\myvec{1 & 1 & 1 & 0 \\ 1 & 1 & 0 & 1\\ 0 & 0 & 0 & 0\\0 & 0 & 0 & 0}\myvec{1\\1\\0\\1} = \myvec{2\\3\\0\\0} \not = \myvec{0\\0\\0\\0}$ \\
		&Hence, the given vectors do not form the basis.\\
		\hline
	\end{tabular}
\caption{}
\label{eq:solutions/2017/dec/27/tab}
\end{table*}

\item Consider the matrix
\begin{align}
A(x) = \myvec{1+x^2&7&11\\3x&2x&4\\8x&17&13} & ;x\in \vec{R}.
\end{align}
Then,
\begin{enumerate}
\item A(x) has eigenvalue 0 for some $x\in \vec{R}$.
\item 0 is not an eigenvalue of A(x) for any $x\in \vec{R}$.
\item A(x) has eigenvalue 0 $\forall x\in \vec{R}$.
\item A(x) is invertible $\forall x\in \vec{R}$.
\end{enumerate}
%
\solution
The nullspace is given by 
\begin{align}
	\myvec{1 & 1 & 1 & 0 \\ 1 & 1 & 0 & 1\\ 0 & 0 & 0 & 0\\0 & 0 & 0 & 0}\myvec{x\\y\\z\\w} = \myvec{0 \\ 0 \\ 0 \\ 0}
\end{align}	
Row reducing the above matrix we get,
\begin{align}
	\myvec{1 & 1 & 1 & 0 \\ 1 & 1 & 0 & 1\\ 0 & 0 & 0 & 0\\0 & 0 & 0 & 0}
	\xleftrightarrow[R_2 \leftarrow R_2 \times -1]{R_2 \leftarrow R_2 - R_1}
	\myvec{1 & 1 & 1 & 0 \\ 0 & 0 & 1 & -1\\ 0 & 0 & 0 & 0\\0 & 0 & 0 & 0}\\
	\xleftrightarrow{R_1 \leftarrow R_1- R_2}
	\myvec{1 & 1 & 0 & 1 \\ 0 & 0 & 1 & -1\\ 0 & 0 & 0 & 0\\0 & 0 & 0 & 0} \label{eq:solutions/2017/dec/27/eq:rref}
\end{align}
See Table \ref{eq:solutions/2017/dec/27/tab}

\begin{table*}[!ht]
	\begin{tabular}{|m{4.5cm}|l|}
		\hline
		&\\
		dim(C$(\vec{A})) = 1$ 
		& \textbf{False}. Because the number of pivot variables are 2 as obtained in \eqref{eq:solutions/2017/dec/27/eq:rref}\\
		&\\
		\hline
		&\\
		dim(C$(\vec{A})) = 2$
		& \textbf{True}. Since the number of pivot variables are 2, the rank of $\vec{A}$ is 2.\\
		&$\therefore dim(C(\vec{A})) = 2 \quad [\because dim(C(\vec{A})) = rank(\vec{A})]$ \\
		&\\
		\hline
		&\\
	     rank$(\vec{A}) = 1$
		& \textbf{False}. Because the rank$(\vec{A}) = 2$, as the number of pivot variables are 2\\
		&\\
		\hline
		&\\
		$\vec{S}$ = $\cbrak{(1, 1, 1, 0), (1, 1, 0, 1)}$ is a basis of $N(\vec{A})$
		& \textbf{False}. \\
		& Let, \\
		&  $\vec{u} = \myvec{1\\1\\1\\0}, \vec{v} = \myvec{1\\1\\0\\1}$\\ 
		&Consider, \\
		&$\myvec{1 & 1 & 1 & 0 \\ 1 & 1 & 0 & 1\\ 0 & 0 & 0 & 0\\0 & 0 & 0 & 0}\myvec{1\\1\\1\\0} = \myvec{3\\2\\0\\0} \not = \myvec{0\\0\\0\\0}$\\
		& Similarly,\\
		&$\myvec{1 & 1 & 1 & 0 \\ 1 & 1 & 0 & 1\\ 0 & 0 & 0 & 0\\0 & 0 & 0 & 0}\myvec{1\\1\\0\\1} = \myvec{2\\3\\0\\0} \not = \myvec{0\\0\\0\\0}$ \\
		&Hence, the given vectors do not form the basis.\\
		\hline
	\end{tabular}
\caption{}
\label{eq:solutions/2017/dec/27/tab}
\end{table*}

\end{enumerate}
