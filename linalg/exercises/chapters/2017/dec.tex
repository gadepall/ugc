\renewcommand{\theequation}{\theenumi}
\renewcommand{\thefigure}{\theenumi}
\renewcommand{\thetable}{\theenumi}
\begin{enumerate}[label=\thesection.\arabic*.,ref=\thesection.\theenumi]
\numberwithin{equation}{enumi}
\numberwithin{figure}{enumi}
\numberwithin{table}{enumi}

\item Let $\vec{A}$ be a real symmetric matrix and  $\vec{B}=\vec{I}+i\vec{A}$, where $i^2=-1$. Then choose the correct option.
\begin{enumerate}
	\item  $\vec{B}$ is invertible if and only if $\vec{A}$ is invertible.
	\item All Eigenvalues of $\vec{B}$ are necessarily real.
	\item $\vec{B}-\vec{I}$ is necessarily invertible.
	\item $\vec{B}$ is necessarily invertible.
\end{enumerate}
%
\solution
The nullspace is given by 
\begin{align}
	\myvec{1 & 1 & 1 & 0 \\ 1 & 1 & 0 & 1\\ 0 & 0 & 0 & 0\\0 & 0 & 0 & 0}\myvec{x\\y\\z\\w} = \myvec{0 \\ 0 \\ 0 \\ 0}
\end{align}	
Row reducing the above matrix we get,
\begin{align}
	\myvec{1 & 1 & 1 & 0 \\ 1 & 1 & 0 & 1\\ 0 & 0 & 0 & 0\\0 & 0 & 0 & 0}
	\xleftrightarrow[R_2 \leftarrow R_2 \times -1]{R_2 \leftarrow R_2 - R_1}
	\myvec{1 & 1 & 1 & 0 \\ 0 & 0 & 1 & -1\\ 0 & 0 & 0 & 0\\0 & 0 & 0 & 0}\\
	\xleftrightarrow{R_1 \leftarrow R_1- R_2}
	\myvec{1 & 1 & 0 & 1 \\ 0 & 0 & 1 & -1\\ 0 & 0 & 0 & 0\\0 & 0 & 0 & 0} \label{eq:solutions/2017/dec/27/eq:rref}
\end{align}
See Table \ref{eq:solutions/2017/dec/27/tab}

\begin{table*}[!ht]
	\begin{tabular}{|m{4.5cm}|l|}
		\hline
		&\\
		dim(C$(\vec{A})) = 1$ 
		& \textbf{False}. Because the number of pivot variables are 2 as obtained in \eqref{eq:solutions/2017/dec/27/eq:rref}\\
		&\\
		\hline
		&\\
		dim(C$(\vec{A})) = 2$
		& \textbf{True}. Since the number of pivot variables are 2, the rank of $\vec{A}$ is 2.\\
		&$\therefore dim(C(\vec{A})) = 2 \quad [\because dim(C(\vec{A})) = rank(\vec{A})]$ \\
		&\\
		\hline
		&\\
	     rank$(\vec{A}) = 1$
		& \textbf{False}. Because the rank$(\vec{A}) = 2$, as the number of pivot variables are 2\\
		&\\
		\hline
		&\\
		$\vec{S}$ = $\cbrak{(1, 1, 1, 0), (1, 1, 0, 1)}$ is a basis of $N(\vec{A})$
		& \textbf{False}. \\
		& Let, \\
		&  $\vec{u} = \myvec{1\\1\\1\\0}, \vec{v} = \myvec{1\\1\\0\\1}$\\ 
		&Consider, \\
		&$\myvec{1 & 1 & 1 & 0 \\ 1 & 1 & 0 & 1\\ 0 & 0 & 0 & 0\\0 & 0 & 0 & 0}\myvec{1\\1\\1\\0} = \myvec{3\\2\\0\\0} \not = \myvec{0\\0\\0\\0}$\\
		& Similarly,\\
		&$\myvec{1 & 1 & 1 & 0 \\ 1 & 1 & 0 & 1\\ 0 & 0 & 0 & 0\\0 & 0 & 0 & 0}\myvec{1\\1\\0\\1} = \myvec{2\\3\\0\\0} \not = \myvec{0\\0\\0\\0}$ \\
		&Hence, the given vectors do not form the basis.\\
		\hline
	\end{tabular}
\caption{}
\label{eq:solutions/2017/dec/27/tab}
\end{table*}

\twocolumn
\item Let $\vec{A}=\myvec{0&1\\-1&1}$. Then the smallest positive integer n such that $\vec{A}^n=\vec{I}$ is
%
\\
\solution
The nullspace is given by 
\begin{align}
	\myvec{1 & 1 & 1 & 0 \\ 1 & 1 & 0 & 1\\ 0 & 0 & 0 & 0\\0 & 0 & 0 & 0}\myvec{x\\y\\z\\w} = \myvec{0 \\ 0 \\ 0 \\ 0}
\end{align}	
Row reducing the above matrix we get,
\begin{align}
	\myvec{1 & 1 & 1 & 0 \\ 1 & 1 & 0 & 1\\ 0 & 0 & 0 & 0\\0 & 0 & 0 & 0}
	\xleftrightarrow[R_2 \leftarrow R_2 \times -1]{R_2 \leftarrow R_2 - R_1}
	\myvec{1 & 1 & 1 & 0 \\ 0 & 0 & 1 & -1\\ 0 & 0 & 0 & 0\\0 & 0 & 0 & 0}\\
	\xleftrightarrow{R_1 \leftarrow R_1- R_2}
	\myvec{1 & 1 & 0 & 1 \\ 0 & 0 & 1 & -1\\ 0 & 0 & 0 & 0\\0 & 0 & 0 & 0} \label{eq:solutions/2017/dec/27/eq:rref}
\end{align}
See Table \ref{eq:solutions/2017/dec/27/tab}

\begin{table*}[!ht]
	\begin{tabular}{|m{4.5cm}|l|}
		\hline
		&\\
		dim(C$(\vec{A})) = 1$ 
		& \textbf{False}. Because the number of pivot variables are 2 as obtained in \eqref{eq:solutions/2017/dec/27/eq:rref}\\
		&\\
		\hline
		&\\
		dim(C$(\vec{A})) = 2$
		& \textbf{True}. Since the number of pivot variables are 2, the rank of $\vec{A}$ is 2.\\
		&$\therefore dim(C(\vec{A})) = 2 \quad [\because dim(C(\vec{A})) = rank(\vec{A})]$ \\
		&\\
		\hline
		&\\
	     rank$(\vec{A}) = 1$
		& \textbf{False}. Because the rank$(\vec{A}) = 2$, as the number of pivot variables are 2\\
		&\\
		\hline
		&\\
		$\vec{S}$ = $\cbrak{(1, 1, 1, 0), (1, 1, 0, 1)}$ is a basis of $N(\vec{A})$
		& \textbf{False}. \\
		& Let, \\
		&  $\vec{u} = \myvec{1\\1\\1\\0}, \vec{v} = \myvec{1\\1\\0\\1}$\\ 
		&Consider, \\
		&$\myvec{1 & 1 & 1 & 0 \\ 1 & 1 & 0 & 1\\ 0 & 0 & 0 & 0\\0 & 0 & 0 & 0}\myvec{1\\1\\1\\0} = \myvec{3\\2\\0\\0} \not = \myvec{0\\0\\0\\0}$\\
		& Similarly,\\
		&$\myvec{1 & 1 & 1 & 0 \\ 1 & 1 & 0 & 1\\ 0 & 0 & 0 & 0\\0 & 0 & 0 & 0}\myvec{1\\1\\0\\1} = \myvec{2\\3\\0\\0} \not = \myvec{0\\0\\0\\0}$ \\
		&Hence, the given vectors do not form the basis.\\
		\hline
	\end{tabular}
\caption{}
\label{eq:solutions/2017/dec/27/tab}
\end{table*}

%
\item Let $\vec{A} = \myvec{1&-1&1\\1&1&1\\2&3&\alpha}$ and $\vec{b}= \myvec{1\\3\\\beta}$. Then the system $\vec{AX}=\vec{b}$ over the real numbers has\\
\begin{enumerate}
    \item No solution when $\beta \ne 7$
    \item Infinite number of solutions when $\alpha \ne 2$
    \item Infinite number of solutions when $\alpha = 2$ and $\beta \ne 7$
    \item A unique solution if $\alpha \ne 2$
\end{enumerate}
%
\solution
The nullspace is given by 
\begin{align}
	\myvec{1 & 1 & 1 & 0 \\ 1 & 1 & 0 & 1\\ 0 & 0 & 0 & 0\\0 & 0 & 0 & 0}\myvec{x\\y\\z\\w} = \myvec{0 \\ 0 \\ 0 \\ 0}
\end{align}	
Row reducing the above matrix we get,
\begin{align}
	\myvec{1 & 1 & 1 & 0 \\ 1 & 1 & 0 & 1\\ 0 & 0 & 0 & 0\\0 & 0 & 0 & 0}
	\xleftrightarrow[R_2 \leftarrow R_2 \times -1]{R_2 \leftarrow R_2 - R_1}
	\myvec{1 & 1 & 1 & 0 \\ 0 & 0 & 1 & -1\\ 0 & 0 & 0 & 0\\0 & 0 & 0 & 0}\\
	\xleftrightarrow{R_1 \leftarrow R_1- R_2}
	\myvec{1 & 1 & 0 & 1 \\ 0 & 0 & 1 & -1\\ 0 & 0 & 0 & 0\\0 & 0 & 0 & 0} \label{eq:solutions/2017/dec/27/eq:rref}
\end{align}
See Table \ref{eq:solutions/2017/dec/27/tab}

\begin{table*}[!ht]
	\begin{tabular}{|m{4.5cm}|l|}
		\hline
		&\\
		dim(C$(\vec{A})) = 1$ 
		& \textbf{False}. Because the number of pivot variables are 2 as obtained in \eqref{eq:solutions/2017/dec/27/eq:rref}\\
		&\\
		\hline
		&\\
		dim(C$(\vec{A})) = 2$
		& \textbf{True}. Since the number of pivot variables are 2, the rank of $\vec{A}$ is 2.\\
		&$\therefore dim(C(\vec{A})) = 2 \quad [\because dim(C(\vec{A})) = rank(\vec{A})]$ \\
		&\\
		\hline
		&\\
	     rank$(\vec{A}) = 1$
		& \textbf{False}. Because the rank$(\vec{A}) = 2$, as the number of pivot variables are 2\\
		&\\
		\hline
		&\\
		$\vec{S}$ = $\cbrak{(1, 1, 1, 0), (1, 1, 0, 1)}$ is a basis of $N(\vec{A})$
		& \textbf{False}. \\
		& Let, \\
		&  $\vec{u} = \myvec{1\\1\\1\\0}, \vec{v} = \myvec{1\\1\\0\\1}$\\ 
		&Consider, \\
		&$\myvec{1 & 1 & 1 & 0 \\ 1 & 1 & 0 & 1\\ 0 & 0 & 0 & 0\\0 & 0 & 0 & 0}\myvec{1\\1\\1\\0} = \myvec{3\\2\\0\\0} \not = \myvec{0\\0\\0\\0}$\\
		& Similarly,\\
		&$\myvec{1 & 1 & 1 & 0 \\ 1 & 1 & 0 & 1\\ 0 & 0 & 0 & 0\\0 & 0 & 0 & 0}\myvec{1\\1\\0\\1} = \myvec{2\\3\\0\\0} \not = \myvec{0\\0\\0\\0}$ \\
		&Hence, the given vectors do not form the basis.\\
		\hline
	\end{tabular}
\caption{}
\label{eq:solutions/2017/dec/27/tab}
\end{table*}

%
\item Consider a Markov chain $\{X_n \: | \: n \geq 0\}$ with state space $\{1,2,3\}$ and transition matrix
\begin{align}
    \vec{P} = \myvec{0 & \frac{1}{2} & \frac{1}{2} \\
    \frac{1}{2} & 0 & \frac{1}{2} \\
    \frac{1}{2} & \frac{1}{2} & 0} \nonumber
\end{align}
Then, $P(X_3 = 1 \: | \: X_0 = 1)$ equals
%
\\
\solution
The nullspace is given by 
\begin{align}
	\myvec{1 & 1 & 1 & 0 \\ 1 & 1 & 0 & 1\\ 0 & 0 & 0 & 0\\0 & 0 & 0 & 0}\myvec{x\\y\\z\\w} = \myvec{0 \\ 0 \\ 0 \\ 0}
\end{align}	
Row reducing the above matrix we get,
\begin{align}
	\myvec{1 & 1 & 1 & 0 \\ 1 & 1 & 0 & 1\\ 0 & 0 & 0 & 0\\0 & 0 & 0 & 0}
	\xleftrightarrow[R_2 \leftarrow R_2 \times -1]{R_2 \leftarrow R_2 - R_1}
	\myvec{1 & 1 & 1 & 0 \\ 0 & 0 & 1 & -1\\ 0 & 0 & 0 & 0\\0 & 0 & 0 & 0}\\
	\xleftrightarrow{R_1 \leftarrow R_1- R_2}
	\myvec{1 & 1 & 0 & 1 \\ 0 & 0 & 1 & -1\\ 0 & 0 & 0 & 0\\0 & 0 & 0 & 0} \label{eq:solutions/2017/dec/27/eq:rref}
\end{align}
See Table \ref{eq:solutions/2017/dec/27/tab}

\begin{table*}[!ht]
	\begin{tabular}{|m{4.5cm}|l|}
		\hline
		&\\
		dim(C$(\vec{A})) = 1$ 
		& \textbf{False}. Because the number of pivot variables are 2 as obtained in \eqref{eq:solutions/2017/dec/27/eq:rref}\\
		&\\
		\hline
		&\\
		dim(C$(\vec{A})) = 2$
		& \textbf{True}. Since the number of pivot variables are 2, the rank of $\vec{A}$ is 2.\\
		&$\therefore dim(C(\vec{A})) = 2 \quad [\because dim(C(\vec{A})) = rank(\vec{A})]$ \\
		&\\
		\hline
		&\\
	     rank$(\vec{A}) = 1$
		& \textbf{False}. Because the rank$(\vec{A}) = 2$, as the number of pivot variables are 2\\
		&\\
		\hline
		&\\
		$\vec{S}$ = $\cbrak{(1, 1, 1, 0), (1, 1, 0, 1)}$ is a basis of $N(\vec{A})$
		& \textbf{False}. \\
		& Let, \\
		&  $\vec{u} = \myvec{1\\1\\1\\0}, \vec{v} = \myvec{1\\1\\0\\1}$\\ 
		&Consider, \\
		&$\myvec{1 & 1 & 1 & 0 \\ 1 & 1 & 0 & 1\\ 0 & 0 & 0 & 0\\0 & 0 & 0 & 0}\myvec{1\\1\\1\\0} = \myvec{3\\2\\0\\0} \not = \myvec{0\\0\\0\\0}$\\
		& Similarly,\\
		&$\myvec{1 & 1 & 1 & 0 \\ 1 & 1 & 0 & 1\\ 0 & 0 & 0 & 0\\0 & 0 & 0 & 0}\myvec{1\\1\\0\\1} = \myvec{2\\3\\0\\0} \not = \myvec{0\\0\\0\\0}$ \\
		&Hence, the given vectors do not form the basis.\\
		\hline
	\end{tabular}
\caption{}
\label{eq:solutions/2017/dec/27/tab}
\end{table*}

%
%\item 
%
%\\
%\solution
%The nullspace is given by 
\begin{align}
	\myvec{1 & 1 & 1 & 0 \\ 1 & 1 & 0 & 1\\ 0 & 0 & 0 & 0\\0 & 0 & 0 & 0}\myvec{x\\y\\z\\w} = \myvec{0 \\ 0 \\ 0 \\ 0}
\end{align}	
Row reducing the above matrix we get,
\begin{align}
	\myvec{1 & 1 & 1 & 0 \\ 1 & 1 & 0 & 1\\ 0 & 0 & 0 & 0\\0 & 0 & 0 & 0}
	\xleftrightarrow[R_2 \leftarrow R_2 \times -1]{R_2 \leftarrow R_2 - R_1}
	\myvec{1 & 1 & 1 & 0 \\ 0 & 0 & 1 & -1\\ 0 & 0 & 0 & 0\\0 & 0 & 0 & 0}\\
	\xleftrightarrow{R_1 \leftarrow R_1- R_2}
	\myvec{1 & 1 & 0 & 1 \\ 0 & 0 & 1 & -1\\ 0 & 0 & 0 & 0\\0 & 0 & 0 & 0} \label{eq:solutions/2017/dec/27/eq:rref}
\end{align}
See Table \ref{eq:solutions/2017/dec/27/tab}

\begin{table*}[!ht]
	\begin{tabular}{|m{4.5cm}|l|}
		\hline
		&\\
		dim(C$(\vec{A})) = 1$ 
		& \textbf{False}. Because the number of pivot variables are 2 as obtained in \eqref{eq:solutions/2017/dec/27/eq:rref}\\
		&\\
		\hline
		&\\
		dim(C$(\vec{A})) = 2$
		& \textbf{True}. Since the number of pivot variables are 2, the rank of $\vec{A}$ is 2.\\
		&$\therefore dim(C(\vec{A})) = 2 \quad [\because dim(C(\vec{A})) = rank(\vec{A})]$ \\
		&\\
		\hline
		&\\
	     rank$(\vec{A}) = 1$
		& \textbf{False}. Because the rank$(\vec{A}) = 2$, as the number of pivot variables are 2\\
		&\\
		\hline
		&\\
		$\vec{S}$ = $\cbrak{(1, 1, 1, 0), (1, 1, 0, 1)}$ is a basis of $N(\vec{A})$
		& \textbf{False}. \\
		& Let, \\
		&  $\vec{u} = \myvec{1\\1\\1\\0}, \vec{v} = \myvec{1\\1\\0\\1}$\\ 
		&Consider, \\
		&$\myvec{1 & 1 & 1 & 0 \\ 1 & 1 & 0 & 1\\ 0 & 0 & 0 & 0\\0 & 0 & 0 & 0}\myvec{1\\1\\1\\0} = \myvec{3\\2\\0\\0} \not = \myvec{0\\0\\0\\0}$\\
		& Similarly,\\
		&$\myvec{1 & 1 & 1 & 0 \\ 1 & 1 & 0 & 1\\ 0 & 0 & 0 & 0\\0 & 0 & 0 & 0}\myvec{1\\1\\0\\1} = \myvec{2\\3\\0\\0} \not = \myvec{0\\0\\0\\0}$ \\
		&Hence, the given vectors do not form the basis.\\
		\hline
	\end{tabular}
\caption{}
\label{eq:solutions/2017/dec/27/tab}
\end{table*}

\item 	Let $\vec{A}$ be an $m\times n$ matrix with rank $r$. If the linear system $\vec{A}\vec{X} = \vec{b}$ has a solution for each $\vec{b} \in \mathbf{R}^{m}$, then
	\begin{enumerate}
		\item $m=r$
		\item the column space of $\vec{A}$ is a proper subspace of $\mathbf{R}^{m}$ 
		\item the null space of $\vec{A}$ is a non-trivial subspace of $\mathbf{R}^{n}$ whenever $m=n$
		\item $m\geq n$ implies $m=n$
	\end{enumerate}
%
\solution
The nullspace is given by 
\begin{align}
	\myvec{1 & 1 & 1 & 0 \\ 1 & 1 & 0 & 1\\ 0 & 0 & 0 & 0\\0 & 0 & 0 & 0}\myvec{x\\y\\z\\w} = \myvec{0 \\ 0 \\ 0 \\ 0}
\end{align}	
Row reducing the above matrix we get,
\begin{align}
	\myvec{1 & 1 & 1 & 0 \\ 1 & 1 & 0 & 1\\ 0 & 0 & 0 & 0\\0 & 0 & 0 & 0}
	\xleftrightarrow[R_2 \leftarrow R_2 \times -1]{R_2 \leftarrow R_2 - R_1}
	\myvec{1 & 1 & 1 & 0 \\ 0 & 0 & 1 & -1\\ 0 & 0 & 0 & 0\\0 & 0 & 0 & 0}\\
	\xleftrightarrow{R_1 \leftarrow R_1- R_2}
	\myvec{1 & 1 & 0 & 1 \\ 0 & 0 & 1 & -1\\ 0 & 0 & 0 & 0\\0 & 0 & 0 & 0} \label{eq:solutions/2017/dec/27/eq:rref}
\end{align}
See Table \ref{eq:solutions/2017/dec/27/tab}

\begin{table*}[!ht]
	\begin{tabular}{|m{4.5cm}|l|}
		\hline
		&\\
		dim(C$(\vec{A})) = 1$ 
		& \textbf{False}. Because the number of pivot variables are 2 as obtained in \eqref{eq:solutions/2017/dec/27/eq:rref}\\
		&\\
		\hline
		&\\
		dim(C$(\vec{A})) = 2$
		& \textbf{True}. Since the number of pivot variables are 2, the rank of $\vec{A}$ is 2.\\
		&$\therefore dim(C(\vec{A})) = 2 \quad [\because dim(C(\vec{A})) = rank(\vec{A})]$ \\
		&\\
		\hline
		&\\
	     rank$(\vec{A}) = 1$
		& \textbf{False}. Because the rank$(\vec{A}) = 2$, as the number of pivot variables are 2\\
		&\\
		\hline
		&\\
		$\vec{S}$ = $\cbrak{(1, 1, 1, 0), (1, 1, 0, 1)}$ is a basis of $N(\vec{A})$
		& \textbf{False}. \\
		& Let, \\
		&  $\vec{u} = \myvec{1\\1\\1\\0}, \vec{v} = \myvec{1\\1\\0\\1}$\\ 
		&Consider, \\
		&$\myvec{1 & 1 & 1 & 0 \\ 1 & 1 & 0 & 1\\ 0 & 0 & 0 & 0\\0 & 0 & 0 & 0}\myvec{1\\1\\1\\0} = \myvec{3\\2\\0\\0} \not = \myvec{0\\0\\0\\0}$\\
		& Similarly,\\
		&$\myvec{1 & 1 & 1 & 0 \\ 1 & 1 & 0 & 1\\ 0 & 0 & 0 & 0\\0 & 0 & 0 & 0}\myvec{1\\1\\0\\1} = \myvec{2\\3\\0\\0} \not = \myvec{0\\0\\0\\0}$ \\
		&Hence, the given vectors do not form the basis.\\
		\hline
	\end{tabular}
\caption{}
\label{eq:solutions/2017/dec/27/tab}
\end{table*}

\item Let $\vec{M}$ =$\cbrak{\myvec{a&b\\c&d} : a,b,c,d \in \mathbb{Z} \text{ and eigen values of} \vec{A} \in \mathbb{Q}}$ \label{main}
\begin{enumerate}
    \item $\vec{M}$ is empty
    \item $\vec{M}$ =\cbrak{\myvec{a&b\\c&d} : a,b,c,d \in \mathbb{Z}}
    \item If $\vec{A}$ $\in$ $\vec{M}$ then the eigen values of $\vec{A}$ $\in$ $\mathbb{Z}$
    \item If $\vec{A}$,$\vec{B}$ $\in$ $\vec{M}$ such that $\vec{A} \vec{B}$=$\vec{I}$ then $\mydet{\vec{A}}$ $\in$ \{+1,-1\}
\end{enumerate}
%
\solution
The nullspace is given by 
\begin{align}
	\myvec{1 & 1 & 1 & 0 \\ 1 & 1 & 0 & 1\\ 0 & 0 & 0 & 0\\0 & 0 & 0 & 0}\myvec{x\\y\\z\\w} = \myvec{0 \\ 0 \\ 0 \\ 0}
\end{align}	
Row reducing the above matrix we get,
\begin{align}
	\myvec{1 & 1 & 1 & 0 \\ 1 & 1 & 0 & 1\\ 0 & 0 & 0 & 0\\0 & 0 & 0 & 0}
	\xleftrightarrow[R_2 \leftarrow R_2 \times -1]{R_2 \leftarrow R_2 - R_1}
	\myvec{1 & 1 & 1 & 0 \\ 0 & 0 & 1 & -1\\ 0 & 0 & 0 & 0\\0 & 0 & 0 & 0}\\
	\xleftrightarrow{R_1 \leftarrow R_1- R_2}
	\myvec{1 & 1 & 0 & 1 \\ 0 & 0 & 1 & -1\\ 0 & 0 & 0 & 0\\0 & 0 & 0 & 0} \label{eq:solutions/2017/dec/27/eq:rref}
\end{align}
See Table \ref{eq:solutions/2017/dec/27/tab}

\begin{table*}[!ht]
	\begin{tabular}{|m{4.5cm}|l|}
		\hline
		&\\
		dim(C$(\vec{A})) = 1$ 
		& \textbf{False}. Because the number of pivot variables are 2 as obtained in \eqref{eq:solutions/2017/dec/27/eq:rref}\\
		&\\
		\hline
		&\\
		dim(C$(\vec{A})) = 2$
		& \textbf{True}. Since the number of pivot variables are 2, the rank of $\vec{A}$ is 2.\\
		&$\therefore dim(C(\vec{A})) = 2 \quad [\because dim(C(\vec{A})) = rank(\vec{A})]$ \\
		&\\
		\hline
		&\\
	     rank$(\vec{A}) = 1$
		& \textbf{False}. Because the rank$(\vec{A}) = 2$, as the number of pivot variables are 2\\
		&\\
		\hline
		&\\
		$\vec{S}$ = $\cbrak{(1, 1, 1, 0), (1, 1, 0, 1)}$ is a basis of $N(\vec{A})$
		& \textbf{False}. \\
		& Let, \\
		&  $\vec{u} = \myvec{1\\1\\1\\0}, \vec{v} = \myvec{1\\1\\0\\1}$\\ 
		&Consider, \\
		&$\myvec{1 & 1 & 1 & 0 \\ 1 & 1 & 0 & 1\\ 0 & 0 & 0 & 0\\0 & 0 & 0 & 0}\myvec{1\\1\\1\\0} = \myvec{3\\2\\0\\0} \not = \myvec{0\\0\\0\\0}$\\
		& Similarly,\\
		&$\myvec{1 & 1 & 1 & 0 \\ 1 & 1 & 0 & 1\\ 0 & 0 & 0 & 0\\0 & 0 & 0 & 0}\myvec{1\\1\\0\\1} = \myvec{2\\3\\0\\0} \not = \myvec{0\\0\\0\\0}$ \\
		&Hence, the given vectors do not form the basis.\\
		\hline
	\end{tabular}
\caption{}
\label{eq:solutions/2017/dec/27/tab}
\end{table*}

%
\item Let A be a $3\times 3$  matrix  with real entries.Identify  the correct statements.
\begin{enumerate}

\item A  is necessarily diagonalizable over $\vec{R}$

\item If A has distinct real  eigen values than  it is diagonalizable over$\vec{R}$

\item If A has distinct eigen values than  it is diagonalizable over $\vec{C}$

\item If all eigen values are non zero than it is diagonalizable over $\vec{C}$
\end{enumerate}
%
\solution
The nullspace is given by 
\begin{align}
	\myvec{1 & 1 & 1 & 0 \\ 1 & 1 & 0 & 1\\ 0 & 0 & 0 & 0\\0 & 0 & 0 & 0}\myvec{x\\y\\z\\w} = \myvec{0 \\ 0 \\ 0 \\ 0}
\end{align}	
Row reducing the above matrix we get,
\begin{align}
	\myvec{1 & 1 & 1 & 0 \\ 1 & 1 & 0 & 1\\ 0 & 0 & 0 & 0\\0 & 0 & 0 & 0}
	\xleftrightarrow[R_2 \leftarrow R_2 \times -1]{R_2 \leftarrow R_2 - R_1}
	\myvec{1 & 1 & 1 & 0 \\ 0 & 0 & 1 & -1\\ 0 & 0 & 0 & 0\\0 & 0 & 0 & 0}\\
	\xleftrightarrow{R_1 \leftarrow R_1- R_2}
	\myvec{1 & 1 & 0 & 1 \\ 0 & 0 & 1 & -1\\ 0 & 0 & 0 & 0\\0 & 0 & 0 & 0} \label{eq:solutions/2017/dec/27/eq:rref}
\end{align}
See Table \ref{eq:solutions/2017/dec/27/tab}

\begin{table*}[!ht]
	\begin{tabular}{|m{4.5cm}|l|}
		\hline
		&\\
		dim(C$(\vec{A})) = 1$ 
		& \textbf{False}. Because the number of pivot variables are 2 as obtained in \eqref{eq:solutions/2017/dec/27/eq:rref}\\
		&\\
		\hline
		&\\
		dim(C$(\vec{A})) = 2$
		& \textbf{True}. Since the number of pivot variables are 2, the rank of $\vec{A}$ is 2.\\
		&$\therefore dim(C(\vec{A})) = 2 \quad [\because dim(C(\vec{A})) = rank(\vec{A})]$ \\
		&\\
		\hline
		&\\
	     rank$(\vec{A}) = 1$
		& \textbf{False}. Because the rank$(\vec{A}) = 2$, as the number of pivot variables are 2\\
		&\\
		\hline
		&\\
		$\vec{S}$ = $\cbrak{(1, 1, 1, 0), (1, 1, 0, 1)}$ is a basis of $N(\vec{A})$
		& \textbf{False}. \\
		& Let, \\
		&  $\vec{u} = \myvec{1\\1\\1\\0}, \vec{v} = \myvec{1\\1\\0\\1}$\\ 
		&Consider, \\
		&$\myvec{1 & 1 & 1 & 0 \\ 1 & 1 & 0 & 1\\ 0 & 0 & 0 & 0\\0 & 0 & 0 & 0}\myvec{1\\1\\1\\0} = \myvec{3\\2\\0\\0} \not = \myvec{0\\0\\0\\0}$\\
		& Similarly,\\
		&$\myvec{1 & 1 & 1 & 0 \\ 1 & 1 & 0 & 1\\ 0 & 0 & 0 & 0\\0 & 0 & 0 & 0}\myvec{1\\1\\0\\1} = \myvec{2\\3\\0\\0} \not = \myvec{0\\0\\0\\0}$ \\
		&Hence, the given vectors do not form the basis.\\
		\hline
	\end{tabular}
\caption{}
\label{eq:solutions/2017/dec/27/tab}
\end{table*}

\twocolumn
\item Let $V$ be a vector space over $C$ of all the polynomials in a variable $X$ of degree atmost 3. Let $D:V \xrightarrow{} V$ be the linear operator given by differentiation with respect to $X$. Let $A$ be the matrix of $D$ with respect to some 
basis for $V$. Which of the following are true? \\
\begin{enumerate}
\item A is nilpotent matrix \\
\item A is diagonalizable matrix \\ 
\item the rank of A is 2 \\
\item the Jordan canonical form of A is
\begin{align}\nonumber
    \myvec{0&1&0&0\\
       0&0&1&0\\
       0&0&0&1\\
       0&0&0&0}
\end{align}
\end{enumerate}
%
\solution
The nullspace is given by 
\begin{align}
	\myvec{1 & 1 & 1 & 0 \\ 1 & 1 & 0 & 1\\ 0 & 0 & 0 & 0\\0 & 0 & 0 & 0}\myvec{x\\y\\z\\w} = \myvec{0 \\ 0 \\ 0 \\ 0}
\end{align}	
Row reducing the above matrix we get,
\begin{align}
	\myvec{1 & 1 & 1 & 0 \\ 1 & 1 & 0 & 1\\ 0 & 0 & 0 & 0\\0 & 0 & 0 & 0}
	\xleftrightarrow[R_2 \leftarrow R_2 \times -1]{R_2 \leftarrow R_2 - R_1}
	\myvec{1 & 1 & 1 & 0 \\ 0 & 0 & 1 & -1\\ 0 & 0 & 0 & 0\\0 & 0 & 0 & 0}\\
	\xleftrightarrow{R_1 \leftarrow R_1- R_2}
	\myvec{1 & 1 & 0 & 1 \\ 0 & 0 & 1 & -1\\ 0 & 0 & 0 & 0\\0 & 0 & 0 & 0} \label{eq:solutions/2017/dec/27/eq:rref}
\end{align}
See Table \ref{eq:solutions/2017/dec/27/tab}

\begin{table*}[!ht]
	\begin{tabular}{|m{4.5cm}|l|}
		\hline
		&\\
		dim(C$(\vec{A})) = 1$ 
		& \textbf{False}. Because the number of pivot variables are 2 as obtained in \eqref{eq:solutions/2017/dec/27/eq:rref}\\
		&\\
		\hline
		&\\
		dim(C$(\vec{A})) = 2$
		& \textbf{True}. Since the number of pivot variables are 2, the rank of $\vec{A}$ is 2.\\
		&$\therefore dim(C(\vec{A})) = 2 \quad [\because dim(C(\vec{A})) = rank(\vec{A})]$ \\
		&\\
		\hline
		&\\
	     rank$(\vec{A}) = 1$
		& \textbf{False}. Because the rank$(\vec{A}) = 2$, as the number of pivot variables are 2\\
		&\\
		\hline
		&\\
		$\vec{S}$ = $\cbrak{(1, 1, 1, 0), (1, 1, 0, 1)}$ is a basis of $N(\vec{A})$
		& \textbf{False}. \\
		& Let, \\
		&  $\vec{u} = \myvec{1\\1\\1\\0}, \vec{v} = \myvec{1\\1\\0\\1}$\\ 
		&Consider, \\
		&$\myvec{1 & 1 & 1 & 0 \\ 1 & 1 & 0 & 1\\ 0 & 0 & 0 & 0\\0 & 0 & 0 & 0}\myvec{1\\1\\1\\0} = \myvec{3\\2\\0\\0} \not = \myvec{0\\0\\0\\0}$\\
		& Similarly,\\
		&$\myvec{1 & 1 & 1 & 0 \\ 1 & 1 & 0 & 1\\ 0 & 0 & 0 & 0\\0 & 0 & 0 & 0}\myvec{1\\1\\0\\1} = \myvec{2\\3\\0\\0} \not = \myvec{0\\0\\0\\0}$ \\
		&Hence, the given vectors do not form the basis.\\
		\hline
	\end{tabular}
\caption{}
\label{eq:solutions/2017/dec/27/tab}
\end{table*}

\item For every $4 \times 4$ real symmetric non-singular matrix $\vec{A}$ there 
exists a positive integer $p$ such that

    \begin{enumerate}
        \item $p\vec{I}+\vec{A}$ is positive definite
        \item $\vec{A}^p$ is positive definite
        \item $\vec{A}^{-p}$ is positive definite
        \item $\text{exp}(p\vec{A})-\vec{I}$ is positive definite
        \end{enumerate}
\solution
The nullspace is given by 
\begin{align}
	\myvec{1 & 1 & 1 & 0 \\ 1 & 1 & 0 & 1\\ 0 & 0 & 0 & 0\\0 & 0 & 0 & 0}\myvec{x\\y\\z\\w} = \myvec{0 \\ 0 \\ 0 \\ 0}
\end{align}	
Row reducing the above matrix we get,
\begin{align}
	\myvec{1 & 1 & 1 & 0 \\ 1 & 1 & 0 & 1\\ 0 & 0 & 0 & 0\\0 & 0 & 0 & 0}
	\xleftrightarrow[R_2 \leftarrow R_2 \times -1]{R_2 \leftarrow R_2 - R_1}
	\myvec{1 & 1 & 1 & 0 \\ 0 & 0 & 1 & -1\\ 0 & 0 & 0 & 0\\0 & 0 & 0 & 0}\\
	\xleftrightarrow{R_1 \leftarrow R_1- R_2}
	\myvec{1 & 1 & 0 & 1 \\ 0 & 0 & 1 & -1\\ 0 & 0 & 0 & 0\\0 & 0 & 0 & 0} \label{eq:solutions/2017/dec/27/eq:rref}
\end{align}
See Table \ref{eq:solutions/2017/dec/27/tab}

\begin{table*}[!ht]
	\begin{tabular}{|m{4.5cm}|l|}
		\hline
		&\\
		dim(C$(\vec{A})) = 1$ 
		& \textbf{False}. Because the number of pivot variables are 2 as obtained in \eqref{eq:solutions/2017/dec/27/eq:rref}\\
		&\\
		\hline
		&\\
		dim(C$(\vec{A})) = 2$
		& \textbf{True}. Since the number of pivot variables are 2, the rank of $\vec{A}$ is 2.\\
		&$\therefore dim(C(\vec{A})) = 2 \quad [\because dim(C(\vec{A})) = rank(\vec{A})]$ \\
		&\\
		\hline
		&\\
	     rank$(\vec{A}) = 1$
		& \textbf{False}. Because the rank$(\vec{A}) = 2$, as the number of pivot variables are 2\\
		&\\
		\hline
		&\\
		$\vec{S}$ = $\cbrak{(1, 1, 1, 0), (1, 1, 0, 1)}$ is a basis of $N(\vec{A})$
		& \textbf{False}. \\
		& Let, \\
		&  $\vec{u} = \myvec{1\\1\\1\\0}, \vec{v} = \myvec{1\\1\\0\\1}$\\ 
		&Consider, \\
		&$\myvec{1 & 1 & 1 & 0 \\ 1 & 1 & 0 & 1\\ 0 & 0 & 0 & 0\\0 & 0 & 0 & 0}\myvec{1\\1\\1\\0} = \myvec{3\\2\\0\\0} \not = \myvec{0\\0\\0\\0}$\\
		& Similarly,\\
		&$\myvec{1 & 1 & 1 & 0 \\ 1 & 1 & 0 & 1\\ 0 & 0 & 0 & 0\\0 & 0 & 0 & 0}\myvec{1\\1\\0\\1} = \myvec{2\\3\\0\\0} \not = \myvec{0\\0\\0\\0}$ \\
		&Hence, the given vectors do not form the basis.\\
		\hline
	\end{tabular}
\caption{}
\label{eq:solutions/2017/dec/27/tab}
\end{table*}

\item Let $\vec{A}$ be an $m \times n$ matrix of rank $m$ with $n>m$. If for some non-zero real number $\alpha$, we have $\vec{x^TAA^Tx} = \alpha\vec{x^Tx}$, for all $x \in \vec{R^m}$, then $\vec{A^TA}$ has,
\begin{enumerate}

\item  exactly two distinct eigenvalues.

\item  0 as an eigenvalue with multiplicity $n-m$.

\item  $\alpha$ as a non-zero eigenvalue.

\item  exactly two non-zero distinct eigenvalues.
\end{enumerate}
%
\solution
The nullspace is given by 
\begin{align}
	\myvec{1 & 1 & 1 & 0 \\ 1 & 1 & 0 & 1\\ 0 & 0 & 0 & 0\\0 & 0 & 0 & 0}\myvec{x\\y\\z\\w} = \myvec{0 \\ 0 \\ 0 \\ 0}
\end{align}	
Row reducing the above matrix we get,
\begin{align}
	\myvec{1 & 1 & 1 & 0 \\ 1 & 1 & 0 & 1\\ 0 & 0 & 0 & 0\\0 & 0 & 0 & 0}
	\xleftrightarrow[R_2 \leftarrow R_2 \times -1]{R_2 \leftarrow R_2 - R_1}
	\myvec{1 & 1 & 1 & 0 \\ 0 & 0 & 1 & -1\\ 0 & 0 & 0 & 0\\0 & 0 & 0 & 0}\\
	\xleftrightarrow{R_1 \leftarrow R_1- R_2}
	\myvec{1 & 1 & 0 & 1 \\ 0 & 0 & 1 & -1\\ 0 & 0 & 0 & 0\\0 & 0 & 0 & 0} \label{eq:solutions/2017/dec/27/eq:rref}
\end{align}
See Table \ref{eq:solutions/2017/dec/27/tab}

\begin{table*}[!ht]
	\begin{tabular}{|m{4.5cm}|l|}
		\hline
		&\\
		dim(C$(\vec{A})) = 1$ 
		& \textbf{False}. Because the number of pivot variables are 2 as obtained in \eqref{eq:solutions/2017/dec/27/eq:rref}\\
		&\\
		\hline
		&\\
		dim(C$(\vec{A})) = 2$
		& \textbf{True}. Since the number of pivot variables are 2, the rank of $\vec{A}$ is 2.\\
		&$\therefore dim(C(\vec{A})) = 2 \quad [\because dim(C(\vec{A})) = rank(\vec{A})]$ \\
		&\\
		\hline
		&\\
	     rank$(\vec{A}) = 1$
		& \textbf{False}. Because the rank$(\vec{A}) = 2$, as the number of pivot variables are 2\\
		&\\
		\hline
		&\\
		$\vec{S}$ = $\cbrak{(1, 1, 1, 0), (1, 1, 0, 1)}$ is a basis of $N(\vec{A})$
		& \textbf{False}. \\
		& Let, \\
		&  $\vec{u} = \myvec{1\\1\\1\\0}, \vec{v} = \myvec{1\\1\\0\\1}$\\ 
		&Consider, \\
		&$\myvec{1 & 1 & 1 & 0 \\ 1 & 1 & 0 & 1\\ 0 & 0 & 0 & 0\\0 & 0 & 0 & 0}\myvec{1\\1\\1\\0} = \myvec{3\\2\\0\\0} \not = \myvec{0\\0\\0\\0}$\\
		& Similarly,\\
		&$\myvec{1 & 1 & 1 & 0 \\ 1 & 1 & 0 & 1\\ 0 & 0 & 0 & 0\\0 & 0 & 0 & 0}\myvec{1\\1\\0\\1} = \myvec{2\\3\\0\\0} \not = \myvec{0\\0\\0\\0}$ \\
		&Hence, the given vectors do not form the basis.\\
		\hline
	\end{tabular}
\caption{}
\label{eq:solutions/2017/dec/27/tab}
\end{table*}

\item %
Consider a Markov chain with five states $\{1,2,3,4,5\}$ and transition matrix
\begin{align}
    P=\myvec{\frac{1}{2} & 0 & 0 & \frac{1}{2} & 0\\
            0 & \frac{1}{7} & 0 & 0&\frac{6}{7}\\
              \frac{1}{5} & \frac{1}{5} & \frac{1}{5} & \frac{1}{5} & \frac{1}{5}\\ \frac{1}{3} & 0 & 0 & \frac{2}{3} & 0 \\
              0 & \frac{5}{8} & 0 & 0 & \frac{3}{8}}
\end{align}
Which of the following are true?
\begin{enumerate}
\item 3 and 1 are in the same communicating class
\item 1 and 4 are in the same communicating class
\item 4 and 2 are in the same communicating class
\item 2 and 5 are in the same communicating class
\end{enumerate}
%
\solution
The nullspace is given by 
\begin{align}
	\myvec{1 & 1 & 1 & 0 \\ 1 & 1 & 0 & 1\\ 0 & 0 & 0 & 0\\0 & 0 & 0 & 0}\myvec{x\\y\\z\\w} = \myvec{0 \\ 0 \\ 0 \\ 0}
\end{align}	
Row reducing the above matrix we get,
\begin{align}
	\myvec{1 & 1 & 1 & 0 \\ 1 & 1 & 0 & 1\\ 0 & 0 & 0 & 0\\0 & 0 & 0 & 0}
	\xleftrightarrow[R_2 \leftarrow R_2 \times -1]{R_2 \leftarrow R_2 - R_1}
	\myvec{1 & 1 & 1 & 0 \\ 0 & 0 & 1 & -1\\ 0 & 0 & 0 & 0\\0 & 0 & 0 & 0}\\
	\xleftrightarrow{R_1 \leftarrow R_1- R_2}
	\myvec{1 & 1 & 0 & 1 \\ 0 & 0 & 1 & -1\\ 0 & 0 & 0 & 0\\0 & 0 & 0 & 0} \label{eq:solutions/2017/dec/27/eq:rref}
\end{align}
See Table \ref{eq:solutions/2017/dec/27/tab}

\begin{table*}[!ht]
	\begin{tabular}{|m{4.5cm}|l|}
		\hline
		&\\
		dim(C$(\vec{A})) = 1$ 
		& \textbf{False}. Because the number of pivot variables are 2 as obtained in \eqref{eq:solutions/2017/dec/27/eq:rref}\\
		&\\
		\hline
		&\\
		dim(C$(\vec{A})) = 2$
		& \textbf{True}. Since the number of pivot variables are 2, the rank of $\vec{A}$ is 2.\\
		&$\therefore dim(C(\vec{A})) = 2 \quad [\because dim(C(\vec{A})) = rank(\vec{A})]$ \\
		&\\
		\hline
		&\\
	     rank$(\vec{A}) = 1$
		& \textbf{False}. Because the rank$(\vec{A}) = 2$, as the number of pivot variables are 2\\
		&\\
		\hline
		&\\
		$\vec{S}$ = $\cbrak{(1, 1, 1, 0), (1, 1, 0, 1)}$ is a basis of $N(\vec{A})$
		& \textbf{False}. \\
		& Let, \\
		&  $\vec{u} = \myvec{1\\1\\1\\0}, \vec{v} = \myvec{1\\1\\0\\1}$\\ 
		&Consider, \\
		&$\myvec{1 & 1 & 1 & 0 \\ 1 & 1 & 0 & 1\\ 0 & 0 & 0 & 0\\0 & 0 & 0 & 0}\myvec{1\\1\\1\\0} = \myvec{3\\2\\0\\0} \not = \myvec{0\\0\\0\\0}$\\
		& Similarly,\\
		&$\myvec{1 & 1 & 1 & 0 \\ 1 & 1 & 0 & 1\\ 0 & 0 & 0 & 0\\0 & 0 & 0 & 0}\myvec{1\\1\\0\\1} = \myvec{2\\3\\0\\0} \not = \myvec{0\\0\\0\\0}$ \\
		&Hence, the given vectors do not form the basis.\\
		\hline
	\end{tabular}
\caption{}
\label{eq:solutions/2017/dec/27/tab}
\end{table*}



%\item Consider a Markov Chain with state space $\cbrak{0,1,2}$ and transition matrix
%\begin{align}
%P = 
%\begin{blockarray}{c@{\hspace{1pt}}rrr@{\hspace{3pt}}}
%         & 0   & 1   & 2 \\
%        \begin{block}{r@{\hspace{3pt}}@{\hspace{1pt}}
%    (@{\hspace{1pt}}rrr@{\hspace{1pt}}@{\hspace{1pt}})}
%        0 & \frac{1}{2} & \frac{1}{2} & 0  \\
%        1 & 0 &\frac{1}{2}  & \frac{3}{4}  \\
%%
%        2 &  \frac{1}{3} & \frac{1}{3} & \frac{1}{3}  \\
%        \end{block}
%    \end{blockarray}
%\end{align}
%For any two states $i$ and $j$, let $p_{ij}^{(n)}$ denote the $n$-step transition probability of going from $i$ to $j$.  Identify correct statements.
%\begin{enumerate}
%\item $\lim_{n \to \infty} p_{11}^{(n)} = \frac{2}{9}$
%\item $\lim_{n \to \infty} p_{21}^{(n)} = 0$
%\item $\lim_{n \to \infty} p_{32}^{(n)} = \frac{1}{3}$
%\item $\lim_{n \to \infty} p_{13}^{(n)} = \frac{1}{3}$
%\end{enumerate}

\end{enumerate}
