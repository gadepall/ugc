See Tables     \ref{eq:solutions/2018/dec/78/table:1}
    \ref{eq:solutions/2018/dec/78/table:2}

\onecolumn
\begin{longtable}{|l|l|}
	\hline
	\multirow{3}{*}{Matrix representation} 
	& \\
	& The Matrix representation of quadratic forms\\
	&\parbox{10cm}
	{\begin{align}
	Q(x,y) = ax^2+2bxy+cy^2=\myvec{x&y}\myvec{a&b\\b&c}\myvec{x\\y} =\vec{X}^{T}\vec{A}\vec{X}\label{eq:solutions/2018/dec/78/eq:1}
	\end{align}}\\
	&The symmetric matrix of the quadratic form is\\
	&\parbox{10cm}
	{\begin{align}
	\vec{A}=\myvec{a&b\\b&c}
	\end{align}}\\ 
	&\\
	\hline
	\multirow{3}{*}{Equivalent condition} 
	&\\
	& Two quadratic forms $\vec{X}^{T}\vec{A}\vec{X}$ and $\vec{Y}^{T}\vec{B}\vec{Y}$ are called equivalent if their matrices,\\
	& A and B are congruent.\\
	&\\
	& Two real quadratic forms are equivalent over the real field iff they have\\
	& the same rank and the same index.\\
	&\\
    \hline
    \multirow{3}{*}{Rank} 
	&\\
	& The rank of a quadratic form is the rank of its associated matrix.\\
	&\\
	\hline
    \multirow{3}{*}{Index} 
	&\\
	& The index of the quadratic form is equal to the number of positive eigen\\ &values of the matrix of quadratic form.\\
	&\\
    \hline
    \caption{Definitions and results used}
    \label{eq:solutions/2018/dec/78/table:1}
\end{longtable}
\begin{table*}[ht]
\begin{tabular}{|l|l|l|l|l|}
\hline
\multicolumn{1}{|c|}{} &
  \textbf{Matrix} &
  \textbf{Rank} &
  \textbf{Eigen Values} &
  \textbf{Index} \\ \hline
$Q_{1}(x,y)$ &
  \multicolumn{1}{c|}{$\vec{A}_1=\myvec{0&\frac{1}{2}\\\frac{1}{2}&0}$} &
  \begin{tabular}[c]{@{}l@{}}
  $\myvec{0&\frac{1}{2}\\\frac{1}{2}&0}\xleftrightarrow[R_2\leftarrow R_1]{R_1\leftarrow R_2}
  \myvec{\frac{1}{2}&0\\0&\frac{1}{2}}$\\
\\
  $\text{rank}(\vec{A}_1)=2$
  \\
  \end{tabular} &
  \begin{tabular}[c]{@{}l@{}}
$\quad \quad\mydet{\Vec{A}_1-\lambda\Vec{I}}=0$\\ \\
$\implies\mydet{-\lambda&\frac{1}{2}\\\frac{1}{2}&-\lambda}=0$\\ \\
$\implies\brak{\lambda-\frac{1}{2}}\brak{\lambda+\frac{1}{2}}=0$\\
\\
$\implies \lambda_1=\frac{1}{2},\lambda_2=-\frac{1}{2}$
\\  \end{tabular} &
 $\text{Index of $\vec{A}_1$}  =1$ \\ \hline
$Q_{2}(x,y)$ &
  $\vec{A}_2=\myvec{1&1\\1&1}$ &
  \begin{tabular}[c]{@{}l@{}}
  $\myvec{1&1\\1&1}\xleftrightarrow[]{R_2\leftarrow R_2-R_1}\myvec{1&1\\0&0}$\\
  \\
  $\text{rank}(\vec{A}_2)=1$
  \\
  \end{tabular} &
  \begin{tabular}[c]{@{}l@{}}
  $\quad \quad\mydet{\Vec{A}_2-\lambda\Vec{I}}=0$\\ \\
$\implies\mydet{1-\lambda&1\\1&1-\lambda}=0$\\\\
$\implies\brak{\lambda}\brak{\lambda-2}=0$ \\
\\$\implies \lambda_1=0,\lambda_2=2$
\\
  \end{tabular} &
  \text{Index of $\vec{A}_2$}=2 \\ \hline
$Q_{3}(x,y)$ &
  $\vec{A}_3=\myvec{1&\frac{3}{2}\\\frac{3}{2}&2}$  &
  \begin{tabular}[c]{@{}l@{}}
  $\myvec{1&\frac{3}{2}\\\frac{3}{2}&2}
\xleftrightarrow[]{R_2\leftarrow R_2-\frac{3}{2} R_1}
\myvec{1&\frac{3}{2}\\0&-\frac{1}{4}}$\\
  \\
  $\text{rank}(\vec{A}_3)=2$
  \\
  \end{tabular} &
  \begin{tabular}[c]{@{}l@{}}
  $\quad\quad\mydet{\Vec{A}_3-\lambda\Vec{I}}=0$\\ \\
$\implies\mydet{1-\lambda&\frac{3}{2}\\\frac{3}{2}&2-\lambda}=0$\\ \\
$\implies\brak{\lambda-\frac{\sqrt{10}+3}{2}}\brak{\lambda+\frac{\sqrt{10}-3}{2}}=0$\\ \\
$\implies \lambda_1=\frac{3+\sqrt{10}}{2},\lambda_2=\frac{3-\sqrt{10}}{2}$\\
  \end{tabular} &
   $\text{Index of $\vec{A}_3$}  =1$  \\ \hline
Conclusion &
  \multicolumn{4}{l|}{We can say that $Q_{1}(x,y)$ and $Q_{3}(x,y)$ are equivalent as the rank and index are same.} \\ \hline
\end{tabular}
\caption{Finding which quadratic forms are equivalent}
\label{eq:solutions/2018/dec/78/table:2}
\end{table*}
\twocolumn
