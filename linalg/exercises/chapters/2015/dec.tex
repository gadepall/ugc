\renewcommand{\theequation}{\theenumi}
\renewcommand{\thefigure}{\theenumi}
\begin{enumerate}[label=\thesection.\arabic*.,ref=\thesection.\theenumi]
\numberwithin{equation}{enumi}
\numberwithin{figure}{enumi}
\numberwithin{table}{enumi}

\item Let $\vec{V}$ be the vector space of polynomials over $\mathbb{R}$ of degree less than or equal to $n$. For $p(x)=a_0+a_{n-1}x+..+a_nx^{n}$ in $\vec{V}$, define a linear transformation $\vec{T}:\vec{V}\rightarrow \vec{V}$ by $\brak{\vec{T}p}(x)=a_n+a_{n-1}x+..+a_0x^n.$ Then \\
\begin{enumerate}
    \item $\vec{T}$ is one to one.
    \item $\vec{T}$ is onto.
    \item $\vec{T}$ is invertible.
    \item $\det{\vec{T}}=\pm 1$.
\end{enumerate}
%
%
\solution
See Tables \ref{eq:solutions/2018/dec/106/table0} and \ref{eq:solutions/2018/dec/106/table1}


\onecolumn
	\begin{longtable}{|l|l|}
		\hline
		\multirow{3}{*}{Irreducible Markov Chain} 
		& \\
		& A Markov chain is $\textbf{irreducible}$ if all the states communicate with each other,\\
		& i.e., if there is only one communication class.\\
		&\\
		\hline
		\multirow{3}{*}{Aperiodic Markov Chain} & \\
		& If there is a self-transition in the chain ($p^{ii}>0$ for some i), then the chain is\\
		& called as $\textbf{aperiodic}$\\
		& \\
		\hline
		\multirow{3}{*}{Stationary Distribution} & \\
		& A stationary distribution of a Markov chain is a probability distribution that\\
		& remains unchanged in the Markov chain as time progresses. Typically, it is\\
		& represented as a row vector $\Vec{\pi}$ whose entries are probabilities summing to 1,\\ 
		& and given transition matrix $\textbf{P}$, it satisfies\\
		& \\
		&  \qquad \qquad  \qquad$\Vec{\pi} = \Vec{\pi} \textbf{P}$\\
		& \\
		\hline
\caption{}
\label{eq:solutions/2018/dec/106/table0}
	\end{longtable}
	\begin{longtable}{|l|l|}
		\hline
		\multirow{3}{*}{Drawing Transition diagram} 
		& \\
		& 
		
		$\begin{tikzpicture}[shorten >=1pt,node distance=2cm, scale =3, auto]
			\tikzstyle{every state}=[fill={rgb:black,1;white,10}]
			
			\node[state]   (q_1)                          {$1$};
			\node[state]   (q_2)  [right of=q_1]          {$2$};
			\node[state]   (q_3)  [below right of=q_1]          {$3$};
			
			\path[->]
			(q_1) edge [loop above] node {$\frac{1}{2}$}    (   )
			edge [bend left]  node {$\frac{1}{2}$}    (q_2)
			(q_2) edge [bend left]  node {$\frac{1}{2}$}    (q_3)
			edge [loop above] node {$\frac{1}{2}$}    ()
			(q_3) edge [bend left]  node {$\frac{1}{3}$}    (q_2)
			edge [bend left]  node {$\frac{1}{3}$}    (q_1)
			edge [loop below] node {$\frac{1}{3}$}    ();
		\end{tikzpicture}$
		
		\\  
		&\\
		&\\
		\hline
		\multirow{3}{*}{Checking whether the  } & \\
		& Here,\\chain is Irreducible
		& All the states are accessible to one another. \\and Aperiodic
		& $\implies$ They are in the same communication class. So, it is Irreducible.\\
		& \\
		& There exists the non- zero self-transition, which means that the chain \\
		& is Aperiodic.\\
		&\\ 
		& We know that if the Markov Chain is irreducible and aperiodic then \\
		& \qquad \qquad \qquad $\Vec{\pi}_{j} = \lim_{n \to \infty}P\{X_{n} = j\}$, $j = 1,...,N$ \\
		& These are the stationary probabilities. \\
		&\\
		\hline
		\multirow{3}{*}{Finding the Stationary} & \\
		& Stationary Probability can be represented as\\Probability Distributions
		& \qquad \qquad \qquad $\Vec{\pi} = \Vec{\pi} \vec{P}$\\
		& \\
		& \qquad $\implies$ $\myvec{v_{1}&&v_{2}&&v_{3}} = \myvec{v_{1}&&v_{2}&&v_{3}}\vec{P}$ \\
		& \\
		& Equating the above equation we get \\
		& \\
		& \qquad \qquad \qquad $\frac{1}{2}v_{1}-\frac{1}{3}v_{3} = 0$ $\label{eq:solutions/2018/dec/106/eq}$\\
		& \\
		& \qquad \qquad \qquad $\frac{1}{2}v_{1}-\frac{1}{2}v_{2} + \frac{1}{3}v_{3} = 0$\\
		& \\
		& \qquad \qquad \qquad $\frac{1}{2}v_{2}-\frac{2}{3}v_{3} = 0$\\
		& \\\
		& We see that summation of second and the third equation gives us the \\
		& first equation only. \\
		& And we know that the probability distribution will sum up to 1. \\
		& \\
		& \qquad \qquad \qquad $v_{1}+v_{2}+v_{3} = 1$ \\
		& \\
		& Therefore, we get the equation form as \\
		& \\
		& \qquad \qquad \qquad $\myvec{1&1&1\\\frac{1}{2}&0&\frac{-1}{3}\\\frac{1}{2}&\frac{-1}{2}&\frac{1}{3}}\myvec{v_{1}\\v_{2}\\v_{3}} = \myvec{1\\0\\0}$ \\
		& \\
		\hline
		\multirow{3}{*}{Solving the linear} & \\
		& The above linear equation can be solved using Gauss-Jordan method as\\equtions
		& \\
		& \qquad \qquad \qquad $\myvec{1&1&1&\vrule&1\\\frac{1}{2}&0&\frac{-1}{3}&\vrule&0\\\frac{1}{2}&\frac{-1}{2}&\frac{1}{3}&\vrule&0}$\\
		& \\
		& \qquad $\xleftrightarrow[]{R_2 \leftarrow R_2 - \frac{1}{2}R_1}$
		$\myvec{1&1&1&\vrule&1\\0&\frac{-1}{2}&\frac{-5}{6}&\vrule&\frac{-1}{2}\\\frac{1}{2}&\frac{-1}{2}&\frac{1}{3}&\vrule&0}$\\
		&\\
		& \qquad $\xleftrightarrow[]{R_3 \leftarrow R_3 - \frac{1}{2}R_1}$
		$\myvec{1&1&1&\vrule&1\\0&\frac{-1}{2}&\frac{-5}{6}&\vrule&\frac{-1}{2}\\0&-1&\frac{-1}{6}&\vrule&\frac{-1}{2}}$\\
		&\\
		& \qquad $\xleftrightarrow[]{R_2 \leftarrow \frac{-1}{2}R_2}$
		$\myvec{1&1&1&\vrule&1\\0&1&\frac{5}{3}&\vrule&1\\0&-1&\frac{-1}{6}&\vrule&\frac{-1}{2}}$\\
		&\\
		& \qquad $\xleftrightarrow[]{R_3 \leftarrow R_3 + R_2}$
		$\myvec{1&1&1&\vrule&1\\0&1&\frac{5}{3}&\vrule&1\\0&0&\frac{3}{2}&\vrule&\frac{1}{2}}$\\
		&\\
		& \qquad $\xleftrightarrow[]{R_3 \leftarrow \frac{3}{2}R_3}$
		$\myvec{1&1&1&\vrule&1\\0&1&\frac{5}{3}&\vrule&1\\0&0&1&\vrule&\frac{1}{3}}$\\
		&\\
		& \qquad $\xleftrightarrow[]{R_2 \leftarrow R_2 - \frac{5}{3}R_3}$
		$\myvec{1&1&1&\vrule&1\\0&1&0&\vrule&\frac{4}{9}\\0&0&1&\vrule&\frac{1}{3}}$\\
		&\\
		& \qquad $\xleftrightarrow[]{R_1 \leftarrow R_1 - R_3}$
		$\myvec{1&1&0&\vrule&\frac{2}{3}\\0&1&0&\vrule&\frac{4}{9}\\0&0&1&\vrule&\frac{1}{3}}$\\
		&\\
		& \qquad $\xleftrightarrow[]{R_1 \leftarrow R_1 - R_2}$
		$\myvec{1&0&0&\vrule&\frac{2}{9}\\0&1&0&\vrule&\frac{4}{9}\\0&0&1&\vrule&\frac{1}{3}}$\\
		&\\
		& $\therefore$, stationary probability distribution $\pi$ is given by \\
		& \qquad \qquad $\pi = \myvec{\frac{2}{9} & \frac{4}{9} & \frac{1}{3}}$ \\
		& \\
		\hline
		\multirow{3}{*}{Observations} & \\
		
		
		& Since the given transition probability matrix $\vec{P}$ is irreducible and aperiodic, \\
		& then $\lim_{n \to \infty} \vec{P}^{n}$ converges to a matrix with all rows identical and equal to $\vec{\pi}$. \\
		& \\
		& We were able to find $\vec{\pi}$ as $\myvec{\frac{2}{9} & \frac{4}{9} & \frac{1}{3}}$ \\
		& \\
		& $\lim_{n \to \infty} \vec{P}^{n} = \myvec{\frac{2}{9}&\frac{4}{9}&\frac{1}{3}\\\frac{2}{9}&\frac{4}{9}&\frac{1}{3}\\\frac{2}{9}&\frac{4}{9}&\frac{1}{3}}$\\
		& \\
		& From the above matrix, we get \\
		& \\
		& $\lim_{n \to \infty} \vec{P}^{n}_{11} = \frac{2}{9}$ \\
		&\\
		& $\lim_{n \to \infty} \vec{P}^{n}_{21} = \frac{2}{9}$ \\
		&\\
		& $\lim_{n \to \infty} \vec{P}^{n}_{32} = \frac{4}{9}$ \\
		&\\
		& $\lim_{n \to \infty} \vec{P}^{n}_{13} = \frac{1}{3}$ \\
		&\\
		\hline
		\multirow{3}{*}{Conclusion} & \\
		& From our observation we see that \\
		&\\
		& Options 1) and 4) are True.\\
		& \\
		\hline
\caption{}
\label{eq:solutions/2018/dec/106/table1}
	\end{longtable}
\twocolumn

%
\item Let $\vec{V}$ be a finite dimensional vector space over $\mathbb{R}$. Let $T:\vec{V}\rightarrow\vec{V}$ be a linear transformation such that $rank(\vec{T}^2)=rank(\vec{T})$. Then,
\begin{enumerate}
    \item $Kernel(\vec{T}^2)=Kernel(\vec{T})$
    \item $Range(\vec{T}^2)=Range(\vec{T})$
    \item $Kernel(\vec{T})\cap Range(\vec{T})=\cbrak{0}$.
    \item $Kernel(\vec{T}^2)\cap Range(\vec{T}^2)=\cbrak{0}$.
\end{enumerate}
%
\solution
See Tables \ref{eq:solutions/2018/dec/106/table0} and \ref{eq:solutions/2018/dec/106/table1}


\onecolumn
	\begin{longtable}{|l|l|}
		\hline
		\multirow{3}{*}{Irreducible Markov Chain} 
		& \\
		& A Markov chain is $\textbf{irreducible}$ if all the states communicate with each other,\\
		& i.e., if there is only one communication class.\\
		&\\
		\hline
		\multirow{3}{*}{Aperiodic Markov Chain} & \\
		& If there is a self-transition in the chain ($p^{ii}>0$ for some i), then the chain is\\
		& called as $\textbf{aperiodic}$\\
		& \\
		\hline
		\multirow{3}{*}{Stationary Distribution} & \\
		& A stationary distribution of a Markov chain is a probability distribution that\\
		& remains unchanged in the Markov chain as time progresses. Typically, it is\\
		& represented as a row vector $\Vec{\pi}$ whose entries are probabilities summing to 1,\\ 
		& and given transition matrix $\textbf{P}$, it satisfies\\
		& \\
		&  \qquad \qquad  \qquad$\Vec{\pi} = \Vec{\pi} \textbf{P}$\\
		& \\
		\hline
\caption{}
\label{eq:solutions/2018/dec/106/table0}
	\end{longtable}
	\begin{longtable}{|l|l|}
		\hline
		\multirow{3}{*}{Drawing Transition diagram} 
		& \\
		& 
		
		$\begin{tikzpicture}[shorten >=1pt,node distance=2cm, scale =3, auto]
			\tikzstyle{every state}=[fill={rgb:black,1;white,10}]
			
			\node[state]   (q_1)                          {$1$};
			\node[state]   (q_2)  [right of=q_1]          {$2$};
			\node[state]   (q_3)  [below right of=q_1]          {$3$};
			
			\path[->]
			(q_1) edge [loop above] node {$\frac{1}{2}$}    (   )
			edge [bend left]  node {$\frac{1}{2}$}    (q_2)
			(q_2) edge [bend left]  node {$\frac{1}{2}$}    (q_3)
			edge [loop above] node {$\frac{1}{2}$}    ()
			(q_3) edge [bend left]  node {$\frac{1}{3}$}    (q_2)
			edge [bend left]  node {$\frac{1}{3}$}    (q_1)
			edge [loop below] node {$\frac{1}{3}$}    ();
		\end{tikzpicture}$
		
		\\  
		&\\
		&\\
		\hline
		\multirow{3}{*}{Checking whether the  } & \\
		& Here,\\chain is Irreducible
		& All the states are accessible to one another. \\and Aperiodic
		& $\implies$ They are in the same communication class. So, it is Irreducible.\\
		& \\
		& There exists the non- zero self-transition, which means that the chain \\
		& is Aperiodic.\\
		&\\ 
		& We know that if the Markov Chain is irreducible and aperiodic then \\
		& \qquad \qquad \qquad $\Vec{\pi}_{j} = \lim_{n \to \infty}P\{X_{n} = j\}$, $j = 1,...,N$ \\
		& These are the stationary probabilities. \\
		&\\
		\hline
		\multirow{3}{*}{Finding the Stationary} & \\
		& Stationary Probability can be represented as\\Probability Distributions
		& \qquad \qquad \qquad $\Vec{\pi} = \Vec{\pi} \vec{P}$\\
		& \\
		& \qquad $\implies$ $\myvec{v_{1}&&v_{2}&&v_{3}} = \myvec{v_{1}&&v_{2}&&v_{3}}\vec{P}$ \\
		& \\
		& Equating the above equation we get \\
		& \\
		& \qquad \qquad \qquad $\frac{1}{2}v_{1}-\frac{1}{3}v_{3} = 0$ $\label{eq:solutions/2018/dec/106/eq}$\\
		& \\
		& \qquad \qquad \qquad $\frac{1}{2}v_{1}-\frac{1}{2}v_{2} + \frac{1}{3}v_{3} = 0$\\
		& \\
		& \qquad \qquad \qquad $\frac{1}{2}v_{2}-\frac{2}{3}v_{3} = 0$\\
		& \\\
		& We see that summation of second and the third equation gives us the \\
		& first equation only. \\
		& And we know that the probability distribution will sum up to 1. \\
		& \\
		& \qquad \qquad \qquad $v_{1}+v_{2}+v_{3} = 1$ \\
		& \\
		& Therefore, we get the equation form as \\
		& \\
		& \qquad \qquad \qquad $\myvec{1&1&1\\\frac{1}{2}&0&\frac{-1}{3}\\\frac{1}{2}&\frac{-1}{2}&\frac{1}{3}}\myvec{v_{1}\\v_{2}\\v_{3}} = \myvec{1\\0\\0}$ \\
		& \\
		\hline
		\multirow{3}{*}{Solving the linear} & \\
		& The above linear equation can be solved using Gauss-Jordan method as\\equtions
		& \\
		& \qquad \qquad \qquad $\myvec{1&1&1&\vrule&1\\\frac{1}{2}&0&\frac{-1}{3}&\vrule&0\\\frac{1}{2}&\frac{-1}{2}&\frac{1}{3}&\vrule&0}$\\
		& \\
		& \qquad $\xleftrightarrow[]{R_2 \leftarrow R_2 - \frac{1}{2}R_1}$
		$\myvec{1&1&1&\vrule&1\\0&\frac{-1}{2}&\frac{-5}{6}&\vrule&\frac{-1}{2}\\\frac{1}{2}&\frac{-1}{2}&\frac{1}{3}&\vrule&0}$\\
		&\\
		& \qquad $\xleftrightarrow[]{R_3 \leftarrow R_3 - \frac{1}{2}R_1}$
		$\myvec{1&1&1&\vrule&1\\0&\frac{-1}{2}&\frac{-5}{6}&\vrule&\frac{-1}{2}\\0&-1&\frac{-1}{6}&\vrule&\frac{-1}{2}}$\\
		&\\
		& \qquad $\xleftrightarrow[]{R_2 \leftarrow \frac{-1}{2}R_2}$
		$\myvec{1&1&1&\vrule&1\\0&1&\frac{5}{3}&\vrule&1\\0&-1&\frac{-1}{6}&\vrule&\frac{-1}{2}}$\\
		&\\
		& \qquad $\xleftrightarrow[]{R_3 \leftarrow R_3 + R_2}$
		$\myvec{1&1&1&\vrule&1\\0&1&\frac{5}{3}&\vrule&1\\0&0&\frac{3}{2}&\vrule&\frac{1}{2}}$\\
		&\\
		& \qquad $\xleftrightarrow[]{R_3 \leftarrow \frac{3}{2}R_3}$
		$\myvec{1&1&1&\vrule&1\\0&1&\frac{5}{3}&\vrule&1\\0&0&1&\vrule&\frac{1}{3}}$\\
		&\\
		& \qquad $\xleftrightarrow[]{R_2 \leftarrow R_2 - \frac{5}{3}R_3}$
		$\myvec{1&1&1&\vrule&1\\0&1&0&\vrule&\frac{4}{9}\\0&0&1&\vrule&\frac{1}{3}}$\\
		&\\
		& \qquad $\xleftrightarrow[]{R_1 \leftarrow R_1 - R_3}$
		$\myvec{1&1&0&\vrule&\frac{2}{3}\\0&1&0&\vrule&\frac{4}{9}\\0&0&1&\vrule&\frac{1}{3}}$\\
		&\\
		& \qquad $\xleftrightarrow[]{R_1 \leftarrow R_1 - R_2}$
		$\myvec{1&0&0&\vrule&\frac{2}{9}\\0&1&0&\vrule&\frac{4}{9}\\0&0&1&\vrule&\frac{1}{3}}$\\
		&\\
		& $\therefore$, stationary probability distribution $\pi$ is given by \\
		& \qquad \qquad $\pi = \myvec{\frac{2}{9} & \frac{4}{9} & \frac{1}{3}}$ \\
		& \\
		\hline
		\multirow{3}{*}{Observations} & \\
		
		
		& Since the given transition probability matrix $\vec{P}$ is irreducible and aperiodic, \\
		& then $\lim_{n \to \infty} \vec{P}^{n}$ converges to a matrix with all rows identical and equal to $\vec{\pi}$. \\
		& \\
		& We were able to find $\vec{\pi}$ as $\myvec{\frac{2}{9} & \frac{4}{9} & \frac{1}{3}}$ \\
		& \\
		& $\lim_{n \to \infty} \vec{P}^{n} = \myvec{\frac{2}{9}&\frac{4}{9}&\frac{1}{3}\\\frac{2}{9}&\frac{4}{9}&\frac{1}{3}\\\frac{2}{9}&\frac{4}{9}&\frac{1}{3}}$\\
		& \\
		& From the above matrix, we get \\
		& \\
		& $\lim_{n \to \infty} \vec{P}^{n}_{11} = \frac{2}{9}$ \\
		&\\
		& $\lim_{n \to \infty} \vec{P}^{n}_{21} = \frac{2}{9}$ \\
		&\\
		& $\lim_{n \to \infty} \vec{P}^{n}_{32} = \frac{4}{9}$ \\
		&\\
		& $\lim_{n \to \infty} \vec{P}^{n}_{13} = \frac{1}{3}$ \\
		&\\
		\hline
		\multirow{3}{*}{Conclusion} & \\
		& From our observation we see that \\
		&\\
		& Options 1) and 4) are True.\\
		& \\
		\hline
\caption{}
\label{eq:solutions/2018/dec/106/table1}
	\end{longtable}
\twocolumn

\item  Let $\vec{A}$ be an m x n real matrix and $\vec{b}\in \mathbb{R}^m$ with $b\neq 0$.
\begin{enumerate}
    \item The set of all real solutions of $\vec{A}x=\vec{b}$ is a vector space.\\
    \item If u nd v are two solutions of $\vec{A}x=\vec{b}$ then $\lambda u  +\brak{1-\lambda}v$ is also a solution of $\vec{A}x=\vec{b}$\\
    \item For any two solutions u and v of $\vec{A}x=\vec{b}$, the linear combination $\lambda u$ + $\brak{1-\lambda}v$ is also a solution of $\vec{A}x=\vec{b}$ only when $0\leq\lambda\leq1.$\\
    \item If rank of $\vec{A}$ is n ,then $\vec{A}x=\vec{b}$ has at most one solution.\ 
    \end{enumerate}
%
\solution
See Tables \ref{eq:solutions/2018/dec/106/table0} and \ref{eq:solutions/2018/dec/106/table1}


\onecolumn
	\begin{longtable}{|l|l|}
		\hline
		\multirow{3}{*}{Irreducible Markov Chain} 
		& \\
		& A Markov chain is $\textbf{irreducible}$ if all the states communicate with each other,\\
		& i.e., if there is only one communication class.\\
		&\\
		\hline
		\multirow{3}{*}{Aperiodic Markov Chain} & \\
		& If there is a self-transition in the chain ($p^{ii}>0$ for some i), then the chain is\\
		& called as $\textbf{aperiodic}$\\
		& \\
		\hline
		\multirow{3}{*}{Stationary Distribution} & \\
		& A stationary distribution of a Markov chain is a probability distribution that\\
		& remains unchanged in the Markov chain as time progresses. Typically, it is\\
		& represented as a row vector $\Vec{\pi}$ whose entries are probabilities summing to 1,\\ 
		& and given transition matrix $\textbf{P}$, it satisfies\\
		& \\
		&  \qquad \qquad  \qquad$\Vec{\pi} = \Vec{\pi} \textbf{P}$\\
		& \\
		\hline
\caption{}
\label{eq:solutions/2018/dec/106/table0}
	\end{longtable}
	\begin{longtable}{|l|l|}
		\hline
		\multirow{3}{*}{Drawing Transition diagram} 
		& \\
		& 
		
		$\begin{tikzpicture}[shorten >=1pt,node distance=2cm, scale =3, auto]
			\tikzstyle{every state}=[fill={rgb:black,1;white,10}]
			
			\node[state]   (q_1)                          {$1$};
			\node[state]   (q_2)  [right of=q_1]          {$2$};
			\node[state]   (q_3)  [below right of=q_1]          {$3$};
			
			\path[->]
			(q_1) edge [loop above] node {$\frac{1}{2}$}    (   )
			edge [bend left]  node {$\frac{1}{2}$}    (q_2)
			(q_2) edge [bend left]  node {$\frac{1}{2}$}    (q_3)
			edge [loop above] node {$\frac{1}{2}$}    ()
			(q_3) edge [bend left]  node {$\frac{1}{3}$}    (q_2)
			edge [bend left]  node {$\frac{1}{3}$}    (q_1)
			edge [loop below] node {$\frac{1}{3}$}    ();
		\end{tikzpicture}$
		
		\\  
		&\\
		&\\
		\hline
		\multirow{3}{*}{Checking whether the  } & \\
		& Here,\\chain is Irreducible
		& All the states are accessible to one another. \\and Aperiodic
		& $\implies$ They are in the same communication class. So, it is Irreducible.\\
		& \\
		& There exists the non- zero self-transition, which means that the chain \\
		& is Aperiodic.\\
		&\\ 
		& We know that if the Markov Chain is irreducible and aperiodic then \\
		& \qquad \qquad \qquad $\Vec{\pi}_{j} = \lim_{n \to \infty}P\{X_{n} = j\}$, $j = 1,...,N$ \\
		& These are the stationary probabilities. \\
		&\\
		\hline
		\multirow{3}{*}{Finding the Stationary} & \\
		& Stationary Probability can be represented as\\Probability Distributions
		& \qquad \qquad \qquad $\Vec{\pi} = \Vec{\pi} \vec{P}$\\
		& \\
		& \qquad $\implies$ $\myvec{v_{1}&&v_{2}&&v_{3}} = \myvec{v_{1}&&v_{2}&&v_{3}}\vec{P}$ \\
		& \\
		& Equating the above equation we get \\
		& \\
		& \qquad \qquad \qquad $\frac{1}{2}v_{1}-\frac{1}{3}v_{3} = 0$ $\label{eq:solutions/2018/dec/106/eq}$\\
		& \\
		& \qquad \qquad \qquad $\frac{1}{2}v_{1}-\frac{1}{2}v_{2} + \frac{1}{3}v_{3} = 0$\\
		& \\
		& \qquad \qquad \qquad $\frac{1}{2}v_{2}-\frac{2}{3}v_{3} = 0$\\
		& \\\
		& We see that summation of second and the third equation gives us the \\
		& first equation only. \\
		& And we know that the probability distribution will sum up to 1. \\
		& \\
		& \qquad \qquad \qquad $v_{1}+v_{2}+v_{3} = 1$ \\
		& \\
		& Therefore, we get the equation form as \\
		& \\
		& \qquad \qquad \qquad $\myvec{1&1&1\\\frac{1}{2}&0&\frac{-1}{3}\\\frac{1}{2}&\frac{-1}{2}&\frac{1}{3}}\myvec{v_{1}\\v_{2}\\v_{3}} = \myvec{1\\0\\0}$ \\
		& \\
		\hline
		\multirow{3}{*}{Solving the linear} & \\
		& The above linear equation can be solved using Gauss-Jordan method as\\equtions
		& \\
		& \qquad \qquad \qquad $\myvec{1&1&1&\vrule&1\\\frac{1}{2}&0&\frac{-1}{3}&\vrule&0\\\frac{1}{2}&\frac{-1}{2}&\frac{1}{3}&\vrule&0}$\\
		& \\
		& \qquad $\xleftrightarrow[]{R_2 \leftarrow R_2 - \frac{1}{2}R_1}$
		$\myvec{1&1&1&\vrule&1\\0&\frac{-1}{2}&\frac{-5}{6}&\vrule&\frac{-1}{2}\\\frac{1}{2}&\frac{-1}{2}&\frac{1}{3}&\vrule&0}$\\
		&\\
		& \qquad $\xleftrightarrow[]{R_3 \leftarrow R_3 - \frac{1}{2}R_1}$
		$\myvec{1&1&1&\vrule&1\\0&\frac{-1}{2}&\frac{-5}{6}&\vrule&\frac{-1}{2}\\0&-1&\frac{-1}{6}&\vrule&\frac{-1}{2}}$\\
		&\\
		& \qquad $\xleftrightarrow[]{R_2 \leftarrow \frac{-1}{2}R_2}$
		$\myvec{1&1&1&\vrule&1\\0&1&\frac{5}{3}&\vrule&1\\0&-1&\frac{-1}{6}&\vrule&\frac{-1}{2}}$\\
		&\\
		& \qquad $\xleftrightarrow[]{R_3 \leftarrow R_3 + R_2}$
		$\myvec{1&1&1&\vrule&1\\0&1&\frac{5}{3}&\vrule&1\\0&0&\frac{3}{2}&\vrule&\frac{1}{2}}$\\
		&\\
		& \qquad $\xleftrightarrow[]{R_3 \leftarrow \frac{3}{2}R_3}$
		$\myvec{1&1&1&\vrule&1\\0&1&\frac{5}{3}&\vrule&1\\0&0&1&\vrule&\frac{1}{3}}$\\
		&\\
		& \qquad $\xleftrightarrow[]{R_2 \leftarrow R_2 - \frac{5}{3}R_3}$
		$\myvec{1&1&1&\vrule&1\\0&1&0&\vrule&\frac{4}{9}\\0&0&1&\vrule&\frac{1}{3}}$\\
		&\\
		& \qquad $\xleftrightarrow[]{R_1 \leftarrow R_1 - R_3}$
		$\myvec{1&1&0&\vrule&\frac{2}{3}\\0&1&0&\vrule&\frac{4}{9}\\0&0&1&\vrule&\frac{1}{3}}$\\
		&\\
		& \qquad $\xleftrightarrow[]{R_1 \leftarrow R_1 - R_2}$
		$\myvec{1&0&0&\vrule&\frac{2}{9}\\0&1&0&\vrule&\frac{4}{9}\\0&0&1&\vrule&\frac{1}{3}}$\\
		&\\
		& $\therefore$, stationary probability distribution $\pi$ is given by \\
		& \qquad \qquad $\pi = \myvec{\frac{2}{9} & \frac{4}{9} & \frac{1}{3}}$ \\
		& \\
		\hline
		\multirow{3}{*}{Observations} & \\
		
		
		& Since the given transition probability matrix $\vec{P}$ is irreducible and aperiodic, \\
		& then $\lim_{n \to \infty} \vec{P}^{n}$ converges to a matrix with all rows identical and equal to $\vec{\pi}$. \\
		& \\
		& We were able to find $\vec{\pi}$ as $\myvec{\frac{2}{9} & \frac{4}{9} & \frac{1}{3}}$ \\
		& \\
		& $\lim_{n \to \infty} \vec{P}^{n} = \myvec{\frac{2}{9}&\frac{4}{9}&\frac{1}{3}\\\frac{2}{9}&\frac{4}{9}&\frac{1}{3}\\\frac{2}{9}&\frac{4}{9}&\frac{1}{3}}$\\
		& \\
		& From the above matrix, we get \\
		& \\
		& $\lim_{n \to \infty} \vec{P}^{n}_{11} = \frac{2}{9}$ \\
		&\\
		& $\lim_{n \to \infty} \vec{P}^{n}_{21} = \frac{2}{9}$ \\
		&\\
		& $\lim_{n \to \infty} \vec{P}^{n}_{32} = \frac{4}{9}$ \\
		&\\
		& $\lim_{n \to \infty} \vec{P}^{n}_{13} = \frac{1}{3}$ \\
		&\\
		\hline
		\multirow{3}{*}{Conclusion} & \\
		& From our observation we see that \\
		&\\
		& Options 1) and 4) are True.\\
		& \\
		\hline
\caption{}
\label{eq:solutions/2018/dec/106/table1}
	\end{longtable}
\twocolumn

\item Let $\vec{A}$ be an $n\times n$ matrix over $\mathbb{C}$ such that every non-zero vector $\mathbb{C}^n$ is an eigen vector of $\vec{A}$.Then
\begin{enumerate}
    \item All eigen values of $\vec{A}$ are equal.
    \item All eigen values of $\vec{A}$ are distinct.
    \item $\vec{A}=\lambda\vec{I}$ for some $\lambda \in \mathbb{C}$, where $\vec{I}$ is the $n\times n$ identity matrix.
    \item If $\chi_\vec{A}$ and $m_\vec{A}$ denote the characteristic polynomial and the minimal polynomial respectively, then $\chi_\vec{A}=m_\vec{A}$
\end{enumerate}
%

\solution
See Tables \ref{eq:solutions/2018/dec/106/table0} and \ref{eq:solutions/2018/dec/106/table1}


\onecolumn
	\begin{longtable}{|l|l|}
		\hline
		\multirow{3}{*}{Irreducible Markov Chain} 
		& \\
		& A Markov chain is $\textbf{irreducible}$ if all the states communicate with each other,\\
		& i.e., if there is only one communication class.\\
		&\\
		\hline
		\multirow{3}{*}{Aperiodic Markov Chain} & \\
		& If there is a self-transition in the chain ($p^{ii}>0$ for some i), then the chain is\\
		& called as $\textbf{aperiodic}$\\
		& \\
		\hline
		\multirow{3}{*}{Stationary Distribution} & \\
		& A stationary distribution of a Markov chain is a probability distribution that\\
		& remains unchanged in the Markov chain as time progresses. Typically, it is\\
		& represented as a row vector $\Vec{\pi}$ whose entries are probabilities summing to 1,\\ 
		& and given transition matrix $\textbf{P}$, it satisfies\\
		& \\
		&  \qquad \qquad  \qquad$\Vec{\pi} = \Vec{\pi} \textbf{P}$\\
		& \\
		\hline
\caption{}
\label{eq:solutions/2018/dec/106/table0}
	\end{longtable}
	\begin{longtable}{|l|l|}
		\hline
		\multirow{3}{*}{Drawing Transition diagram} 
		& \\
		& 
		
		$\begin{tikzpicture}[shorten >=1pt,node distance=2cm, scale =3, auto]
			\tikzstyle{every state}=[fill={rgb:black,1;white,10}]
			
			\node[state]   (q_1)                          {$1$};
			\node[state]   (q_2)  [right of=q_1]          {$2$};
			\node[state]   (q_3)  [below right of=q_1]          {$3$};
			
			\path[->]
			(q_1) edge [loop above] node {$\frac{1}{2}$}    (   )
			edge [bend left]  node {$\frac{1}{2}$}    (q_2)
			(q_2) edge [bend left]  node {$\frac{1}{2}$}    (q_3)
			edge [loop above] node {$\frac{1}{2}$}    ()
			(q_3) edge [bend left]  node {$\frac{1}{3}$}    (q_2)
			edge [bend left]  node {$\frac{1}{3}$}    (q_1)
			edge [loop below] node {$\frac{1}{3}$}    ();
		\end{tikzpicture}$
		
		\\  
		&\\
		&\\
		\hline
		\multirow{3}{*}{Checking whether the  } & \\
		& Here,\\chain is Irreducible
		& All the states are accessible to one another. \\and Aperiodic
		& $\implies$ They are in the same communication class. So, it is Irreducible.\\
		& \\
		& There exists the non- zero self-transition, which means that the chain \\
		& is Aperiodic.\\
		&\\ 
		& We know that if the Markov Chain is irreducible and aperiodic then \\
		& \qquad \qquad \qquad $\Vec{\pi}_{j} = \lim_{n \to \infty}P\{X_{n} = j\}$, $j = 1,...,N$ \\
		& These are the stationary probabilities. \\
		&\\
		\hline
		\multirow{3}{*}{Finding the Stationary} & \\
		& Stationary Probability can be represented as\\Probability Distributions
		& \qquad \qquad \qquad $\Vec{\pi} = \Vec{\pi} \vec{P}$\\
		& \\
		& \qquad $\implies$ $\myvec{v_{1}&&v_{2}&&v_{3}} = \myvec{v_{1}&&v_{2}&&v_{3}}\vec{P}$ \\
		& \\
		& Equating the above equation we get \\
		& \\
		& \qquad \qquad \qquad $\frac{1}{2}v_{1}-\frac{1}{3}v_{3} = 0$ $\label{eq:solutions/2018/dec/106/eq}$\\
		& \\
		& \qquad \qquad \qquad $\frac{1}{2}v_{1}-\frac{1}{2}v_{2} + \frac{1}{3}v_{3} = 0$\\
		& \\
		& \qquad \qquad \qquad $\frac{1}{2}v_{2}-\frac{2}{3}v_{3} = 0$\\
		& \\\
		& We see that summation of second and the third equation gives us the \\
		& first equation only. \\
		& And we know that the probability distribution will sum up to 1. \\
		& \\
		& \qquad \qquad \qquad $v_{1}+v_{2}+v_{3} = 1$ \\
		& \\
		& Therefore, we get the equation form as \\
		& \\
		& \qquad \qquad \qquad $\myvec{1&1&1\\\frac{1}{2}&0&\frac{-1}{3}\\\frac{1}{2}&\frac{-1}{2}&\frac{1}{3}}\myvec{v_{1}\\v_{2}\\v_{3}} = \myvec{1\\0\\0}$ \\
		& \\
		\hline
		\multirow{3}{*}{Solving the linear} & \\
		& The above linear equation can be solved using Gauss-Jordan method as\\equtions
		& \\
		& \qquad \qquad \qquad $\myvec{1&1&1&\vrule&1\\\frac{1}{2}&0&\frac{-1}{3}&\vrule&0\\\frac{1}{2}&\frac{-1}{2}&\frac{1}{3}&\vrule&0}$\\
		& \\
		& \qquad $\xleftrightarrow[]{R_2 \leftarrow R_2 - \frac{1}{2}R_1}$
		$\myvec{1&1&1&\vrule&1\\0&\frac{-1}{2}&\frac{-5}{6}&\vrule&\frac{-1}{2}\\\frac{1}{2}&\frac{-1}{2}&\frac{1}{3}&\vrule&0}$\\
		&\\
		& \qquad $\xleftrightarrow[]{R_3 \leftarrow R_3 - \frac{1}{2}R_1}$
		$\myvec{1&1&1&\vrule&1\\0&\frac{-1}{2}&\frac{-5}{6}&\vrule&\frac{-1}{2}\\0&-1&\frac{-1}{6}&\vrule&\frac{-1}{2}}$\\
		&\\
		& \qquad $\xleftrightarrow[]{R_2 \leftarrow \frac{-1}{2}R_2}$
		$\myvec{1&1&1&\vrule&1\\0&1&\frac{5}{3}&\vrule&1\\0&-1&\frac{-1}{6}&\vrule&\frac{-1}{2}}$\\
		&\\
		& \qquad $\xleftrightarrow[]{R_3 \leftarrow R_3 + R_2}$
		$\myvec{1&1&1&\vrule&1\\0&1&\frac{5}{3}&\vrule&1\\0&0&\frac{3}{2}&\vrule&\frac{1}{2}}$\\
		&\\
		& \qquad $\xleftrightarrow[]{R_3 \leftarrow \frac{3}{2}R_3}$
		$\myvec{1&1&1&\vrule&1\\0&1&\frac{5}{3}&\vrule&1\\0&0&1&\vrule&\frac{1}{3}}$\\
		&\\
		& \qquad $\xleftrightarrow[]{R_2 \leftarrow R_2 - \frac{5}{3}R_3}$
		$\myvec{1&1&1&\vrule&1\\0&1&0&\vrule&\frac{4}{9}\\0&0&1&\vrule&\frac{1}{3}}$\\
		&\\
		& \qquad $\xleftrightarrow[]{R_1 \leftarrow R_1 - R_3}$
		$\myvec{1&1&0&\vrule&\frac{2}{3}\\0&1&0&\vrule&\frac{4}{9}\\0&0&1&\vrule&\frac{1}{3}}$\\
		&\\
		& \qquad $\xleftrightarrow[]{R_1 \leftarrow R_1 - R_2}$
		$\myvec{1&0&0&\vrule&\frac{2}{9}\\0&1&0&\vrule&\frac{4}{9}\\0&0&1&\vrule&\frac{1}{3}}$\\
		&\\
		& $\therefore$, stationary probability distribution $\pi$ is given by \\
		& \qquad \qquad $\pi = \myvec{\frac{2}{9} & \frac{4}{9} & \frac{1}{3}}$ \\
		& \\
		\hline
		\multirow{3}{*}{Observations} & \\
		
		
		& Since the given transition probability matrix $\vec{P}$ is irreducible and aperiodic, \\
		& then $\lim_{n \to \infty} \vec{P}^{n}$ converges to a matrix with all rows identical and equal to $\vec{\pi}$. \\
		& \\
		& We were able to find $\vec{\pi}$ as $\myvec{\frac{2}{9} & \frac{4}{9} & \frac{1}{3}}$ \\
		& \\
		& $\lim_{n \to \infty} \vec{P}^{n} = \myvec{\frac{2}{9}&\frac{4}{9}&\frac{1}{3}\\\frac{2}{9}&\frac{4}{9}&\frac{1}{3}\\\frac{2}{9}&\frac{4}{9}&\frac{1}{3}}$\\
		& \\
		& From the above matrix, we get \\
		& \\
		& $\lim_{n \to \infty} \vec{P}^{n}_{11} = \frac{2}{9}$ \\
		&\\
		& $\lim_{n \to \infty} \vec{P}^{n}_{21} = \frac{2}{9}$ \\
		&\\
		& $\lim_{n \to \infty} \vec{P}^{n}_{32} = \frac{4}{9}$ \\
		&\\
		& $\lim_{n \to \infty} \vec{P}^{n}_{13} = \frac{1}{3}$ \\
		&\\
		\hline
		\multirow{3}{*}{Conclusion} & \\
		& From our observation we see that \\
		&\\
		& Options 1) and 4) are True.\\
		& \\
		\hline
\caption{}
\label{eq:solutions/2018/dec/106/table1}
	\end{longtable}
\twocolumn

\item 	Consider a matrix,
	\begin{align}
	\vec{A} = \myvec{2 & 2 & 1 \\0 & 2 & -1 \\ 0 & 0 & 3}\\ \intertext{and,} \vec{B} = \myvec{2 & 1 & 0 \\ 0 & 2 & 0 \\ 0 & 0 & 3}
	\end{align}
	
Then which of following is true,
\begin{enumerate}
\item $\vec{A}$ and $\vec{B}$ is similar over the field of rational numbers.
\item $\vec{A}$ is diagonalizable over the field of rational numbers $\mathbb{Q}$.
\item $\vec{B}$ is the Jordan canonical form of $\vec{A}$.
\item The minimal polynomial and the characteristic polynomial of $\vec{A}$ are the same.
\end{enumerate}
%
\solution
See Tables \ref{eq:solutions/2018/dec/106/table0} and \ref{eq:solutions/2018/dec/106/table1}


\onecolumn
	\begin{longtable}{|l|l|}
		\hline
		\multirow{3}{*}{Irreducible Markov Chain} 
		& \\
		& A Markov chain is $\textbf{irreducible}$ if all the states communicate with each other,\\
		& i.e., if there is only one communication class.\\
		&\\
		\hline
		\multirow{3}{*}{Aperiodic Markov Chain} & \\
		& If there is a self-transition in the chain ($p^{ii}>0$ for some i), then the chain is\\
		& called as $\textbf{aperiodic}$\\
		& \\
		\hline
		\multirow{3}{*}{Stationary Distribution} & \\
		& A stationary distribution of a Markov chain is a probability distribution that\\
		& remains unchanged in the Markov chain as time progresses. Typically, it is\\
		& represented as a row vector $\Vec{\pi}$ whose entries are probabilities summing to 1,\\ 
		& and given transition matrix $\textbf{P}$, it satisfies\\
		& \\
		&  \qquad \qquad  \qquad$\Vec{\pi} = \Vec{\pi} \textbf{P}$\\
		& \\
		\hline
\caption{}
\label{eq:solutions/2018/dec/106/table0}
	\end{longtable}
	\begin{longtable}{|l|l|}
		\hline
		\multirow{3}{*}{Drawing Transition diagram} 
		& \\
		& 
		
		$\begin{tikzpicture}[shorten >=1pt,node distance=2cm, scale =3, auto]
			\tikzstyle{every state}=[fill={rgb:black,1;white,10}]
			
			\node[state]   (q_1)                          {$1$};
			\node[state]   (q_2)  [right of=q_1]          {$2$};
			\node[state]   (q_3)  [below right of=q_1]          {$3$};
			
			\path[->]
			(q_1) edge [loop above] node {$\frac{1}{2}$}    (   )
			edge [bend left]  node {$\frac{1}{2}$}    (q_2)
			(q_2) edge [bend left]  node {$\frac{1}{2}$}    (q_3)
			edge [loop above] node {$\frac{1}{2}$}    ()
			(q_3) edge [bend left]  node {$\frac{1}{3}$}    (q_2)
			edge [bend left]  node {$\frac{1}{3}$}    (q_1)
			edge [loop below] node {$\frac{1}{3}$}    ();
		\end{tikzpicture}$
		
		\\  
		&\\
		&\\
		\hline
		\multirow{3}{*}{Checking whether the  } & \\
		& Here,\\chain is Irreducible
		& All the states are accessible to one another. \\and Aperiodic
		& $\implies$ They are in the same communication class. So, it is Irreducible.\\
		& \\
		& There exists the non- zero self-transition, which means that the chain \\
		& is Aperiodic.\\
		&\\ 
		& We know that if the Markov Chain is irreducible and aperiodic then \\
		& \qquad \qquad \qquad $\Vec{\pi}_{j} = \lim_{n \to \infty}P\{X_{n} = j\}$, $j = 1,...,N$ \\
		& These are the stationary probabilities. \\
		&\\
		\hline
		\multirow{3}{*}{Finding the Stationary} & \\
		& Stationary Probability can be represented as\\Probability Distributions
		& \qquad \qquad \qquad $\Vec{\pi} = \Vec{\pi} \vec{P}$\\
		& \\
		& \qquad $\implies$ $\myvec{v_{1}&&v_{2}&&v_{3}} = \myvec{v_{1}&&v_{2}&&v_{3}}\vec{P}$ \\
		& \\
		& Equating the above equation we get \\
		& \\
		& \qquad \qquad \qquad $\frac{1}{2}v_{1}-\frac{1}{3}v_{3} = 0$ $\label{eq:solutions/2018/dec/106/eq}$\\
		& \\
		& \qquad \qquad \qquad $\frac{1}{2}v_{1}-\frac{1}{2}v_{2} + \frac{1}{3}v_{3} = 0$\\
		& \\
		& \qquad \qquad \qquad $\frac{1}{2}v_{2}-\frac{2}{3}v_{3} = 0$\\
		& \\\
		& We see that summation of second and the third equation gives us the \\
		& first equation only. \\
		& And we know that the probability distribution will sum up to 1. \\
		& \\
		& \qquad \qquad \qquad $v_{1}+v_{2}+v_{3} = 1$ \\
		& \\
		& Therefore, we get the equation form as \\
		& \\
		& \qquad \qquad \qquad $\myvec{1&1&1\\\frac{1}{2}&0&\frac{-1}{3}\\\frac{1}{2}&\frac{-1}{2}&\frac{1}{3}}\myvec{v_{1}\\v_{2}\\v_{3}} = \myvec{1\\0\\0}$ \\
		& \\
		\hline
		\multirow{3}{*}{Solving the linear} & \\
		& The above linear equation can be solved using Gauss-Jordan method as\\equtions
		& \\
		& \qquad \qquad \qquad $\myvec{1&1&1&\vrule&1\\\frac{1}{2}&0&\frac{-1}{3}&\vrule&0\\\frac{1}{2}&\frac{-1}{2}&\frac{1}{3}&\vrule&0}$\\
		& \\
		& \qquad $\xleftrightarrow[]{R_2 \leftarrow R_2 - \frac{1}{2}R_1}$
		$\myvec{1&1&1&\vrule&1\\0&\frac{-1}{2}&\frac{-5}{6}&\vrule&\frac{-1}{2}\\\frac{1}{2}&\frac{-1}{2}&\frac{1}{3}&\vrule&0}$\\
		&\\
		& \qquad $\xleftrightarrow[]{R_3 \leftarrow R_3 - \frac{1}{2}R_1}$
		$\myvec{1&1&1&\vrule&1\\0&\frac{-1}{2}&\frac{-5}{6}&\vrule&\frac{-1}{2}\\0&-1&\frac{-1}{6}&\vrule&\frac{-1}{2}}$\\
		&\\
		& \qquad $\xleftrightarrow[]{R_2 \leftarrow \frac{-1}{2}R_2}$
		$\myvec{1&1&1&\vrule&1\\0&1&\frac{5}{3}&\vrule&1\\0&-1&\frac{-1}{6}&\vrule&\frac{-1}{2}}$\\
		&\\
		& \qquad $\xleftrightarrow[]{R_3 \leftarrow R_3 + R_2}$
		$\myvec{1&1&1&\vrule&1\\0&1&\frac{5}{3}&\vrule&1\\0&0&\frac{3}{2}&\vrule&\frac{1}{2}}$\\
		&\\
		& \qquad $\xleftrightarrow[]{R_3 \leftarrow \frac{3}{2}R_3}$
		$\myvec{1&1&1&\vrule&1\\0&1&\frac{5}{3}&\vrule&1\\0&0&1&\vrule&\frac{1}{3}}$\\
		&\\
		& \qquad $\xleftrightarrow[]{R_2 \leftarrow R_2 - \frac{5}{3}R_3}$
		$\myvec{1&1&1&\vrule&1\\0&1&0&\vrule&\frac{4}{9}\\0&0&1&\vrule&\frac{1}{3}}$\\
		&\\
		& \qquad $\xleftrightarrow[]{R_1 \leftarrow R_1 - R_3}$
		$\myvec{1&1&0&\vrule&\frac{2}{3}\\0&1&0&\vrule&\frac{4}{9}\\0&0&1&\vrule&\frac{1}{3}}$\\
		&\\
		& \qquad $\xleftrightarrow[]{R_1 \leftarrow R_1 - R_2}$
		$\myvec{1&0&0&\vrule&\frac{2}{9}\\0&1&0&\vrule&\frac{4}{9}\\0&0&1&\vrule&\frac{1}{3}}$\\
		&\\
		& $\therefore$, stationary probability distribution $\pi$ is given by \\
		& \qquad \qquad $\pi = \myvec{\frac{2}{9} & \frac{4}{9} & \frac{1}{3}}$ \\
		& \\
		\hline
		\multirow{3}{*}{Observations} & \\
		
		
		& Since the given transition probability matrix $\vec{P}$ is irreducible and aperiodic, \\
		& then $\lim_{n \to \infty} \vec{P}^{n}$ converges to a matrix with all rows identical and equal to $\vec{\pi}$. \\
		& \\
		& We were able to find $\vec{\pi}$ as $\myvec{\frac{2}{9} & \frac{4}{9} & \frac{1}{3}}$ \\
		& \\
		& $\lim_{n \to \infty} \vec{P}^{n} = \myvec{\frac{2}{9}&\frac{4}{9}&\frac{1}{3}\\\frac{2}{9}&\frac{4}{9}&\frac{1}{3}\\\frac{2}{9}&\frac{4}{9}&\frac{1}{3}}$\\
		& \\
		& From the above matrix, we get \\
		& \\
		& $\lim_{n \to \infty} \vec{P}^{n}_{11} = \frac{2}{9}$ \\
		&\\
		& $\lim_{n \to \infty} \vec{P}^{n}_{21} = \frac{2}{9}$ \\
		&\\
		& $\lim_{n \to \infty} \vec{P}^{n}_{32} = \frac{4}{9}$ \\
		&\\
		& $\lim_{n \to \infty} \vec{P}^{n}_{13} = \frac{1}{3}$ \\
		&\\
		\hline
		\multirow{3}{*}{Conclusion} & \\
		& From our observation we see that \\
		&\\
		& Options 1) and 4) are True.\\
		& \\
		\hline
\caption{}
\label{eq:solutions/2018/dec/106/table1}
	\end{longtable}
\twocolumn


\end{enumerate}
