\renewcommand{\theequation}{\theenumi}
\renewcommand{\thefigure}{\theenumi}
\begin{enumerate}[label=\thesection.\arabic*.,ref=\thesection.\theenumi]
\numberwithin{equation}{enumi}
\numberwithin{figure}{enumi}
\numberwithin{table}{enumi}

\item Let $\vec{V}$ be the vector space of polynomials over $\mathbb{R}$ of degree less than or equal to $n$. For $p(x)=a_0+a_{n-1}x+..+a_nx^{n}$ in $\vec{V}$, define a linear transformation $\vec{T}:\vec{V}\rightarrow \vec{V}$ by $\brak{\vec{T}p}(x)=a_n+a_{n-1}x+..+a_0x^n.$ Then \\
\begin{enumerate}
    \item $\vec{T}$ is one to one.
    \item $\vec{T}$ is onto.
    \item $\vec{T}$ is invertible.
    \item $\det{\vec{T}}=\pm 1$.
\end{enumerate}
%
%
\solution
The nullspace is given by 
\begin{align}
	\myvec{1 & 1 & 1 & 0 \\ 1 & 1 & 0 & 1\\ 0 & 0 & 0 & 0\\0 & 0 & 0 & 0}\myvec{x\\y\\z\\w} = \myvec{0 \\ 0 \\ 0 \\ 0}
\end{align}	
Row reducing the above matrix we get,
\begin{align}
	\myvec{1 & 1 & 1 & 0 \\ 1 & 1 & 0 & 1\\ 0 & 0 & 0 & 0\\0 & 0 & 0 & 0}
	\xleftrightarrow[R_2 \leftarrow R_2 \times -1]{R_2 \leftarrow R_2 - R_1}
	\myvec{1 & 1 & 1 & 0 \\ 0 & 0 & 1 & -1\\ 0 & 0 & 0 & 0\\0 & 0 & 0 & 0}\\
	\xleftrightarrow{R_1 \leftarrow R_1- R_2}
	\myvec{1 & 1 & 0 & 1 \\ 0 & 0 & 1 & -1\\ 0 & 0 & 0 & 0\\0 & 0 & 0 & 0} \label{eq:solutions/2017/dec/27/eq:rref}
\end{align}
See Table \ref{eq:solutions/2017/dec/27/tab}

\begin{table*}[!ht]
	\begin{tabular}{|m{4.5cm}|l|}
		\hline
		&\\
		dim(C$(\vec{A})) = 1$ 
		& \textbf{False}. Because the number of pivot variables are 2 as obtained in \eqref{eq:solutions/2017/dec/27/eq:rref}\\
		&\\
		\hline
		&\\
		dim(C$(\vec{A})) = 2$
		& \textbf{True}. Since the number of pivot variables are 2, the rank of $\vec{A}$ is 2.\\
		&$\therefore dim(C(\vec{A})) = 2 \quad [\because dim(C(\vec{A})) = rank(\vec{A})]$ \\
		&\\
		\hline
		&\\
	     rank$(\vec{A}) = 1$
		& \textbf{False}. Because the rank$(\vec{A}) = 2$, as the number of pivot variables are 2\\
		&\\
		\hline
		&\\
		$\vec{S}$ = $\cbrak{(1, 1, 1, 0), (1, 1, 0, 1)}$ is a basis of $N(\vec{A})$
		& \textbf{False}. \\
		& Let, \\
		&  $\vec{u} = \myvec{1\\1\\1\\0}, \vec{v} = \myvec{1\\1\\0\\1}$\\ 
		&Consider, \\
		&$\myvec{1 & 1 & 1 & 0 \\ 1 & 1 & 0 & 1\\ 0 & 0 & 0 & 0\\0 & 0 & 0 & 0}\myvec{1\\1\\1\\0} = \myvec{3\\2\\0\\0} \not = \myvec{0\\0\\0\\0}$\\
		& Similarly,\\
		&$\myvec{1 & 1 & 1 & 0 \\ 1 & 1 & 0 & 1\\ 0 & 0 & 0 & 0\\0 & 0 & 0 & 0}\myvec{1\\1\\0\\1} = \myvec{2\\3\\0\\0} \not = \myvec{0\\0\\0\\0}$ \\
		&Hence, the given vectors do not form the basis.\\
		\hline
	\end{tabular}
\caption{}
\label{eq:solutions/2017/dec/27/tab}
\end{table*}

%
\item Let $\vec{V}$ be a finite dimensional vector space over $\mathbb{R}$. Let $T:\vec{V}\rightarrow\vec{V}$ be a linear transformation such that $rank(\vec{T}^2)=rank(\vec{T})$. Then,
\begin{enumerate}
    \item $Kernel(\vec{T}^2)=Kernel(\vec{T})$
    \item $Range(\vec{T}^2)=Range(\vec{T})$
    \item $Kernel(\vec{T})\cap Range(\vec{T})=\cbrak{0}$.
    \item $Kernel(\vec{T}^2)\cap Range(\vec{T}^2)=\cbrak{0}$.
\end{enumerate}
%
\solution
The nullspace is given by 
\begin{align}
	\myvec{1 & 1 & 1 & 0 \\ 1 & 1 & 0 & 1\\ 0 & 0 & 0 & 0\\0 & 0 & 0 & 0}\myvec{x\\y\\z\\w} = \myvec{0 \\ 0 \\ 0 \\ 0}
\end{align}	
Row reducing the above matrix we get,
\begin{align}
	\myvec{1 & 1 & 1 & 0 \\ 1 & 1 & 0 & 1\\ 0 & 0 & 0 & 0\\0 & 0 & 0 & 0}
	\xleftrightarrow[R_2 \leftarrow R_2 \times -1]{R_2 \leftarrow R_2 - R_1}
	\myvec{1 & 1 & 1 & 0 \\ 0 & 0 & 1 & -1\\ 0 & 0 & 0 & 0\\0 & 0 & 0 & 0}\\
	\xleftrightarrow{R_1 \leftarrow R_1- R_2}
	\myvec{1 & 1 & 0 & 1 \\ 0 & 0 & 1 & -1\\ 0 & 0 & 0 & 0\\0 & 0 & 0 & 0} \label{eq:solutions/2017/dec/27/eq:rref}
\end{align}
See Table \ref{eq:solutions/2017/dec/27/tab}

\begin{table*}[!ht]
	\begin{tabular}{|m{4.5cm}|l|}
		\hline
		&\\
		dim(C$(\vec{A})) = 1$ 
		& \textbf{False}. Because the number of pivot variables are 2 as obtained in \eqref{eq:solutions/2017/dec/27/eq:rref}\\
		&\\
		\hline
		&\\
		dim(C$(\vec{A})) = 2$
		& \textbf{True}. Since the number of pivot variables are 2, the rank of $\vec{A}$ is 2.\\
		&$\therefore dim(C(\vec{A})) = 2 \quad [\because dim(C(\vec{A})) = rank(\vec{A})]$ \\
		&\\
		\hline
		&\\
	     rank$(\vec{A}) = 1$
		& \textbf{False}. Because the rank$(\vec{A}) = 2$, as the number of pivot variables are 2\\
		&\\
		\hline
		&\\
		$\vec{S}$ = $\cbrak{(1, 1, 1, 0), (1, 1, 0, 1)}$ is a basis of $N(\vec{A})$
		& \textbf{False}. \\
		& Let, \\
		&  $\vec{u} = \myvec{1\\1\\1\\0}, \vec{v} = \myvec{1\\1\\0\\1}$\\ 
		&Consider, \\
		&$\myvec{1 & 1 & 1 & 0 \\ 1 & 1 & 0 & 1\\ 0 & 0 & 0 & 0\\0 & 0 & 0 & 0}\myvec{1\\1\\1\\0} = \myvec{3\\2\\0\\0} \not = \myvec{0\\0\\0\\0}$\\
		& Similarly,\\
		&$\myvec{1 & 1 & 1 & 0 \\ 1 & 1 & 0 & 1\\ 0 & 0 & 0 & 0\\0 & 0 & 0 & 0}\myvec{1\\1\\0\\1} = \myvec{2\\3\\0\\0} \not = \myvec{0\\0\\0\\0}$ \\
		&Hence, the given vectors do not form the basis.\\
		\hline
	\end{tabular}
\caption{}
\label{eq:solutions/2017/dec/27/tab}
\end{table*}

\item  Let $\vec{A}$ be an m x n real matrix and $\vec{b}\in \mathbb{R}^m$ with $b\neq 0$.
\begin{enumerate}
    \item The set of all real solutions of $\vec{A}x=\vec{b}$ is a vector space.\\
    \item If u nd v are two solutions of $\vec{A}x=\vec{b}$ then $\lambda u  +\brak{1-\lambda}v$ is also a solution of $\vec{A}x=\vec{b}$\\
    \item For any two solutions u and v of $\vec{A}x=\vec{b}$, the linear combination $\lambda u$ + $\brak{1-\lambda}v$ is also a solution of $\vec{A}x=\vec{b}$ only when $0\leq\lambda\leq1.$\\
    \item If rank of $\vec{A}$ is n ,then $\vec{A}x=\vec{b}$ has at most one solution.\ 
    \end{enumerate}
%
\solution
The nullspace is given by 
\begin{align}
	\myvec{1 & 1 & 1 & 0 \\ 1 & 1 & 0 & 1\\ 0 & 0 & 0 & 0\\0 & 0 & 0 & 0}\myvec{x\\y\\z\\w} = \myvec{0 \\ 0 \\ 0 \\ 0}
\end{align}	
Row reducing the above matrix we get,
\begin{align}
	\myvec{1 & 1 & 1 & 0 \\ 1 & 1 & 0 & 1\\ 0 & 0 & 0 & 0\\0 & 0 & 0 & 0}
	\xleftrightarrow[R_2 \leftarrow R_2 \times -1]{R_2 \leftarrow R_2 - R_1}
	\myvec{1 & 1 & 1 & 0 \\ 0 & 0 & 1 & -1\\ 0 & 0 & 0 & 0\\0 & 0 & 0 & 0}\\
	\xleftrightarrow{R_1 \leftarrow R_1- R_2}
	\myvec{1 & 1 & 0 & 1 \\ 0 & 0 & 1 & -1\\ 0 & 0 & 0 & 0\\0 & 0 & 0 & 0} \label{eq:solutions/2017/dec/27/eq:rref}
\end{align}
See Table \ref{eq:solutions/2017/dec/27/tab}

\begin{table*}[!ht]
	\begin{tabular}{|m{4.5cm}|l|}
		\hline
		&\\
		dim(C$(\vec{A})) = 1$ 
		& \textbf{False}. Because the number of pivot variables are 2 as obtained in \eqref{eq:solutions/2017/dec/27/eq:rref}\\
		&\\
		\hline
		&\\
		dim(C$(\vec{A})) = 2$
		& \textbf{True}. Since the number of pivot variables are 2, the rank of $\vec{A}$ is 2.\\
		&$\therefore dim(C(\vec{A})) = 2 \quad [\because dim(C(\vec{A})) = rank(\vec{A})]$ \\
		&\\
		\hline
		&\\
	     rank$(\vec{A}) = 1$
		& \textbf{False}. Because the rank$(\vec{A}) = 2$, as the number of pivot variables are 2\\
		&\\
		\hline
		&\\
		$\vec{S}$ = $\cbrak{(1, 1, 1, 0), (1, 1, 0, 1)}$ is a basis of $N(\vec{A})$
		& \textbf{False}. \\
		& Let, \\
		&  $\vec{u} = \myvec{1\\1\\1\\0}, \vec{v} = \myvec{1\\1\\0\\1}$\\ 
		&Consider, \\
		&$\myvec{1 & 1 & 1 & 0 \\ 1 & 1 & 0 & 1\\ 0 & 0 & 0 & 0\\0 & 0 & 0 & 0}\myvec{1\\1\\1\\0} = \myvec{3\\2\\0\\0} \not = \myvec{0\\0\\0\\0}$\\
		& Similarly,\\
		&$\myvec{1 & 1 & 1 & 0 \\ 1 & 1 & 0 & 1\\ 0 & 0 & 0 & 0\\0 & 0 & 0 & 0}\myvec{1\\1\\0\\1} = \myvec{2\\3\\0\\0} \not = \myvec{0\\0\\0\\0}$ \\
		&Hence, the given vectors do not form the basis.\\
		\hline
	\end{tabular}
\caption{}
\label{eq:solutions/2017/dec/27/tab}
\end{table*}

\item Let $\vec{A}$ be an $n\times n$ matrix over $\mathbb{C}$ such that every non-zero vector $\mathbb{C}^n$ is an eigen vector of $\vec{A}$.Then
\begin{enumerate}
    \item All eigen values of $\vec{A}$ are equal.
    \item All eigen values of $\vec{A}$ are distinct.
    \item $\vec{A}=\lambda\vec{I}$ for some $\lambda \in \mathbb{C}$, where $\vec{I}$ is the $n\times n$ identity matrix.
    \item If $\chi_\vec{A}$ and $m_\vec{A}$ denote the characteristic polynomial and the minimal polynomial respectively, then $\chi_\vec{A}=m_\vec{A}$
\end{enumerate}
%

\solution
The nullspace is given by 
\begin{align}
	\myvec{1 & 1 & 1 & 0 \\ 1 & 1 & 0 & 1\\ 0 & 0 & 0 & 0\\0 & 0 & 0 & 0}\myvec{x\\y\\z\\w} = \myvec{0 \\ 0 \\ 0 \\ 0}
\end{align}	
Row reducing the above matrix we get,
\begin{align}
	\myvec{1 & 1 & 1 & 0 \\ 1 & 1 & 0 & 1\\ 0 & 0 & 0 & 0\\0 & 0 & 0 & 0}
	\xleftrightarrow[R_2 \leftarrow R_2 \times -1]{R_2 \leftarrow R_2 - R_1}
	\myvec{1 & 1 & 1 & 0 \\ 0 & 0 & 1 & -1\\ 0 & 0 & 0 & 0\\0 & 0 & 0 & 0}\\
	\xleftrightarrow{R_1 \leftarrow R_1- R_2}
	\myvec{1 & 1 & 0 & 1 \\ 0 & 0 & 1 & -1\\ 0 & 0 & 0 & 0\\0 & 0 & 0 & 0} \label{eq:solutions/2017/dec/27/eq:rref}
\end{align}
See Table \ref{eq:solutions/2017/dec/27/tab}

\begin{table*}[!ht]
	\begin{tabular}{|m{4.5cm}|l|}
		\hline
		&\\
		dim(C$(\vec{A})) = 1$ 
		& \textbf{False}. Because the number of pivot variables are 2 as obtained in \eqref{eq:solutions/2017/dec/27/eq:rref}\\
		&\\
		\hline
		&\\
		dim(C$(\vec{A})) = 2$
		& \textbf{True}. Since the number of pivot variables are 2, the rank of $\vec{A}$ is 2.\\
		&$\therefore dim(C(\vec{A})) = 2 \quad [\because dim(C(\vec{A})) = rank(\vec{A})]$ \\
		&\\
		\hline
		&\\
	     rank$(\vec{A}) = 1$
		& \textbf{False}. Because the rank$(\vec{A}) = 2$, as the number of pivot variables are 2\\
		&\\
		\hline
		&\\
		$\vec{S}$ = $\cbrak{(1, 1, 1, 0), (1, 1, 0, 1)}$ is a basis of $N(\vec{A})$
		& \textbf{False}. \\
		& Let, \\
		&  $\vec{u} = \myvec{1\\1\\1\\0}, \vec{v} = \myvec{1\\1\\0\\1}$\\ 
		&Consider, \\
		&$\myvec{1 & 1 & 1 & 0 \\ 1 & 1 & 0 & 1\\ 0 & 0 & 0 & 0\\0 & 0 & 0 & 0}\myvec{1\\1\\1\\0} = \myvec{3\\2\\0\\0} \not = \myvec{0\\0\\0\\0}$\\
		& Similarly,\\
		&$\myvec{1 & 1 & 1 & 0 \\ 1 & 1 & 0 & 1\\ 0 & 0 & 0 & 0\\0 & 0 & 0 & 0}\myvec{1\\1\\0\\1} = \myvec{2\\3\\0\\0} \not = \myvec{0\\0\\0\\0}$ \\
		&Hence, the given vectors do not form the basis.\\
		\hline
	\end{tabular}
\caption{}
\label{eq:solutions/2017/dec/27/tab}
\end{table*}

\item 	Consider a matrix,
	\begin{align}
	\vec{A} = \myvec{2 & 2 & 1 \\0 & 2 & -1 \\ 0 & 0 & 3}\\ \intertext{and,} \vec{B} = \myvec{2 & 1 & 0 \\ 0 & 2 & 0 \\ 0 & 0 & 3}
	\end{align}
	
Then which of following is true,
\begin{enumerate}
\item $\vec{A}$ and $\vec{B}$ is similar over the field of rational numbers.
\item $\vec{A}$ is diagonalizable over the field of rational numbers $\mathbb{Q}$.
\item $\vec{B}$ is the Jordan canonical form of $\vec{A}$.
\item The minimal polynomial and the characteristic polynomial of $\vec{A}$ are the same.
\end{enumerate}
%
\solution
The nullspace is given by 
\begin{align}
	\myvec{1 & 1 & 1 & 0 \\ 1 & 1 & 0 & 1\\ 0 & 0 & 0 & 0\\0 & 0 & 0 & 0}\myvec{x\\y\\z\\w} = \myvec{0 \\ 0 \\ 0 \\ 0}
\end{align}	
Row reducing the above matrix we get,
\begin{align}
	\myvec{1 & 1 & 1 & 0 \\ 1 & 1 & 0 & 1\\ 0 & 0 & 0 & 0\\0 & 0 & 0 & 0}
	\xleftrightarrow[R_2 \leftarrow R_2 \times -1]{R_2 \leftarrow R_2 - R_1}
	\myvec{1 & 1 & 1 & 0 \\ 0 & 0 & 1 & -1\\ 0 & 0 & 0 & 0\\0 & 0 & 0 & 0}\\
	\xleftrightarrow{R_1 \leftarrow R_1- R_2}
	\myvec{1 & 1 & 0 & 1 \\ 0 & 0 & 1 & -1\\ 0 & 0 & 0 & 0\\0 & 0 & 0 & 0} \label{eq:solutions/2017/dec/27/eq:rref}
\end{align}
See Table \ref{eq:solutions/2017/dec/27/tab}

\begin{table*}[!ht]
	\begin{tabular}{|m{4.5cm}|l|}
		\hline
		&\\
		dim(C$(\vec{A})) = 1$ 
		& \textbf{False}. Because the number of pivot variables are 2 as obtained in \eqref{eq:solutions/2017/dec/27/eq:rref}\\
		&\\
		\hline
		&\\
		dim(C$(\vec{A})) = 2$
		& \textbf{True}. Since the number of pivot variables are 2, the rank of $\vec{A}$ is 2.\\
		&$\therefore dim(C(\vec{A})) = 2 \quad [\because dim(C(\vec{A})) = rank(\vec{A})]$ \\
		&\\
		\hline
		&\\
	     rank$(\vec{A}) = 1$
		& \textbf{False}. Because the rank$(\vec{A}) = 2$, as the number of pivot variables are 2\\
		&\\
		\hline
		&\\
		$\vec{S}$ = $\cbrak{(1, 1, 1, 0), (1, 1, 0, 1)}$ is a basis of $N(\vec{A})$
		& \textbf{False}. \\
		& Let, \\
		&  $\vec{u} = \myvec{1\\1\\1\\0}, \vec{v} = \myvec{1\\1\\0\\1}$\\ 
		&Consider, \\
		&$\myvec{1 & 1 & 1 & 0 \\ 1 & 1 & 0 & 1\\ 0 & 0 & 0 & 0\\0 & 0 & 0 & 0}\myvec{1\\1\\1\\0} = \myvec{3\\2\\0\\0} \not = \myvec{0\\0\\0\\0}$\\
		& Similarly,\\
		&$\myvec{1 & 1 & 1 & 0 \\ 1 & 1 & 0 & 1\\ 0 & 0 & 0 & 0\\0 & 0 & 0 & 0}\myvec{1\\1\\0\\1} = \myvec{2\\3\\0\\0} \not = \myvec{0\\0\\0\\0}$ \\
		&Hence, the given vectors do not form the basis.\\
		\hline
	\end{tabular}
\caption{}
\label{eq:solutions/2017/dec/27/tab}
\end{table*}


\end{enumerate}
