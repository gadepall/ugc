See Tables \ref{eq:solutions/2014/dec/71/table:1}
\ref{eq:solutions/2014/dec/71/table:2} and 
\ref{eq:solutions/2014/dec/71/table:3}.

\begin{table*}[ht!]
\centering
\begin{tabular}{|c|l|}
    \hline
	\multirow{3}{*}{Characteristic Polynomial} 
	& \\
	& For an $n\times n$ matrix $\vec{A}$, characteristic polynomial is defined by,\\
	&\\
	& $\qquad\qquad\qquad p\brak{x}=\mydet{x\Vec{I}-\Vec{A}}$\\
	&\\
	\hline
	\multirow{3}{*}{Cayley-Hamilton Theorem}
    &\\
    & If $p\brak{x}$ is the characteristic polynomial of an $n\times n$ matrix $\vec{A}$, then,\\
    &\\
    &$\qquad \qquad \qquad p\brak{\vec{A}}=\vec{0}$\\
    &\\
    \hline
	\multirow{3}{*}{Minimal Polynomial} 
	&\\
	& Minimal polynomial $m\brak{x}$ is the smallest factor of\\
	&characteristic polynomial $p\brak{x}$ such that,\\
	&\\
	& $\qquad \qquad \qquad m\brak{\vec{A}}=0$\\
	& \\
	& Every root of characteristic polynomial should be the root of\\
	&minimal polynomial\\
	&\\
    \hline
\end{tabular}
    \caption{Definitions}
\label{eq:solutions/2014/dec/71/table:1}
\end{table*}

\onecolumn
\begin{longtable}{|l|l|}
\hline
\multirow{3}{*}{} & \\
Statement&Solution\\
\hline
&\\
1.&\\
&\parbox{6cm}{\begin{align*}
    \mbox{Let }\vec{A}&=\myvec{0&0&1\\0&0&0\\0&0&0}\\
\end{align*}}\\
&Since $\vec{A}$ is upper triangular matrix, $\therefore \lambda_{1}=0,\lambda_{2}=0,\lambda_{3}=0$ \\
&\parbox{6cm}{\begin{align*}
    \mbox{Therefore, }p(x)&=(x)^3\\
    \mbox{Solving }\vec{A}^3&=\myvec{0&0&0\\0&0&0\\0&0&0}\\
    \mbox{Solving }\vec{A}^2&=\myvec{0&0&0\\0&0&0\\0&0&0}\\
    \mbox{Since }\vec{A}&\neq \vec{0}\\
    \mbox{Therefore, }m(x)&=(x)^2\\
    \end{align*}}\\
Justification&Hence, the Jordan form of $\vec{A}$ is a $3 \times 3$ matrix consisting of two block:\\
&one block of order 2 with principal diagonal value as $\lambda = 0$ and super\\
&diagonal of the block (i.e the set of elements that lies directly above the\\
&elements comprising the principal diagonal) contains 1.\\
&And one block of order 1 with $\lambda=0$.\\
&Hence the required Jordan form of $\vec{A}$ is,\\
&\parbox{6cm}{\begin{align*}
    \therefore \vec{J}&=\myvec{0&1&0\\0&0&0\\0&0&0}
\end{align*}}\\
&\\
\hline
&\\
Conclusion&Therefore option 1 is true.\\
&\\
\hline
\pagebreak
\hline
&\\
2.&\\
&\parbox{6cm}{\begin{align*}
    \mbox{Let }\vec{A}&=\myvec{0&0&1\\0&0&1\\0&0&0}\\
\end{align*}}\\
&Since $\vec{A}$ is upper triangular matrix, $\therefore \lambda_{1}=0,\lambda_{2}=0,\lambda_{3}=0$ \\
&\parbox{6cm}{\begin{align*}
    \mbox{Therefore, }p(x)&=(x)^3\\
    \mbox{Solving }\vec{A}^3&=\myvec{0&0&0\\0&0&0\\0&0&0}\\
    \mbox{Solving }\vec{A}^2&=\myvec{0&0&0\\0&0&0\\0&0&0}\\
    \mbox{Since }\vec{A}&\neq \vec{0}\\
    \mbox{Therefore, }m(x)&=(x)^2\\
    \end{align*}}\\
Justification&Hence, the Jordan form of $\vec{A}$ is a $3 \times 3$ matrix consisting of two block:\\
&one block of order 2 with principal diagonal value as $\lambda = 0$ and super\\
&diagonal of the block (i.e the set of elements that lies directly above the\\
&elements comprising the principal diagonal) contains 1.\\
&And one block of order 1 with $\lambda=0$.\\
&Hence the required Jordan form of $\vec{A}$ is,\\
&\parbox{6cm}{\begin{align*}
    \therefore \vec{J}&=\myvec{0&1&0\\0&0&0\\0&0&0}
\end{align*}}\\
&\\
\hline
&\\
Conclusion&Therefore option 2 is true.\\
&\\
\hline
\pagebreak
\hline
&\\
3.&\\
&\parbox{6cm}{\begin{align*}
    \mbox{Let }\vec{A}&=\myvec{0&1&1\\0&0&0\\0&0&0}\\
\end{align*}}\\
&Since $\vec{A}$ is upper triangular matrix, $\therefore \lambda_{1}=0,\lambda_{2}=0,\lambda_{3}=0$ \\
&\parbox{6cm}{\begin{align*}
    \mbox{Therefore, }p(x)&=(x)^3\\
    \mbox{Solving }\vec{A}^3&=\myvec{0&0&0\\0&0&0\\0&0&0}\\
    \mbox{Solving }\vec{A}^2&=\myvec{0&0&0\\0&0&0\\0&0&0}\\
    \mbox{Since }\vec{A}&\neq \vec{0}\\
    \mbox{Therefore, }m(x)&=(x)^2\\
    \end{align*}}\\
Justification&Hence, the Jordan form of $\vec{A}$ is a $3 \times 3$ matrix consisting of two block:\\
&one block of order 2 with principal diagonal value as $\lambda = 0$ and super\\
&diagonal of the block (i.e the set of elements that lies directly above the\\
&elements comprising the principal diagonal) contains 1.\\
&And one block of order 1 with $\lambda=0$.\\
&Hence the required Jordan form of $\vec{A}$ is,\\
&\parbox{6cm}{\begin{align*}
    \therefore \vec{J}&=\myvec{0&1&0\\0&0&0\\0&0&0}
\end{align*}}\\
&\\
\hline
&\\
Conclusion&Therefore option 3 is true.\\
&\\
\hline
\pagebreak
\hline
&\\
4.&\\
&\parbox{6cm}{\begin{align*}
    \mbox{Let }\vec{A}&=\myvec{0&1&1\\0&0&1\\0&0&0}\\
\end{align*}}\\
&Since $\vec{A}$ is upper triangular matrix, $\therefore \lambda_{1}=0,\lambda_{2}=0,\lambda_{3}=0$ \\
&\parbox{6cm}{\begin{align*}
    \mbox{Therefore, }p(x)&=(x)^3\\
    \mbox{Solving }\vec{A}^3&=\myvec{0&0&0\\0&0&0\\0&0&0}\\
    \mbox{Solving }\vec{A}^2&=\myvec{0&0&1\\0&0&0\\0&0&0}\\
    \mbox{Since }\vec{A}^2&\neq \vec{0}\\
    \mbox{Therefore, }m(x)&=(x)^3\\
    \end{align*}}\\
Justification&Hence, the Jordan form of $\vec{A}$ is a $3 \times 3$ matrix consisting of only\\
&one block with principal diagonal values as $\lambda = 0$ and super diagonal\\
&of the matrix (i.e the set of elements that lies directly above the\\
&elements comprising the principal diagonal) contains 1.\\
&Hence the required Jordan form of $\vec{A}$ is,\\
&\parbox{6cm}{\begin{align*}
    \therefore \vec{J}&=\myvec{0&1&0\\0&0&1\\0&0&0}
\end{align*}}\\
&\\
\hline
&\\
Conclusion&Therefore option 4 is false.\\
&\\
\hline
\caption{Solution}
\label{eq:solutions/2014/dec/71/table:2}
\end{longtable}
\begin{longtable}{|c|l|}
    \hline
	\multirow{1}{*}{For given jordan form:}
	&\parbox{3cm}{\begin{align*}
	    \vec{A}&=\myvec{0&1&0\\0&0&0\\0&0&0}
	\end{align*}}\\
    We have two blocks:&one block is of order 2.\\
    &And one block is of order 1.\\
    &And eigenvalues are all $\lambda=0$\\
    &$\therefore$ Algebraic Multiplicity of 0 is 3.\\
    &The rank of the matrix is 1.\\
    &\parbox{6cm}{\begin{align*}
        \mbox{Geometric Multiplicity of 0}&=n-\mbox{Rank}(\vec{A}-\lambda\vec{I})\\
        &=n-\mbox{Rank}(\vec{A})\\
        &=2
    \end{align*}}\\
	&\\
	\hline
	&\\
	1.&The eigenvalue order of 0 in the characteristic polynomial = 3.\\
	&$\therefore$ Algebraic Multiplicity of 0 is 3.\\
	&The eigenvalue order of 0 in the minimal polynomial = 2.\\
    &The rank of the matrix is 1.\\
    &$\therefore$ The Geometric Multiplicity of 0 = 2.\\
    &Therefore the matrix gives the same jordan form\\
	& \\
	\hline
	&\\
	2.&The eigenvalue order of 0 in the characteristic polynomial = 3.\\
	&$\therefore$ Algebraic Multiplicity of 0 is 3.\\
	&The eigenvalue order of 0 in the minimal polynomial = 2.\\
    &The rank of the matrix is 1.\\
    &$\therefore$ The Geometric Multiplicity of 0 = 2.\\
    &Therefore the matrix gives the same jordan form\\
	& \\
	\hline
	&\\
	3.&The eigenvalue order of 0 in the characteristic polynomial = 3.\\
	&$\therefore$ Algebraic Multiplicity of 0 is 3.\\
	&The eigenvalue order of 0 in the minimal polynomial = 2.\\
    &The rank of the matrix is 1.\\
    &$\therefore$ The Geometric Multiplicity of 0 = 2.\\
    &Therefore the matrix gives the same jordan form\\
	& \\
	\hline
	&\\
	4.&The eigenvalue order of 0 in the characteristic polynomial = 3.\\
	&$\therefore$ Algebraic Multiplicity of 0 is 3.\\
	&The eigenvalue order of 0 in the minimal polynomial = 3.\\
    &The rank of the matrix is 2.\\
    &$\therefore$ The Geometric Multiplicity of 0 = 1.\\
    &Therefore the matrix gives different jordan form\\
	\hline
	\caption{Conclusion of above Results}
    \label{eq:solutions/2014/dec/71/table:3}
\end{longtable}

\twocolumn
