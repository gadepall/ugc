 Let
 \begin{align}
    \vec{x} = \myvec{x&y}^T 
 \end{align}
 Then
 \begin{align}
     B(x,y) = \vec{x}^T\frac{\vec{R}}{2}\vec{x}\label{eq:solutions/2018/dec/32/1} 
 \end{align}
 where $\vec{R}$ is the reflection matrix defined as:-
 \begin{align}
  \myvec{0 & 1 \\ 1 & 0}
 \end{align}
 \eqref{eq:solutions/2018/dec/32/1} represent Quadratic form of B(x,y).  See Table \ref{eq:solutions/2018/dec/32/table1}

\begin{table*}[ht!]
\begin{center}
\begin{tabular}{|c|c|}
\hline
\textbf{Options} & \textbf{Explanation} \\
\hline
\text{$B$ is a linear transformation} & 
Let the transformation be $B: \mathbb{R} \times \mathbb{R} \xrightarrow[]{} \mathbb{R}$ such that, 
\\& $B(\vec{x}) = xy$ where $\vec{x} = \myvec{x\\y}$
\\& Now $B(\vec{e}) = ab$ where $\vec{e} = \myvec{a\\b}$ 
\\& Hence, $B(c\vec{e}) = c^2B(\vec{e})$
\\& Hence $B$ is not a linear transformation.\\
& Hence incorrect.
\\
\hline
\text{$B$ is a positive definite bilinear form} & 
$f: \mathbb{V} \times \mathbb{V} \xrightarrow[]{} \mathbb{F}$ where $\mathbb{V}$ is a vector space and $\mathbb{F}$ is a field\\
Bilinear Form&$f$ is a bilinear if the following holds true - 
\\& If one variable is fixed then other should be linear  
\\&Let's say $x$ is fixed,$x$=c
\\&$\eqref{eq:solutions/2018/dec/32/1}$ becomes $B(x,y)=cy$,$y$ is linear
\\&Let's say $y$ is fixed,$y$=c
\\&$\eqref{eq:solutions/2018/dec/32/1}$ becomes $B(x,y)=cx$,$x$ is linear
\\& Hence $B$ is a bilinear form.
\\Symmetric & Again a bilinear form $f$ is symmetric if $f(\alpha,\beta) = f(\beta,\alpha)$
\\& Here, $B(a,b) = ab$,from $\eqref{eq:solutions/2018/dec/32/1}$
\\& $B(b,a) = ba$,from $\eqref{eq:solutions/2018/dec/32/1}$
\\& $ba=ab$,Hence $B$ is symmetric.
\\Positive Definite& A symmetric bilinear $f$ is positive definite if
\\& $f(\alpha,\alpha) >0$ $\forall \alpha \ne 0$
\\& Here, $B(a,a) = a^2$ from $\eqref{eq:solutions/2018/dec/32/1}$
\\& $a^2 > 0$ $\forall a\ne0$
\\& \textbf{Conclusion:} $B$ is symmetric and positive definite bilinear form.\\
& Hence Correct.
\\
\hline
\text{$B$ is symmetric but not positive definite}
& From previous proof it is obvious that
\\& $B$ is both symmetric as well as positive definite\\
& Hence incorrect
\\
\hline
\text{$B$ neither linear nor bilinear}
& From previous proofs it is obvious that
\\& $B$ is bilinear.\\
& Hence incorrect.
\\
\hline
\text{Result}
& $B$ is symmetric and positive definite bilinear form
\\
\hline
\end{tabular}
\end{center}
\caption{Finding Correct Option}
\label{eq:solutions/2018/dec/32/table1}
\end{table*}

 
