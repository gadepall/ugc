See Tables \ref{eq:solutions/2018/dec/71/table:1} and \ref{eq:solutions/2018/dec/71/table:2}


\begin{table*}[ht!]
\centering
\begin{tabular}{|c|l|}
    \hline
	\multirow{3}{*}{Characteristic Polynomial} 
	& \\
	& For an $n\times n$ matrix $\vec{A}$, characteristic polynomial is defined by,\\
	&\\
	& $\qquad\qquad\qquad p\brak{x}=\mydet{x\Vec{I}-\Vec{A}}$\\
	&\\
	\hline
	\multirow{3}{*}{Cayley-Hamilton Theorem}
    &\\
    & If $p\brak{x}$ is the characteristic polynomial of an $n\times n$ matrix $\vec{A}$, then,\\
    &\\
    &$\qquad \qquad \qquad p\brak{\vec{A}}=\vec{0}$\\
    &\\
    \hline
	\multirow{3}{*}{Minimal Polynomial} 
	&\\
	& Minimal polynomial $m\brak{x}$ is the smallest factor of\\
	&characteristic polynomial $p\brak{x}$ such that,\\
	&\\
	& $\qquad \qquad \qquad m\brak{\vec{A}}=0$\\
	& \\
	& Every root of characteristic polynomial should be the root of\\
	&minimal polynomial\\
	&\\
    \hline
\end{tabular}
    \caption{Definitions}
\label{eq:solutions/2018/dec/71/table:1}
\end{table*}

\onecolumn
\begin{longtable}{|l|l|}
\hline
\multirow{3}{*}{} & \\
Statement&Solution\\
\hline
&\\
1.&\\
&Given that $\vec{T}:\mathbb{R}^n \rightarrow \mathbb{R}^n$\\
&Since $\vec{T}$ is a linear map from $\mathbb{R}^n$ to $\mathbb{R}^n$ therefore the matrix\\
&corresponding to it is of order $n \times n$.\\
&\parbox{6cm}{\begin{align*}
    \mbox{Since }\vec{T}^2&=\vec{T}-\vec{I}_{n}\\
    \therefore\vec{T}^2-\vec{T}+\vec{I}_{n}&=\vec{0}
\end{align*}}\\
&$\implies p(x)=x^2-x+1$ will be annihilating polynomial.\\
&$\therefore p(\vec{T})=\vec{T}^2-\vec{T}+\vec{I}_{n}=\vec{0}$\\
&We know that minimal polynomial always divides annihilating polynomial.\\
&$\therefore$ The roots of minimal polynomial are as follows:\\
&\parbox{6cm}{\begin{align}
    x&=\frac{1\pm\sqrt{3}i}{2}\label{eq:solutions/2018/dec/71/eq:root}
\end{align}}\\
&Therefore any eigenvalue of $\vec{T}$ is a root of its minimal polynomial.\\
&Since 0 is not a root of p(x), Therefore 0 is not an eigen value for $\vec{T}$.\\
&Since $\vec{T}$ is not invertible iff there exists an eigen value which is zero.\\
&\parbox{6cm}{\begin{align}
    \therefore\vec{T}\mbox{ is invertible.}
\end{align}}\\
&\\
\hline
&\\
Conclusion&Therefore the statement is true.\\
&\\
\hline
&\\
2.&\\
& From equation \eqref{eq:solutions/2018/dec/71/eq:root} ,\\
&Since 1 is not a root of p(x), Therefore 1 is not an eigen value for $\vec{T}$.\\
&Therefore, 0 is not an eigen values of  $\vec{T}-\vec{I}_{n}$.\\
&\parbox{6cm}{\begin{align}
    \therefore\vec{T}-\vec{I}_{n}\mbox{ is invertible.}
\end{align}}\\
&\\
\hline
&\\
Conclusion&Therefore the statement is false.\\
&\\
\hline
\pagebreak
\hline
&\\
3.&\\
& From equation \eqref{eq:solutions/2018/dec/71/eq:root} ,\\
&Therefore any eigenvalue of $\vec{T}$ is a root of its minimal polynomial.\\
&But the roots of minimal polynomial are not real.\\
&Therefore $\vec{T}$ cant have a real eigen value.\\
&\\
\hline
&\\
Conclusion&Therefore the statement is false.\\
&\\
\hline
&\\
4.&\\
&\parbox{6cm}{\begin{align}
    \mbox{Since }\vec{T}^2&=\vec{T}-\vec{I}_{n}\\
    \vec{T}^3&=\vec{T}(\vec{T}-\vec{I}_{n})\\
    \therefore\vec{T}^3&=\vec{T}^2-\vec{T}\\
    \therefore \vec{T}^3&=-\vec{I}_{n}\label{eq:solutions/2018/dec/71/eq:T_cube}
\end{align}}\\
&\\
\hline
&\\
Conclusion&Therefore the statement is true.\\
&\\
\hline
\caption{Solution summary}
\label{eq:solutions/2018/dec/71/table:2}
\end{longtable}
\twocolumn
