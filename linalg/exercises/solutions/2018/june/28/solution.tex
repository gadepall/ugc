See Table     \ref{eq:solutions/2018/june/28/table:1}

\onecolumn
\begin{longtable}{|p{5cm}|p{13cm}|}
\hline
    \multirow{3}{*}{Given} 
    &\\
     & $\vec{A}$ be a $2\times2$ matrix over $\mathbb{R}$ with\\
     &\\
      &$\qquad\qquad \qquad \det\brak{\vec{A}+\vec{I}}=1+\det(\vec{A})$\\
      &\\
     \hline
     \multirow{3}{*}{Explanation} &\\
     & If $\vec{X}$ is an eigen vector of matrix $\vec{A}$
     corresponding to the eigen value $\lambda$ i.e \\
     &\\
     & $\qquad\qquad\qquad\vec{A}\vec{X}=\lambda\vec{X}$\\
     &\\
     & then, $\brak{\vec{I}+\vec{A}}\vec{X}=\brak{1+\lambda}\vec{X}$\\
     &\\
     &Thus, $\vec{X}$ is an eigen vector of $\brak{\vec{A}+\vec{I}}$ corresponding to the eigen value $\brak{1+\lambda}$.\\
     
     &\\
     
     & Let $\lambda_1,\lambda_2$ be two eigen values of $\vec{A}$ and $\brak{1+\lambda_1},\brak{1+\lambda_2}$ be the eigen values of $\brak{\vec{A}+\vec{I}}.$\\
    &\\
    & $\implies$ Eigen value of $\vec{A}=\lambda_1,\lambda_2$\\
    &\\
    &  $\implies$ Eigen value of $\brak{\vec{A}+\vec{I}} = \lambda_1+1,\lambda_2+1$\\
    &\\
    & Since,\\
    \hline
    &$\qquad\qquad \qquad \det\brak{\vec{A}+\vec{I}}=1+\det(\vec{A})$\\
    &\\
     & Trace of any matrix is sum of its eigen values. \\
    &\\
    & Determinant of matrix is product of its eigen values \\
    &\\
    & $\qquad\qquad\implies \brak{\lambda_1+1}\brak{\lambda_2+1}=1+\brak{\lambda_1\lambda_2}$\\
    &\\
    &$\qquad\qquad\implies\boxed{ \lambda_1+\lambda_2 = 0}$\\
    &\\
   
    &$\qquad\qquad\implies\boxed{ tr(\vec{A})=0}$\\
    &\\
    \hline
    \multirow{3}{*}{\textbf{Statement 1} : $\det\vec{A}=0$ } 
	& \\
	& False\\
	\hline
	&\\
	& Let, $\vec{A}=\myvec{1&0\\0&-1}$\\
	&\\
	& Here, $\det{\vec{A}}=-1$ and $\det(\vec{A}+\vec{I})=0$\\
	&\\
	& Thus,$1+\det(\vec{A})=\det(\vec{A}+\vec{I})$\\
	&\\
	& In this case, \\
	& $\det\vec{A}\neq 0$\\ 
	&but satisfy the given condition i.e $1+\det(\vec{A})=\det(\vec{A}+\vec{I})$\\ 
	&\\
	\hline
	\multirow{3}{*}{\textbf{Statement 2} : $\vec{A}=\vec{0}$} & \\
	& False\\
	\hline
	&\\
	& Let , $\vec{A}= \myvec{0&1\\0&0}$\\
	&\\
	& Here, $\det{\vec{A}}=0$ and $\det(\vec{A}+\vec{I})=1$\\
	&\\
	& Thus,$1+\det(\vec{A})=\det(\vec{A}+\vec{I})$\\
	&\\
	& In this case, \\
	& $\vec{A}\neq \vec{0}$\\ 
	& But , satisfy the given condition i.e $ 1+\det(\vec{A})=\det(\vec{A}+\vec{I})$\\
	&\\
	\hline
	\multirow{3}{*}{\textbf{Statement 3}: $tr(\vec{A})=0$}&\\
    & True\\
	\hline
	&\\
	& The given statement is true for all possible matrices.\\
	&\\
	& If $tr\vec{A}\neq0$ then the given condition i.e $1
	+\det(\vec{A})=\det(\vec{A}+\vec{I})$ doesn't satisy.\\
	& Let , $\vec{A}= \myvec{1&0\\0&0}$\\
	\hline
	& Here, $\det{\vec{A}}=0$ , $\det(\vec{A}+\vec{I})=2 , tr{\vec{A}} \neq 0$\\
	&\\
	& Thus, $ 1+\det(\vec{A})\neq\det(\vec{A}+\vec{I})$\\
	&\\
	\hline
	\multirow{3}{*}{\textbf{Statement4}:$\vec{A}$ is non singular} &\\
    & False\\
	\hline
	&\\

   & Non Singular Matrix:
   A non-singular matrix is a square one whose determinant is not zero.non-singular matrix is also  a full rank matrix.\\
   &\\
   & Let, $\vec{A}=\myvec{0&0\\0&0}$\\
	&\\
	& Here, $\det{\vec{A}}=0$ and $\det(\vec{A}+\vec{I})=1$\\
	&\\
	& Thus,$1+\det(\vec{A})=\det(\vec{A}+\vec{I})$\\
	&\\
	& In this case, \\
	& $\vec{A}$ is Singular,\\ & But satisfy the given condition i.e $1+\det(\vec{A})=\det(\vec{A}+\vec{I})$ \\ 
	&\\
	\hline
	\multirow{3}{*}{Conclusion} &\\
	& Thus, we can conclude Statement 3 is true for all possible matrices which satisfy the given condition i.e 
	$1+\det(\vec{A})=\det(\vec{A}+\vec{I})$\\
	&\\
	\hline
	\caption{Solution Summary}
    \label{eq:solutions/2018/june/28/table:1}
\end{longtable}

\twocolumn
