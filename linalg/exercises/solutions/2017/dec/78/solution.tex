Refer Table \ref{eq:solutions/2017/dec/78/table:1}.


\begin{table*}[ht!]
\begin{center}
\resizebox{\columnwidth}{!}
{
\begin{tabular}{|l|l|}
\hline
\textbf{Given} & \textbf{Derivation} \\[0.5ex]
\hline
\text{Given} & 
\text{$\vec{A}$ is a $m \times n$ matrix of rank $m$ with $n>m$}. \\
& A non-zero real number $\alpha$.\\
& To find eigenvalues of $\vec{A^TA}$.
\\ [0.5ex]
\hline
\text{Eigenvalues of $\vec{AA^T}$} & 
\text{$\vec{AA^T}$ is a $m \times m$ matrix and $\vec{A^TA}$ is a $n \times n$ matrix.}\\
& Let, $\lambda$ be a non-zero eigen value of $\vec{A^TA}$.\\
& \parbox{10cm}{\begin{align}
    \vec{A^TAv} = \lambda \vec{v} \quad \vec{v} \in \vec{R^n}\\
    \vec{AA^TAv} = \lambda \vec{Av}\\
    \text{Let,} \quad \vec{x} = \vec{Av} \quad \vec{x} \in \vec{R^m}\\
    \vec{AA^Tx} = \lambda \vec{x}\\
    \vec{x^TAA^Tx} = \lambda \vec{x^Tx}\\
    \text{Given}, \quad \vec{x^TAA^Tx} = \alpha\vec{x^Tx}\\
    \implies \alpha\vec{x^Tx} = \lambda \vec{x^Tx} \label{eq:solutions/2017/dec/78/eq:eq18}
\end{align}} \\
& From equation \eqref{eq:solutions/2017/dec/78/eq:eq18}, $\lambda = \alpha$ as $\norm{\vec{x}} \not = 0$\\
& As rank($\vec{A^TA}$) = rank($\vec{A}$) = $m$ and equation \eqref{eq:solutions/2017/dec/78/eq:eq18} satisfies the condition in question.\\
&Therefore the only non-zero eigen value is $\alpha$\\
& $\vec{A^TA}$ has an eigenvalue $\alpha$ with multiplicity $m$.
\\ [0.5ex]
\hline
\text{Eigenvalues of $\vec{A^TA}$} & 
$\vec{A^TA}$ is a $n \times n$ matrix. Given $n > m$, \\\\
&We know that, $\vec{A^TA}$ and $\vec{AA^T}$ have same number of non-zero eigenvalues\\& and if one of them has more number of eigenvalues than the other \\&then these eigenvalues are zero.\\
& 1. From above, as $\alpha$ is non-zero, $\vec{A^TA}$ has $\alpha$ as its eigenvalue with multiplicity $m$ \\
& 2. $\vec{A^TA}$ has $0$ as its eigenvalue with multiplicity $n-m$\\
& 3. Therefore, the two distinct eigenvalues of $\vec{A^TA}$ are $\alpha$ and $0$.
\\ [0.5ex]
\hline
\end{tabular}
}
\end{center}

\caption{Explanation}
\label{eq:solutions/2017/dec/78/table:1}
\end{table*}

Refer Table \ref{eq:solutions/2017/dec/78/table:2}.

\begin{table*}[ht]
\begin{center}
\resizebox{\columnwidth}{!}
{
\begin{tabular}{|c|c|}
\hline
& \\
$\vec{A^TA}$ has exactly two distinct eigenvalues.
& True statement\\
\hline
& \\
$\vec{A^TA}$ has 0 as an eigenvalue with multiplicity $n-m$
& True statement\\
\hline
& \\
$\vec{A^TA}$ has $\alpha$ as a non-zero eigenvalue
&  True statement\\
\hline
& \\
$\vec{A^TA}$ has exactly two non-zero distinct eigenvalues.
& False statement\\
\hline
\end{tabular}
}
\end{center}
\caption{Solution}
\label{eq:solutions/2017/dec/78/table:2}
\end{table*}
