See Tables     \ref{eq:solutions/2014/dec/73/tab:Cons},     \ref{eq:solutions/2014/dec/73/tab:Ans}
and     \ref{eq:solutions/2014/dec/73/tab:example}


\onecolumn
\begin{longtable}{|l|l|}
    \hline
    Definition&A bilinear form on a vector space $\vec{V}$ is a function $f$, which assigns to each ordered pair\\
    of bilinear&of vectors $\alpha,\beta$ in $\vec{V}$ a scalar $f\brak{\alpha,\beta}$ in field $\vec{F}$ which satisfies\\
    form&\qquad \qquad $i)$ $f\brak{c\alpha_1+\alpha_2,\beta}=cf\brak{\alpha_1,\beta}+f\brak{\alpha_2,\beta}$\\
    &\qquad \qquad $ii)$ $f\brak{\alpha,c\beta_1+\beta_2}=cf\brak{\alpha,\beta_1}+f\brak{\alpha,\beta_2}$\\
    \hline
    Symmetric&A bilinear form on the vector space $\vec{V}$ is symmetric if\\
    bilinear&\qquad \qquad \qquad $f\brak{\alpha,\beta}=f\brak{\beta,\alpha}$\\
    form& for all vectors $\alpha,\beta\in\vec{V}$\\
    \hline
    Matrix of &Let $\alpha,\beta\in\mathbb{R}^3$ be two vectors, which are represented in standard basis as\\
    bilinear&$\alpha=\alpha_1\vec{e_1}+\alpha_2\vec{e_2}+\alpha_3\vec{e_3}$ and $\beta=\beta_1\vec{e_1}+\beta_2\vec{e_2}+\beta_3\vec{e_3}$,therefore $f\brak{\alpha,\beta}$ can be represented \\
    form&in matrix form as\\
    &$f\brak{\alpha,\beta}=f\brak{\alpha_1\vec{e_1}+\alpha_2\vec{e_2}+\alpha_3\vec{e_3},\beta_1\vec{e_1}+\beta_2\vec{e_2}+\beta_3\vec{e_3}}$\\
    &$=\myvec{\alpha_1&\alpha_2&\alpha_3}\myvec{f\brak{\vec{e_1},\vec{e_1}}&f\brak{\vec{e_1},\vec{e_2}}&f\brak{\vec{e_1},\vec{e_3}}\\f\brak{\vec{e_2},\vec{e_1}}&f\brak{\vec{e_2},\vec{e_2}}&f\brak{\vec{e_2},\vec{e_3}}\\f\brak{\vec{e_3},\vec{e_1}}&f\brak{\vec{e_3},\vec{e_2}}&f\brak{\vec{e_3},\vec{e_3}}}\myvec{\beta_1\\\beta_2\\\beta_3}$\\
    \hline
    Given &Given a non-zero symmetric bilinear form $f$ such that $f\brak{\alpha,\beta}=T_1\brak{\alpha}T_2\brak{\beta}$ where\\
    &$\alpha,\beta\in\mathbb{R}^3$. So the symmetric bilinear form can be represented on matrix form as\\
    &$f\brak{\alpha,\beta}=\myvec{\alpha_1&\alpha_2&\alpha_3}\myvec{f\brak{\vec{e_1},\vec{e_1}}&f\brak{\vec{e_1},\vec{e_2}}&f\brak{\vec{e_1},\vec{e_3}}\\f\brak{\vec{e_2},\vec{e_1}}&f\brak{\vec{e_2},\vec{e_2}}&f\brak{\vec{e_2},\vec{e_3}}\\f\brak{\vec{e_3},\vec{e_1}}&f\brak{\vec{e_3},\vec{e_2}}&f\brak{\vec{e_3},\vec{e_3}}}\myvec{\beta_1\\\beta_2\\\beta_3}$\\
    &$f\brak{\alpha,\beta}=\myvec{\alpha_1&\alpha_2&\alpha_3}\myvec{T_1\brak{\vec{e_1}}T_2\brak{\vec{e_1}}&T_1\brak{\vec{e_1}}T_2\brak{\vec{e_2}}&T_1\brak{\vec{e_1}}T_2\brak{\vec{e_3}}\\T_1\brak{\vec{e_2}}T_2\brak{\vec{e_1}}&T_1\brak{\vec{e_2}}T_2\brak{\vec{e_2}}&T_1\brak{\vec{e_2}}T_2\brak{\vec{e_3}}\\T_1\brak{\vec{e_3}}T_2\brak{\vec{e_1}}&T_1\brak{\vec{e_3}}T_2\brak{\vec{e_2}}&T_1\brak{\vec{e_3}}T_2\brak{\vec{e_3}}\\}\myvec{\beta_1\\\beta_2\\\beta_3}$\\
    &$f\brak{\alpha,\beta}=\myvec{\alpha_1&\alpha_2&\alpha_3}\myvec{T_1\brak{\vec{e_1}}\\T_1\brak{\vec{e_2}}\\T_1\brak{\vec{e_3}}}\myvec{T_2\brak{\vec{e_1}}&T_2\brak{\vec{e_2}}&T_2\brak{\vec{e_3}}}\myvec{\beta_1\\\beta_2\\\beta_3}=\alpha^T\vec{T_1}\vec{T_2}^T\beta$\\
    &where $\vec{T_1}=\myvec{T_1\brak{\vec{e_1}}\\T_1\brak{\vec{e_2}}\\T_1\brak{\vec{e_3}}}$ and $\vec{T_2}=\myvec{T_2\brak{\vec{e_1}}\\T_2\brak{\vec{e_2}}\\T_2\brak{\vec{e_3}}}$ are the matrix representation of the linear\\
    &transformations $T_1,T_2$.So, the matrix representation of $f$ is $\vec{T_1}\vec{T_2}^T$ or $\vec{T_2}\vec{T_1}^T$ since\\
    &$f$ is symmetric.\\
    &$note:$ Since $f$ is non-zero symmetric bilinear form $rank\brak{\vec{T_1}}=rank\brak{\vec{T_2}}=1$\\
    \hline
    \caption{Construction}
    \label{eq:solutions/2014/dec/73/tab:Cons}
\end{longtable}
\begin{longtable}{|l|l|}
    \hline
    Option 1& By using the property of rank of product of two matrices, we get\\
    &$rank\brak{f}=rank\brak{\vec{T_1}\vec{T_2}^T}\leq min\brak{rank\brak{\vec{T_1}}, rank\brak{\vec{T_2}}}\leq 1$.\\
    &Since $f$ is non-zero the $rank\brak{f}\neq 0$. Hence the $rank\brak{f}=1$\\
    \hline
    Option 2& $\beta\in\mathbb{R}^3:f\brak{\alpha,\beta}=0$ for all $\alpha\in\mathbb{R}^3\implies \beta\in\mathbb{R}^3:T_2\brak{\beta}=0$ for all $\alpha\in\mathbb{R}^3$ because \\
    &$T_1\brak{\alpha}\neq 0$ for all $\alpha\in\mathbb{R}^3$. By using rank nullity theorem \\
    \hline
    & $rank\{T_2\}+dim\{Nullspace\brak{T_2}\}=3\implies dim\{Nullspace\brak{T_2}\}=2$. Similarly for $T_1$, we\\
    &get dim\{Nullspace\brak{T_1}\}=2. Therefore \\
    &dim $\{\beta\in\mathbb{R}^3:f\brak{\alpha,\beta}=0$ for all $\alpha\in\mathbb{R}^3\}=dim\{Nullspace\brak{T_1}\}=dim\{Nullspace\brak{T_2}\}=2$\\
    \hline
    Option 3&By using rank nullity theorem we get $rank\brak{f}+dim\{nullspace\brak{f}\}=3$. We know that \\
    &$rank\brak{f}=1\implies dim\{nullspace\brak{f}\}=2$. Therefore two eigen values of $f$ will be $0$.\\
    &Since the matrix is a symmetric matrix the eigen values are real. So, the third eigen value\\
    & can be either positive or negative. So, the matrix will be either positive semi-definite\\
    & or negative semi-definite accordingly.This option is correct.\\
    \hline
    Option 4&$\{\alpha:f\brak{\alpha,\alpha}=0\}$ is a linear subspace of dimension 2. Since the $dim\{nullspace\brak{f}\}=2$,\\
    &and $f$ is diagonalizable,since it is a symmetric, the two eigen vectors corresponding to $0$\\
    & eigen values form a subspace of dimension 2.\\
    \hline
    \caption{Answer}
    \label{eq:solutions/2014/dec/73/tab:Ans}
\end{longtable}
\begin{longtable}{|l|l|}
    \hline
    Construction &Consider the non-zero symmetric bilinear form $f\brak{\alpha,\beta}=T_1\brak{\alpha}T_2\brak{\beta}$ on $\mathbb{R}^3$ where\\
    &Where the matrix of linear transformations are $\vec{T_1}=\myvec{1\\0\\1}$ and $\vec{T_2}=\myvec{2\\0\\2}$.\\
    &The matrix of symmetric bilinear form is $f=\myvec{2&0&2\\0&0&0\\2&0&2}$.The $rank\brak{f}=1$.\\
    &$f\brak{\alpha,\beta}=\alpha^T\myvec{2&0&2\\0&0&0\\2&0&2}\beta$\\
    &The characteristic equation is  $\abs{f-\lambda\vec{I}}=\lambda^2\brak{\lambda-4}$. So the eigen values are $0,0,4$\\
    \hline
    &Therefore $f$ is positive semi-definite.\\
    &$f\brak{\alpha,\beta}=0$ for all $\alpha\in\mathbb{R}^3$, then $\beta=xe_1+ye_2$ where $e_1=\myvec{0\\1\\0}$ and $e_2=\myvec{1\\0\\-1}$.Therefore\\
    &dim $\{\beta\in\mathbb{R}^3:f\brak{\alpha,\beta}=0$ for all $\alpha\in\mathbb{R}^3\}=2$\\
    &$\alpha:f\brak{\alpha,\alpha}=0$ also has a dimension of 2 which forms the nullspace of $f$, where \\
    &nullspace of $f$ is the $span\{e_1,e_2\}$\\ 
    \hline
    \caption{Example}
    \label{eq:solutions/2014/dec/73/tab:example}
\end{longtable}
\twocolumn
