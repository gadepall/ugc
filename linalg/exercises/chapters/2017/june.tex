\renewcommand{\theequation}{\theenumi}
\renewcommand{\thefigure}{\theenumi}
\begin{enumerate}[label=\thesection.\arabic*.,ref=\thesection.\theenumi]
\numberwithin{equation}{enumi}
\numberwithin{figure}{enumi}

\item 	Let $\vec{A}$ be a $4 \times 4$ matrix. Suppose that the null space $N(\vec{A})$ of $\vec{A}$ is
	\begin{align}
		\cbrak{(x,y,z,w) \in \vec{R}^4 : x + y + z = 0 , x + y + w = 0}
	\end{align}
Then which one of the following is correct
\begin{enumerate}
\item  dim(column space$(\vec{A})) = 1$ 
\item dim(column space$(\vec{A})) = 2$
\item rank$(\vec{A}) = 1$
\item $\vec{S}= \cbrak{(1, 1, 1, 0), (1, 1, 0, 1)}$ is a basis of $N(\vec{A})$
\end{enumerate}
%
%
\solution
See Tables \ref{eq:solutions/2018/dec/106/table0} and \ref{eq:solutions/2018/dec/106/table1}


\onecolumn
	\begin{longtable}{|l|l|}
		\hline
		\multirow{3}{*}{Irreducible Markov Chain} 
		& \\
		& A Markov chain is $\textbf{irreducible}$ if all the states communicate with each other,\\
		& i.e., if there is only one communication class.\\
		&\\
		\hline
		\multirow{3}{*}{Aperiodic Markov Chain} & \\
		& If there is a self-transition in the chain ($p^{ii}>0$ for some i), then the chain is\\
		& called as $\textbf{aperiodic}$\\
		& \\
		\hline
		\multirow{3}{*}{Stationary Distribution} & \\
		& A stationary distribution of a Markov chain is a probability distribution that\\
		& remains unchanged in the Markov chain as time progresses. Typically, it is\\
		& represented as a row vector $\Vec{\pi}$ whose entries are probabilities summing to 1,\\ 
		& and given transition matrix $\textbf{P}$, it satisfies\\
		& \\
		&  \qquad \qquad  \qquad$\Vec{\pi} = \Vec{\pi} \textbf{P}$\\
		& \\
		\hline
\caption{}
\label{eq:solutions/2018/dec/106/table0}
	\end{longtable}
	\begin{longtable}{|l|l|}
		\hline
		\multirow{3}{*}{Drawing Transition diagram} 
		& \\
		& 
		
		$\begin{tikzpicture}[shorten >=1pt,node distance=2cm, scale =3, auto]
			\tikzstyle{every state}=[fill={rgb:black,1;white,10}]
			
			\node[state]   (q_1)                          {$1$};
			\node[state]   (q_2)  [right of=q_1]          {$2$};
			\node[state]   (q_3)  [below right of=q_1]          {$3$};
			
			\path[->]
			(q_1) edge [loop above] node {$\frac{1}{2}$}    (   )
			edge [bend left]  node {$\frac{1}{2}$}    (q_2)
			(q_2) edge [bend left]  node {$\frac{1}{2}$}    (q_3)
			edge [loop above] node {$\frac{1}{2}$}    ()
			(q_3) edge [bend left]  node {$\frac{1}{3}$}    (q_2)
			edge [bend left]  node {$\frac{1}{3}$}    (q_1)
			edge [loop below] node {$\frac{1}{3}$}    ();
		\end{tikzpicture}$
		
		\\  
		&\\
		&\\
		\hline
		\multirow{3}{*}{Checking whether the  } & \\
		& Here,\\chain is Irreducible
		& All the states are accessible to one another. \\and Aperiodic
		& $\implies$ They are in the same communication class. So, it is Irreducible.\\
		& \\
		& There exists the non- zero self-transition, which means that the chain \\
		& is Aperiodic.\\
		&\\ 
		& We know that if the Markov Chain is irreducible and aperiodic then \\
		& \qquad \qquad \qquad $\Vec{\pi}_{j} = \lim_{n \to \infty}P\{X_{n} = j\}$, $j = 1,...,N$ \\
		& These are the stationary probabilities. \\
		&\\
		\hline
		\multirow{3}{*}{Finding the Stationary} & \\
		& Stationary Probability can be represented as\\Probability Distributions
		& \qquad \qquad \qquad $\Vec{\pi} = \Vec{\pi} \vec{P}$\\
		& \\
		& \qquad $\implies$ $\myvec{v_{1}&&v_{2}&&v_{3}} = \myvec{v_{1}&&v_{2}&&v_{3}}\vec{P}$ \\
		& \\
		& Equating the above equation we get \\
		& \\
		& \qquad \qquad \qquad $\frac{1}{2}v_{1}-\frac{1}{3}v_{3} = 0$ $\label{eq:solutions/2018/dec/106/eq}$\\
		& \\
		& \qquad \qquad \qquad $\frac{1}{2}v_{1}-\frac{1}{2}v_{2} + \frac{1}{3}v_{3} = 0$\\
		& \\
		& \qquad \qquad \qquad $\frac{1}{2}v_{2}-\frac{2}{3}v_{3} = 0$\\
		& \\\
		& We see that summation of second and the third equation gives us the \\
		& first equation only. \\
		& And we know that the probability distribution will sum up to 1. \\
		& \\
		& \qquad \qquad \qquad $v_{1}+v_{2}+v_{3} = 1$ \\
		& \\
		& Therefore, we get the equation form as \\
		& \\
		& \qquad \qquad \qquad $\myvec{1&1&1\\\frac{1}{2}&0&\frac{-1}{3}\\\frac{1}{2}&\frac{-1}{2}&\frac{1}{3}}\myvec{v_{1}\\v_{2}\\v_{3}} = \myvec{1\\0\\0}$ \\
		& \\
		\hline
		\multirow{3}{*}{Solving the linear} & \\
		& The above linear equation can be solved using Gauss-Jordan method as\\equtions
		& \\
		& \qquad \qquad \qquad $\myvec{1&1&1&\vrule&1\\\frac{1}{2}&0&\frac{-1}{3}&\vrule&0\\\frac{1}{2}&\frac{-1}{2}&\frac{1}{3}&\vrule&0}$\\
		& \\
		& \qquad $\xleftrightarrow[]{R_2 \leftarrow R_2 - \frac{1}{2}R_1}$
		$\myvec{1&1&1&\vrule&1\\0&\frac{-1}{2}&\frac{-5}{6}&\vrule&\frac{-1}{2}\\\frac{1}{2}&\frac{-1}{2}&\frac{1}{3}&\vrule&0}$\\
		&\\
		& \qquad $\xleftrightarrow[]{R_3 \leftarrow R_3 - \frac{1}{2}R_1}$
		$\myvec{1&1&1&\vrule&1\\0&\frac{-1}{2}&\frac{-5}{6}&\vrule&\frac{-1}{2}\\0&-1&\frac{-1}{6}&\vrule&\frac{-1}{2}}$\\
		&\\
		& \qquad $\xleftrightarrow[]{R_2 \leftarrow \frac{-1}{2}R_2}$
		$\myvec{1&1&1&\vrule&1\\0&1&\frac{5}{3}&\vrule&1\\0&-1&\frac{-1}{6}&\vrule&\frac{-1}{2}}$\\
		&\\
		& \qquad $\xleftrightarrow[]{R_3 \leftarrow R_3 + R_2}$
		$\myvec{1&1&1&\vrule&1\\0&1&\frac{5}{3}&\vrule&1\\0&0&\frac{3}{2}&\vrule&\frac{1}{2}}$\\
		&\\
		& \qquad $\xleftrightarrow[]{R_3 \leftarrow \frac{3}{2}R_3}$
		$\myvec{1&1&1&\vrule&1\\0&1&\frac{5}{3}&\vrule&1\\0&0&1&\vrule&\frac{1}{3}}$\\
		&\\
		& \qquad $\xleftrightarrow[]{R_2 \leftarrow R_2 - \frac{5}{3}R_3}$
		$\myvec{1&1&1&\vrule&1\\0&1&0&\vrule&\frac{4}{9}\\0&0&1&\vrule&\frac{1}{3}}$\\
		&\\
		& \qquad $\xleftrightarrow[]{R_1 \leftarrow R_1 - R_3}$
		$\myvec{1&1&0&\vrule&\frac{2}{3}\\0&1&0&\vrule&\frac{4}{9}\\0&0&1&\vrule&\frac{1}{3}}$\\
		&\\
		& \qquad $\xleftrightarrow[]{R_1 \leftarrow R_1 - R_2}$
		$\myvec{1&0&0&\vrule&\frac{2}{9}\\0&1&0&\vrule&\frac{4}{9}\\0&0&1&\vrule&\frac{1}{3}}$\\
		&\\
		& $\therefore$, stationary probability distribution $\pi$ is given by \\
		& \qquad \qquad $\pi = \myvec{\frac{2}{9} & \frac{4}{9} & \frac{1}{3}}$ \\
		& \\
		\hline
		\multirow{3}{*}{Observations} & \\
		
		
		& Since the given transition probability matrix $\vec{P}$ is irreducible and aperiodic, \\
		& then $\lim_{n \to \infty} \vec{P}^{n}$ converges to a matrix with all rows identical and equal to $\vec{\pi}$. \\
		& \\
		& We were able to find $\vec{\pi}$ as $\myvec{\frac{2}{9} & \frac{4}{9} & \frac{1}{3}}$ \\
		& \\
		& $\lim_{n \to \infty} \vec{P}^{n} = \myvec{\frac{2}{9}&\frac{4}{9}&\frac{1}{3}\\\frac{2}{9}&\frac{4}{9}&\frac{1}{3}\\\frac{2}{9}&\frac{4}{9}&\frac{1}{3}}$\\
		& \\
		& From the above matrix, we get \\
		& \\
		& $\lim_{n \to \infty} \vec{P}^{n}_{11} = \frac{2}{9}$ \\
		&\\
		& $\lim_{n \to \infty} \vec{P}^{n}_{21} = \frac{2}{9}$ \\
		&\\
		& $\lim_{n \to \infty} \vec{P}^{n}_{32} = \frac{4}{9}$ \\
		&\\
		& $\lim_{n \to \infty} \vec{P}^{n}_{13} = \frac{1}{3}$ \\
		&\\
		\hline
		\multirow{3}{*}{Conclusion} & \\
		& From our observation we see that \\
		&\\
		& Options 1) and 4) are True.\\
		& \\
		\hline
\caption{}
\label{eq:solutions/2018/dec/106/table1}
	\end{longtable}
\twocolumn

\item Let $\vec{A}$ and $\vec{B}$ be real invertible matrices such that 
\begin{align}
    \vec{AB}=-\vec{BA}\label{eq:eq:solutions/2017/june/28/eq1}.
\end{align}
Then
\begin{enumerate}
    \item trace{$\vec{A}$} = trace($\vec{B}$) = 0
    \item trace{$\vec{A}$} = trace($\vec{B}$) = 1
    \item trace{$\vec{A}$} = 0, trace($\vec{B}$) = 1
    \item trace($\vec{A}$) = 1, trace($\vec{B}$) = 0
\end{enumerate}
%
\solution
See Tables \ref{eq:solutions/2018/dec/106/table0} and \ref{eq:solutions/2018/dec/106/table1}


\onecolumn
	\begin{longtable}{|l|l|}
		\hline
		\multirow{3}{*}{Irreducible Markov Chain} 
		& \\
		& A Markov chain is $\textbf{irreducible}$ if all the states communicate with each other,\\
		& i.e., if there is only one communication class.\\
		&\\
		\hline
		\multirow{3}{*}{Aperiodic Markov Chain} & \\
		& If there is a self-transition in the chain ($p^{ii}>0$ for some i), then the chain is\\
		& called as $\textbf{aperiodic}$\\
		& \\
		\hline
		\multirow{3}{*}{Stationary Distribution} & \\
		& A stationary distribution of a Markov chain is a probability distribution that\\
		& remains unchanged in the Markov chain as time progresses. Typically, it is\\
		& represented as a row vector $\Vec{\pi}$ whose entries are probabilities summing to 1,\\ 
		& and given transition matrix $\textbf{P}$, it satisfies\\
		& \\
		&  \qquad \qquad  \qquad$\Vec{\pi} = \Vec{\pi} \textbf{P}$\\
		& \\
		\hline
\caption{}
\label{eq:solutions/2018/dec/106/table0}
	\end{longtable}
	\begin{longtable}{|l|l|}
		\hline
		\multirow{3}{*}{Drawing Transition diagram} 
		& \\
		& 
		
		$\begin{tikzpicture}[shorten >=1pt,node distance=2cm, scale =3, auto]
			\tikzstyle{every state}=[fill={rgb:black,1;white,10}]
			
			\node[state]   (q_1)                          {$1$};
			\node[state]   (q_2)  [right of=q_1]          {$2$};
			\node[state]   (q_3)  [below right of=q_1]          {$3$};
			
			\path[->]
			(q_1) edge [loop above] node {$\frac{1}{2}$}    (   )
			edge [bend left]  node {$\frac{1}{2}$}    (q_2)
			(q_2) edge [bend left]  node {$\frac{1}{2}$}    (q_3)
			edge [loop above] node {$\frac{1}{2}$}    ()
			(q_3) edge [bend left]  node {$\frac{1}{3}$}    (q_2)
			edge [bend left]  node {$\frac{1}{3}$}    (q_1)
			edge [loop below] node {$\frac{1}{3}$}    ();
		\end{tikzpicture}$
		
		\\  
		&\\
		&\\
		\hline
		\multirow{3}{*}{Checking whether the  } & \\
		& Here,\\chain is Irreducible
		& All the states are accessible to one another. \\and Aperiodic
		& $\implies$ They are in the same communication class. So, it is Irreducible.\\
		& \\
		& There exists the non- zero self-transition, which means that the chain \\
		& is Aperiodic.\\
		&\\ 
		& We know that if the Markov Chain is irreducible and aperiodic then \\
		& \qquad \qquad \qquad $\Vec{\pi}_{j} = \lim_{n \to \infty}P\{X_{n} = j\}$, $j = 1,...,N$ \\
		& These are the stationary probabilities. \\
		&\\
		\hline
		\multirow{3}{*}{Finding the Stationary} & \\
		& Stationary Probability can be represented as\\Probability Distributions
		& \qquad \qquad \qquad $\Vec{\pi} = \Vec{\pi} \vec{P}$\\
		& \\
		& \qquad $\implies$ $\myvec{v_{1}&&v_{2}&&v_{3}} = \myvec{v_{1}&&v_{2}&&v_{3}}\vec{P}$ \\
		& \\
		& Equating the above equation we get \\
		& \\
		& \qquad \qquad \qquad $\frac{1}{2}v_{1}-\frac{1}{3}v_{3} = 0$ $\label{eq:solutions/2018/dec/106/eq}$\\
		& \\
		& \qquad \qquad \qquad $\frac{1}{2}v_{1}-\frac{1}{2}v_{2} + \frac{1}{3}v_{3} = 0$\\
		& \\
		& \qquad \qquad \qquad $\frac{1}{2}v_{2}-\frac{2}{3}v_{3} = 0$\\
		& \\\
		& We see that summation of second and the third equation gives us the \\
		& first equation only. \\
		& And we know that the probability distribution will sum up to 1. \\
		& \\
		& \qquad \qquad \qquad $v_{1}+v_{2}+v_{3} = 1$ \\
		& \\
		& Therefore, we get the equation form as \\
		& \\
		& \qquad \qquad \qquad $\myvec{1&1&1\\\frac{1}{2}&0&\frac{-1}{3}\\\frac{1}{2}&\frac{-1}{2}&\frac{1}{3}}\myvec{v_{1}\\v_{2}\\v_{3}} = \myvec{1\\0\\0}$ \\
		& \\
		\hline
		\multirow{3}{*}{Solving the linear} & \\
		& The above linear equation can be solved using Gauss-Jordan method as\\equtions
		& \\
		& \qquad \qquad \qquad $\myvec{1&1&1&\vrule&1\\\frac{1}{2}&0&\frac{-1}{3}&\vrule&0\\\frac{1}{2}&\frac{-1}{2}&\frac{1}{3}&\vrule&0}$\\
		& \\
		& \qquad $\xleftrightarrow[]{R_2 \leftarrow R_2 - \frac{1}{2}R_1}$
		$\myvec{1&1&1&\vrule&1\\0&\frac{-1}{2}&\frac{-5}{6}&\vrule&\frac{-1}{2}\\\frac{1}{2}&\frac{-1}{2}&\frac{1}{3}&\vrule&0}$\\
		&\\
		& \qquad $\xleftrightarrow[]{R_3 \leftarrow R_3 - \frac{1}{2}R_1}$
		$\myvec{1&1&1&\vrule&1\\0&\frac{-1}{2}&\frac{-5}{6}&\vrule&\frac{-1}{2}\\0&-1&\frac{-1}{6}&\vrule&\frac{-1}{2}}$\\
		&\\
		& \qquad $\xleftrightarrow[]{R_2 \leftarrow \frac{-1}{2}R_2}$
		$\myvec{1&1&1&\vrule&1\\0&1&\frac{5}{3}&\vrule&1\\0&-1&\frac{-1}{6}&\vrule&\frac{-1}{2}}$\\
		&\\
		& \qquad $\xleftrightarrow[]{R_3 \leftarrow R_3 + R_2}$
		$\myvec{1&1&1&\vrule&1\\0&1&\frac{5}{3}&\vrule&1\\0&0&\frac{3}{2}&\vrule&\frac{1}{2}}$\\
		&\\
		& \qquad $\xleftrightarrow[]{R_3 \leftarrow \frac{3}{2}R_3}$
		$\myvec{1&1&1&\vrule&1\\0&1&\frac{5}{3}&\vrule&1\\0&0&1&\vrule&\frac{1}{3}}$\\
		&\\
		& \qquad $\xleftrightarrow[]{R_2 \leftarrow R_2 - \frac{5}{3}R_3}$
		$\myvec{1&1&1&\vrule&1\\0&1&0&\vrule&\frac{4}{9}\\0&0&1&\vrule&\frac{1}{3}}$\\
		&\\
		& \qquad $\xleftrightarrow[]{R_1 \leftarrow R_1 - R_3}$
		$\myvec{1&1&0&\vrule&\frac{2}{3}\\0&1&0&\vrule&\frac{4}{9}\\0&0&1&\vrule&\frac{1}{3}}$\\
		&\\
		& \qquad $\xleftrightarrow[]{R_1 \leftarrow R_1 - R_2}$
		$\myvec{1&0&0&\vrule&\frac{2}{9}\\0&1&0&\vrule&\frac{4}{9}\\0&0&1&\vrule&\frac{1}{3}}$\\
		&\\
		& $\therefore$, stationary probability distribution $\pi$ is given by \\
		& \qquad \qquad $\pi = \myvec{\frac{2}{9} & \frac{4}{9} & \frac{1}{3}}$ \\
		& \\
		\hline
		\multirow{3}{*}{Observations} & \\
		
		
		& Since the given transition probability matrix $\vec{P}$ is irreducible and aperiodic, \\
		& then $\lim_{n \to \infty} \vec{P}^{n}$ converges to a matrix with all rows identical and equal to $\vec{\pi}$. \\
		& \\
		& We were able to find $\vec{\pi}$ as $\myvec{\frac{2}{9} & \frac{4}{9} & \frac{1}{3}}$ \\
		& \\
		& $\lim_{n \to \infty} \vec{P}^{n} = \myvec{\frac{2}{9}&\frac{4}{9}&\frac{1}{3}\\\frac{2}{9}&\frac{4}{9}&\frac{1}{3}\\\frac{2}{9}&\frac{4}{9}&\frac{1}{3}}$\\
		& \\
		& From the above matrix, we get \\
		& \\
		& $\lim_{n \to \infty} \vec{P}^{n}_{11} = \frac{2}{9}$ \\
		&\\
		& $\lim_{n \to \infty} \vec{P}^{n}_{21} = \frac{2}{9}$ \\
		&\\
		& $\lim_{n \to \infty} \vec{P}^{n}_{32} = \frac{4}{9}$ \\
		&\\
		& $\lim_{n \to \infty} \vec{P}^{n}_{13} = \frac{1}{3}$ \\
		&\\
		\hline
		\multirow{3}{*}{Conclusion} & \\
		& From our observation we see that \\
		&\\
		& Options 1) and 4) are True.\\
		& \\
		\hline
\caption{}
\label{eq:solutions/2018/dec/106/table1}
	\end{longtable}
\twocolumn

\item Let $\vec{A}$ be an n $\times$ n self-adjoint matrix with eigenvalues $\lambda_1, \cdots, \lambda_2$.
Let,\begin{align} \norm{\vec{X}}_2=\sqrt{\vert\vec{X}_{1}^{2}\vert+\cdots+\vert\vec{X}_{n}^{2}\vert}\end{align} for $\vec{X}$=$(\vec{X}_{1},\cdots,\vec{X}_{n})\in \mathbb{C}^n$. If \begin{align}
p(\vec{A})=a_0\vec{I}+a_1\vec{A}+\cdots+a_n\vec{A}^n
\end{align}
then $sup_{\norm{\vec{X}}_{2}=1}\norm{p(\vec{A})\vec{X}}_2$ is equal to
%
\\
\solution
See Tables \ref{eq:solutions/2018/dec/106/table0} and \ref{eq:solutions/2018/dec/106/table1}


\onecolumn
	\begin{longtable}{|l|l|}
		\hline
		\multirow{3}{*}{Irreducible Markov Chain} 
		& \\
		& A Markov chain is $\textbf{irreducible}$ if all the states communicate with each other,\\
		& i.e., if there is only one communication class.\\
		&\\
		\hline
		\multirow{3}{*}{Aperiodic Markov Chain} & \\
		& If there is a self-transition in the chain ($p^{ii}>0$ for some i), then the chain is\\
		& called as $\textbf{aperiodic}$\\
		& \\
		\hline
		\multirow{3}{*}{Stationary Distribution} & \\
		& A stationary distribution of a Markov chain is a probability distribution that\\
		& remains unchanged in the Markov chain as time progresses. Typically, it is\\
		& represented as a row vector $\Vec{\pi}$ whose entries are probabilities summing to 1,\\ 
		& and given transition matrix $\textbf{P}$, it satisfies\\
		& \\
		&  \qquad \qquad  \qquad$\Vec{\pi} = \Vec{\pi} \textbf{P}$\\
		& \\
		\hline
\caption{}
\label{eq:solutions/2018/dec/106/table0}
	\end{longtable}
	\begin{longtable}{|l|l|}
		\hline
		\multirow{3}{*}{Drawing Transition diagram} 
		& \\
		& 
		
		$\begin{tikzpicture}[shorten >=1pt,node distance=2cm, scale =3, auto]
			\tikzstyle{every state}=[fill={rgb:black,1;white,10}]
			
			\node[state]   (q_1)                          {$1$};
			\node[state]   (q_2)  [right of=q_1]          {$2$};
			\node[state]   (q_3)  [below right of=q_1]          {$3$};
			
			\path[->]
			(q_1) edge [loop above] node {$\frac{1}{2}$}    (   )
			edge [bend left]  node {$\frac{1}{2}$}    (q_2)
			(q_2) edge [bend left]  node {$\frac{1}{2}$}    (q_3)
			edge [loop above] node {$\frac{1}{2}$}    ()
			(q_3) edge [bend left]  node {$\frac{1}{3}$}    (q_2)
			edge [bend left]  node {$\frac{1}{3}$}    (q_1)
			edge [loop below] node {$\frac{1}{3}$}    ();
		\end{tikzpicture}$
		
		\\  
		&\\
		&\\
		\hline
		\multirow{3}{*}{Checking whether the  } & \\
		& Here,\\chain is Irreducible
		& All the states are accessible to one another. \\and Aperiodic
		& $\implies$ They are in the same communication class. So, it is Irreducible.\\
		& \\
		& There exists the non- zero self-transition, which means that the chain \\
		& is Aperiodic.\\
		&\\ 
		& We know that if the Markov Chain is irreducible and aperiodic then \\
		& \qquad \qquad \qquad $\Vec{\pi}_{j} = \lim_{n \to \infty}P\{X_{n} = j\}$, $j = 1,...,N$ \\
		& These are the stationary probabilities. \\
		&\\
		\hline
		\multirow{3}{*}{Finding the Stationary} & \\
		& Stationary Probability can be represented as\\Probability Distributions
		& \qquad \qquad \qquad $\Vec{\pi} = \Vec{\pi} \vec{P}$\\
		& \\
		& \qquad $\implies$ $\myvec{v_{1}&&v_{2}&&v_{3}} = \myvec{v_{1}&&v_{2}&&v_{3}}\vec{P}$ \\
		& \\
		& Equating the above equation we get \\
		& \\
		& \qquad \qquad \qquad $\frac{1}{2}v_{1}-\frac{1}{3}v_{3} = 0$ $\label{eq:solutions/2018/dec/106/eq}$\\
		& \\
		& \qquad \qquad \qquad $\frac{1}{2}v_{1}-\frac{1}{2}v_{2} + \frac{1}{3}v_{3} = 0$\\
		& \\
		& \qquad \qquad \qquad $\frac{1}{2}v_{2}-\frac{2}{3}v_{3} = 0$\\
		& \\\
		& We see that summation of second and the third equation gives us the \\
		& first equation only. \\
		& And we know that the probability distribution will sum up to 1. \\
		& \\
		& \qquad \qquad \qquad $v_{1}+v_{2}+v_{3} = 1$ \\
		& \\
		& Therefore, we get the equation form as \\
		& \\
		& \qquad \qquad \qquad $\myvec{1&1&1\\\frac{1}{2}&0&\frac{-1}{3}\\\frac{1}{2}&\frac{-1}{2}&\frac{1}{3}}\myvec{v_{1}\\v_{2}\\v_{3}} = \myvec{1\\0\\0}$ \\
		& \\
		\hline
		\multirow{3}{*}{Solving the linear} & \\
		& The above linear equation can be solved using Gauss-Jordan method as\\equtions
		& \\
		& \qquad \qquad \qquad $\myvec{1&1&1&\vrule&1\\\frac{1}{2}&0&\frac{-1}{3}&\vrule&0\\\frac{1}{2}&\frac{-1}{2}&\frac{1}{3}&\vrule&0}$\\
		& \\
		& \qquad $\xleftrightarrow[]{R_2 \leftarrow R_2 - \frac{1}{2}R_1}$
		$\myvec{1&1&1&\vrule&1\\0&\frac{-1}{2}&\frac{-5}{6}&\vrule&\frac{-1}{2}\\\frac{1}{2}&\frac{-1}{2}&\frac{1}{3}&\vrule&0}$\\
		&\\
		& \qquad $\xleftrightarrow[]{R_3 \leftarrow R_3 - \frac{1}{2}R_1}$
		$\myvec{1&1&1&\vrule&1\\0&\frac{-1}{2}&\frac{-5}{6}&\vrule&\frac{-1}{2}\\0&-1&\frac{-1}{6}&\vrule&\frac{-1}{2}}$\\
		&\\
		& \qquad $\xleftrightarrow[]{R_2 \leftarrow \frac{-1}{2}R_2}$
		$\myvec{1&1&1&\vrule&1\\0&1&\frac{5}{3}&\vrule&1\\0&-1&\frac{-1}{6}&\vrule&\frac{-1}{2}}$\\
		&\\
		& \qquad $\xleftrightarrow[]{R_3 \leftarrow R_3 + R_2}$
		$\myvec{1&1&1&\vrule&1\\0&1&\frac{5}{3}&\vrule&1\\0&0&\frac{3}{2}&\vrule&\frac{1}{2}}$\\
		&\\
		& \qquad $\xleftrightarrow[]{R_3 \leftarrow \frac{3}{2}R_3}$
		$\myvec{1&1&1&\vrule&1\\0&1&\frac{5}{3}&\vrule&1\\0&0&1&\vrule&\frac{1}{3}}$\\
		&\\
		& \qquad $\xleftrightarrow[]{R_2 \leftarrow R_2 - \frac{5}{3}R_3}$
		$\myvec{1&1&1&\vrule&1\\0&1&0&\vrule&\frac{4}{9}\\0&0&1&\vrule&\frac{1}{3}}$\\
		&\\
		& \qquad $\xleftrightarrow[]{R_1 \leftarrow R_1 - R_3}$
		$\myvec{1&1&0&\vrule&\frac{2}{3}\\0&1&0&\vrule&\frac{4}{9}\\0&0&1&\vrule&\frac{1}{3}}$\\
		&\\
		& \qquad $\xleftrightarrow[]{R_1 \leftarrow R_1 - R_2}$
		$\myvec{1&0&0&\vrule&\frac{2}{9}\\0&1&0&\vrule&\frac{4}{9}\\0&0&1&\vrule&\frac{1}{3}}$\\
		&\\
		& $\therefore$, stationary probability distribution $\pi$ is given by \\
		& \qquad \qquad $\pi = \myvec{\frac{2}{9} & \frac{4}{9} & \frac{1}{3}}$ \\
		& \\
		\hline
		\multirow{3}{*}{Observations} & \\
		
		
		& Since the given transition probability matrix $\vec{P}$ is irreducible and aperiodic, \\
		& then $\lim_{n \to \infty} \vec{P}^{n}$ converges to a matrix with all rows identical and equal to $\vec{\pi}$. \\
		& \\
		& We were able to find $\vec{\pi}$ as $\myvec{\frac{2}{9} & \frac{4}{9} & \frac{1}{3}}$ \\
		& \\
		& $\lim_{n \to \infty} \vec{P}^{n} = \myvec{\frac{2}{9}&\frac{4}{9}&\frac{1}{3}\\\frac{2}{9}&\frac{4}{9}&\frac{1}{3}\\\frac{2}{9}&\frac{4}{9}&\frac{1}{3}}$\\
		& \\
		& From the above matrix, we get \\
		& \\
		& $\lim_{n \to \infty} \vec{P}^{n}_{11} = \frac{2}{9}$ \\
		&\\
		& $\lim_{n \to \infty} \vec{P}^{n}_{21} = \frac{2}{9}$ \\
		&\\
		& $\lim_{n \to \infty} \vec{P}^{n}_{32} = \frac{4}{9}$ \\
		&\\
		& $\lim_{n \to \infty} \vec{P}^{n}_{13} = \frac{1}{3}$ \\
		&\\
		\hline
		\multirow{3}{*}{Conclusion} & \\
		& From our observation we see that \\
		&\\
		& Options 1) and 4) are True.\\
		& \\
		\hline
\caption{}
\label{eq:solutions/2018/dec/106/table1}
	\end{longtable}
\twocolumn

\item Let $p \brak{x}= \alpha x^2+\beta x + \gamma$ be a polynomial, where $\alpha,\beta,\gamma \epsilon R$. Fix $X_0 \epsilon R$. Let $S=\{\brak{a,b,c}  \epsilon R^3: p \brak{x}= a \brak{x-x_0}^2+b \brak{x-x_0}+ c\}$ for all $x\epsilon R$. Find the number of elements in S is
\begin{enumerate}
    \item 0
    \item 1
    \item Strictly greater than 1 but finite
    \item Infinite
\end{enumerate}
%
\solution
See Tables \ref{eq:solutions/2018/dec/106/table0} and \ref{eq:solutions/2018/dec/106/table1}


\onecolumn
	\begin{longtable}{|l|l|}
		\hline
		\multirow{3}{*}{Irreducible Markov Chain} 
		& \\
		& A Markov chain is $\textbf{irreducible}$ if all the states communicate with each other,\\
		& i.e., if there is only one communication class.\\
		&\\
		\hline
		\multirow{3}{*}{Aperiodic Markov Chain} & \\
		& If there is a self-transition in the chain ($p^{ii}>0$ for some i), then the chain is\\
		& called as $\textbf{aperiodic}$\\
		& \\
		\hline
		\multirow{3}{*}{Stationary Distribution} & \\
		& A stationary distribution of a Markov chain is a probability distribution that\\
		& remains unchanged in the Markov chain as time progresses. Typically, it is\\
		& represented as a row vector $\Vec{\pi}$ whose entries are probabilities summing to 1,\\ 
		& and given transition matrix $\textbf{P}$, it satisfies\\
		& \\
		&  \qquad \qquad  \qquad$\Vec{\pi} = \Vec{\pi} \textbf{P}$\\
		& \\
		\hline
\caption{}
\label{eq:solutions/2018/dec/106/table0}
	\end{longtable}
	\begin{longtable}{|l|l|}
		\hline
		\multirow{3}{*}{Drawing Transition diagram} 
		& \\
		& 
		
		$\begin{tikzpicture}[shorten >=1pt,node distance=2cm, scale =3, auto]
			\tikzstyle{every state}=[fill={rgb:black,1;white,10}]
			
			\node[state]   (q_1)                          {$1$};
			\node[state]   (q_2)  [right of=q_1]          {$2$};
			\node[state]   (q_3)  [below right of=q_1]          {$3$};
			
			\path[->]
			(q_1) edge [loop above] node {$\frac{1}{2}$}    (   )
			edge [bend left]  node {$\frac{1}{2}$}    (q_2)
			(q_2) edge [bend left]  node {$\frac{1}{2}$}    (q_3)
			edge [loop above] node {$\frac{1}{2}$}    ()
			(q_3) edge [bend left]  node {$\frac{1}{3}$}    (q_2)
			edge [bend left]  node {$\frac{1}{3}$}    (q_1)
			edge [loop below] node {$\frac{1}{3}$}    ();
		\end{tikzpicture}$
		
		\\  
		&\\
		&\\
		\hline
		\multirow{3}{*}{Checking whether the  } & \\
		& Here,\\chain is Irreducible
		& All the states are accessible to one another. \\and Aperiodic
		& $\implies$ They are in the same communication class. So, it is Irreducible.\\
		& \\
		& There exists the non- zero self-transition, which means that the chain \\
		& is Aperiodic.\\
		&\\ 
		& We know that if the Markov Chain is irreducible and aperiodic then \\
		& \qquad \qquad \qquad $\Vec{\pi}_{j} = \lim_{n \to \infty}P\{X_{n} = j\}$, $j = 1,...,N$ \\
		& These are the stationary probabilities. \\
		&\\
		\hline
		\multirow{3}{*}{Finding the Stationary} & \\
		& Stationary Probability can be represented as\\Probability Distributions
		& \qquad \qquad \qquad $\Vec{\pi} = \Vec{\pi} \vec{P}$\\
		& \\
		& \qquad $\implies$ $\myvec{v_{1}&&v_{2}&&v_{3}} = \myvec{v_{1}&&v_{2}&&v_{3}}\vec{P}$ \\
		& \\
		& Equating the above equation we get \\
		& \\
		& \qquad \qquad \qquad $\frac{1}{2}v_{1}-\frac{1}{3}v_{3} = 0$ $\label{eq:solutions/2018/dec/106/eq}$\\
		& \\
		& \qquad \qquad \qquad $\frac{1}{2}v_{1}-\frac{1}{2}v_{2} + \frac{1}{3}v_{3} = 0$\\
		& \\
		& \qquad \qquad \qquad $\frac{1}{2}v_{2}-\frac{2}{3}v_{3} = 0$\\
		& \\\
		& We see that summation of second and the third equation gives us the \\
		& first equation only. \\
		& And we know that the probability distribution will sum up to 1. \\
		& \\
		& \qquad \qquad \qquad $v_{1}+v_{2}+v_{3} = 1$ \\
		& \\
		& Therefore, we get the equation form as \\
		& \\
		& \qquad \qquad \qquad $\myvec{1&1&1\\\frac{1}{2}&0&\frac{-1}{3}\\\frac{1}{2}&\frac{-1}{2}&\frac{1}{3}}\myvec{v_{1}\\v_{2}\\v_{3}} = \myvec{1\\0\\0}$ \\
		& \\
		\hline
		\multirow{3}{*}{Solving the linear} & \\
		& The above linear equation can be solved using Gauss-Jordan method as\\equtions
		& \\
		& \qquad \qquad \qquad $\myvec{1&1&1&\vrule&1\\\frac{1}{2}&0&\frac{-1}{3}&\vrule&0\\\frac{1}{2}&\frac{-1}{2}&\frac{1}{3}&\vrule&0}$\\
		& \\
		& \qquad $\xleftrightarrow[]{R_2 \leftarrow R_2 - \frac{1}{2}R_1}$
		$\myvec{1&1&1&\vrule&1\\0&\frac{-1}{2}&\frac{-5}{6}&\vrule&\frac{-1}{2}\\\frac{1}{2}&\frac{-1}{2}&\frac{1}{3}&\vrule&0}$\\
		&\\
		& \qquad $\xleftrightarrow[]{R_3 \leftarrow R_3 - \frac{1}{2}R_1}$
		$\myvec{1&1&1&\vrule&1\\0&\frac{-1}{2}&\frac{-5}{6}&\vrule&\frac{-1}{2}\\0&-1&\frac{-1}{6}&\vrule&\frac{-1}{2}}$\\
		&\\
		& \qquad $\xleftrightarrow[]{R_2 \leftarrow \frac{-1}{2}R_2}$
		$\myvec{1&1&1&\vrule&1\\0&1&\frac{5}{3}&\vrule&1\\0&-1&\frac{-1}{6}&\vrule&\frac{-1}{2}}$\\
		&\\
		& \qquad $\xleftrightarrow[]{R_3 \leftarrow R_3 + R_2}$
		$\myvec{1&1&1&\vrule&1\\0&1&\frac{5}{3}&\vrule&1\\0&0&\frac{3}{2}&\vrule&\frac{1}{2}}$\\
		&\\
		& \qquad $\xleftrightarrow[]{R_3 \leftarrow \frac{3}{2}R_3}$
		$\myvec{1&1&1&\vrule&1\\0&1&\frac{5}{3}&\vrule&1\\0&0&1&\vrule&\frac{1}{3}}$\\
		&\\
		& \qquad $\xleftrightarrow[]{R_2 \leftarrow R_2 - \frac{5}{3}R_3}$
		$\myvec{1&1&1&\vrule&1\\0&1&0&\vrule&\frac{4}{9}\\0&0&1&\vrule&\frac{1}{3}}$\\
		&\\
		& \qquad $\xleftrightarrow[]{R_1 \leftarrow R_1 - R_3}$
		$\myvec{1&1&0&\vrule&\frac{2}{3}\\0&1&0&\vrule&\frac{4}{9}\\0&0&1&\vrule&\frac{1}{3}}$\\
		&\\
		& \qquad $\xleftrightarrow[]{R_1 \leftarrow R_1 - R_2}$
		$\myvec{1&0&0&\vrule&\frac{2}{9}\\0&1&0&\vrule&\frac{4}{9}\\0&0&1&\vrule&\frac{1}{3}}$\\
		&\\
		& $\therefore$, stationary probability distribution $\pi$ is given by \\
		& \qquad \qquad $\pi = \myvec{\frac{2}{9} & \frac{4}{9} & \frac{1}{3}}$ \\
		& \\
		\hline
		\multirow{3}{*}{Observations} & \\
		
		
		& Since the given transition probability matrix $\vec{P}$ is irreducible and aperiodic, \\
		& then $\lim_{n \to \infty} \vec{P}^{n}$ converges to a matrix with all rows identical and equal to $\vec{\pi}$. \\
		& \\
		& We were able to find $\vec{\pi}$ as $\myvec{\frac{2}{9} & \frac{4}{9} & \frac{1}{3}}$ \\
		& \\
		& $\lim_{n \to \infty} \vec{P}^{n} = \myvec{\frac{2}{9}&\frac{4}{9}&\frac{1}{3}\\\frac{2}{9}&\frac{4}{9}&\frac{1}{3}\\\frac{2}{9}&\frac{4}{9}&\frac{1}{3}}$\\
		& \\
		& From the above matrix, we get \\
		& \\
		& $\lim_{n \to \infty} \vec{P}^{n}_{11} = \frac{2}{9}$ \\
		&\\
		& $\lim_{n \to \infty} \vec{P}^{n}_{21} = \frac{2}{9}$ \\
		&\\
		& $\lim_{n \to \infty} \vec{P}^{n}_{32} = \frac{4}{9}$ \\
		&\\
		& $\lim_{n \to \infty} \vec{P}^{n}_{13} = \frac{1}{3}$ \\
		&\\
		\hline
		\multirow{3}{*}{Conclusion} & \\
		& From our observation we see that \\
		&\\
		& Options 1) and 4) are True.\\
		& \\
		\hline
\caption{}
\label{eq:solutions/2018/dec/106/table1}
	\end{longtable}
\twocolumn

\item Let
\begin{align}
\vec{A}=\myvec{1&0&2\\1&-2&0\\0&0&-3}
\end{align}
and $\vec{I}$ be the $3\times3$ identity matrix. If 
\begin{align}
6\vec{A}^{-1}=a\vec{A}^2+b\vec{A}+c\vec{I} \label{eq:solutions/2017/june/31/eq:1}
\end{align} for $a,b,c \in \mathbb{R}$ then (a,b,c) equals
\begin{enumerate}
\item (1,2,1)\\
\item (1,-1,2)\\
\item (4,1,1)\\
\item (1,4,1)
\end{enumerate}
\solution
See Tables \ref{eq:solutions/2018/dec/106/table0} and \ref{eq:solutions/2018/dec/106/table1}


\onecolumn
	\begin{longtable}{|l|l|}
		\hline
		\multirow{3}{*}{Irreducible Markov Chain} 
		& \\
		& A Markov chain is $\textbf{irreducible}$ if all the states communicate with each other,\\
		& i.e., if there is only one communication class.\\
		&\\
		\hline
		\multirow{3}{*}{Aperiodic Markov Chain} & \\
		& If there is a self-transition in the chain ($p^{ii}>0$ for some i), then the chain is\\
		& called as $\textbf{aperiodic}$\\
		& \\
		\hline
		\multirow{3}{*}{Stationary Distribution} & \\
		& A stationary distribution of a Markov chain is a probability distribution that\\
		& remains unchanged in the Markov chain as time progresses. Typically, it is\\
		& represented as a row vector $\Vec{\pi}$ whose entries are probabilities summing to 1,\\ 
		& and given transition matrix $\textbf{P}$, it satisfies\\
		& \\
		&  \qquad \qquad  \qquad$\Vec{\pi} = \Vec{\pi} \textbf{P}$\\
		& \\
		\hline
\caption{}
\label{eq:solutions/2018/dec/106/table0}
	\end{longtable}
	\begin{longtable}{|l|l|}
		\hline
		\multirow{3}{*}{Drawing Transition diagram} 
		& \\
		& 
		
		$\begin{tikzpicture}[shorten >=1pt,node distance=2cm, scale =3, auto]
			\tikzstyle{every state}=[fill={rgb:black,1;white,10}]
			
			\node[state]   (q_1)                          {$1$};
			\node[state]   (q_2)  [right of=q_1]          {$2$};
			\node[state]   (q_3)  [below right of=q_1]          {$3$};
			
			\path[->]
			(q_1) edge [loop above] node {$\frac{1}{2}$}    (   )
			edge [bend left]  node {$\frac{1}{2}$}    (q_2)
			(q_2) edge [bend left]  node {$\frac{1}{2}$}    (q_3)
			edge [loop above] node {$\frac{1}{2}$}    ()
			(q_3) edge [bend left]  node {$\frac{1}{3}$}    (q_2)
			edge [bend left]  node {$\frac{1}{3}$}    (q_1)
			edge [loop below] node {$\frac{1}{3}$}    ();
		\end{tikzpicture}$
		
		\\  
		&\\
		&\\
		\hline
		\multirow{3}{*}{Checking whether the  } & \\
		& Here,\\chain is Irreducible
		& All the states are accessible to one another. \\and Aperiodic
		& $\implies$ They are in the same communication class. So, it is Irreducible.\\
		& \\
		& There exists the non- zero self-transition, which means that the chain \\
		& is Aperiodic.\\
		&\\ 
		& We know that if the Markov Chain is irreducible and aperiodic then \\
		& \qquad \qquad \qquad $\Vec{\pi}_{j} = \lim_{n \to \infty}P\{X_{n} = j\}$, $j = 1,...,N$ \\
		& These are the stationary probabilities. \\
		&\\
		\hline
		\multirow{3}{*}{Finding the Stationary} & \\
		& Stationary Probability can be represented as\\Probability Distributions
		& \qquad \qquad \qquad $\Vec{\pi} = \Vec{\pi} \vec{P}$\\
		& \\
		& \qquad $\implies$ $\myvec{v_{1}&&v_{2}&&v_{3}} = \myvec{v_{1}&&v_{2}&&v_{3}}\vec{P}$ \\
		& \\
		& Equating the above equation we get \\
		& \\
		& \qquad \qquad \qquad $\frac{1}{2}v_{1}-\frac{1}{3}v_{3} = 0$ $\label{eq:solutions/2018/dec/106/eq}$\\
		& \\
		& \qquad \qquad \qquad $\frac{1}{2}v_{1}-\frac{1}{2}v_{2} + \frac{1}{3}v_{3} = 0$\\
		& \\
		& \qquad \qquad \qquad $\frac{1}{2}v_{2}-\frac{2}{3}v_{3} = 0$\\
		& \\\
		& We see that summation of second and the third equation gives us the \\
		& first equation only. \\
		& And we know that the probability distribution will sum up to 1. \\
		& \\
		& \qquad \qquad \qquad $v_{1}+v_{2}+v_{3} = 1$ \\
		& \\
		& Therefore, we get the equation form as \\
		& \\
		& \qquad \qquad \qquad $\myvec{1&1&1\\\frac{1}{2}&0&\frac{-1}{3}\\\frac{1}{2}&\frac{-1}{2}&\frac{1}{3}}\myvec{v_{1}\\v_{2}\\v_{3}} = \myvec{1\\0\\0}$ \\
		& \\
		\hline
		\multirow{3}{*}{Solving the linear} & \\
		& The above linear equation can be solved using Gauss-Jordan method as\\equtions
		& \\
		& \qquad \qquad \qquad $\myvec{1&1&1&\vrule&1\\\frac{1}{2}&0&\frac{-1}{3}&\vrule&0\\\frac{1}{2}&\frac{-1}{2}&\frac{1}{3}&\vrule&0}$\\
		& \\
		& \qquad $\xleftrightarrow[]{R_2 \leftarrow R_2 - \frac{1}{2}R_1}$
		$\myvec{1&1&1&\vrule&1\\0&\frac{-1}{2}&\frac{-5}{6}&\vrule&\frac{-1}{2}\\\frac{1}{2}&\frac{-1}{2}&\frac{1}{3}&\vrule&0}$\\
		&\\
		& \qquad $\xleftrightarrow[]{R_3 \leftarrow R_3 - \frac{1}{2}R_1}$
		$\myvec{1&1&1&\vrule&1\\0&\frac{-1}{2}&\frac{-5}{6}&\vrule&\frac{-1}{2}\\0&-1&\frac{-1}{6}&\vrule&\frac{-1}{2}}$\\
		&\\
		& \qquad $\xleftrightarrow[]{R_2 \leftarrow \frac{-1}{2}R_2}$
		$\myvec{1&1&1&\vrule&1\\0&1&\frac{5}{3}&\vrule&1\\0&-1&\frac{-1}{6}&\vrule&\frac{-1}{2}}$\\
		&\\
		& \qquad $\xleftrightarrow[]{R_3 \leftarrow R_3 + R_2}$
		$\myvec{1&1&1&\vrule&1\\0&1&\frac{5}{3}&\vrule&1\\0&0&\frac{3}{2}&\vrule&\frac{1}{2}}$\\
		&\\
		& \qquad $\xleftrightarrow[]{R_3 \leftarrow \frac{3}{2}R_3}$
		$\myvec{1&1&1&\vrule&1\\0&1&\frac{5}{3}&\vrule&1\\0&0&1&\vrule&\frac{1}{3}}$\\
		&\\
		& \qquad $\xleftrightarrow[]{R_2 \leftarrow R_2 - \frac{5}{3}R_3}$
		$\myvec{1&1&1&\vrule&1\\0&1&0&\vrule&\frac{4}{9}\\0&0&1&\vrule&\frac{1}{3}}$\\
		&\\
		& \qquad $\xleftrightarrow[]{R_1 \leftarrow R_1 - R_3}$
		$\myvec{1&1&0&\vrule&\frac{2}{3}\\0&1&0&\vrule&\frac{4}{9}\\0&0&1&\vrule&\frac{1}{3}}$\\
		&\\
		& \qquad $\xleftrightarrow[]{R_1 \leftarrow R_1 - R_2}$
		$\myvec{1&0&0&\vrule&\frac{2}{9}\\0&1&0&\vrule&\frac{4}{9}\\0&0&1&\vrule&\frac{1}{3}}$\\
		&\\
		& $\therefore$, stationary probability distribution $\pi$ is given by \\
		& \qquad \qquad $\pi = \myvec{\frac{2}{9} & \frac{4}{9} & \frac{1}{3}}$ \\
		& \\
		\hline
		\multirow{3}{*}{Observations} & \\
		
		
		& Since the given transition probability matrix $\vec{P}$ is irreducible and aperiodic, \\
		& then $\lim_{n \to \infty} \vec{P}^{n}$ converges to a matrix with all rows identical and equal to $\vec{\pi}$. \\
		& \\
		& We were able to find $\vec{\pi}$ as $\myvec{\frac{2}{9} & \frac{4}{9} & \frac{1}{3}}$ \\
		& \\
		& $\lim_{n \to \infty} \vec{P}^{n} = \myvec{\frac{2}{9}&\frac{4}{9}&\frac{1}{3}\\\frac{2}{9}&\frac{4}{9}&\frac{1}{3}\\\frac{2}{9}&\frac{4}{9}&\frac{1}{3}}$\\
		& \\
		& From the above matrix, we get \\
		& \\
		& $\lim_{n \to \infty} \vec{P}^{n}_{11} = \frac{2}{9}$ \\
		&\\
		& $\lim_{n \to \infty} \vec{P}^{n}_{21} = \frac{2}{9}$ \\
		&\\
		& $\lim_{n \to \infty} \vec{P}^{n}_{32} = \frac{4}{9}$ \\
		&\\
		& $\lim_{n \to \infty} \vec{P}^{n}_{13} = \frac{1}{3}$ \\
		&\\
		\hline
		\multirow{3}{*}{Conclusion} & \\
		& From our observation we see that \\
		&\\
		& Options 1) and 4) are True.\\
		& \\
		\hline
\caption{}
\label{eq:solutions/2018/dec/106/table1}
	\end{longtable}
\twocolumn


%
\item Find the Eigenvalues of the matrix,
\begin{align}
\vec{A} = \myvec{1 & 1 & 2 \\ 1 & -2 & 5 \\ 2 & 5 & -3 }\label{eq:solutions/2017/june/32/eq:1}
\end{align}
\begin{enumerate}
\item -4, 3, -3
\item 4, 3, 1
\item 4, -4$\pm\sqrt{13}$
\item 4, -2$\pm\sqrt{7}$
\end{enumerate}
%
%
\solution
See Tables \ref{eq:solutions/2018/dec/106/table0} and \ref{eq:solutions/2018/dec/106/table1}


\onecolumn
	\begin{longtable}{|l|l|}
		\hline
		\multirow{3}{*}{Irreducible Markov Chain} 
		& \\
		& A Markov chain is $\textbf{irreducible}$ if all the states communicate with each other,\\
		& i.e., if there is only one communication class.\\
		&\\
		\hline
		\multirow{3}{*}{Aperiodic Markov Chain} & \\
		& If there is a self-transition in the chain ($p^{ii}>0$ for some i), then the chain is\\
		& called as $\textbf{aperiodic}$\\
		& \\
		\hline
		\multirow{3}{*}{Stationary Distribution} & \\
		& A stationary distribution of a Markov chain is a probability distribution that\\
		& remains unchanged in the Markov chain as time progresses. Typically, it is\\
		& represented as a row vector $\Vec{\pi}$ whose entries are probabilities summing to 1,\\ 
		& and given transition matrix $\textbf{P}$, it satisfies\\
		& \\
		&  \qquad \qquad  \qquad$\Vec{\pi} = \Vec{\pi} \textbf{P}$\\
		& \\
		\hline
\caption{}
\label{eq:solutions/2018/dec/106/table0}
	\end{longtable}
	\begin{longtable}{|l|l|}
		\hline
		\multirow{3}{*}{Drawing Transition diagram} 
		& \\
		& 
		
		$\begin{tikzpicture}[shorten >=1pt,node distance=2cm, scale =3, auto]
			\tikzstyle{every state}=[fill={rgb:black,1;white,10}]
			
			\node[state]   (q_1)                          {$1$};
			\node[state]   (q_2)  [right of=q_1]          {$2$};
			\node[state]   (q_3)  [below right of=q_1]          {$3$};
			
			\path[->]
			(q_1) edge [loop above] node {$\frac{1}{2}$}    (   )
			edge [bend left]  node {$\frac{1}{2}$}    (q_2)
			(q_2) edge [bend left]  node {$\frac{1}{2}$}    (q_3)
			edge [loop above] node {$\frac{1}{2}$}    ()
			(q_3) edge [bend left]  node {$\frac{1}{3}$}    (q_2)
			edge [bend left]  node {$\frac{1}{3}$}    (q_1)
			edge [loop below] node {$\frac{1}{3}$}    ();
		\end{tikzpicture}$
		
		\\  
		&\\
		&\\
		\hline
		\multirow{3}{*}{Checking whether the  } & \\
		& Here,\\chain is Irreducible
		& All the states are accessible to one another. \\and Aperiodic
		& $\implies$ They are in the same communication class. So, it is Irreducible.\\
		& \\
		& There exists the non- zero self-transition, which means that the chain \\
		& is Aperiodic.\\
		&\\ 
		& We know that if the Markov Chain is irreducible and aperiodic then \\
		& \qquad \qquad \qquad $\Vec{\pi}_{j} = \lim_{n \to \infty}P\{X_{n} = j\}$, $j = 1,...,N$ \\
		& These are the stationary probabilities. \\
		&\\
		\hline
		\multirow{3}{*}{Finding the Stationary} & \\
		& Stationary Probability can be represented as\\Probability Distributions
		& \qquad \qquad \qquad $\Vec{\pi} = \Vec{\pi} \vec{P}$\\
		& \\
		& \qquad $\implies$ $\myvec{v_{1}&&v_{2}&&v_{3}} = \myvec{v_{1}&&v_{2}&&v_{3}}\vec{P}$ \\
		& \\
		& Equating the above equation we get \\
		& \\
		& \qquad \qquad \qquad $\frac{1}{2}v_{1}-\frac{1}{3}v_{3} = 0$ $\label{eq:solutions/2018/dec/106/eq}$\\
		& \\
		& \qquad \qquad \qquad $\frac{1}{2}v_{1}-\frac{1}{2}v_{2} + \frac{1}{3}v_{3} = 0$\\
		& \\
		& \qquad \qquad \qquad $\frac{1}{2}v_{2}-\frac{2}{3}v_{3} = 0$\\
		& \\\
		& We see that summation of second and the third equation gives us the \\
		& first equation only. \\
		& And we know that the probability distribution will sum up to 1. \\
		& \\
		& \qquad \qquad \qquad $v_{1}+v_{2}+v_{3} = 1$ \\
		& \\
		& Therefore, we get the equation form as \\
		& \\
		& \qquad \qquad \qquad $\myvec{1&1&1\\\frac{1}{2}&0&\frac{-1}{3}\\\frac{1}{2}&\frac{-1}{2}&\frac{1}{3}}\myvec{v_{1}\\v_{2}\\v_{3}} = \myvec{1\\0\\0}$ \\
		& \\
		\hline
		\multirow{3}{*}{Solving the linear} & \\
		& The above linear equation can be solved using Gauss-Jordan method as\\equtions
		& \\
		& \qquad \qquad \qquad $\myvec{1&1&1&\vrule&1\\\frac{1}{2}&0&\frac{-1}{3}&\vrule&0\\\frac{1}{2}&\frac{-1}{2}&\frac{1}{3}&\vrule&0}$\\
		& \\
		& \qquad $\xleftrightarrow[]{R_2 \leftarrow R_2 - \frac{1}{2}R_1}$
		$\myvec{1&1&1&\vrule&1\\0&\frac{-1}{2}&\frac{-5}{6}&\vrule&\frac{-1}{2}\\\frac{1}{2}&\frac{-1}{2}&\frac{1}{3}&\vrule&0}$\\
		&\\
		& \qquad $\xleftrightarrow[]{R_3 \leftarrow R_3 - \frac{1}{2}R_1}$
		$\myvec{1&1&1&\vrule&1\\0&\frac{-1}{2}&\frac{-5}{6}&\vrule&\frac{-1}{2}\\0&-1&\frac{-1}{6}&\vrule&\frac{-1}{2}}$\\
		&\\
		& \qquad $\xleftrightarrow[]{R_2 \leftarrow \frac{-1}{2}R_2}$
		$\myvec{1&1&1&\vrule&1\\0&1&\frac{5}{3}&\vrule&1\\0&-1&\frac{-1}{6}&\vrule&\frac{-1}{2}}$\\
		&\\
		& \qquad $\xleftrightarrow[]{R_3 \leftarrow R_3 + R_2}$
		$\myvec{1&1&1&\vrule&1\\0&1&\frac{5}{3}&\vrule&1\\0&0&\frac{3}{2}&\vrule&\frac{1}{2}}$\\
		&\\
		& \qquad $\xleftrightarrow[]{R_3 \leftarrow \frac{3}{2}R_3}$
		$\myvec{1&1&1&\vrule&1\\0&1&\frac{5}{3}&\vrule&1\\0&0&1&\vrule&\frac{1}{3}}$\\
		&\\
		& \qquad $\xleftrightarrow[]{R_2 \leftarrow R_2 - \frac{5}{3}R_3}$
		$\myvec{1&1&1&\vrule&1\\0&1&0&\vrule&\frac{4}{9}\\0&0&1&\vrule&\frac{1}{3}}$\\
		&\\
		& \qquad $\xleftrightarrow[]{R_1 \leftarrow R_1 - R_3}$
		$\myvec{1&1&0&\vrule&\frac{2}{3}\\0&1&0&\vrule&\frac{4}{9}\\0&0&1&\vrule&\frac{1}{3}}$\\
		&\\
		& \qquad $\xleftrightarrow[]{R_1 \leftarrow R_1 - R_2}$
		$\myvec{1&0&0&\vrule&\frac{2}{9}\\0&1&0&\vrule&\frac{4}{9}\\0&0&1&\vrule&\frac{1}{3}}$\\
		&\\
		& $\therefore$, stationary probability distribution $\pi$ is given by \\
		& \qquad \qquad $\pi = \myvec{\frac{2}{9} & \frac{4}{9} & \frac{1}{3}}$ \\
		& \\
		\hline
		\multirow{3}{*}{Observations} & \\
		
		
		& Since the given transition probability matrix $\vec{P}$ is irreducible and aperiodic, \\
		& then $\lim_{n \to \infty} \vec{P}^{n}$ converges to a matrix with all rows identical and equal to $\vec{\pi}$. \\
		& \\
		& We were able to find $\vec{\pi}$ as $\myvec{\frac{2}{9} & \frac{4}{9} & \frac{1}{3}}$ \\
		& \\
		& $\lim_{n \to \infty} \vec{P}^{n} = \myvec{\frac{2}{9}&\frac{4}{9}&\frac{1}{3}\\\frac{2}{9}&\frac{4}{9}&\frac{1}{3}\\\frac{2}{9}&\frac{4}{9}&\frac{1}{3}}$\\
		& \\
		& From the above matrix, we get \\
		& \\
		& $\lim_{n \to \infty} \vec{P}^{n}_{11} = \frac{2}{9}$ \\
		&\\
		& $\lim_{n \to \infty} \vec{P}^{n}_{21} = \frac{2}{9}$ \\
		&\\
		& $\lim_{n \to \infty} \vec{P}^{n}_{32} = \frac{4}{9}$ \\
		&\\
		& $\lim_{n \to \infty} \vec{P}^{n}_{13} = \frac{1}{3}$ \\
		&\\
		\hline
		\multirow{3}{*}{Conclusion} & \\
		& From our observation we see that \\
		&\\
		& Options 1) and 4) are True.\\
		& \\
		\hline
\caption{}
\label{eq:solutions/2018/dec/106/table1}
	\end{longtable}
\twocolumn

\item Consider the vector space V of real polynomials of degree less than or equal to n. Fix distinct real numbers $a_0, a_1, \cdots, a_k$. For $p \in V$
\begin{align}
    max\cbrak{\abs{p(a_j)}: 0\leq j \leq k}
\end{align}
defines a norm on V
\begin{enumerate}
    \item only if $k<n$
    \item only if $k\ge n$
    \item if $ k+1\leq n$ 
    \item if $k \ge n+1$
\end{enumerate}
%
\solution
See Tables \ref{eq:solutions/2018/dec/106/table0} and \ref{eq:solutions/2018/dec/106/table1}


\onecolumn
	\begin{longtable}{|l|l|}
		\hline
		\multirow{3}{*}{Irreducible Markov Chain} 
		& \\
		& A Markov chain is $\textbf{irreducible}$ if all the states communicate with each other,\\
		& i.e., if there is only one communication class.\\
		&\\
		\hline
		\multirow{3}{*}{Aperiodic Markov Chain} & \\
		& If there is a self-transition in the chain ($p^{ii}>0$ for some i), then the chain is\\
		& called as $\textbf{aperiodic}$\\
		& \\
		\hline
		\multirow{3}{*}{Stationary Distribution} & \\
		& A stationary distribution of a Markov chain is a probability distribution that\\
		& remains unchanged in the Markov chain as time progresses. Typically, it is\\
		& represented as a row vector $\Vec{\pi}$ whose entries are probabilities summing to 1,\\ 
		& and given transition matrix $\textbf{P}$, it satisfies\\
		& \\
		&  \qquad \qquad  \qquad$\Vec{\pi} = \Vec{\pi} \textbf{P}$\\
		& \\
		\hline
\caption{}
\label{eq:solutions/2018/dec/106/table0}
	\end{longtable}
	\begin{longtable}{|l|l|}
		\hline
		\multirow{3}{*}{Drawing Transition diagram} 
		& \\
		& 
		
		$\begin{tikzpicture}[shorten >=1pt,node distance=2cm, scale =3, auto]
			\tikzstyle{every state}=[fill={rgb:black,1;white,10}]
			
			\node[state]   (q_1)                          {$1$};
			\node[state]   (q_2)  [right of=q_1]          {$2$};
			\node[state]   (q_3)  [below right of=q_1]          {$3$};
			
			\path[->]
			(q_1) edge [loop above] node {$\frac{1}{2}$}    (   )
			edge [bend left]  node {$\frac{1}{2}$}    (q_2)
			(q_2) edge [bend left]  node {$\frac{1}{2}$}    (q_3)
			edge [loop above] node {$\frac{1}{2}$}    ()
			(q_3) edge [bend left]  node {$\frac{1}{3}$}    (q_2)
			edge [bend left]  node {$\frac{1}{3}$}    (q_1)
			edge [loop below] node {$\frac{1}{3}$}    ();
		\end{tikzpicture}$
		
		\\  
		&\\
		&\\
		\hline
		\multirow{3}{*}{Checking whether the  } & \\
		& Here,\\chain is Irreducible
		& All the states are accessible to one another. \\and Aperiodic
		& $\implies$ They are in the same communication class. So, it is Irreducible.\\
		& \\
		& There exists the non- zero self-transition, which means that the chain \\
		& is Aperiodic.\\
		&\\ 
		& We know that if the Markov Chain is irreducible and aperiodic then \\
		& \qquad \qquad \qquad $\Vec{\pi}_{j} = \lim_{n \to \infty}P\{X_{n} = j\}$, $j = 1,...,N$ \\
		& These are the stationary probabilities. \\
		&\\
		\hline
		\multirow{3}{*}{Finding the Stationary} & \\
		& Stationary Probability can be represented as\\Probability Distributions
		& \qquad \qquad \qquad $\Vec{\pi} = \Vec{\pi} \vec{P}$\\
		& \\
		& \qquad $\implies$ $\myvec{v_{1}&&v_{2}&&v_{3}} = \myvec{v_{1}&&v_{2}&&v_{3}}\vec{P}$ \\
		& \\
		& Equating the above equation we get \\
		& \\
		& \qquad \qquad \qquad $\frac{1}{2}v_{1}-\frac{1}{3}v_{3} = 0$ $\label{eq:solutions/2018/dec/106/eq}$\\
		& \\
		& \qquad \qquad \qquad $\frac{1}{2}v_{1}-\frac{1}{2}v_{2} + \frac{1}{3}v_{3} = 0$\\
		& \\
		& \qquad \qquad \qquad $\frac{1}{2}v_{2}-\frac{2}{3}v_{3} = 0$\\
		& \\\
		& We see that summation of second and the third equation gives us the \\
		& first equation only. \\
		& And we know that the probability distribution will sum up to 1. \\
		& \\
		& \qquad \qquad \qquad $v_{1}+v_{2}+v_{3} = 1$ \\
		& \\
		& Therefore, we get the equation form as \\
		& \\
		& \qquad \qquad \qquad $\myvec{1&1&1\\\frac{1}{2}&0&\frac{-1}{3}\\\frac{1}{2}&\frac{-1}{2}&\frac{1}{3}}\myvec{v_{1}\\v_{2}\\v_{3}} = \myvec{1\\0\\0}$ \\
		& \\
		\hline
		\multirow{3}{*}{Solving the linear} & \\
		& The above linear equation can be solved using Gauss-Jordan method as\\equtions
		& \\
		& \qquad \qquad \qquad $\myvec{1&1&1&\vrule&1\\\frac{1}{2}&0&\frac{-1}{3}&\vrule&0\\\frac{1}{2}&\frac{-1}{2}&\frac{1}{3}&\vrule&0}$\\
		& \\
		& \qquad $\xleftrightarrow[]{R_2 \leftarrow R_2 - \frac{1}{2}R_1}$
		$\myvec{1&1&1&\vrule&1\\0&\frac{-1}{2}&\frac{-5}{6}&\vrule&\frac{-1}{2}\\\frac{1}{2}&\frac{-1}{2}&\frac{1}{3}&\vrule&0}$\\
		&\\
		& \qquad $\xleftrightarrow[]{R_3 \leftarrow R_3 - \frac{1}{2}R_1}$
		$\myvec{1&1&1&\vrule&1\\0&\frac{-1}{2}&\frac{-5}{6}&\vrule&\frac{-1}{2}\\0&-1&\frac{-1}{6}&\vrule&\frac{-1}{2}}$\\
		&\\
		& \qquad $\xleftrightarrow[]{R_2 \leftarrow \frac{-1}{2}R_2}$
		$\myvec{1&1&1&\vrule&1\\0&1&\frac{5}{3}&\vrule&1\\0&-1&\frac{-1}{6}&\vrule&\frac{-1}{2}}$\\
		&\\
		& \qquad $\xleftrightarrow[]{R_3 \leftarrow R_3 + R_2}$
		$\myvec{1&1&1&\vrule&1\\0&1&\frac{5}{3}&\vrule&1\\0&0&\frac{3}{2}&\vrule&\frac{1}{2}}$\\
		&\\
		& \qquad $\xleftrightarrow[]{R_3 \leftarrow \frac{3}{2}R_3}$
		$\myvec{1&1&1&\vrule&1\\0&1&\frac{5}{3}&\vrule&1\\0&0&1&\vrule&\frac{1}{3}}$\\
		&\\
		& \qquad $\xleftrightarrow[]{R_2 \leftarrow R_2 - \frac{5}{3}R_3}$
		$\myvec{1&1&1&\vrule&1\\0&1&0&\vrule&\frac{4}{9}\\0&0&1&\vrule&\frac{1}{3}}$\\
		&\\
		& \qquad $\xleftrightarrow[]{R_1 \leftarrow R_1 - R_3}$
		$\myvec{1&1&0&\vrule&\frac{2}{3}\\0&1&0&\vrule&\frac{4}{9}\\0&0&1&\vrule&\frac{1}{3}}$\\
		&\\
		& \qquad $\xleftrightarrow[]{R_1 \leftarrow R_1 - R_2}$
		$\myvec{1&0&0&\vrule&\frac{2}{9}\\0&1&0&\vrule&\frac{4}{9}\\0&0&1&\vrule&\frac{1}{3}}$\\
		&\\
		& $\therefore$, stationary probability distribution $\pi$ is given by \\
		& \qquad \qquad $\pi = \myvec{\frac{2}{9} & \frac{4}{9} & \frac{1}{3}}$ \\
		& \\
		\hline
		\multirow{3}{*}{Observations} & \\
		
		
		& Since the given transition probability matrix $\vec{P}$ is irreducible and aperiodic, \\
		& then $\lim_{n \to \infty} \vec{P}^{n}$ converges to a matrix with all rows identical and equal to $\vec{\pi}$. \\
		& \\
		& We were able to find $\vec{\pi}$ as $\myvec{\frac{2}{9} & \frac{4}{9} & \frac{1}{3}}$ \\
		& \\
		& $\lim_{n \to \infty} \vec{P}^{n} = \myvec{\frac{2}{9}&\frac{4}{9}&\frac{1}{3}\\\frac{2}{9}&\frac{4}{9}&\frac{1}{3}\\\frac{2}{9}&\frac{4}{9}&\frac{1}{3}}$\\
		& \\
		& From the above matrix, we get \\
		& \\
		& $\lim_{n \to \infty} \vec{P}^{n}_{11} = \frac{2}{9}$ \\
		&\\
		& $\lim_{n \to \infty} \vec{P}^{n}_{21} = \frac{2}{9}$ \\
		&\\
		& $\lim_{n \to \infty} \vec{P}^{n}_{32} = \frac{4}{9}$ \\
		&\\
		& $\lim_{n \to \infty} \vec{P}^{n}_{13} = \frac{1}{3}$ \\
		&\\
		\hline
		\multirow{3}{*}{Conclusion} & \\
		& From our observation we see that \\
		&\\
		& Options 1) and 4) are True.\\
		& \\
		\hline
\caption{}
\label{eq:solutions/2018/dec/106/table1}
	\end{longtable}
\twocolumn

\item Let \textbf{V} be the vector space of polynomials of degree at most 3 in a variable x with coefficients in $\mathbb{R}$. Let \textbf{T}=d/dx be the linear transformation of \textbf{V} to itself given by differentiation.\\

Which of the following are correct?\\
\begin{enumerate}
\item $\vec{T}$ is invertible
\item 0 is an eigenvalue of $\vec{T}$
\item There is a basis with respect to which the matrix of \textbf{T} is nilpotent.
\item The matrix of \textbf{T} with respect to the basis \myvec{1,1+x,1+x+x^2,1+x+x^2+x^3} is diagonal.
\end{enumerate}
\solution
See Tables \ref{eq:solutions/2018/dec/106/table0} and \ref{eq:solutions/2018/dec/106/table1}


\onecolumn
	\begin{longtable}{|l|l|}
		\hline
		\multirow{3}{*}{Irreducible Markov Chain} 
		& \\
		& A Markov chain is $\textbf{irreducible}$ if all the states communicate with each other,\\
		& i.e., if there is only one communication class.\\
		&\\
		\hline
		\multirow{3}{*}{Aperiodic Markov Chain} & \\
		& If there is a self-transition in the chain ($p^{ii}>0$ for some i), then the chain is\\
		& called as $\textbf{aperiodic}$\\
		& \\
		\hline
		\multirow{3}{*}{Stationary Distribution} & \\
		& A stationary distribution of a Markov chain is a probability distribution that\\
		& remains unchanged in the Markov chain as time progresses. Typically, it is\\
		& represented as a row vector $\Vec{\pi}$ whose entries are probabilities summing to 1,\\ 
		& and given transition matrix $\textbf{P}$, it satisfies\\
		& \\
		&  \qquad \qquad  \qquad$\Vec{\pi} = \Vec{\pi} \textbf{P}$\\
		& \\
		\hline
\caption{}
\label{eq:solutions/2018/dec/106/table0}
	\end{longtable}
	\begin{longtable}{|l|l|}
		\hline
		\multirow{3}{*}{Drawing Transition diagram} 
		& \\
		& 
		
		$\begin{tikzpicture}[shorten >=1pt,node distance=2cm, scale =3, auto]
			\tikzstyle{every state}=[fill={rgb:black,1;white,10}]
			
			\node[state]   (q_1)                          {$1$};
			\node[state]   (q_2)  [right of=q_1]          {$2$};
			\node[state]   (q_3)  [below right of=q_1]          {$3$};
			
			\path[->]
			(q_1) edge [loop above] node {$\frac{1}{2}$}    (   )
			edge [bend left]  node {$\frac{1}{2}$}    (q_2)
			(q_2) edge [bend left]  node {$\frac{1}{2}$}    (q_3)
			edge [loop above] node {$\frac{1}{2}$}    ()
			(q_3) edge [bend left]  node {$\frac{1}{3}$}    (q_2)
			edge [bend left]  node {$\frac{1}{3}$}    (q_1)
			edge [loop below] node {$\frac{1}{3}$}    ();
		\end{tikzpicture}$
		
		\\  
		&\\
		&\\
		\hline
		\multirow{3}{*}{Checking whether the  } & \\
		& Here,\\chain is Irreducible
		& All the states are accessible to one another. \\and Aperiodic
		& $\implies$ They are in the same communication class. So, it is Irreducible.\\
		& \\
		& There exists the non- zero self-transition, which means that the chain \\
		& is Aperiodic.\\
		&\\ 
		& We know that if the Markov Chain is irreducible and aperiodic then \\
		& \qquad \qquad \qquad $\Vec{\pi}_{j} = \lim_{n \to \infty}P\{X_{n} = j\}$, $j = 1,...,N$ \\
		& These are the stationary probabilities. \\
		&\\
		\hline
		\multirow{3}{*}{Finding the Stationary} & \\
		& Stationary Probability can be represented as\\Probability Distributions
		& \qquad \qquad \qquad $\Vec{\pi} = \Vec{\pi} \vec{P}$\\
		& \\
		& \qquad $\implies$ $\myvec{v_{1}&&v_{2}&&v_{3}} = \myvec{v_{1}&&v_{2}&&v_{3}}\vec{P}$ \\
		& \\
		& Equating the above equation we get \\
		& \\
		& \qquad \qquad \qquad $\frac{1}{2}v_{1}-\frac{1}{3}v_{3} = 0$ $\label{eq:solutions/2018/dec/106/eq}$\\
		& \\
		& \qquad \qquad \qquad $\frac{1}{2}v_{1}-\frac{1}{2}v_{2} + \frac{1}{3}v_{3} = 0$\\
		& \\
		& \qquad \qquad \qquad $\frac{1}{2}v_{2}-\frac{2}{3}v_{3} = 0$\\
		& \\\
		& We see that summation of second and the third equation gives us the \\
		& first equation only. \\
		& And we know that the probability distribution will sum up to 1. \\
		& \\
		& \qquad \qquad \qquad $v_{1}+v_{2}+v_{3} = 1$ \\
		& \\
		& Therefore, we get the equation form as \\
		& \\
		& \qquad \qquad \qquad $\myvec{1&1&1\\\frac{1}{2}&0&\frac{-1}{3}\\\frac{1}{2}&\frac{-1}{2}&\frac{1}{3}}\myvec{v_{1}\\v_{2}\\v_{3}} = \myvec{1\\0\\0}$ \\
		& \\
		\hline
		\multirow{3}{*}{Solving the linear} & \\
		& The above linear equation can be solved using Gauss-Jordan method as\\equtions
		& \\
		& \qquad \qquad \qquad $\myvec{1&1&1&\vrule&1\\\frac{1}{2}&0&\frac{-1}{3}&\vrule&0\\\frac{1}{2}&\frac{-1}{2}&\frac{1}{3}&\vrule&0}$\\
		& \\
		& \qquad $\xleftrightarrow[]{R_2 \leftarrow R_2 - \frac{1}{2}R_1}$
		$\myvec{1&1&1&\vrule&1\\0&\frac{-1}{2}&\frac{-5}{6}&\vrule&\frac{-1}{2}\\\frac{1}{2}&\frac{-1}{2}&\frac{1}{3}&\vrule&0}$\\
		&\\
		& \qquad $\xleftrightarrow[]{R_3 \leftarrow R_3 - \frac{1}{2}R_1}$
		$\myvec{1&1&1&\vrule&1\\0&\frac{-1}{2}&\frac{-5}{6}&\vrule&\frac{-1}{2}\\0&-1&\frac{-1}{6}&\vrule&\frac{-1}{2}}$\\
		&\\
		& \qquad $\xleftrightarrow[]{R_2 \leftarrow \frac{-1}{2}R_2}$
		$\myvec{1&1&1&\vrule&1\\0&1&\frac{5}{3}&\vrule&1\\0&-1&\frac{-1}{6}&\vrule&\frac{-1}{2}}$\\
		&\\
		& \qquad $\xleftrightarrow[]{R_3 \leftarrow R_3 + R_2}$
		$\myvec{1&1&1&\vrule&1\\0&1&\frac{5}{3}&\vrule&1\\0&0&\frac{3}{2}&\vrule&\frac{1}{2}}$\\
		&\\
		& \qquad $\xleftrightarrow[]{R_3 \leftarrow \frac{3}{2}R_3}$
		$\myvec{1&1&1&\vrule&1\\0&1&\frac{5}{3}&\vrule&1\\0&0&1&\vrule&\frac{1}{3}}$\\
		&\\
		& \qquad $\xleftrightarrow[]{R_2 \leftarrow R_2 - \frac{5}{3}R_3}$
		$\myvec{1&1&1&\vrule&1\\0&1&0&\vrule&\frac{4}{9}\\0&0&1&\vrule&\frac{1}{3}}$\\
		&\\
		& \qquad $\xleftrightarrow[]{R_1 \leftarrow R_1 - R_3}$
		$\myvec{1&1&0&\vrule&\frac{2}{3}\\0&1&0&\vrule&\frac{4}{9}\\0&0&1&\vrule&\frac{1}{3}}$\\
		&\\
		& \qquad $\xleftrightarrow[]{R_1 \leftarrow R_1 - R_2}$
		$\myvec{1&0&0&\vrule&\frac{2}{9}\\0&1&0&\vrule&\frac{4}{9}\\0&0&1&\vrule&\frac{1}{3}}$\\
		&\\
		& $\therefore$, stationary probability distribution $\pi$ is given by \\
		& \qquad \qquad $\pi = \myvec{\frac{2}{9} & \frac{4}{9} & \frac{1}{3}}$ \\
		& \\
		\hline
		\multirow{3}{*}{Observations} & \\
		
		
		& Since the given transition probability matrix $\vec{P}$ is irreducible and aperiodic, \\
		& then $\lim_{n \to \infty} \vec{P}^{n}$ converges to a matrix with all rows identical and equal to $\vec{\pi}$. \\
		& \\
		& We were able to find $\vec{\pi}$ as $\myvec{\frac{2}{9} & \frac{4}{9} & \frac{1}{3}}$ \\
		& \\
		& $\lim_{n \to \infty} \vec{P}^{n} = \myvec{\frac{2}{9}&\frac{4}{9}&\frac{1}{3}\\\frac{2}{9}&\frac{4}{9}&\frac{1}{3}\\\frac{2}{9}&\frac{4}{9}&\frac{1}{3}}$\\
		& \\
		& From the above matrix, we get \\
		& \\
		& $\lim_{n \to \infty} \vec{P}^{n}_{11} = \frac{2}{9}$ \\
		&\\
		& $\lim_{n \to \infty} \vec{P}^{n}_{21} = \frac{2}{9}$ \\
		&\\
		& $\lim_{n \to \infty} \vec{P}^{n}_{32} = \frac{4}{9}$ \\
		&\\
		& $\lim_{n \to \infty} \vec{P}^{n}_{13} = \frac{1}{3}$ \\
		&\\
		\hline
		\multirow{3}{*}{Conclusion} & \\
		& From our observation we see that \\
		&\\
		& Options 1) and 4) are True.\\
		& \\
		\hline
\caption{}
\label{eq:solutions/2018/dec/106/table1}
	\end{longtable}
\twocolumn

\item Let $m,n,r$ be natural numbers. Let $A$ be an $m\times n$ matrix with real entries such that $(AA^t)^r = I$, where $I$ is the $m \times m$ is identity matrix and $A^t$ is the transpose of the matrix $A$. We can conclude that\\
\begin{enumerate}
\item
$m = n$\\
\item
$AA^t$ is invertible\\
\item
$A^tA$ is invertible\\
\item
if $m=n$, then $A$ is invertible
\end{enumerate}
%
\solution
See Tables \ref{eq:solutions/2018/dec/106/table0} and \ref{eq:solutions/2018/dec/106/table1}


\onecolumn
	\begin{longtable}{|l|l|}
		\hline
		\multirow{3}{*}{Irreducible Markov Chain} 
		& \\
		& A Markov chain is $\textbf{irreducible}$ if all the states communicate with each other,\\
		& i.e., if there is only one communication class.\\
		&\\
		\hline
		\multirow{3}{*}{Aperiodic Markov Chain} & \\
		& If there is a self-transition in the chain ($p^{ii}>0$ for some i), then the chain is\\
		& called as $\textbf{aperiodic}$\\
		& \\
		\hline
		\multirow{3}{*}{Stationary Distribution} & \\
		& A stationary distribution of a Markov chain is a probability distribution that\\
		& remains unchanged in the Markov chain as time progresses. Typically, it is\\
		& represented as a row vector $\Vec{\pi}$ whose entries are probabilities summing to 1,\\ 
		& and given transition matrix $\textbf{P}$, it satisfies\\
		& \\
		&  \qquad \qquad  \qquad$\Vec{\pi} = \Vec{\pi} \textbf{P}$\\
		& \\
		\hline
\caption{}
\label{eq:solutions/2018/dec/106/table0}
	\end{longtable}
	\begin{longtable}{|l|l|}
		\hline
		\multirow{3}{*}{Drawing Transition diagram} 
		& \\
		& 
		
		$\begin{tikzpicture}[shorten >=1pt,node distance=2cm, scale =3, auto]
			\tikzstyle{every state}=[fill={rgb:black,1;white,10}]
			
			\node[state]   (q_1)                          {$1$};
			\node[state]   (q_2)  [right of=q_1]          {$2$};
			\node[state]   (q_3)  [below right of=q_1]          {$3$};
			
			\path[->]
			(q_1) edge [loop above] node {$\frac{1}{2}$}    (   )
			edge [bend left]  node {$\frac{1}{2}$}    (q_2)
			(q_2) edge [bend left]  node {$\frac{1}{2}$}    (q_3)
			edge [loop above] node {$\frac{1}{2}$}    ()
			(q_3) edge [bend left]  node {$\frac{1}{3}$}    (q_2)
			edge [bend left]  node {$\frac{1}{3}$}    (q_1)
			edge [loop below] node {$\frac{1}{3}$}    ();
		\end{tikzpicture}$
		
		\\  
		&\\
		&\\
		\hline
		\multirow{3}{*}{Checking whether the  } & \\
		& Here,\\chain is Irreducible
		& All the states are accessible to one another. \\and Aperiodic
		& $\implies$ They are in the same communication class. So, it is Irreducible.\\
		& \\
		& There exists the non- zero self-transition, which means that the chain \\
		& is Aperiodic.\\
		&\\ 
		& We know that if the Markov Chain is irreducible and aperiodic then \\
		& \qquad \qquad \qquad $\Vec{\pi}_{j} = \lim_{n \to \infty}P\{X_{n} = j\}$, $j = 1,...,N$ \\
		& These are the stationary probabilities. \\
		&\\
		\hline
		\multirow{3}{*}{Finding the Stationary} & \\
		& Stationary Probability can be represented as\\Probability Distributions
		& \qquad \qquad \qquad $\Vec{\pi} = \Vec{\pi} \vec{P}$\\
		& \\
		& \qquad $\implies$ $\myvec{v_{1}&&v_{2}&&v_{3}} = \myvec{v_{1}&&v_{2}&&v_{3}}\vec{P}$ \\
		& \\
		& Equating the above equation we get \\
		& \\
		& \qquad \qquad \qquad $\frac{1}{2}v_{1}-\frac{1}{3}v_{3} = 0$ $\label{eq:solutions/2018/dec/106/eq}$\\
		& \\
		& \qquad \qquad \qquad $\frac{1}{2}v_{1}-\frac{1}{2}v_{2} + \frac{1}{3}v_{3} = 0$\\
		& \\
		& \qquad \qquad \qquad $\frac{1}{2}v_{2}-\frac{2}{3}v_{3} = 0$\\
		& \\\
		& We see that summation of second and the third equation gives us the \\
		& first equation only. \\
		& And we know that the probability distribution will sum up to 1. \\
		& \\
		& \qquad \qquad \qquad $v_{1}+v_{2}+v_{3} = 1$ \\
		& \\
		& Therefore, we get the equation form as \\
		& \\
		& \qquad \qquad \qquad $\myvec{1&1&1\\\frac{1}{2}&0&\frac{-1}{3}\\\frac{1}{2}&\frac{-1}{2}&\frac{1}{3}}\myvec{v_{1}\\v_{2}\\v_{3}} = \myvec{1\\0\\0}$ \\
		& \\
		\hline
		\multirow{3}{*}{Solving the linear} & \\
		& The above linear equation can be solved using Gauss-Jordan method as\\equtions
		& \\
		& \qquad \qquad \qquad $\myvec{1&1&1&\vrule&1\\\frac{1}{2}&0&\frac{-1}{3}&\vrule&0\\\frac{1}{2}&\frac{-1}{2}&\frac{1}{3}&\vrule&0}$\\
		& \\
		& \qquad $\xleftrightarrow[]{R_2 \leftarrow R_2 - \frac{1}{2}R_1}$
		$\myvec{1&1&1&\vrule&1\\0&\frac{-1}{2}&\frac{-5}{6}&\vrule&\frac{-1}{2}\\\frac{1}{2}&\frac{-1}{2}&\frac{1}{3}&\vrule&0}$\\
		&\\
		& \qquad $\xleftrightarrow[]{R_3 \leftarrow R_3 - \frac{1}{2}R_1}$
		$\myvec{1&1&1&\vrule&1\\0&\frac{-1}{2}&\frac{-5}{6}&\vrule&\frac{-1}{2}\\0&-1&\frac{-1}{6}&\vrule&\frac{-1}{2}}$\\
		&\\
		& \qquad $\xleftrightarrow[]{R_2 \leftarrow \frac{-1}{2}R_2}$
		$\myvec{1&1&1&\vrule&1\\0&1&\frac{5}{3}&\vrule&1\\0&-1&\frac{-1}{6}&\vrule&\frac{-1}{2}}$\\
		&\\
		& \qquad $\xleftrightarrow[]{R_3 \leftarrow R_3 + R_2}$
		$\myvec{1&1&1&\vrule&1\\0&1&\frac{5}{3}&\vrule&1\\0&0&\frac{3}{2}&\vrule&\frac{1}{2}}$\\
		&\\
		& \qquad $\xleftrightarrow[]{R_3 \leftarrow \frac{3}{2}R_3}$
		$\myvec{1&1&1&\vrule&1\\0&1&\frac{5}{3}&\vrule&1\\0&0&1&\vrule&\frac{1}{3}}$\\
		&\\
		& \qquad $\xleftrightarrow[]{R_2 \leftarrow R_2 - \frac{5}{3}R_3}$
		$\myvec{1&1&1&\vrule&1\\0&1&0&\vrule&\frac{4}{9}\\0&0&1&\vrule&\frac{1}{3}}$\\
		&\\
		& \qquad $\xleftrightarrow[]{R_1 \leftarrow R_1 - R_3}$
		$\myvec{1&1&0&\vrule&\frac{2}{3}\\0&1&0&\vrule&\frac{4}{9}\\0&0&1&\vrule&\frac{1}{3}}$\\
		&\\
		& \qquad $\xleftrightarrow[]{R_1 \leftarrow R_1 - R_2}$
		$\myvec{1&0&0&\vrule&\frac{2}{9}\\0&1&0&\vrule&\frac{4}{9}\\0&0&1&\vrule&\frac{1}{3}}$\\
		&\\
		& $\therefore$, stationary probability distribution $\pi$ is given by \\
		& \qquad \qquad $\pi = \myvec{\frac{2}{9} & \frac{4}{9} & \frac{1}{3}}$ \\
		& \\
		\hline
		\multirow{3}{*}{Observations} & \\
		
		
		& Since the given transition probability matrix $\vec{P}$ is irreducible and aperiodic, \\
		& then $\lim_{n \to \infty} \vec{P}^{n}$ converges to a matrix with all rows identical and equal to $\vec{\pi}$. \\
		& \\
		& We were able to find $\vec{\pi}$ as $\myvec{\frac{2}{9} & \frac{4}{9} & \frac{1}{3}}$ \\
		& \\
		& $\lim_{n \to \infty} \vec{P}^{n} = \myvec{\frac{2}{9}&\frac{4}{9}&\frac{1}{3}\\\frac{2}{9}&\frac{4}{9}&\frac{1}{3}\\\frac{2}{9}&\frac{4}{9}&\frac{1}{3}}$\\
		& \\
		& From the above matrix, we get \\
		& \\
		& $\lim_{n \to \infty} \vec{P}^{n}_{11} = \frac{2}{9}$ \\
		&\\
		& $\lim_{n \to \infty} \vec{P}^{n}_{21} = \frac{2}{9}$ \\
		&\\
		& $\lim_{n \to \infty} \vec{P}^{n}_{32} = \frac{4}{9}$ \\
		&\\
		& $\lim_{n \to \infty} \vec{P}^{n}_{13} = \frac{1}{3}$ \\
		&\\
		\hline
		\multirow{3}{*}{Conclusion} & \\
		& From our observation we see that \\
		&\\
		& Options 1) and 4) are True.\\
		& \\
		\hline
\caption{}
\label{eq:solutions/2018/dec/106/table1}
	\end{longtable}
\twocolumn

\item Let $\vec{A}$ be a $n\times n$ real matrix with $\vec{A}^2=\vec{A}$. Then
\begin{enumerate}
	\item the eigenvalues of $\vec{A}$ are either 0 or 1
	\item $\vec{A}$ is a diagonal matrix with diagonal entries 0 or 1
	\item $rank(\vec{A})=trace(\vec{A})$
	\item if $rank(\vec{I-A})=trace(\vec{I-A})$
\end{enumerate}
%
%
\solution
See Tables \ref{eq:solutions/2018/dec/106/table0} and \ref{eq:solutions/2018/dec/106/table1}


\onecolumn
	\begin{longtable}{|l|l|}
		\hline
		\multirow{3}{*}{Irreducible Markov Chain} 
		& \\
		& A Markov chain is $\textbf{irreducible}$ if all the states communicate with each other,\\
		& i.e., if there is only one communication class.\\
		&\\
		\hline
		\multirow{3}{*}{Aperiodic Markov Chain} & \\
		& If there is a self-transition in the chain ($p^{ii}>0$ for some i), then the chain is\\
		& called as $\textbf{aperiodic}$\\
		& \\
		\hline
		\multirow{3}{*}{Stationary Distribution} & \\
		& A stationary distribution of a Markov chain is a probability distribution that\\
		& remains unchanged in the Markov chain as time progresses. Typically, it is\\
		& represented as a row vector $\Vec{\pi}$ whose entries are probabilities summing to 1,\\ 
		& and given transition matrix $\textbf{P}$, it satisfies\\
		& \\
		&  \qquad \qquad  \qquad$\Vec{\pi} = \Vec{\pi} \textbf{P}$\\
		& \\
		\hline
\caption{}
\label{eq:solutions/2018/dec/106/table0}
	\end{longtable}
	\begin{longtable}{|l|l|}
		\hline
		\multirow{3}{*}{Drawing Transition diagram} 
		& \\
		& 
		
		$\begin{tikzpicture}[shorten >=1pt,node distance=2cm, scale =3, auto]
			\tikzstyle{every state}=[fill={rgb:black,1;white,10}]
			
			\node[state]   (q_1)                          {$1$};
			\node[state]   (q_2)  [right of=q_1]          {$2$};
			\node[state]   (q_3)  [below right of=q_1]          {$3$};
			
			\path[->]
			(q_1) edge [loop above] node {$\frac{1}{2}$}    (   )
			edge [bend left]  node {$\frac{1}{2}$}    (q_2)
			(q_2) edge [bend left]  node {$\frac{1}{2}$}    (q_3)
			edge [loop above] node {$\frac{1}{2}$}    ()
			(q_3) edge [bend left]  node {$\frac{1}{3}$}    (q_2)
			edge [bend left]  node {$\frac{1}{3}$}    (q_1)
			edge [loop below] node {$\frac{1}{3}$}    ();
		\end{tikzpicture}$
		
		\\  
		&\\
		&\\
		\hline
		\multirow{3}{*}{Checking whether the  } & \\
		& Here,\\chain is Irreducible
		& All the states are accessible to one another. \\and Aperiodic
		& $\implies$ They are in the same communication class. So, it is Irreducible.\\
		& \\
		& There exists the non- zero self-transition, which means that the chain \\
		& is Aperiodic.\\
		&\\ 
		& We know that if the Markov Chain is irreducible and aperiodic then \\
		& \qquad \qquad \qquad $\Vec{\pi}_{j} = \lim_{n \to \infty}P\{X_{n} = j\}$, $j = 1,...,N$ \\
		& These are the stationary probabilities. \\
		&\\
		\hline
		\multirow{3}{*}{Finding the Stationary} & \\
		& Stationary Probability can be represented as\\Probability Distributions
		& \qquad \qquad \qquad $\Vec{\pi} = \Vec{\pi} \vec{P}$\\
		& \\
		& \qquad $\implies$ $\myvec{v_{1}&&v_{2}&&v_{3}} = \myvec{v_{1}&&v_{2}&&v_{3}}\vec{P}$ \\
		& \\
		& Equating the above equation we get \\
		& \\
		& \qquad \qquad \qquad $\frac{1}{2}v_{1}-\frac{1}{3}v_{3} = 0$ $\label{eq:solutions/2018/dec/106/eq}$\\
		& \\
		& \qquad \qquad \qquad $\frac{1}{2}v_{1}-\frac{1}{2}v_{2} + \frac{1}{3}v_{3} = 0$\\
		& \\
		& \qquad \qquad \qquad $\frac{1}{2}v_{2}-\frac{2}{3}v_{3} = 0$\\
		& \\\
		& We see that summation of second and the third equation gives us the \\
		& first equation only. \\
		& And we know that the probability distribution will sum up to 1. \\
		& \\
		& \qquad \qquad \qquad $v_{1}+v_{2}+v_{3} = 1$ \\
		& \\
		& Therefore, we get the equation form as \\
		& \\
		& \qquad \qquad \qquad $\myvec{1&1&1\\\frac{1}{2}&0&\frac{-1}{3}\\\frac{1}{2}&\frac{-1}{2}&\frac{1}{3}}\myvec{v_{1}\\v_{2}\\v_{3}} = \myvec{1\\0\\0}$ \\
		& \\
		\hline
		\multirow{3}{*}{Solving the linear} & \\
		& The above linear equation can be solved using Gauss-Jordan method as\\equtions
		& \\
		& \qquad \qquad \qquad $\myvec{1&1&1&\vrule&1\\\frac{1}{2}&0&\frac{-1}{3}&\vrule&0\\\frac{1}{2}&\frac{-1}{2}&\frac{1}{3}&\vrule&0}$\\
		& \\
		& \qquad $\xleftrightarrow[]{R_2 \leftarrow R_2 - \frac{1}{2}R_1}$
		$\myvec{1&1&1&\vrule&1\\0&\frac{-1}{2}&\frac{-5}{6}&\vrule&\frac{-1}{2}\\\frac{1}{2}&\frac{-1}{2}&\frac{1}{3}&\vrule&0}$\\
		&\\
		& \qquad $\xleftrightarrow[]{R_3 \leftarrow R_3 - \frac{1}{2}R_1}$
		$\myvec{1&1&1&\vrule&1\\0&\frac{-1}{2}&\frac{-5}{6}&\vrule&\frac{-1}{2}\\0&-1&\frac{-1}{6}&\vrule&\frac{-1}{2}}$\\
		&\\
		& \qquad $\xleftrightarrow[]{R_2 \leftarrow \frac{-1}{2}R_2}$
		$\myvec{1&1&1&\vrule&1\\0&1&\frac{5}{3}&\vrule&1\\0&-1&\frac{-1}{6}&\vrule&\frac{-1}{2}}$\\
		&\\
		& \qquad $\xleftrightarrow[]{R_3 \leftarrow R_3 + R_2}$
		$\myvec{1&1&1&\vrule&1\\0&1&\frac{5}{3}&\vrule&1\\0&0&\frac{3}{2}&\vrule&\frac{1}{2}}$\\
		&\\
		& \qquad $\xleftrightarrow[]{R_3 \leftarrow \frac{3}{2}R_3}$
		$\myvec{1&1&1&\vrule&1\\0&1&\frac{5}{3}&\vrule&1\\0&0&1&\vrule&\frac{1}{3}}$\\
		&\\
		& \qquad $\xleftrightarrow[]{R_2 \leftarrow R_2 - \frac{5}{3}R_3}$
		$\myvec{1&1&1&\vrule&1\\0&1&0&\vrule&\frac{4}{9}\\0&0&1&\vrule&\frac{1}{3}}$\\
		&\\
		& \qquad $\xleftrightarrow[]{R_1 \leftarrow R_1 - R_3}$
		$\myvec{1&1&0&\vrule&\frac{2}{3}\\0&1&0&\vrule&\frac{4}{9}\\0&0&1&\vrule&\frac{1}{3}}$\\
		&\\
		& \qquad $\xleftrightarrow[]{R_1 \leftarrow R_1 - R_2}$
		$\myvec{1&0&0&\vrule&\frac{2}{9}\\0&1&0&\vrule&\frac{4}{9}\\0&0&1&\vrule&\frac{1}{3}}$\\
		&\\
		& $\therefore$, stationary probability distribution $\pi$ is given by \\
		& \qquad \qquad $\pi = \myvec{\frac{2}{9} & \frac{4}{9} & \frac{1}{3}}$ \\
		& \\
		\hline
		\multirow{3}{*}{Observations} & \\
		
		
		& Since the given transition probability matrix $\vec{P}$ is irreducible and aperiodic, \\
		& then $\lim_{n \to \infty} \vec{P}^{n}$ converges to a matrix with all rows identical and equal to $\vec{\pi}$. \\
		& \\
		& We were able to find $\vec{\pi}$ as $\myvec{\frac{2}{9} & \frac{4}{9} & \frac{1}{3}}$ \\
		& \\
		& $\lim_{n \to \infty} \vec{P}^{n} = \myvec{\frac{2}{9}&\frac{4}{9}&\frac{1}{3}\\\frac{2}{9}&\frac{4}{9}&\frac{1}{3}\\\frac{2}{9}&\frac{4}{9}&\frac{1}{3}}$\\
		& \\
		& From the above matrix, we get \\
		& \\
		& $\lim_{n \to \infty} \vec{P}^{n}_{11} = \frac{2}{9}$ \\
		&\\
		& $\lim_{n \to \infty} \vec{P}^{n}_{21} = \frac{2}{9}$ \\
		&\\
		& $\lim_{n \to \infty} \vec{P}^{n}_{32} = \frac{4}{9}$ \\
		&\\
		& $\lim_{n \to \infty} \vec{P}^{n}_{13} = \frac{1}{3}$ \\
		&\\
		\hline
		\multirow{3}{*}{Conclusion} & \\
		& From our observation we see that \\
		&\\
		& Options 1) and 4) are True.\\
		& \\
		\hline
\caption{}
\label{eq:solutions/2018/dec/106/table1}
	\end{longtable}
\twocolumn

\item For any $n\times n$ matrix $B$, let $N(B)=\{X\in \mathbb{R}^n:BX=0\}$ be the null space of $B$. Let $A$ be a $4\times 4$ matrix with $dim(N(A-4I))=2, dim(N(A-2I))=1$ and $rank(A)=3$
Which of the following are true?
\begin{enumerate}
\item 0,2 and 4 are eigenvalues of A
\item determinant(A)=0
\item A is not diagonalizable
\item trace(A)=8
\end{enumerate}
%
\solution
See Tables \ref{eq:solutions/2018/dec/106/table0} and \ref{eq:solutions/2018/dec/106/table1}


\onecolumn
	\begin{longtable}{|l|l|}
		\hline
		\multirow{3}{*}{Irreducible Markov Chain} 
		& \\
		& A Markov chain is $\textbf{irreducible}$ if all the states communicate with each other,\\
		& i.e., if there is only one communication class.\\
		&\\
		\hline
		\multirow{3}{*}{Aperiodic Markov Chain} & \\
		& If there is a self-transition in the chain ($p^{ii}>0$ for some i), then the chain is\\
		& called as $\textbf{aperiodic}$\\
		& \\
		\hline
		\multirow{3}{*}{Stationary Distribution} & \\
		& A stationary distribution of a Markov chain is a probability distribution that\\
		& remains unchanged in the Markov chain as time progresses. Typically, it is\\
		& represented as a row vector $\Vec{\pi}$ whose entries are probabilities summing to 1,\\ 
		& and given transition matrix $\textbf{P}$, it satisfies\\
		& \\
		&  \qquad \qquad  \qquad$\Vec{\pi} = \Vec{\pi} \textbf{P}$\\
		& \\
		\hline
\caption{}
\label{eq:solutions/2018/dec/106/table0}
	\end{longtable}
	\begin{longtable}{|l|l|}
		\hline
		\multirow{3}{*}{Drawing Transition diagram} 
		& \\
		& 
		
		$\begin{tikzpicture}[shorten >=1pt,node distance=2cm, scale =3, auto]
			\tikzstyle{every state}=[fill={rgb:black,1;white,10}]
			
			\node[state]   (q_1)                          {$1$};
			\node[state]   (q_2)  [right of=q_1]          {$2$};
			\node[state]   (q_3)  [below right of=q_1]          {$3$};
			
			\path[->]
			(q_1) edge [loop above] node {$\frac{1}{2}$}    (   )
			edge [bend left]  node {$\frac{1}{2}$}    (q_2)
			(q_2) edge [bend left]  node {$\frac{1}{2}$}    (q_3)
			edge [loop above] node {$\frac{1}{2}$}    ()
			(q_3) edge [bend left]  node {$\frac{1}{3}$}    (q_2)
			edge [bend left]  node {$\frac{1}{3}$}    (q_1)
			edge [loop below] node {$\frac{1}{3}$}    ();
		\end{tikzpicture}$
		
		\\  
		&\\
		&\\
		\hline
		\multirow{3}{*}{Checking whether the  } & \\
		& Here,\\chain is Irreducible
		& All the states are accessible to one another. \\and Aperiodic
		& $\implies$ They are in the same communication class. So, it is Irreducible.\\
		& \\
		& There exists the non- zero self-transition, which means that the chain \\
		& is Aperiodic.\\
		&\\ 
		& We know that if the Markov Chain is irreducible and aperiodic then \\
		& \qquad \qquad \qquad $\Vec{\pi}_{j} = \lim_{n \to \infty}P\{X_{n} = j\}$, $j = 1,...,N$ \\
		& These are the stationary probabilities. \\
		&\\
		\hline
		\multirow{3}{*}{Finding the Stationary} & \\
		& Stationary Probability can be represented as\\Probability Distributions
		& \qquad \qquad \qquad $\Vec{\pi} = \Vec{\pi} \vec{P}$\\
		& \\
		& \qquad $\implies$ $\myvec{v_{1}&&v_{2}&&v_{3}} = \myvec{v_{1}&&v_{2}&&v_{3}}\vec{P}$ \\
		& \\
		& Equating the above equation we get \\
		& \\
		& \qquad \qquad \qquad $\frac{1}{2}v_{1}-\frac{1}{3}v_{3} = 0$ $\label{eq:solutions/2018/dec/106/eq}$\\
		& \\
		& \qquad \qquad \qquad $\frac{1}{2}v_{1}-\frac{1}{2}v_{2} + \frac{1}{3}v_{3} = 0$\\
		& \\
		& \qquad \qquad \qquad $\frac{1}{2}v_{2}-\frac{2}{3}v_{3} = 0$\\
		& \\\
		& We see that summation of second and the third equation gives us the \\
		& first equation only. \\
		& And we know that the probability distribution will sum up to 1. \\
		& \\
		& \qquad \qquad \qquad $v_{1}+v_{2}+v_{3} = 1$ \\
		& \\
		& Therefore, we get the equation form as \\
		& \\
		& \qquad \qquad \qquad $\myvec{1&1&1\\\frac{1}{2}&0&\frac{-1}{3}\\\frac{1}{2}&\frac{-1}{2}&\frac{1}{3}}\myvec{v_{1}\\v_{2}\\v_{3}} = \myvec{1\\0\\0}$ \\
		& \\
		\hline
		\multirow{3}{*}{Solving the linear} & \\
		& The above linear equation can be solved using Gauss-Jordan method as\\equtions
		& \\
		& \qquad \qquad \qquad $\myvec{1&1&1&\vrule&1\\\frac{1}{2}&0&\frac{-1}{3}&\vrule&0\\\frac{1}{2}&\frac{-1}{2}&\frac{1}{3}&\vrule&0}$\\
		& \\
		& \qquad $\xleftrightarrow[]{R_2 \leftarrow R_2 - \frac{1}{2}R_1}$
		$\myvec{1&1&1&\vrule&1\\0&\frac{-1}{2}&\frac{-5}{6}&\vrule&\frac{-1}{2}\\\frac{1}{2}&\frac{-1}{2}&\frac{1}{3}&\vrule&0}$\\
		&\\
		& \qquad $\xleftrightarrow[]{R_3 \leftarrow R_3 - \frac{1}{2}R_1}$
		$\myvec{1&1&1&\vrule&1\\0&\frac{-1}{2}&\frac{-5}{6}&\vrule&\frac{-1}{2}\\0&-1&\frac{-1}{6}&\vrule&\frac{-1}{2}}$\\
		&\\
		& \qquad $\xleftrightarrow[]{R_2 \leftarrow \frac{-1}{2}R_2}$
		$\myvec{1&1&1&\vrule&1\\0&1&\frac{5}{3}&\vrule&1\\0&-1&\frac{-1}{6}&\vrule&\frac{-1}{2}}$\\
		&\\
		& \qquad $\xleftrightarrow[]{R_3 \leftarrow R_3 + R_2}$
		$\myvec{1&1&1&\vrule&1\\0&1&\frac{5}{3}&\vrule&1\\0&0&\frac{3}{2}&\vrule&\frac{1}{2}}$\\
		&\\
		& \qquad $\xleftrightarrow[]{R_3 \leftarrow \frac{3}{2}R_3}$
		$\myvec{1&1&1&\vrule&1\\0&1&\frac{5}{3}&\vrule&1\\0&0&1&\vrule&\frac{1}{3}}$\\
		&\\
		& \qquad $\xleftrightarrow[]{R_2 \leftarrow R_2 - \frac{5}{3}R_3}$
		$\myvec{1&1&1&\vrule&1\\0&1&0&\vrule&\frac{4}{9}\\0&0&1&\vrule&\frac{1}{3}}$\\
		&\\
		& \qquad $\xleftrightarrow[]{R_1 \leftarrow R_1 - R_3}$
		$\myvec{1&1&0&\vrule&\frac{2}{3}\\0&1&0&\vrule&\frac{4}{9}\\0&0&1&\vrule&\frac{1}{3}}$\\
		&\\
		& \qquad $\xleftrightarrow[]{R_1 \leftarrow R_1 - R_2}$
		$\myvec{1&0&0&\vrule&\frac{2}{9}\\0&1&0&\vrule&\frac{4}{9}\\0&0&1&\vrule&\frac{1}{3}}$\\
		&\\
		& $\therefore$, stationary probability distribution $\pi$ is given by \\
		& \qquad \qquad $\pi = \myvec{\frac{2}{9} & \frac{4}{9} & \frac{1}{3}}$ \\
		& \\
		\hline
		\multirow{3}{*}{Observations} & \\
		
		
		& Since the given transition probability matrix $\vec{P}$ is irreducible and aperiodic, \\
		& then $\lim_{n \to \infty} \vec{P}^{n}$ converges to a matrix with all rows identical and equal to $\vec{\pi}$. \\
		& \\
		& We were able to find $\vec{\pi}$ as $\myvec{\frac{2}{9} & \frac{4}{9} & \frac{1}{3}}$ \\
		& \\
		& $\lim_{n \to \infty} \vec{P}^{n} = \myvec{\frac{2}{9}&\frac{4}{9}&\frac{1}{3}\\\frac{2}{9}&\frac{4}{9}&\frac{1}{3}\\\frac{2}{9}&\frac{4}{9}&\frac{1}{3}}$\\
		& \\
		& From the above matrix, we get \\
		& \\
		& $\lim_{n \to \infty} \vec{P}^{n}_{11} = \frac{2}{9}$ \\
		&\\
		& $\lim_{n \to \infty} \vec{P}^{n}_{21} = \frac{2}{9}$ \\
		&\\
		& $\lim_{n \to \infty} \vec{P}^{n}_{32} = \frac{4}{9}$ \\
		&\\
		& $\lim_{n \to \infty} \vec{P}^{n}_{13} = \frac{1}{3}$ \\
		&\\
		\hline
		\multirow{3}{*}{Conclusion} & \\
		& From our observation we see that \\
		&\\
		& Options 1) and 4) are True.\\
		& \\
		\hline
\caption{}
\label{eq:solutions/2018/dec/106/table1}
	\end{longtable}
\twocolumn

\item Which of the following 3x3 matrices are diagonizable over $\mathbb{R}?$\\
\begin{enumerate}
    \item \myvec{1&2&3\\0&4&5\\0&0&6}
    \item \myvec{0&1&0\\-1&0&0\\0&0&1}
    \item \myvec{1&2&3\\2&1&4\\3&4&1}
    \item \myvec{0&1&2\\0&0&1\\0&0&0}
\end{enumerate}
%
\solution
See Tables \ref{eq:solutions/2018/dec/106/table0} and \ref{eq:solutions/2018/dec/106/table1}


\onecolumn
	\begin{longtable}{|l|l|}
		\hline
		\multirow{3}{*}{Irreducible Markov Chain} 
		& \\
		& A Markov chain is $\textbf{irreducible}$ if all the states communicate with each other,\\
		& i.e., if there is only one communication class.\\
		&\\
		\hline
		\multirow{3}{*}{Aperiodic Markov Chain} & \\
		& If there is a self-transition in the chain ($p^{ii}>0$ for some i), then the chain is\\
		& called as $\textbf{aperiodic}$\\
		& \\
		\hline
		\multirow{3}{*}{Stationary Distribution} & \\
		& A stationary distribution of a Markov chain is a probability distribution that\\
		& remains unchanged in the Markov chain as time progresses. Typically, it is\\
		& represented as a row vector $\Vec{\pi}$ whose entries are probabilities summing to 1,\\ 
		& and given transition matrix $\textbf{P}$, it satisfies\\
		& \\
		&  \qquad \qquad  \qquad$\Vec{\pi} = \Vec{\pi} \textbf{P}$\\
		& \\
		\hline
\caption{}
\label{eq:solutions/2018/dec/106/table0}
	\end{longtable}
	\begin{longtable}{|l|l|}
		\hline
		\multirow{3}{*}{Drawing Transition diagram} 
		& \\
		& 
		
		$\begin{tikzpicture}[shorten >=1pt,node distance=2cm, scale =3, auto]
			\tikzstyle{every state}=[fill={rgb:black,1;white,10}]
			
			\node[state]   (q_1)                          {$1$};
			\node[state]   (q_2)  [right of=q_1]          {$2$};
			\node[state]   (q_3)  [below right of=q_1]          {$3$};
			
			\path[->]
			(q_1) edge [loop above] node {$\frac{1}{2}$}    (   )
			edge [bend left]  node {$\frac{1}{2}$}    (q_2)
			(q_2) edge [bend left]  node {$\frac{1}{2}$}    (q_3)
			edge [loop above] node {$\frac{1}{2}$}    ()
			(q_3) edge [bend left]  node {$\frac{1}{3}$}    (q_2)
			edge [bend left]  node {$\frac{1}{3}$}    (q_1)
			edge [loop below] node {$\frac{1}{3}$}    ();
		\end{tikzpicture}$
		
		\\  
		&\\
		&\\
		\hline
		\multirow{3}{*}{Checking whether the  } & \\
		& Here,\\chain is Irreducible
		& All the states are accessible to one another. \\and Aperiodic
		& $\implies$ They are in the same communication class. So, it is Irreducible.\\
		& \\
		& There exists the non- zero self-transition, which means that the chain \\
		& is Aperiodic.\\
		&\\ 
		& We know that if the Markov Chain is irreducible and aperiodic then \\
		& \qquad \qquad \qquad $\Vec{\pi}_{j} = \lim_{n \to \infty}P\{X_{n} = j\}$, $j = 1,...,N$ \\
		& These are the stationary probabilities. \\
		&\\
		\hline
		\multirow{3}{*}{Finding the Stationary} & \\
		& Stationary Probability can be represented as\\Probability Distributions
		& \qquad \qquad \qquad $\Vec{\pi} = \Vec{\pi} \vec{P}$\\
		& \\
		& \qquad $\implies$ $\myvec{v_{1}&&v_{2}&&v_{3}} = \myvec{v_{1}&&v_{2}&&v_{3}}\vec{P}$ \\
		& \\
		& Equating the above equation we get \\
		& \\
		& \qquad \qquad \qquad $\frac{1}{2}v_{1}-\frac{1}{3}v_{3} = 0$ $\label{eq:solutions/2018/dec/106/eq}$\\
		& \\
		& \qquad \qquad \qquad $\frac{1}{2}v_{1}-\frac{1}{2}v_{2} + \frac{1}{3}v_{3} = 0$\\
		& \\
		& \qquad \qquad \qquad $\frac{1}{2}v_{2}-\frac{2}{3}v_{3} = 0$\\
		& \\\
		& We see that summation of second and the third equation gives us the \\
		& first equation only. \\
		& And we know that the probability distribution will sum up to 1. \\
		& \\
		& \qquad \qquad \qquad $v_{1}+v_{2}+v_{3} = 1$ \\
		& \\
		& Therefore, we get the equation form as \\
		& \\
		& \qquad \qquad \qquad $\myvec{1&1&1\\\frac{1}{2}&0&\frac{-1}{3}\\\frac{1}{2}&\frac{-1}{2}&\frac{1}{3}}\myvec{v_{1}\\v_{2}\\v_{3}} = \myvec{1\\0\\0}$ \\
		& \\
		\hline
		\multirow{3}{*}{Solving the linear} & \\
		& The above linear equation can be solved using Gauss-Jordan method as\\equtions
		& \\
		& \qquad \qquad \qquad $\myvec{1&1&1&\vrule&1\\\frac{1}{2}&0&\frac{-1}{3}&\vrule&0\\\frac{1}{2}&\frac{-1}{2}&\frac{1}{3}&\vrule&0}$\\
		& \\
		& \qquad $\xleftrightarrow[]{R_2 \leftarrow R_2 - \frac{1}{2}R_1}$
		$\myvec{1&1&1&\vrule&1\\0&\frac{-1}{2}&\frac{-5}{6}&\vrule&\frac{-1}{2}\\\frac{1}{2}&\frac{-1}{2}&\frac{1}{3}&\vrule&0}$\\
		&\\
		& \qquad $\xleftrightarrow[]{R_3 \leftarrow R_3 - \frac{1}{2}R_1}$
		$\myvec{1&1&1&\vrule&1\\0&\frac{-1}{2}&\frac{-5}{6}&\vrule&\frac{-1}{2}\\0&-1&\frac{-1}{6}&\vrule&\frac{-1}{2}}$\\
		&\\
		& \qquad $\xleftrightarrow[]{R_2 \leftarrow \frac{-1}{2}R_2}$
		$\myvec{1&1&1&\vrule&1\\0&1&\frac{5}{3}&\vrule&1\\0&-1&\frac{-1}{6}&\vrule&\frac{-1}{2}}$\\
		&\\
		& \qquad $\xleftrightarrow[]{R_3 \leftarrow R_3 + R_2}$
		$\myvec{1&1&1&\vrule&1\\0&1&\frac{5}{3}&\vrule&1\\0&0&\frac{3}{2}&\vrule&\frac{1}{2}}$\\
		&\\
		& \qquad $\xleftrightarrow[]{R_3 \leftarrow \frac{3}{2}R_3}$
		$\myvec{1&1&1&\vrule&1\\0&1&\frac{5}{3}&\vrule&1\\0&0&1&\vrule&\frac{1}{3}}$\\
		&\\
		& \qquad $\xleftrightarrow[]{R_2 \leftarrow R_2 - \frac{5}{3}R_3}$
		$\myvec{1&1&1&\vrule&1\\0&1&0&\vrule&\frac{4}{9}\\0&0&1&\vrule&\frac{1}{3}}$\\
		&\\
		& \qquad $\xleftrightarrow[]{R_1 \leftarrow R_1 - R_3}$
		$\myvec{1&1&0&\vrule&\frac{2}{3}\\0&1&0&\vrule&\frac{4}{9}\\0&0&1&\vrule&\frac{1}{3}}$\\
		&\\
		& \qquad $\xleftrightarrow[]{R_1 \leftarrow R_1 - R_2}$
		$\myvec{1&0&0&\vrule&\frac{2}{9}\\0&1&0&\vrule&\frac{4}{9}\\0&0&1&\vrule&\frac{1}{3}}$\\
		&\\
		& $\therefore$, stationary probability distribution $\pi$ is given by \\
		& \qquad \qquad $\pi = \myvec{\frac{2}{9} & \frac{4}{9} & \frac{1}{3}}$ \\
		& \\
		\hline
		\multirow{3}{*}{Observations} & \\
		
		
		& Since the given transition probability matrix $\vec{P}$ is irreducible and aperiodic, \\
		& then $\lim_{n \to \infty} \vec{P}^{n}$ converges to a matrix with all rows identical and equal to $\vec{\pi}$. \\
		& \\
		& We were able to find $\vec{\pi}$ as $\myvec{\frac{2}{9} & \frac{4}{9} & \frac{1}{3}}$ \\
		& \\
		& $\lim_{n \to \infty} \vec{P}^{n} = \myvec{\frac{2}{9}&\frac{4}{9}&\frac{1}{3}\\\frac{2}{9}&\frac{4}{9}&\frac{1}{3}\\\frac{2}{9}&\frac{4}{9}&\frac{1}{3}}$\\
		& \\
		& From the above matrix, we get \\
		& \\
		& $\lim_{n \to \infty} \vec{P}^{n}_{11} = \frac{2}{9}$ \\
		&\\
		& $\lim_{n \to \infty} \vec{P}^{n}_{21} = \frac{2}{9}$ \\
		&\\
		& $\lim_{n \to \infty} \vec{P}^{n}_{32} = \frac{4}{9}$ \\
		&\\
		& $\lim_{n \to \infty} \vec{P}^{n}_{13} = \frac{1}{3}$ \\
		&\\
		\hline
		\multirow{3}{*}{Conclusion} & \\
		& From our observation we see that \\
		&\\
		& Options 1) and 4) are True.\\
		& \\
		\hline
\caption{}
\label{eq:solutions/2018/dec/106/table1}
	\end{longtable}
\twocolumn

\twocolumn
\item Let $\vec{A} = \myvec{3 & 1 & 2 \\ 1 & 2 & 3 \\ 2 & 3 & 1  }$ and $\vec{Q(X) = X^TAX}$ for $\vec{X} \in \mathbb{R}^{3}$. Then
\begin{enumerate}
	\item $\vec{A}$ has exactly two positive eigen values.
	\item all the eigen values of $\vec{A}$ are positive.
	\item $\vec{Q(X)} \geq 0 $ $\forall$ $\vec{X}$ $\in$ $\mathbb{R}^3$
	\item $\vec{Q(X)} < 0 $ for some $\vec{X}$ $\in$ $\mathbb{R}^3$
\end{enumerate}
%
%
\solution
See Tables \ref{eq:solutions/2018/dec/106/table0} and \ref{eq:solutions/2018/dec/106/table1}


\onecolumn
	\begin{longtable}{|l|l|}
		\hline
		\multirow{3}{*}{Irreducible Markov Chain} 
		& \\
		& A Markov chain is $\textbf{irreducible}$ if all the states communicate with each other,\\
		& i.e., if there is only one communication class.\\
		&\\
		\hline
		\multirow{3}{*}{Aperiodic Markov Chain} & \\
		& If there is a self-transition in the chain ($p^{ii}>0$ for some i), then the chain is\\
		& called as $\textbf{aperiodic}$\\
		& \\
		\hline
		\multirow{3}{*}{Stationary Distribution} & \\
		& A stationary distribution of a Markov chain is a probability distribution that\\
		& remains unchanged in the Markov chain as time progresses. Typically, it is\\
		& represented as a row vector $\Vec{\pi}$ whose entries are probabilities summing to 1,\\ 
		& and given transition matrix $\textbf{P}$, it satisfies\\
		& \\
		&  \qquad \qquad  \qquad$\Vec{\pi} = \Vec{\pi} \textbf{P}$\\
		& \\
		\hline
\caption{}
\label{eq:solutions/2018/dec/106/table0}
	\end{longtable}
	\begin{longtable}{|l|l|}
		\hline
		\multirow{3}{*}{Drawing Transition diagram} 
		& \\
		& 
		
		$\begin{tikzpicture}[shorten >=1pt,node distance=2cm, scale =3, auto]
			\tikzstyle{every state}=[fill={rgb:black,1;white,10}]
			
			\node[state]   (q_1)                          {$1$};
			\node[state]   (q_2)  [right of=q_1]          {$2$};
			\node[state]   (q_3)  [below right of=q_1]          {$3$};
			
			\path[->]
			(q_1) edge [loop above] node {$\frac{1}{2}$}    (   )
			edge [bend left]  node {$\frac{1}{2}$}    (q_2)
			(q_2) edge [bend left]  node {$\frac{1}{2}$}    (q_3)
			edge [loop above] node {$\frac{1}{2}$}    ()
			(q_3) edge [bend left]  node {$\frac{1}{3}$}    (q_2)
			edge [bend left]  node {$\frac{1}{3}$}    (q_1)
			edge [loop below] node {$\frac{1}{3}$}    ();
		\end{tikzpicture}$
		
		\\  
		&\\
		&\\
		\hline
		\multirow{3}{*}{Checking whether the  } & \\
		& Here,\\chain is Irreducible
		& All the states are accessible to one another. \\and Aperiodic
		& $\implies$ They are in the same communication class. So, it is Irreducible.\\
		& \\
		& There exists the non- zero self-transition, which means that the chain \\
		& is Aperiodic.\\
		&\\ 
		& We know that if the Markov Chain is irreducible and aperiodic then \\
		& \qquad \qquad \qquad $\Vec{\pi}_{j} = \lim_{n \to \infty}P\{X_{n} = j\}$, $j = 1,...,N$ \\
		& These are the stationary probabilities. \\
		&\\
		\hline
		\multirow{3}{*}{Finding the Stationary} & \\
		& Stationary Probability can be represented as\\Probability Distributions
		& \qquad \qquad \qquad $\Vec{\pi} = \Vec{\pi} \vec{P}$\\
		& \\
		& \qquad $\implies$ $\myvec{v_{1}&&v_{2}&&v_{3}} = \myvec{v_{1}&&v_{2}&&v_{3}}\vec{P}$ \\
		& \\
		& Equating the above equation we get \\
		& \\
		& \qquad \qquad \qquad $\frac{1}{2}v_{1}-\frac{1}{3}v_{3} = 0$ $\label{eq:solutions/2018/dec/106/eq}$\\
		& \\
		& \qquad \qquad \qquad $\frac{1}{2}v_{1}-\frac{1}{2}v_{2} + \frac{1}{3}v_{3} = 0$\\
		& \\
		& \qquad \qquad \qquad $\frac{1}{2}v_{2}-\frac{2}{3}v_{3} = 0$\\
		& \\\
		& We see that summation of second and the third equation gives us the \\
		& first equation only. \\
		& And we know that the probability distribution will sum up to 1. \\
		& \\
		& \qquad \qquad \qquad $v_{1}+v_{2}+v_{3} = 1$ \\
		& \\
		& Therefore, we get the equation form as \\
		& \\
		& \qquad \qquad \qquad $\myvec{1&1&1\\\frac{1}{2}&0&\frac{-1}{3}\\\frac{1}{2}&\frac{-1}{2}&\frac{1}{3}}\myvec{v_{1}\\v_{2}\\v_{3}} = \myvec{1\\0\\0}$ \\
		& \\
		\hline
		\multirow{3}{*}{Solving the linear} & \\
		& The above linear equation can be solved using Gauss-Jordan method as\\equtions
		& \\
		& \qquad \qquad \qquad $\myvec{1&1&1&\vrule&1\\\frac{1}{2}&0&\frac{-1}{3}&\vrule&0\\\frac{1}{2}&\frac{-1}{2}&\frac{1}{3}&\vrule&0}$\\
		& \\
		& \qquad $\xleftrightarrow[]{R_2 \leftarrow R_2 - \frac{1}{2}R_1}$
		$\myvec{1&1&1&\vrule&1\\0&\frac{-1}{2}&\frac{-5}{6}&\vrule&\frac{-1}{2}\\\frac{1}{2}&\frac{-1}{2}&\frac{1}{3}&\vrule&0}$\\
		&\\
		& \qquad $\xleftrightarrow[]{R_3 \leftarrow R_3 - \frac{1}{2}R_1}$
		$\myvec{1&1&1&\vrule&1\\0&\frac{-1}{2}&\frac{-5}{6}&\vrule&\frac{-1}{2}\\0&-1&\frac{-1}{6}&\vrule&\frac{-1}{2}}$\\
		&\\
		& \qquad $\xleftrightarrow[]{R_2 \leftarrow \frac{-1}{2}R_2}$
		$\myvec{1&1&1&\vrule&1\\0&1&\frac{5}{3}&\vrule&1\\0&-1&\frac{-1}{6}&\vrule&\frac{-1}{2}}$\\
		&\\
		& \qquad $\xleftrightarrow[]{R_3 \leftarrow R_3 + R_2}$
		$\myvec{1&1&1&\vrule&1\\0&1&\frac{5}{3}&\vrule&1\\0&0&\frac{3}{2}&\vrule&\frac{1}{2}}$\\
		&\\
		& \qquad $\xleftrightarrow[]{R_3 \leftarrow \frac{3}{2}R_3}$
		$\myvec{1&1&1&\vrule&1\\0&1&\frac{5}{3}&\vrule&1\\0&0&1&\vrule&\frac{1}{3}}$\\
		&\\
		& \qquad $\xleftrightarrow[]{R_2 \leftarrow R_2 - \frac{5}{3}R_3}$
		$\myvec{1&1&1&\vrule&1\\0&1&0&\vrule&\frac{4}{9}\\0&0&1&\vrule&\frac{1}{3}}$\\
		&\\
		& \qquad $\xleftrightarrow[]{R_1 \leftarrow R_1 - R_3}$
		$\myvec{1&1&0&\vrule&\frac{2}{3}\\0&1&0&\vrule&\frac{4}{9}\\0&0&1&\vrule&\frac{1}{3}}$\\
		&\\
		& \qquad $\xleftrightarrow[]{R_1 \leftarrow R_1 - R_2}$
		$\myvec{1&0&0&\vrule&\frac{2}{9}\\0&1&0&\vrule&\frac{4}{9}\\0&0&1&\vrule&\frac{1}{3}}$\\
		&\\
		& $\therefore$, stationary probability distribution $\pi$ is given by \\
		& \qquad \qquad $\pi = \myvec{\frac{2}{9} & \frac{4}{9} & \frac{1}{3}}$ \\
		& \\
		\hline
		\multirow{3}{*}{Observations} & \\
		
		
		& Since the given transition probability matrix $\vec{P}$ is irreducible and aperiodic, \\
		& then $\lim_{n \to \infty} \vec{P}^{n}$ converges to a matrix with all rows identical and equal to $\vec{\pi}$. \\
		& \\
		& We were able to find $\vec{\pi}$ as $\myvec{\frac{2}{9} & \frac{4}{9} & \frac{1}{3}}$ \\
		& \\
		& $\lim_{n \to \infty} \vec{P}^{n} = \myvec{\frac{2}{9}&\frac{4}{9}&\frac{1}{3}\\\frac{2}{9}&\frac{4}{9}&\frac{1}{3}\\\frac{2}{9}&\frac{4}{9}&\frac{1}{3}}$\\
		& \\
		& From the above matrix, we get \\
		& \\
		& $\lim_{n \to \infty} \vec{P}^{n}_{11} = \frac{2}{9}$ \\
		&\\
		& $\lim_{n \to \infty} \vec{P}^{n}_{21} = \frac{2}{9}$ \\
		&\\
		& $\lim_{n \to \infty} \vec{P}^{n}_{32} = \frac{4}{9}$ \\
		&\\
		& $\lim_{n \to \infty} \vec{P}^{n}_{13} = \frac{1}{3}$ \\
		&\\
		\hline
		\multirow{3}{*}{Conclusion} & \\
		& From our observation we see that \\
		&\\
		& Options 1) and 4) are True.\\
		& \\
		\hline
\caption{}
\label{eq:solutions/2018/dec/106/table1}
	\end{longtable}
\twocolumn

\item Consider the matrix
\begin{align}
A(x) = \myvec{1+x^2&7&11\\3x&2x&4\\8x&17&13} & ;x\in \vec{R}.
\end{align}
Then,
\begin{enumerate}
\item A(x) has eigenvalue 0 for some $x\in \vec{R}$.
\item 0 is not an eigenvalue of A(x) for any $x\in \vec{R}$.
\item A(x) has eigenvalue 0 $\forall x\in \vec{R}$.
\item A(x) is invertible $\forall x\in \vec{R}$.
\end{enumerate}
%
\solution
See Tables \ref{eq:solutions/2018/dec/106/table0} and \ref{eq:solutions/2018/dec/106/table1}


\onecolumn
	\begin{longtable}{|l|l|}
		\hline
		\multirow{3}{*}{Irreducible Markov Chain} 
		& \\
		& A Markov chain is $\textbf{irreducible}$ if all the states communicate with each other,\\
		& i.e., if there is only one communication class.\\
		&\\
		\hline
		\multirow{3}{*}{Aperiodic Markov Chain} & \\
		& If there is a self-transition in the chain ($p^{ii}>0$ for some i), then the chain is\\
		& called as $\textbf{aperiodic}$\\
		& \\
		\hline
		\multirow{3}{*}{Stationary Distribution} & \\
		& A stationary distribution of a Markov chain is a probability distribution that\\
		& remains unchanged in the Markov chain as time progresses. Typically, it is\\
		& represented as a row vector $\Vec{\pi}$ whose entries are probabilities summing to 1,\\ 
		& and given transition matrix $\textbf{P}$, it satisfies\\
		& \\
		&  \qquad \qquad  \qquad$\Vec{\pi} = \Vec{\pi} \textbf{P}$\\
		& \\
		\hline
\caption{}
\label{eq:solutions/2018/dec/106/table0}
	\end{longtable}
	\begin{longtable}{|l|l|}
		\hline
		\multirow{3}{*}{Drawing Transition diagram} 
		& \\
		& 
		
		$\begin{tikzpicture}[shorten >=1pt,node distance=2cm, scale =3, auto]
			\tikzstyle{every state}=[fill={rgb:black,1;white,10}]
			
			\node[state]   (q_1)                          {$1$};
			\node[state]   (q_2)  [right of=q_1]          {$2$};
			\node[state]   (q_3)  [below right of=q_1]          {$3$};
			
			\path[->]
			(q_1) edge [loop above] node {$\frac{1}{2}$}    (   )
			edge [bend left]  node {$\frac{1}{2}$}    (q_2)
			(q_2) edge [bend left]  node {$\frac{1}{2}$}    (q_3)
			edge [loop above] node {$\frac{1}{2}$}    ()
			(q_3) edge [bend left]  node {$\frac{1}{3}$}    (q_2)
			edge [bend left]  node {$\frac{1}{3}$}    (q_1)
			edge [loop below] node {$\frac{1}{3}$}    ();
		\end{tikzpicture}$
		
		\\  
		&\\
		&\\
		\hline
		\multirow{3}{*}{Checking whether the  } & \\
		& Here,\\chain is Irreducible
		& All the states are accessible to one another. \\and Aperiodic
		& $\implies$ They are in the same communication class. So, it is Irreducible.\\
		& \\
		& There exists the non- zero self-transition, which means that the chain \\
		& is Aperiodic.\\
		&\\ 
		& We know that if the Markov Chain is irreducible and aperiodic then \\
		& \qquad \qquad \qquad $\Vec{\pi}_{j} = \lim_{n \to \infty}P\{X_{n} = j\}$, $j = 1,...,N$ \\
		& These are the stationary probabilities. \\
		&\\
		\hline
		\multirow{3}{*}{Finding the Stationary} & \\
		& Stationary Probability can be represented as\\Probability Distributions
		& \qquad \qquad \qquad $\Vec{\pi} = \Vec{\pi} \vec{P}$\\
		& \\
		& \qquad $\implies$ $\myvec{v_{1}&&v_{2}&&v_{3}} = \myvec{v_{1}&&v_{2}&&v_{3}}\vec{P}$ \\
		& \\
		& Equating the above equation we get \\
		& \\
		& \qquad \qquad \qquad $\frac{1}{2}v_{1}-\frac{1}{3}v_{3} = 0$ $\label{eq:solutions/2018/dec/106/eq}$\\
		& \\
		& \qquad \qquad \qquad $\frac{1}{2}v_{1}-\frac{1}{2}v_{2} + \frac{1}{3}v_{3} = 0$\\
		& \\
		& \qquad \qquad \qquad $\frac{1}{2}v_{2}-\frac{2}{3}v_{3} = 0$\\
		& \\\
		& We see that summation of second and the third equation gives us the \\
		& first equation only. \\
		& And we know that the probability distribution will sum up to 1. \\
		& \\
		& \qquad \qquad \qquad $v_{1}+v_{2}+v_{3} = 1$ \\
		& \\
		& Therefore, we get the equation form as \\
		& \\
		& \qquad \qquad \qquad $\myvec{1&1&1\\\frac{1}{2}&0&\frac{-1}{3}\\\frac{1}{2}&\frac{-1}{2}&\frac{1}{3}}\myvec{v_{1}\\v_{2}\\v_{3}} = \myvec{1\\0\\0}$ \\
		& \\
		\hline
		\multirow{3}{*}{Solving the linear} & \\
		& The above linear equation can be solved using Gauss-Jordan method as\\equtions
		& \\
		& \qquad \qquad \qquad $\myvec{1&1&1&\vrule&1\\\frac{1}{2}&0&\frac{-1}{3}&\vrule&0\\\frac{1}{2}&\frac{-1}{2}&\frac{1}{3}&\vrule&0}$\\
		& \\
		& \qquad $\xleftrightarrow[]{R_2 \leftarrow R_2 - \frac{1}{2}R_1}$
		$\myvec{1&1&1&\vrule&1\\0&\frac{-1}{2}&\frac{-5}{6}&\vrule&\frac{-1}{2}\\\frac{1}{2}&\frac{-1}{2}&\frac{1}{3}&\vrule&0}$\\
		&\\
		& \qquad $\xleftrightarrow[]{R_3 \leftarrow R_3 - \frac{1}{2}R_1}$
		$\myvec{1&1&1&\vrule&1\\0&\frac{-1}{2}&\frac{-5}{6}&\vrule&\frac{-1}{2}\\0&-1&\frac{-1}{6}&\vrule&\frac{-1}{2}}$\\
		&\\
		& \qquad $\xleftrightarrow[]{R_2 \leftarrow \frac{-1}{2}R_2}$
		$\myvec{1&1&1&\vrule&1\\0&1&\frac{5}{3}&\vrule&1\\0&-1&\frac{-1}{6}&\vrule&\frac{-1}{2}}$\\
		&\\
		& \qquad $\xleftrightarrow[]{R_3 \leftarrow R_3 + R_2}$
		$\myvec{1&1&1&\vrule&1\\0&1&\frac{5}{3}&\vrule&1\\0&0&\frac{3}{2}&\vrule&\frac{1}{2}}$\\
		&\\
		& \qquad $\xleftrightarrow[]{R_3 \leftarrow \frac{3}{2}R_3}$
		$\myvec{1&1&1&\vrule&1\\0&1&\frac{5}{3}&\vrule&1\\0&0&1&\vrule&\frac{1}{3}}$\\
		&\\
		& \qquad $\xleftrightarrow[]{R_2 \leftarrow R_2 - \frac{5}{3}R_3}$
		$\myvec{1&1&1&\vrule&1\\0&1&0&\vrule&\frac{4}{9}\\0&0&1&\vrule&\frac{1}{3}}$\\
		&\\
		& \qquad $\xleftrightarrow[]{R_1 \leftarrow R_1 - R_3}$
		$\myvec{1&1&0&\vrule&\frac{2}{3}\\0&1&0&\vrule&\frac{4}{9}\\0&0&1&\vrule&\frac{1}{3}}$\\
		&\\
		& \qquad $\xleftrightarrow[]{R_1 \leftarrow R_1 - R_2}$
		$\myvec{1&0&0&\vrule&\frac{2}{9}\\0&1&0&\vrule&\frac{4}{9}\\0&0&1&\vrule&\frac{1}{3}}$\\
		&\\
		& $\therefore$, stationary probability distribution $\pi$ is given by \\
		& \qquad \qquad $\pi = \myvec{\frac{2}{9} & \frac{4}{9} & \frac{1}{3}}$ \\
		& \\
		\hline
		\multirow{3}{*}{Observations} & \\
		
		
		& Since the given transition probability matrix $\vec{P}$ is irreducible and aperiodic, \\
		& then $\lim_{n \to \infty} \vec{P}^{n}$ converges to a matrix with all rows identical and equal to $\vec{\pi}$. \\
		& \\
		& We were able to find $\vec{\pi}$ as $\myvec{\frac{2}{9} & \frac{4}{9} & \frac{1}{3}}$ \\
		& \\
		& $\lim_{n \to \infty} \vec{P}^{n} = \myvec{\frac{2}{9}&\frac{4}{9}&\frac{1}{3}\\\frac{2}{9}&\frac{4}{9}&\frac{1}{3}\\\frac{2}{9}&\frac{4}{9}&\frac{1}{3}}$\\
		& \\
		& From the above matrix, we get \\
		& \\
		& $\lim_{n \to \infty} \vec{P}^{n}_{11} = \frac{2}{9}$ \\
		&\\
		& $\lim_{n \to \infty} \vec{P}^{n}_{21} = \frac{2}{9}$ \\
		&\\
		& $\lim_{n \to \infty} \vec{P}^{n}_{32} = \frac{4}{9}$ \\
		&\\
		& $\lim_{n \to \infty} \vec{P}^{n}_{13} = \frac{1}{3}$ \\
		&\\
		\hline
		\multirow{3}{*}{Conclusion} & \\
		& From our observation we see that \\
		&\\
		& Options 1) and 4) are True.\\
		& \\
		\hline
\caption{}
\label{eq:solutions/2018/dec/106/table1}
	\end{longtable}
\twocolumn

\end{enumerate}
