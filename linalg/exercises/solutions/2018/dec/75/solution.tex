Given 
\begin{align}
T(p(x)) = p(x+1)+p(x-1).
\end{align}
The matrix of T with respect to the standard basis $\vec{B}=\{1, x, x^2, x^3\}$ is given by:
\begin{align}
p(x)=1 &\implies T(1) = 1 + 1 \nonumber \\
&= 2\\
p(x)=x &\implies T(x) = x+1 + x-1 \nonumber\\
&= 2x \\
p(x)=x^2 &\implies T(x^2) = (x+1)^2 + (x-1)^2 \nonumber\\
&= 2 + 2x^2 \\
p(x)=x^3 &\implies T(x^3) = (x+1)^3 + (x-1)^3 \nonumber\\
&= 6x + 2x^3 
\end{align}

Hence, matrix of $T$ is:
\begin{align}
	\myvec{2 & 0 & 2 & 0 \\ 0 & 2 & 0 & 6\\ 0 & 0 & 2 & 0\\0 & 0 & 0 & 2}\label{eq:solutions/2018/dec/75/Tmatrix}
\end{align} 
See Table \ref{eq:solutions/2018/dec/75/table}
%
\begin{table}[h]
	\begin{tabular}{|m{3cm}|m{5cm}|}
		\hline
		&\\
		$\det(T)$ = 0 
		& \textbf{False}. From \eqref{eq:solutions/2018/dec/75/Tmatrix}, it is found that the determinant is not zero as the eigenvalues are nonzero.\\
		& \\
		\hline
		&\\
		$(T-2\vec{I})^4=0$ but $(T-2\vec{I})^3 \ne 0$
		& \textbf{False}. 
		$(T-2\vec{I})$
		 = $\myvec{0 & 0 & 2 & 0 \\ 0 & 0 & 0 & 6\\ 0 & 0 & 0 & 0\\0 & 0 & 0 & 0}$\\
		& $\implies (T - 2\vec{I})^2 = 0$\\
		&and hence $(T-2\vec{I})^4=0$ and $(T-2\vec{I})^3 = 0$ \\
		&\\
		\hline
		&\\
	    $(T-2\vec{I})^3=0$ but $(T-2\vec{I})^2 \ne 0$
		& \textbf{False}. Because $(T-2\vec{I})^3=0$ and $(T-2\vec{I})^2 = 0$\\
		&\\
		\hline
		&\\
		2 is an eigenvalue with multiplicity 4.
		& \textbf{True}. It is noted that the matrix of $T$ is an upper triangular matrix having the value 2 along its principal diagonal and hence 2 is an eigenvalue with algebraic multiplicity 4.\\
		\hline
	\end{tabular}
\caption{}
\label{eq:solutions/2018/dec/75/table}
\end{table}
