See Tables     \ref{eq:solutions/2015/june/71/table:1}
and     \ref{eq:solutions/2015/june/71/table:2}.

\onecolumn
\begin{longtable}{|l|l|}
	\hline
	\multirow{3}{*}{Proof 1}	& \\
	&Let $\vec{A}x$=0 and $\mathbb{N(\vec{A})}$ is the null space of $\vec{A}$\\
	&\\
	$Rank(\vec{A})=Rank(\vec{A}^T\vec{A})$&Then $\vec{A}^T\vec{A}$x=0 which means $\mathbb{N(\vec{A})}\subset \mathbb{N(\vec{A}^T\vec{A})}$ \\
	&\\
	&Thus if $\vec{A}^T\vec{A}$x=0 ,then\\
	&\\
	&$x^T\vec{A}^T\vec{A}x=0\implies\lVert\vec{A}x\rVert=0$\\
	&\\
	&Which means $\vec{A}x=0$ thus\\
	&\\
	& $\mathbb{N(\vec{A}^T\vec{A})}\subset\mathbb{N(\vec{A})}$\\
	&\\
	&From the Above two condition we can say that ${N(\vec{A}^T\vec{A})}=\mathbb{N(\vec{A})}$\\
	&\\
	&$rank(\vec{A})=n-\mathbb{N(\vec{A})}$\\
	&\\
	&$rank(\vec{A})=rank(\vec{A}^T\vec{A})$\\
	&\\
	&Hence Proved.\\
	&\\
	\hline
	\multirow{3}{*}{Proof 2} 
	&\\
	&Suppose $\vec{A}=\myvec{\vec{a_1}&\hdots&\vec{a_n}}$ where $\vec{a_i}$ is the column vector of $\vec{A}$\\
	&\\
	Rowspace$(\vec{A}^T\vec{A})$=Rowspace($\vec{A}$) & $\vec{A}^T\vec{A}=\vec{A}^T\myvec{\vec{a_1}&\hdots&\vec{a_n}}=\myvec{\vec{A}^T\vec{a_1}&\hdots\vec{A}^T\vec{a_n}}$\\
	&\\
	&For each column of $\vec{A}^T\vec{A}$\\
	&\\
	&$\vec{A}^T\vec{a_i}=\myvec{\vec{b_1}&\hdots\vec{b_n}}\vec{a_i}$where $\vec{b_i}$ is the column vector of $\vec{A}^T$ and Row of $\vec{A}$\\
	&\\
	&$=\myvec{\vec{b_1}&\hdots\vec{b_n}}\myvec{a_{i1}\\ \vdots \\a_{in}}=\sum_{j=1}^{n}a_{ij}b_j$\\
	&\\
	&So column of $\vec{A}^T\vec{A}$ is the linear combination of rows of $\vec{A}$.\\
	&\\
	&Since rank$(\vec{A}^T)$=rank$(\vec{A})$ so,\\
	&\\
	&Row$(\vec{A}^T\vec{A})=$Column$(\vec{A}^T\vec{A})$=Row$(\vec{A})$\\
	&\\
	&Hence Proved.\\

	&\\
\hline
  
    \caption{Proofs}
    \label{eq:solutions/2015/june/71/table:1}
\end{longtable}
\begin{longtable}{|l|l|}
	\hline
	\multirow{3}{*}{Option 1} & \\
	&From Proof 2,Set $S$ contained a set of matrix whose First Column is Non-zero. \\ 
    & \\
    Nilpotent Matrix check&$S\in$ Set$\myvec{1&0&0\\0&0&0\\0&0&0}$,$\myvec{0&0&0\\1&0&0\\0&0&0}$,$\myvec{0&0&0\\0&0&0\\1&0&0}$\\
    &\\
    &Given $\vec{A}^T\vec{A}=\myvec{1&0&0\\0&0&0\\0&0&0}$\\
    &\\
    &So the only matrix $\vec{A}$ which satisfy $\vec{A}^T\vec{A}=\myvec{1&0&0\\0&0&0\\0&0&0}$, $\vec{A}^2=0$ such that $\vec{A}\in S$\\
    &\\
    &$\vec{A}=\myvec{0&0&0\\1&0&0\\0&0&0}\in S$\\
    &\\
    &$\vec{A}^T\vec{A}=\myvec{0&1&0\\0&0&0\\0&0&0}\myvec{0&0&0\\1&0&0\\0&0&0}=\myvec{1&0&0\\0&0&0\\0&0&0}$\\
    
    &\\
    &$\vec{A}^2=\myvec{0&0&0\\1&0&0\\0&0&0}\myvec{0&0&0\\1&0&0\\0&0&0}=\myvec{0&0&0\\0&0&0\\0&0&0}$ which is a nilpotent matrix\\
    &\\
    &Option 1 is correct.\\
    &\\
    \hline
	\multirow{3}{*}{Option 2}
	& \\
    &In Proof 1 we already prove that $Rank(\vec{A})=Rank(\vec{A}^T\vec{A})$\\
    &\\
    matrix of rank one check &Since the $Rank(\vec{A}^T\vec{A})=1$ so the $Rank(\vec{A})=1$ \\ 
	&\\
	&There fore Set S always contains only Rank 1 matrices.\\
	&\\
	&Hence Option 2 is correct.\\
	&\\
	\hline
	\multirow{3}{*}{Option 3}
	&\\
    &Since set S contain only rank 1 matrices and none of rank 2 matrices \\
    &\\
    matrix of rank two check&as already proved above therefore\\
    &\\
    &Option 3 is incorrect.\\
    &\\
    
    \hline
	\multirow{3}{*}{Option 4}
	&\\
	&Proved by contradiction\\
	&\\
    non-zero skew .&Assume Rank of $\vec{A}$ is 1 so $\vec{A}$ can be written as $\vec{A}=\vec{u}\vec{v}^T$ for any non-zero\\
    &\\
    symmetric matrix check&Columns vectors $\vec{u}$ , $\vec{v}$ with n entries. If A is skew symmetric,we have:-\\
    &\\
    &$\vec{A}^T=-\vec{A}$\\
    &\\
    &$(\vec{u}\vec{v})^T=-\vec{u}\vec{v}^T$ $\vec{v}\vec{u}^T=-\vec{u}\vec{v}^T$\\
    &\\
    &The Column space of these matrices is same.The column space of $\vec{v}\vec{u}^T$\\ 
    &is span of $\vec{v}$,where as the column space of $\vec{u}\vec{v}^T$ is the span of $\vec{u}$,\\
    &\\
    &So we must have $\vec{v}=k\vec{u}$ for some $k\in\mathbb{R}$.So the equation becomes\\
    &\\
    &$k\vec{u}\vec{u}^T=-k\vec{u}\vec{u}^T$ \\
    &\\
    &and since $\vec{u}\neq 0$;We can conclude that k=0,which means $\vec{v}=0$ therefore $\vec{A}=0$.\\
    &\\
    &This Contradicts our assumption that $\vec{A}$has rank 1.\\
    &\\
    &Thus real skew symmentric matrix can never have rank=1.\\
    &\\
    &Hence option 4 is incorrect.\\
    &\\
	\hline
	\multirow{3}{*}{Answers}
	&\\
&Option 1 and Option 2 are correct.\\
&\\
	\hline
	
	\caption{Solution Table}
    \label{eq:solutions/2015/june/71/table:2}
\end{longtable}
\twocolumn
