See Tables \ref{eq:solutions/2014/dec/75/deftab}
 and \ref{eq:solutions/2014/dec/75/deftab1}.

\onecolumn
\begin{longtable}{|l|l|}
\hline
\endhead
\textbf{Theorem}&If $\vec{M}$ and $\vec{N}$ are two matrices whose ranks are $rank(\vec{M})$ and $rank(\vec{N})$ respectively. Then\\&\parbox{14cm}{\begin{align}
    rank(\vec{M}+\vec{N})\leq rank(\vec{M})+rank(\vec{N})\label{eq:solutions/2014/dec/75/prop}
\end{align}}\\
\hline
\caption{Definitions and theorem used}
\label{eq:solutions/2014/dec/75/deftab}
\end{longtable}

\begin{longtable}{|l|l|l|}
\hline
\endhead
\textbf{Option}&\textbf{Solution}&\textbf{True/}\\&&\textbf{False}\\
\hline
1.&Given matrix $\vec{A}$ has rank $r$ and $\vec{B}$ has rank s.&\\&Also given matrix $\vec{B}$ is obtained by changing only one element of $\vec{A}$.&\\&Lets assume another matrix $\vec{P}$ whose addition to matrix $\vec{A}$ results to matrix $\vec{B}$&\\&as below.&\\&\parbox{14cm}{\begin{align}
    \vec{A}+\vec{P}=\vec{B}\label{eq:solutions/2014/dec/75/APB}
\end{align}}&\\&Since matrix $\vec{P}$ consists only single element we can say that $rank(\vec{P})=1$&True\\&From \eqref{eq:solutions/2014/dec/75/prop}, \eqref{eq:solutions/2014/dec/75/APB}, we get&\\&\parbox{14cm}{\begin{align}
    rank(\vec{A}+\vec{P})&\leq rank(\vec{A})+rank(\vec{P})\\
    \implies rank(\vec{B})&\leq rank(\vec{A})+rank(\vec{P})\\
    \implies s&\leq r+1\label{eq:solutions/2014/dec/75/p1}
\end{align}}&\\&\textbf{Example:}&\\&Let matrices $\vec{A}$ and $\vec{B}$ be as below&\\&\parbox{14cm}{\begin{align}
    \vec{A}=\myvec{2&-3&6&2&5\\-2&3&-3&-3&-4\\4&-6&9&5&9\\-2&3&3&-4&1\\6&-9&12&8&13}\label{eq:solutions/2014/dec/75/A}\\
    \vec{B}=\myvec{2&-3&6&2&5\\-2&3&-3&-3&4\\4&-6&9&5&9\\-2&3&3&-4&1\\6&-9&12&8&13}\label{eq:solutions/2014/dec/75/B}
\end{align}}&\\&lets calculate rank of matrix $\vec{A}$&\\&\parbox{14cm}{\begin{align}
    &\myvec{2&-3&6&2&5\\-2&3&-3&-3&-4\\4&-6&9&5&9\\-2&3&3&-4&1\\6&-9&12&8&13}\xleftrightarrow[R_3\leftarrow R_3-2R_1]{R_2\leftarrow R_2+R_1}\myvec{2&-3&6&2&5\\0&0&3&-1&1\\0&0&-3&1&-1\\-2&3&3&-4&1\\6&-9&12&8&13}\\&\xleftrightarrow[R_5\leftarrow R_5-3R_1]{R_4\leftarrow R_4+R_1}\myvec{2&-3&6&2&5\\0&0&3&-1&1\\0&0&-3&1&-1\\0&0&9&-2&6\\0&0&-6&2&-2}\xleftrightarrow[R_5\leftarrow R_5-2R_3]{R_4\leftarrow R_4+3R_3}\myvec{2&-3&6&2&5\\0&0&3&-1&1\\0&0&-3&1&-1\\0&0&0&1&3\\0&0&0&0&0}\\&\xleftrightarrow[]{R_3\leftarrow R_3+R_1}\myvec{2&-3&6&2&5\\0&0&3&-1&1\\0&0&0&0&0\\0&0&0&1&3\\0&0&0&0&0}\xleftrightarrow[]{R_3\leftrightarrow R_4}\myvec{2&-3&6&2&5\\0&0&3&-1&1\\0&0&0&1&3\\0&0&0&0&0\\0&0&0&0&0}
\end{align}}&\\&\parbox{14cm}{\begin{align}
    \implies rank(\vec{A})=3=r\label{eq:solutions/2014/dec/75/ra}
\end{align}}&\\&Now lets calculate rank of matrix $\vec{B}$&\\&\parbox{14cm}{\begin{align}
    &\myvec{2&-3&6&2&5\\-2&3&-3&-3&4\\4&-6&9&5&9\\-2&3&3&-4&1\\6&-9&12&8&13}\xleftrightarrow[R_3\leftarrow R_3-2R_1]{R_2\leftarrow R_2+R_1}\myvec{2&-3&6&2&5\\0&0&3&-1&9\\0&0&-3&1&-1\\-2&3&3&-4&1\\6&-9&12&8&13}\\
    &\xleftrightarrow[R_5\leftarrow R_5-3R_1]{R_4\leftarrow R_4+R_1}\myvec{2&-3&6&2&5\\0&0&3&-1&9\\0&0&-3&1&-1\\0&0&9&-2&6\\0&0&-6&2&-2}\xleftrightarrow[R_5\leftarrow R_5-2R_3]{R_4\leftarrow R_4+3R_3}\myvec{2&-3&6&2&5\\0&0&3&-1&9\\0&0&-3&1&-1\\0&0&0&1&3\\0&0&0&0&0}
\end{align}}&\\&\parbox{14cm}{\begin{align}
    \implies rank(\vec{B})=4=s\label{eq:solutions/2014/dec/75/rb}
\end{align}}&\\&Now matrix $\vec{P}$ will be&\\&\parbox{14cm}{\begin{align}
    \vec{P}&=\vec{B}-\vec{A}\\
    \implies\vec{P}&=\myvec{0&0&0&0&0\\0&0&0&0&8\\0&0&0&0&0\\0&0&0&0&0\\0&0&0&0&0}\label{eq:solutions/2014/dec/75/P}\\
    \implies rank(\vec{P})&=1
\end{align}}&\\&Now we will see equation \eqref{eq:solutions/2014/dec/75/p1} is satisfied or not&\\&\parbox{14cm}{\begin{align}
    s\leq r+1\implies4\leq3+1\implies4\leq4
\end{align}}&\\&Hence satisfied&\\
\hline2.&From \eqref{eq:solutions/2014/dec/75/APB}, If $\vec{P}=-\vec{Q}$ then we can get as below&\\&\parbox{14cm}{\begin{align}
    \vec{A}-\vec{Q}=\vec{B}\\
    \implies\vec{B}+\vec{Q}=\vec{A}\label{eq:solutions/2014/dec/75/BQP}
\end{align}}&\\&Since matrix $\vec{Q}$ also consists only single element we can say that $rank(\vec{Q})=1$&True\\&From \eqref{eq:solutions/2014/dec/75/prop}, \eqref{eq:solutions/2014/dec/75/BQP}, we get&\\&\parbox{14cm}{\begin{align}
    rank(\vec{B}+\vec{Q})&\leq rank(\vec{B})+rank(\vec{Q})\\
    \implies rank(\vec{A})&\leq rank(\vec{B})+rank(\vec{Q})\\
    \implies r&\leq s+1\\
    \implies r-1&\leq s\label{eq:solutions/2014/dec/75/p2}
\end{align}}&\\&\textbf{Example:}&\\&Let matrix $\vec{A}$ and $\vec{B}$ are considered same as in \eqref{eq:solutions/2014/dec/75/A}, \eqref{eq:solutions/2014/dec/75/B}&\\&From \eqref{eq:solutions/2014/dec/75/ra} and \eqref{eq:solutions/2014/dec/75/rb} we got&\\&\parbox{14cm}{\begin{align}
    rank(\vec{A})=r=3\\
    rank(\vec{B})=s=4\\
\end{align}}&\\&Here matrix $\vec{Q}$ will be&\\&\parbox{14cm}{\begin{align}
    \vec{Q}&=\vec{A}-\vec{B}\\
    \implies\vec{Q}&=\myvec{0&0&0&0&0\\0&0&0&0&-8\\0&0&0&0&0\\0&0&0&0&0\\0&0&0&0&0}\implies\vec{Q}=-\vec{P}\\
    \implies rank(\vec{Q})&=1
\end{align}}&\\&Now we will see equation \eqref{eq:solutions/2014/dec/75/p2} is satisfied or not&\\&\parbox{14cm}{\begin{align}
    r-1\leq s\implies3-1\leq4\implies2\leq4
\end{align}}&\\&Hence satisfied&\\
\hline
3.&Let matrix $\vec{A}$ be identity matrix then $rank(\vec{A})$ is 5 and matrix $\vec{B}$ can be&\\&\parbox{14cm}{\begin{align}
    \vec{A}=\vec{I}_{5\times5}\label{eq:solutions/2014/dec/75/eq1}\\
    \vec{B}=\myvec{1&1&0&0&0\\0&1&0&0&0\\0&0&1&0&0\\0&0&0&1&0\\0&0&0&0&1}\label{eq:solutions/2014/dec/75/eq2}
\end{align}}&False\\&Then $rank(\vec{B})$ is also 5.Therefore $s=r-1$ is always not true.&\\
\hline
4.&Similarly from \eqref{eq:solutions/2014/dec/75/eq1},\eqref{eq:solutions/2014/dec/75/eq2} we can say that $s\neq r$ is not true always.&False\\&&\\
\hline
\caption{Solution}
\label{eq:solutions/2014/dec/75/deftab1}
\end{longtable}
\twocolumn
