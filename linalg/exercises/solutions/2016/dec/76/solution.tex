We can represent our matrix as:
\begin{align}
    \vec{A}=\myvec{0&1&0&.....&0&0\\0&0&1&.....&0&0\\.\\.\\0&0& 0&.....&0&1\\1&0&0&.....&0&0}\label{eq:solutions/2016/dec/76/1}\\
    \vec{A^T}=\myvec{0&0&0&.....&0&1\\1&0&0&.....&0&0\\0&1& 0&.....&0&0\\.\\.\\0&0&0&....&1&0}\label{eq:solutions/2016/dec/76/2}
    \end{align}
$\vec{A}$ is our given matrix.We know that Characteristic Equation of $\vec{A}$ and $\vec{A^T}$ is same.Consider the minimal polynomial
\begin{align}
    x^n+a_{n-1}x^{n-1}+a_{n-2}x^{n-2}+....+a_0\label{eq:solutions/2016/dec/76/4}
\end{align}
We can represent it in $n\times n$ matrix with 1's on sub-diagonals and in last column it has negative of the coefficient,and rest all 0.We represent it using $\vec{C}$.It is known as the companion matrix.
\begin{align}
     \vec{C}=\myvec{0&0&0&....&0&-a_0\\1&0&0&....&0&-a_1\\0&1& 0&....&0&-a_2\\.\\.\\0&0&0&....&1&-a_{n-1}}\label{eq:solutions/2016/dec/76/3}
\end{align}
\eqref{eq:solutions/2016/dec/76/4} is also the characteristic equation of $\vec{C}$\\
Comparing \eqref{eq:solutions/2016/dec/76/2} with \eqref{eq:solutions/2016/dec/76/3} we get:
\begin{align}
    a_0=-1, a_1=a_2=a_3=a_4=...=a_{n-1}=0\label{eq:solutions/2016/dec/76/5}
\end{align}
Substituting \eqref{eq:solutions/2016/dec/76/5} into \eqref{eq:solutions/2016/dec/76/4} we get:
\begin{align}
    x^n-1=0\label{eq:solutions/2016/dec/76/6}
\end{align}
By Cayley-Hamilton Theorem:
\begin{align}
    \lambda^n-1=0\\
\end{align}
 $\lambda$=$n^{th}$ roots of unity. See Table \ref{eq:solutions/2016/dec/76/table1}.

\begin{table*}[ht!]
\begin{center}
\begin{tabular}{|c|c|}
\hline
\textbf{Options} & \textbf{Explanation} \\
\hline
\text{$\vec{A}$ has 1 as an eigen value} & 
One value out of the $n^{th}$ roots of unity is 1.So,correct \\
\hline
\text{$\vec{A}$ has -1 as an eigen value} & 
Since,$n$ is odd.So,-1 cannot be one of the value of $n^{th}$ roots of unity.\\& Hence,incorrect \\
\hline
\text{$\vec{A}$ has atleast one eigenvalue}\\ with multiplicity $\geq2$& All values of $n^{th}$ roots of unity are distinct.\\
& So there is no eigenvalue with multiplicity $\geq2$.\\
& Hence,incorrect.
\\
\hline
\text{$\vec{A}$ has no real eigen values}
& One of the value is 1,which is real.\\
& Hence,incorrect.
\\
\hline
\end{tabular}
\caption{Finding Correct Option}
\label{eq:solutions/2016/dec/76/table1}
\end{center}
\end{table*}

 
