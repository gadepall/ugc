See Tables \ref{eq:solutions/2014/dec/29/table:1} and \ref{eq:solutions/2014/dec/29/table:2}


\onecolumn
\begin{longtable}{|l|l|}
\hline
\multirow{3}{*}{Given} & \\
& n x n permutation matrix\\
& $\myvec{ & & & & & &1\\ & & & & &1& \\ & & & &.& & \\ & & &.& & & \\ & &.& & & & \\ &1& & & & & \\1& & & & & &  }$\\
&\\
\hline
\multirow{3}{*}{Proof of row exchange} & \\
& The given n x n permutation matrix can be converted into \\
& identity matrix of n x n dimension by doing row exchange\\
& operations.\\
& \\
& Let $\vec{A} = \myvec{a_1\\.\\.\\a_i\\a_j\\.\\.\\a_n}$\\
& $ \begin{vmatrix}
a_1\\.\\.\\a_i+a_j\\a_i+a_j\\.\\.\\a_n 
\end{vmatrix} = 0$\\
& since determinant of a any matrix will be zero,\\
& if it has dependent rows.\\
& Expanding the above using linear property of determinants\\
& $ \begin{vmatrix} a_1\\.\\.\\a_i\\a_i\\.\\.\\a_n \end{vmatrix} +
\begin{vmatrix} a_1\\.\\.\\a_j\\a_i\\.\\.\\a_n \end{vmatrix} +
\begin{vmatrix} a_1\\.\\.\\a_i\\a_j\\.\\.\\a_n \end{vmatrix} +
\begin{vmatrix} a_1\\.\\.\\a_j\\a_j\\.\\.\\a_n \end{vmatrix} = 0$\\
& $\implies 0 +
\begin{vmatrix} a_1\\.\\.\\a_j\\a_i\\.\\.\\a_n \end{vmatrix} +
\begin{vmatrix} a_1\\.\\.\\a_i\\a_j\\.\\.\\a_n \end{vmatrix} +
0 = 0$\\
& $\implies \begin{vmatrix} a_1\\.\\.\\a_j\\a_i\\.\\.\\a_n \end{vmatrix} = 
(-1)\begin{vmatrix} a_1\\.\\.\\a_i\\a_j\\.\\.\\a_n \end{vmatrix}$\\
& Hence it is proved that the exchange of rows $a_i \  and \  a_j$ changes the \\
& sign of the determinant.\\
& \\
& $\therefore$ for every row exchange in given permutation matrix\\
& the determinant gets multiplied by -1.\\
& \\
\hline
\multirow{3}{*}{finding no of exchanges} & \\
& Let $\vec{A}= \myvec{a_1&.&a_i&a_{i+1}.&a_n}$\\
& if n is even number then the elements $a_1 \  to \  a_i$ will be exchanged with\\
& $a_{i+1} \  to \ a_n$  where $i = \frac{n}{2} =  \lfloor \frac{n}{2} \rfloor $.\\
& if n is odd, the center element will be $a_{i+1}$ where $i+1 = \lceil \frac{n}{2} \rceil$\\
& then $i = \lfloor \frac{n}{2} \rfloor$  and the elements $a_1 \  to \  a_i$ will be exchanged with\\
& $a_{i+2} \  to \ a_n$.\\
& $\therefore$ The given n x n matrix requires $\lfloor \frac{n}{2} \rfloor$\\
& row exchanges to become identity matrix.\\
& \\
\hline
\multirow{3}{*}{finding determinant} & \\
& from the above results the determinant of given permutation matrix is\\
& $(-1)^{\lfloor \frac{n}{2} \rfloor}\begin{vmatrix}1& & & & & & \\ &1& & & & & \\ & &.& & & & \\ & & &.& & & \\ & & & &.& & \\ & & & & &1& \\ & & & & & &1  \end{vmatrix}$\\
& we know that the determinant of identity matrix, $det(\vec{I})=1$\\
& $\therefore$ the determinant of given n x n permutation matrix = $(-1)^{\lfloor \frac{n}{2} \rfloor}$\\
&\\
\hline
\multirow{3}{*}{Conclusion} & \\
& Option-2 is the right solution\\
&\\
\hline
\caption{Solution}
\label{eq:solutions/2014/dec/29/table:1}
\end{longtable}
\begin{longtable}{|l|l|}
\hline
\multirow{3}{*}{Example-1} & \\
& Let $\vec{A}$ is 5 x 5 permutation matrix, then\\
& $det(\vec{A})= \begin{vmatrix} & & & &1\\ & & &1& \\ & &1& & \\ &1& & & \\1& & & &    \end{vmatrix}$\\
& $ = \xleftrightarrow[]{R_1 \leftrightarrow R_5}
(-1)\begin{vmatrix}
1& & & & \\ & & &1& \\ & &1& & \\ &1& & & \\ & & & &1 
\end{vmatrix}$\\
& $ = \xleftrightarrow[]{R_2 \leftrightarrow R_4}
(-1)(-1)\begin{vmatrix}
1& & & & \\ &1& & & \\ & &1& & \\ & & &1& \\ & & & &1 
\end{vmatrix}$\\
& $ = 1$\\
& \\
& substituting n = 5 in the solution\\
& $(-1)^{\lfloor \frac{5}{2} \rfloor} = 1$\\
& \\
\hline
\multirow{3}{*}{Example-2} & \\
& Let $\vec{A}$ is 6 x 6 permutation matrix, then    \\
& $det(\vec{A})= \begin{vmatrix} & & & & &1\\ & & & &1& \\ & & &1& & \\ & &1& & & \\ &1& & & & \\1& & & & &    \end{vmatrix}$\\
& $ = \xleftrightarrow[]{R_1 \leftrightarrow R_6}
(-1)\begin{vmatrix}
1& & & & & \\ & & & &1& \\ & & &1& & \\ & &1& & & \\ &1& & & & \\ & & & & &1 
\end{vmatrix}$\\
& $ = \xleftrightarrow[]{R_2 \leftrightarrow R_5}
(-1)(-1)\begin{vmatrix}
1& & & & & \\ &1& & & & \\ & & &1& & \\ & &1& & & \\ & & & &1& \\ & & & & &1 
\end{vmatrix}$\\
& $ = \xleftrightarrow[]{R_3 \leftrightarrow R_4}
(-1)(-1)(-1)\begin{vmatrix}
1& & & & & \\ &1& & & & \\ & &1& & & \\ & & &1& & \\ & & & &1& \\ & & & & &1 
\end{vmatrix}$\\
& $ = -1$\\
& \\
& substituting n = 6 in the solution\\
& $(-1)^{\lfloor \frac{6}{2} \rfloor} = -1$\\
& \\
& Hence the proved that the solution is correct.\\
& \\
\hline
\caption{Example}
\label{eq:solutions/2014/dec/29/table:2}
\end{longtable}
\twocolumn
